\documentclass[]{article}
\usepackage[margin=1.0in]{geometry}
\usepackage{amssymb}

%title material
\title{Astronomy 400B Homework \#3}
\author{Please Show Your Work for Full Credit}
\date{Due March 4, 2015 by 9:35am}


%include latex definitions

%average
\newcommand{\ave}[1]{\langle#1\rangle}
%radian
\newcommand{\rad}{\mathrm{rad}}

%astronomical unit
\newcommand{\AU}{\mathrm{AU}}

%centimeter
\newcommand{\cm}{\mathrm{cm}}

%meter
\newcommand{\m}{\mathrm{m}}

%kilometer
\newcommand{\km}{\mathrm{km}}

%parsec
\newcommand{\pc}{\mathrm{pc}}

%kiloparsec
\newcommand{\kpc}{\mathrm{kpc}}

%megaparsec
\newcommand{\Mpc}{\mathrm{Mpc}}

%gigaparsec
\newcommand{\Gpc}{\mathrm{Gpc}}

%light year
\newcommand{\ly}{\mathrm{ly}}

%second
\newcommand{\s}{\mathrm{s}}
\newcommand{\yr}{\mathrm{yr}}
\newcommand{\Gyr}{\mathrm{Gyr}}

%solar mass
\newcommand{\Msun}{M_{\odot}}

%grams
\newcommand{\g}{\mathrm{g}}

%erg
\newcommand{\erg}{\mathrm{erg}}

%solar luminosity
\newcommand{\Lsun}{L_{\odot}}

%jansky 
\newcommand{\Jy}{\mathrm{Jy}}

%flux density
\newcommand{\Fnu}{F_{\nu}}
\newcommand{\Flambda}{F_{\lambda}}
\newcommand{\Lnu}{L_{\nu}}

%hertz 
\newcommand{\Hz}{\mathrm{Hz}}

%angstrom
\newcommand{\Ang}{\mathrm{\r{A}}}

%solar radius
\newcommand{\Rsun}{R_{\odot}}

%stefan-boltzmann
\newcommand{\sigmaSB}{\sigma_{\mathrm{SB}}}

%boltzmann
\newcommand{\kB}{k_{\mathrm{B}}}

%kelvin
\newcommand{\K}{\mathrm{K}}

%T_bandpass
\newcommand{\TBP}{T_{\mathrm{BP}}}

%magnitude
\newcommand{\mg}{\mathrm{mag}}


%arcsecond
\newcommand{\arcsec}{\mathrm{arcsec}}

%begin the document
\begin{document}

%make the title, goes after document begins
\maketitle

\section{Sparke \& Gallagher Problem 3.20}

Effective potentials have many uses. The motion of a star around
a non-rotating black hole of mass $M_{BH}$ is given by
\begin{equation}
\left(\frac{dr}{d\tau}\right)^2 = E^2 - \left(c^2 - \frac{2GM_{BH}}{r}\right)\left(1 + \frac{L^2}{c^2 r^2}\right) \equiv E^2 - 2\Phi_{\mathrm{eff}}(r)
\end{equation}
\noindent
we can interpret $r$ as distance from the center, and $\tau$ as time. (More precisely $r$ is the usual Schwarzschild radial coordinate, $\tau$ is proper time for a static observer at radius $r$, and $E$ and $L$ are, respectively, the energy and angular momentum per unit mass as measured by that observer.) Show that there are no circular orbits at $r<3GM_{BH}/c^2$, and that the stable circular orbits lie at $r>6GM_{BH}/c^2$ with $L>2\sqrt{3}GM_{BH}/c$.

\section{Sparke \& Gallagher Problem 3.24}

Use the divergence theorem to show that the potential at height $z$ above a uniform sheet of matter with surface density $\Sigma$ is
\begin{equation}
\Phi(\vx) = 2 \pi G \Sigma|z|
\end{equation}
\noindent
Show that the vertical force does not depend on $z$, and check that $\nabla^2\Phi=0$ when $z\ne0$. Suppose that the mass of the Galaxy was all in a flat uniform disk; use the equation
\begin{equation}
\frac{d}{dz}\left[n(z)\sigma_z^2\right] = - \frac{\partial\Phi}{\partial z} n(z)
\end{equation}
\noindent
to find the density $n(z)$ of K dwarfs, assuming that they have a constant velocity dispersion $\sigma_z$. As in the Earth's atmosphere, where the acceleration of gravity is also nearly independent of height, show that $n(z)$ drops by a factor of $e$ as $|z|$ increase by $h_z = \sigma_z^2 / (2\pi G \Sigma)$. Estimate $h$ near the Sun, taking $\sigma_z = 20~\km~\s^{-1}$.

\section{Sparke \& Gallagher Problem 3.25}

For stars moving vertically in the Galactic disk, suppose the distribution function $f(z,v_z)$ to be given by the equation
\begin{equation}
f(E_z) = \frac{n_0}{\sqrt{2\pi\sigma^2}} \exp\left(-E_z/\sigma^2\right)~~\mathrm{for}~E_z<0.
\end{equation}
\noindent
When the disk is symmetric about the plane $z=0$, then $d\Phi(z)/dz=0$ at $z=0$, and we can choose $\Phi(0)=0$ too. Find the integral giving the density of stars $n(z)$: what is $n(0)$?

To construct a self-consistent model, let $\Phi(z)=\sigma^{2}\phi(z)$, and let the average mass of the stars be $m$; show from Poisson's equation that
\begin{equation}
2\frac{d^2\phi}{dy^2} = e^{-\phi},~~\mathrm{where}~y=z/z_0~~\mathrm{and}~~z_0^2 = \sigma^2/(8\pi G m n_0).
\end{equation}
\noindent
Integrate this once to find $d\phi/dy$, and then again (substituting $u=\exp[-\phi/2]$) to find $\phi(y)$ and hence $\Phi(z)$.
Show that the number density of stars is $n(z) = n_0 \mathrm{sech}^2[z/(2 z_0)]$. What is its approximate form at large $|z|$?

\end{document}