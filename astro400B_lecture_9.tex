\documentclass[]{article}
\usepackage[margin=1.0in]{geometry}
\usepackage{amssymb}

%title material
\title{Astronomy 400B Lecture 9: Elliptical Galaxies}
\author{Brant Robertson}
\date{March, 2015}


%include latex definitions

%average
\newcommand{\ave}[1]{\langle#1\rangle}
%radian
\newcommand{\rad}{\mathrm{rad}}

%astronomical unit
\newcommand{\AU}{\mathrm{AU}}

%micron
\newcommand{\mum}{\mu\mathrm{m}}

%millimeter
\newcommand{\mm}{\mathrm{mm}}

%centimeter
\newcommand{\cm}{\mathrm{cm}}

%meter
\newcommand{\m}{\mathrm{m}}

%kilometer
\newcommand{\km}{\mathrm{km}}

%parsec
\newcommand{\pc}{\mathrm{pc}}

%kiloparsec
\newcommand{\kpc}{\mathrm{kpc}}

%megaparsec
\newcommand{\Mpc}{\mathrm{Mpc}}

%gigaparsec
\newcommand{\Gpc}{\mathrm{Gpc}}

%light year
\newcommand{\ly}{\mathrm{ly}}

%second
\newcommand{\s}{\mathrm{s}}
\newcommand{\yr}{\mathrm{yr}}
\newcommand{\Gyr}{\mathrm{Gyr}}

%solar mass
\newcommand{\Msun}{M_{\odot}}

%grams
\newcommand{\g}{\mathrm{g}}

%erg
\newcommand{\erg}{\mathrm{erg}}

%solar luminosity
\newcommand{\Lsun}{L_{\odot}}

%jansky 
\newcommand{\Jy}{\mathrm{Jy}}

%flux density
\newcommand{\Fnu}{F_{\nu}}
\newcommand{\Flambda}{F_{\lambda}}
\newcommand{\Lnu}{L_{\nu}}

%hertz 
\newcommand{\Hz}{\mathrm{Hz}}

%angstrom
\newcommand{\Ang}{\mathrm{\r{A}}}

%solar radius
\newcommand{\Rsun}{R_{\odot}}

%stefan-boltzmann
\newcommand{\sigmaSB}{\sigma_{\mathrm{SB}}}

%boltzmann
\newcommand{\kB}{k_{\mathrm{B}}}

%kelvin
\newcommand{\K}{\mathrm{K}}

%T_bandpass
\newcommand{\TBP}{T_{\mathrm{BP}}}

%magnitude
\newcommand{\mg}{\mathrm{mag}}


%arcsecond
\newcommand{\arcsec}{\mathrm{arcsec}}

%critical density
\newcommand{\rhoc}{\rho_{\mathrm{crit}}}

%proton mass
\newcommand{\mproton}{m_{\mathrm{p}}}

%electron volt
\newcommand{\eV}{\mathrm{eV}}

%kiloelectron volt
\newcommand{\keV}{\mathrm{keV}}

%megaelectron volt
\newcommand{\MeV}{\mathrm{MeV}}

%gigaelectron volt
\newcommand{\GeV}{\mathrm{GeV}}

%vector velocity
\newcommand{\vv}{\mathbf{v}}

%vector radius
\newcommand{\vr}{\mathbf{r}}

%vector position
\newcommand{\vx}{\mathbf{x}}

%vector force
\newcommand{\vF}{\mathbf{F}}

%vector surface
\newcommand{\vS}{\mathbf{S}}

%vector angular momentum
\newcommand{\vL}{\mathbf{L}}

%script I -- integral of motion
\newcommand{\cI}{\mathcal{I}}

%effective potential
\newcommand{\Phieff}{\Phi_\mathrm{eff}}


%begin the document
\begin{document}

%make the title, goes after document begins
\maketitle

%first section
\section{Photometry of Ellipticals}

The surface brightness profile of elliptical galaxies typically is similar to the Sersic profile.  Re-writting
the standard Sersic profile in terms of an effective radius $\Reff$, we have

\begin{equation}
I(R) = I(\Reff)\exp\{-b[(R/\Reff)^{1/n}-1]\}
\end{equation}
\noindent
where $b\approx 1.999n-0.327$ for $n>1$ is a constant that gives $\Reff$ containing half the light.


\section{Shapes of Ellipticals}

The density of stars in an oblate spheroid $\rho(\vx)$ can be written
\begin{equation}
\label{eqn:spheroid}
\rho(\vx) = \rho(m^2)~~\mathrm{where}~~m^2 = \frac{x^2 + y^2}{A^2}+\frac{z^2}{B^2}
\end{equation}
\noindent
and $A>B>0$.  The isodensity contours are ellipsoids with $m^2 = $constant.

{\bf See figure 6.8 of Sparke and Gallagher}

When determining the axis ratio $q=b/a$, w place the observer in the $x-z$ plane
looking at an angle $0^\circ < i < 90^\circ$ to the $z$ axis.
The line of sight hits the isodensity surface $m^2 = x^2 / A^2 + z^2/B^2$ 
at the tangent point T, so we have
\begin{equation}
\tan i = dx/dz = -(z/x)(A^2/B^2)
\end{equation}
\noindent
The elliptical image has a semi-major axis $a=mA$.  It's semi-minor
axis is $b=OR=OQ\sin i$.  We then have
\begin{equation}
OQ = OP+PQ = z+(-x)\cot i = B^2m^2/z
\end{equation}
\noindent
and the observed axis ratio is
\begin{equation}
q_{obl} \equiv \frac{b}{a} = \frac{OQ \sin i}{mA} = \frac{B^2 m}{zA}\sin i = \left(\frac{B^2}{A^2} + \cot^2 i\right)^{1/2}\sin i
\end{equation}
\noindent
therefore
\begin{equation}
q_{obl}^2 = (b/a)^2 = (B/A)^2\sin^2 i + \cos^2 i
\end{equation}
\noindent
which means that an oblate spheroid never appears more flattened than its true axis ratio $A/B$.
A prolate spheroid can be described by Equation \ref{eqn:spheroid} with $A<B$.  In this case
\begin{equation}
q_{prol}^2 = [(B/A)^2 \sin^2 i + \cos^2 i]^{-1}
\end{equation}
\noindent
and $q_{prol}>A/B$.  The flattening is not more than the true axis ratio.

We can integrate over inclination $i$ to determine the distribution $f(q)$ of observed axis
ratios in the range $q +\Delta q$ given true axes $A,B$.  In the oblong case:
\begin{equation}
f_{obl}(q) \Delta q = \frac{\sin i \cdot \Delta q}{|dq/di|} = \frac{q\Delta q}{\sqrt{1-(B/A)^2}\sqrt{q^2-(B/A)^2}}
\end{equation}
\noindent
For very flattened systems ($B\ll A$) this distribution is almost uniform.  
For disks, there are not many as flattened as $q=0.1$. There
are no ellipticals flatter than $q=0.3$ (E7).
Galaxies with $M_B~-20$ have $q\approx0.75$. Galaxies with $M_B<-20$ have $q\approx0.85$.

Some bright galaxies are probably triaxial
\begin{equation}
\rho(\vx) = \rho(m^2)~~\mathrm{where}~~m^2 = \frac{x^2}{A^2} + \frac{y^2}{B^2} + \frac{z^2}{C^2}
\end{equation}
\noindent
The cross-section in any plane perpendicular to the principal axes is an ellipse.


\section{Isophotal Shapes}

The isophotes of ellipticals can change their shape as a function of
radius.  There are {\it disky} ellipticals with extra light along the major axis.
Others appear {\it boxy} with more light in the ``corners'' of the ellipse.
Consider a given isophote.  We can write the ellipse of that best matches this isophote as
\begin{equation}
x = a \cos t;~y=b\cos t
\end{equation}
\noindent
Let $\Delta r(t)$ describe the distance between the ellipse and the isophote.  We
can write
\begin{equation}
\Delta r(t) \approx \Sigma_{k\ge3} a_k \cos(k t) + b_k \sin(kt)
\end{equation}
\noindent
where the $k<3$ terms vanish because we are considering the best-fit ellipse.
Terms $a_3$ and $b_3$ are usually small and describe ``egg-shaped'' isophotes.
The term $b_4$ is usually small but $a_4$ is not.  If $a_4>0$, the
isophote is disky and lies beyond the ellipse on the major and minor axes, 
whereas $a_4<0$ gives a boxy isophote that bulges out at $45^{\circ}$ relative
to the major and minor axes.

\section{Fundamental Relations of Elliptical Galaxies}

There is a correlation between elliptical galaxy luminosity and
velocity dispersion called the Faber-Jackson relation
\begin{equation}
\frac{L_V}{2\times10^{10}\Lsun} \approx \left( \frac{\sigma}{200~\km~\s^{-1}}\right)^{4}.
\end{equation}

There is another relation called the {\it Fundamental Plane} that
connects the effective radius, surface brightness within the effective
radius, and velocity dispersion as
\begin{equation}
\Reff \propto \sigma^{1.2} I_{e}^{-0.8}.
\end{equation}
Note that expectations would suggest $\Reff \propto \sigma^2 I_{e}^{-1}$.

\section{Rotation in Ellipticals}

Consider the equation
\begin{equation}
\frac{d}{dt}(m_\alpha \vv_\alpha) = - \sum_{\beta; \alpha \ne \beta} \frac{Gm_\alpha m_\beta}{|\vx_\alpha - \vx_\beta|^3}(\vx_\alpha - \vx_\beta)
\end{equation}
\noindent
Now multiply by the $z$-component by $z$ and sum over stars to find
\begin{equation}
\sum_{\alpha} \frac{d}{dt}(m_\alpha v_{z\alpha})z_\alpha = - \sum_{\alpha,\beta; \alpha \ne \beta} \frac{Gm_\alpha m_\beta}{|\vx_\alpha - \vx_\beta|^3}(z_\alpha - z_\beta)z_\alpha
\end{equation}
\noindent
But we could have started with star $\beta$
\begin{equation}
\sum_{\beta} \frac{d}{dt}(m_\beta v_{z\beta})z_\beta = - \sum_{\alpha,\beta; \alpha \ne \beta} \frac{Gm_\alpha m_\beta}{|\vx_\alpha - \vx_\beta|^3}(z_\beta - z_\alpha)z_\beta
\end{equation}
\noindent
Now, average these equations
\begin{equation}
\frac{1}{2}\frac{d^2 I_{zz}}{dt^2} = 2KE_{zz} + PE_{zz}
\end{equation}
\noindent
where
\begin{equation}
I_{zz} \equiv \sum_{\alpha} m_\alpha z_\alpha z_\alpha,
\end{equation}
\noindent
the $z$-direction kinetic energy is
\begin{equation}
KE_{zz} \equiv \frac{1}{2} \sum_{\alpha} m_{\alpha} v_{z\alpha} v_{z\alpha}
\end{equation}
and the $z$ contribution to the potential energy is
\begin{equation}
PE_{zz} \equiv -\sum_{\alpha,\beta;\alpha\ne\beta}\frac{1}{2} \frac{Gm_\alpha m_\beta}{|\vx_\alpha - \vx_\beta|^3}(z_\alpha-z_\beta)^2.
\end{equation}
\noindent
If all the stars are bound, we can write
\begin{equation}
2\ave{KE_{zz}} + \ave{PE_{zz}} = 0
\end{equation}
\noindent
Assume the galaxy is axisymmetric and rotates about the $z$ direction.  If the rotation speed
$V$ and velocity dispersions $\sigma_x$ and $\sigma_z$ are almost constant, then
\begin{equation}
\frac{\ave{PE_{zz}}}{\ave{PE_{xx}}} = \frac{\ave{KE_{zz}}}{\ave{KE_{xx}}} \approx \frac{\sigma_z^2}{\frac{1}{2}V^2 + \sigma_x^2}
\end{equation}
\noindent
The ratio of potential energies depend only on $B/A$ or $\epsilon = 1-B/A$.  Roughly
\begin{equation}
\frac{\ave{PE_{zz}}}{\ave{PE_{xx}}} \approx (B/A)^{0.9} = (1-\epsilon)^{0.9}
\end{equation}
\noindent
The max velocity is less than the average velocity, $V_{max}\approx\pi V/4$.
If the random motions are isotropic, then we have
\begin{equation}
\left(\frac{V_{max}}{\sigma}\right) =\left(\frac{V}{\sigma}\right)_{iso} \equiv \frac{\pi}{4}\sqrt{2[(1-\epsilon)^{-0.9} -1]} \approx \sqrt{\epsilon/(1-\epsilon)}
\end{equation}
\noindent
Even fairly spherical galaxies should rotate fast -- for example, $B/A=0.7$ should imply $V_{max}/\sigma\approx0.68$. But real galaxies are no where near that {\bf see Figure 6.14 of Sparke and Gallagher}.

\section{Super Massive Black Holes}
There are supermassive black holes at the centers of ellipticals. The
circular velocities near the black holes will exceed the typical velocity 
dispersion of the ellipticals if you get close enough.  In other words
\begin{equation}
V^2(r) \approx \frac{GM_{BH}}{r} \gtrsim \sigma_c^2
\end{equation}
\noindent
so you'd have to get within
\begin{equation}
r_{BH} \approx 45~\pc \times \left(\frac{M_{BH}}{10^8\Msun}\right) \times \left(\frac{\sigma_c^2}{100~\km~\s^{-1}}\right)^{-2}
\end{equation}
\noindent
to see the effect of the black hole.  
We find that for bulges
\begin{equation}
\log_{10}\left(M_{BH}/M_{\odot}\right) = \alpha + \beta \log_{10} (\sigma/\sigma_0)
\end{equation}
\noindent
with $\beta\approx4.02$ and $\alpha\approx8.13$.
\end{document}