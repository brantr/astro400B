\documentclass[]{article}
\usepackage[margin=1.0in]{geometry}
\usepackage{amssymb}

%title material
\title{Astronomy 400B Lecture 11: Structure Formation and Cosmology}
\author{Brant Robertson}
\date{March, 2015}


%include latex definitions

%average
\newcommand{\ave}[1]{\langle#1\rangle}
%radian
\newcommand{\rad}{\mathrm{rad}}

%astronomical unit
\newcommand{\AU}{\mathrm{AU}}

%micron
\newcommand{\mum}{\mu\mathrm{m}}

%millimeter
\newcommand{\mm}{\mathrm{mm}}

%centimeter
\newcommand{\cm}{\mathrm{cm}}

%meter
\newcommand{\m}{\mathrm{m}}

%kilometer
\newcommand{\km}{\mathrm{km}}

%parsec
\newcommand{\pc}{\mathrm{pc}}

%kiloparsec
\newcommand{\kpc}{\mathrm{kpc}}

%megaparsec
\newcommand{\Mpc}{\mathrm{Mpc}}

%gigaparsec
\newcommand{\Gpc}{\mathrm{Gpc}}

%light year
\newcommand{\ly}{\mathrm{ly}}

%second
\newcommand{\s}{\mathrm{s}}
\newcommand{\yr}{\mathrm{yr}}
\newcommand{\Gyr}{\mathrm{Gyr}}

%solar mass
\newcommand{\Msun}{M_{\odot}}

%grams
\newcommand{\g}{\mathrm{g}}

%erg
\newcommand{\erg}{\mathrm{erg}}

%solar luminosity
\newcommand{\Lsun}{L_{\odot}}

%jansky 
\newcommand{\Jy}{\mathrm{Jy}}

%flux density
\newcommand{\Fnu}{F_{\nu}}
\newcommand{\Flambda}{F_{\lambda}}
\newcommand{\Lnu}{L_{\nu}}

%hertz 
\newcommand{\Hz}{\mathrm{Hz}}

%angstrom
\newcommand{\Ang}{\mathrm{\r{A}}}

%solar radius
\newcommand{\Rsun}{R_{\odot}}

%stefan-boltzmann
\newcommand{\sigmaSB}{\sigma_{\mathrm{SB}}}

%boltzmann
\newcommand{\kB}{k_{\mathrm{B}}}

%kelvin
\newcommand{\K}{\mathrm{K}}

%T_bandpass
\newcommand{\TBP}{T_{\mathrm{BP}}}

%magnitude
\newcommand{\mg}{\mathrm{mag}}


%arcsecond
\newcommand{\arcsec}{\mathrm{arcsec}}

%critical density
\newcommand{\rhoc}{\rho_{\mathrm{crit}}}

%proton mass
\newcommand{\mproton}{m_{\mathrm{p}}}

%electron volt
\newcommand{\eV}{\mathrm{eV}}

%kiloelectron volt
\newcommand{\keV}{\mathrm{keV}}

%megaelectron volt
\newcommand{\MeV}{\mathrm{MeV}}

%gigaelectron volt
\newcommand{\GeV}{\mathrm{GeV}}

%vector velocity
\newcommand{\vv}{\mathbf{v}}

%vector radius
\newcommand{\vr}{\mathbf{r}}

%vector position
\newcommand{\vx}{\mathbf{x}}

%vector force
\newcommand{\vF}{\mathbf{F}}

%vector surface
\newcommand{\vS}{\mathbf{S}}

%vector angular momentum
\newcommand{\vL}{\mathbf{L}}

%script I -- integral of motion
\newcommand{\cI}{\mathcal{I}}

%effective potential
\newcommand{\Phieff}{\Phi_\mathrm{eff}}


%begin the document
\begin{document}

%make the title, goes after document begins
\maketitle

%first section
\section{Structure Formation}

Galaxies are not randomly distributed in the
Poisson or white-noise sense. They have a 
clustering strength that changes the probability
that two galaxies will lie near one another.
For instance, if the mean number density of
galaxies is $n$, then the expected number
of galaxies in a volume $\Delta V_1$ is $\sim n\Delta V_1$
as long as $n\Delta V_1\ll1$.  However, the joint
probability of having galaxies in two volumes
$V_1$ and $V_2$ depends on the radial separation $r_{12}$
of the two volumes as
\begin{equation}
\Delta P = n^2 [1 + \xi(r_{12})]\Delta V_1 \Delta V_2.
\end{equation}
\noindent
where $\xi(r)$ is called the {\it correlation function}.
If $\xi(r)>0$, then galaxies are clustered (biased) and
if $\xi(r)<0$ then galaxies are anti-biased.

The form of the correlation function is very roughly
a power-law of the form
\begin{equation}
\xi(r) \approx (r/r_0)^{-\gamma}
\end{equation}
\noindent
where $r_0$ is called the {\it correlation length} and
$\gamma<0$.  This holds on scales $r\lesssim10 h^{-1}\mathrm{Mpc}$.
When $r<r_0$ then the probability of
finding galaxies proximate to one another is much
larger than for a Poisson distribution.

The correlation length for all galaxies is about $r_0\sim 5 h^{-1}\Mpc$, but
a bit larger for ellipticals.  The slope is $\gamma\sim1.7$.  At
$r\lesssim50h^{-1}\Mpc$, $\xi(r)\sim0$ and galaxies are roughly
uniformly randomly distributed.

\subsection{Power Spectrum}

The spatial power spectrum of galaxy counts is given by
\begin{equation}
P(\vk) \equiv \int \xi(\vr) \exp(i\vk\cdot\vr)d^3\vr = 4\pi \int_0^\infty \xi(r) \frac{\sin(kr)}{kr} r^2 dr,
\end{equation}
\noindent
such that the galaxy power spectrum is the Fourier transform of the correlation function.
Remember, small $k$ corresponds to large $r$.  The correlation function has
no units, so the power spectrum has dimensions of volume.

\subsection{Galaxy Number Density}

We can write the galaxy density at some location $\vx$ as
\begin{equation}
\rho(\vx) = \bar{\rho}[1+\delta(\vx)]
\end{equation}
\noindent
where $\bar{\rho}$ is the mean density and $\delta$ is the overdensity
\begin{equation}
\delta(\vx) = \frac{\rho(\vx) - \bar{\rho}}{\bar{\rho}}.
\end{equation}
We can calculate the statistics of the overdensity most readily
by averaging the over all spherical volumes of radius $R$.
We write this quantity $\delta_R$, and the variance of 
this quantity $\sigma_R^2 = \ave{\delta_R^2}$.  Defining
the dimensionless quanity $\Delta_k^2$ as the typical
fluctuation in over density in a volume $k^{-1}~h^{-1}\Mpc$
in radius as
\begin{equation}
\Delta_k^2 \equiv \frac{k^3 P(k)}{2\pi^2},
\end{equation}
\noindent
we then have $\sigma_R^2 \approx \Delta_k^2$ and if
$P(k)\propto k^n$, then $\sigma_R \propto R^{-(n+3)/2}$.
Then if $n>-3$, the universe is lumpiest on small scales.
We often characterize the power spectrum by the 
quantity $\sigma_8\equiv\sigma(R=8~h^{-1}\Mpc)$.  Today, we think 
$\sigma_8\approx0.8$.

Instead of taking the variance in a volume corresponding to
a radius $R$, we can take the variance $\sigma(M)$ on a mass scale
$M \approx 4\pi R^3  \bar{\rho}/3$.  Since $M \propto R^3$, we
have that $\sigma(M) \propto M^{-(n+3)/6}$.

\section{Expansion of a Homogenous Universe}

We know from the CMB that the universe is homogeneous on
large scales.  We can describe the distance between two
points in such a universe by the relation
\begin{equation}
ds^2 = R^2 \left( \frac{d\sigma^2}{1-k\sigma^2} + \sigma^2 d\theta^2 + \sigma^2 \sin^2 \theta d\phi^2\right)
\end{equation}
\noindent
where $\sigma, \theta, \phi$ are spherical polar coordinates in curved space.
The constant $k$ specifies the curvature of space.  For $k=1$ the universe
is {\it closed} with positive curvature.  If $k=-1$ the universe is
{\it open} with negative curvature.  If $k=0$ we have normal Euclidean
flat space.

We can include cosmic expansion just by setting $R = R(t)$. Galaxies
remain at points with fixed values of $\sigma, \theta, \phi$ called
the comoving coordinates.  Two systems a distance $d$ in this coordinate system
move apart from each other at the rate
\begin{equation}
V = \dot{d} = \frac{\dot{R}(t)}{R(t)}d \equiv H(t) d
\end{equation}
\noindent
where $H(t)$ is the Hubble parameter that has $H(t_0) = H_0$
today.

\section{Time in an Expanding Universe}
Owing to the finite speed of light, General Relativity
informs us that the distance between two events happening
at different times in different places depends on  the
motion of an observer.  But all observers will measure
the same proper time $\tau$ along a path through space
and time connecting the events.  The proper time
is found by integrating the relation
\begin{equation}
c^2 d\tau^2 = c^2 dt^2 - ds^2.
\end{equation}
\noindent
Light rays travel paths with $d\tau^2 = 0$, called
null geodesics.

Place yourself at the origin.  Then the light
received from a galaxy at comoving distance
$\sigma_e$ has followed the path
\begin{equation}
\frac{c dt}{R(t)} = -\frac{d\sigma}{\sqrt{1-k\sigma^2}}.
\end{equation}
\noindent
As the universe grows, light covers less and less 
comoving distance per unit time.

\section{Redshift}

Consider a light wave that leaves at time $t_e$
from a distance $\sigma_e$ and arrives today $t_0$.
We have that
\begin{equation}
c\int_{t_e}^{t_0} \frac{dt}{R(t)} = \int_0^{\sigma_e} \frac{d\sigma}{\sqrt{1-k\sigma^2}}.
\end{equation}
\noindent
Now consider another wave leaving at a later time $t_e + \Delta t_e$.
It arrives at $t + \Delta t$.  The comoving position hasn't changed
so we have that
\begin{equation}
\int_{t_e+\Delta t_e}^{t_0 + \Delta t} \frac{dt}{R(t)} = \int_{t_e}^{t_0} \frac{dt}{R(t)},~~\mathrm{so}~\frac{\Delta t_e}{R(t_e)} = \frac{\Delta t}{R(t_0)},
\end{equation}
\noindent
provided that $\Delta t \ll R(t)/\dot{R}(t)$ (e.g., a Hubble time).  So all processes appear to be
slowed by a factor $R(t_0)/R(t_e)$.

If the light was emitted with wavelength $\lambda_e = c \Delta t_e$, it is
received with wavelength $\lambda_{obs} = c \Delta t$.  The
redshift is then
\begin{equation}
1 + z = \frac{\lambda_{obs}}{\lambda_e} = \frac{R(t_0)}{R(t_e)}.
\end{equation}

\section{Expansion History}

The rate of expansion is determined by the gravitational effects
of the energy densities it contains.  We can model the expansion
using Newtonian physics first, and then use general relativity
to revise for a more correct answer.

Consider a sphere of radius $r$ at time $t$ when the 
universe has a typical density $\rho(t)$, and assume
$r$ is much less than any curvature radius in the 
universe.  Assuming symmetry about $r=0$, the
gravitational force at radius $r$ is just supplied by the
mass within the sphere.  If the sphere
is large enough that pressure forces are small, then
the force on a object of mass $m$ at radius $r$ is
\begin{equation}
m \frac{d^2 r}{dt^2} = - \frac{G m M(<r)}{r^2} = - \frac{4\pi Gm}{3}\rho(t)r.
\end{equation}
\noindent
The radius of the sphere of matter is expanding with the
rest of the universe, so $r\propto R(t)$.  The mass of the
cloud cancels, such that
\begin{equation}
\label{eqn:friedman_A}
\ddot{R}(t) = - \frac{4\pi G}{3} \rho(t) R(t).
\end{equation}
\noindent
The mass doesn't change, so $\rho(t) R(t)$ is a constant.
We can multiply by $\dot{R}(t)$ to find
\begin{equation}
\frac{1}{2} \frac{d}{dt}[\dot{R}^2(t)] = - \frac{4 \pi G}{3} \frac{\rho(t_0)R^3(t_0)}{R^2(t)} \dot{R}(t)
\end{equation}
\noindent
where $t_0$ is the present day.  We then integrate to find
\begin{equation}
\label{eqn:friedman_B}
\dot{R}^2(t) = \frac{8\pi G}{3}\rho(t) R^2(t) - kc^2
\end{equation}
\noindent
where $k$ is a constant of integration.  Turns out Equation
\ref{eqn:friedman_B} is valid in GR, and tells us
that $k$ is the same constant as in the metric.

We appeal to thermodynamics to tell us that
as heat $\Delta Q$ flows into a volume $V$ its
internal energy $E$ must increase, or it expands
and does work against pressure
\begin{equation}
\Delta Q = \Delta E + p \Delta V = V \Delta(\rho c^2) + (\rho c^2 + p)\Delta V
\end{equation}
\noindent
where $\rho$ includes all the forms of matter and energy.
But no volume $V$ gains heat at the expense of another and
\begin{equation}
\Delta Q = 0 = \Delta \rho + \left(\rho + \frac{p}{c^2}\right)\frac{\Delta V}{V}
\end{equation}
\noindent
or
\begin{equation}
\frac{d\rho}{dt} = - c \frac{\dot{R}(t)}{R(t)}\left( \rho + \frac{p}{c^2}\right)
\end{equation}
\noindent
Differentiating Equation \ref{eqn:friedman_B} and substituting for $d\rho/dt$
yields
\begin{equation}
\label{eqn:friedman_C}
\ddot{R}(t) = - \frac{4\pi G}{3} R(t) \left[ \rho(t) + \frac{3p(t)}{c^2}\right]
\end{equation}
\noindent
So in GR, the pressure $p$ adds to the gravitational attraction. 
Equations \ref{eqn:friedman_B} and \ref{eqn:friedman_C} are
called the {\it Friedmann equations}.  For cool matter
the pressure $p \sim \rho c_s^2$ where the sound speed $c_s \ll c$,
and the pressure term in equation \ref{eqn:friedman_C} is 0.
Then $\rho(t)\propto R^{-3}$.  For radiation and particles,
$p\approx pc^2/3$ and $\rho(t) \propto R^{-4}(t)$.
For matter and radiation $\rho + 3p/c^2$ is positive, and the
universe decelerates. 

The quantity $\rho(t)R^2(t)$ decreases as $R(t)$ grows, so the
right hand side of eqn \ref{eqn:friedman_B} becomes negative
for large $R$ if $k=1$.  Since $\dot{R}^2$ cannot be negative,
$R$ will reach a maximum and turn around.  For $k\le0$, the
expansion never ends.

GR allows for a {\it vacuum energy} with constant density $\rho_{\Lambda} = \Lambda/(8\pi G)$.  Since $\rho_{\Lambda}$ is a constant, the RHS of the $\Delta Q$ equation must
be zero since $\rho_{\Lambda}$
cannot change and the pressure is $p_{\Lambda} = -\Lambda c^2/(8\pi G)$,
which is more like a tension pulling out than a pressure pushing in.
Once the universe expands so $\rho_{\Lambda}$ is larger than the other
components, the universe expands exponentially.  This kind of 
expansion may have happened in the early universe (inflation), which
enables us to explain away several cosmological puzzles.

\section{Flat Cosmologies}

In the case where $k=0$, we need that the
density is equal to the {\it critical density}, such that
\begin{equation}
\rho(t) = \rhoc(t) \equiv \frac{3 H^2(t)}{8 \pi G}.
\end{equation}
\noindent
Today, $\rhoc(t) = 3.3\times 10^{11} h^2~\Msun~\Mpc^{-3}$.

We can express the abundance of various mass-energy components
relative to the critical density as
\begin{equation}
\Omega(t) \equiv \rho(t)/\rhoc(t).
\end{equation}
\noindent
We typically write $\Omega_0$ as the present value today.
We can re-write equation \ref{eqn:friedman_B} as
\begin{equation}
H^2(t)[1 - \Omega(t)] = -kc^2/R^2(t).
\end{equation}
\noindent
If $k=1$, then $\Omega(t)>1$ and the density exceeds the
critical density always.  If $k=-1$ then $\Omega(t)<1$
and the density is lower than $\rhoc$. For $k=0$,
we have that the density always equals the critical
density.

{\bf See Figure 8.7 of SG.}

We can measure the cosmic expansion relative to the present
through the {\it scale factor} $a\equiv R(t)/R(t_0)$.
The first Friedmann equation becomes
\begin{equation}
\frac{kc^2}{R^2(t_0)} = H_0^2(1 - \Omega) = a^2(t)\left[ H^{2}(t) - \frac{8\pi G}{3}\rho(t)\right].
\end{equation}
\noindent
We can then write an important equation for the Hubble parameter
\begin{equation}
H^2(t) = H_0^2[\Omega_r (1+z)^4 + \Omega_m(1+z)^3 + \Omega_k(1+z)^2 + \Omega_{\Lambda}]
\end{equation}
\noindent
where $\Omega_k = (1-\Omega)$ is the fractional density contributed by curvature with $k\ne0$.

\section{Lookback Time}
Since the universe is redshifting and time can be clocked by
successive troughs in a wave, the rate at which universal
time changes with redshift is
\begin{equation}
\frac{dt}{dz} = - \frac{1}{H(z)}\frac{1}{1+z}
\end{equation}
\noindent
We can integrate this to find the {\it lookback time}
\begin{equation}
t_0 - t_e = \int_{0}^{z} \frac{1}{H(z')}\frac{dz'}{1+z'}.
\end{equation}
\noindent
The appearance of $H(z)$ in the denominator complicates things,
but if $\Omega_m + \Omega_\Lambda = 1$ we can find
\begin{eqnarray}
t_e &=& \int_z^{\infty}\frac{1}{H(z')}\frac{dz'}{1+z'} \nonumber\\
&=&\int_z^{\infty} \frac{dz'}{H_0(1+z')[\Omega_m(1+z')^3 + \Omega_\Lambda]^{1/2}}\nonumber\\
&=&\frac{2}{3 H_0\sqrt{\Omega_\Lambda}}\ln\left(\frac{1+\cos\theta}{\sin\theta}\right)\nonumber\\
\mathrm{where}~\tan^{2}\theta&\equiv&\frac{(1-\Omega_\Lambda)}{\Omega_\Lambda}(1+z)^3
\end{eqnarray}
\noindent
which is accurate to a few percent.

For $\Lambda=0$, we have
\begin{equation}
t_e \int_z^\infty \frac{1}{H_0\sqrt{1+\Omega z}}\frac{dz'}{(1+z')^2}
\end{equation}
\noindent
Which we can solve for $\Omega=1$ or $\Omega=0$, giving
\begin{equation}
t_0 - t_e = \frac{1}{H_0}\int_0^z \frac{dz'}{(1+z')^{5/2}} = \frac{2}{3H_0}\left[1-\frac{1}{(1+z)^{3/2}}\right]~\mathrm{for}~\Omega = 1
\end{equation}
\noindent
and
\begin{equation}
t_0 - t_e = \frac{1}{H_0}\int_0^z \frac{dz'}{(1+z')^{2}} = \frac{1}{H_0}\frac{z}{(1+z)}~\mathrm{for}~\Omega = 0.
\end{equation}
\noindent
In the matter dominated era, we have
\begin{equation}
t_e \approx \frac{2}{3H_0\sqrt{\Omega_m}}\frac{1}{(1+z)^{3/2}}~\mathrm{when}~\frac{\Omega_m}{\Omega_r}\gg1+z \gg \left(\frac{\Omega_{\Lambda}}{\Omega_m}\right)^{1/3}~\mathrm{and}~\frac{1}{\Omega_m}.
\end{equation}

\section{Observing Distant Galaxies}

For nearby objects, the flux is related to luminosity by
\begin{equation}
F = \frac{L}{4\pi d^2},
\end{equation}
\noindent
which still holds for very distant galaxies provided we
use the correct distance that accounts for the expansion
of the universe and the change in energy density.  Actually,
if these effects did not come into play, the sky would be
bright because in a static universe filled with stars light
``accumulate''s without limit {\it Olber's paradox}.

\subsection{Luminosity Distance}
In an expanding universe, we now write
\begin{equation}
F = \frac{L}{4\pi R^2(t_0)\sigma_e^2(1+z)^2} \equiv \frac{L}{4\pi d_L^2}
\end{equation}
\noindent
where
\begin{equation}
d_L = (1+z) R(t_0) \sigma_e
\end{equation}
\noindent
is the {\it luminosity distance}.

\subsection{Angular-Size Distance}
The distance we use to measure an angular size also changes.
We are comfortable with the small angle approximation $\alpha = D/d$,
where $D$ is the transverse size and $d$ is the line-of-sight distance.
In an expanding universe we instead have
\begin{equation}
\alpha~[\mathrm{rad}] = \frac{D}{R(t_e)\sigma_e}\equiv\frac{D}{d_A}
\end{equation}
\noindent
where
\begin{equation}
d_A = R(t_e) \sigma_e = \frac{R(t_0) \sigma_e}{1+z}
\end{equation}
\noindent
is the {\it angular-size distance} or the {\it angular diameter distance}.
Note that the angular-size distance is a factor $(1+z)^2$ less than the
luminosity distance.

\subsection{Surface Brightness}
Owing to the behavior of the angular-size and lumionsity distances,
the surface brightness of galaxies are affected dramatically by the
expansion of the universe.  The surface brightness is
\begin{equation}
I(\vx) = \frac{F}{\alpha^2} = \frac{L/(4\pi d_L^2)}{D^2 / d_A^2} = \frac{L}{4\pi D^2}\left(\frac{d_A}{d_L}\right)^2 = \frac{L}{4\pi D^2}\frac{1}{(1+z)^4}.
\end{equation}

\subsection{Comoving Distance}
To evaluate the luminosity or angular-size distance, we have to
compute the {\it comoving distance} $R(t_0) \sigma_e$.  We can
compute this quantity by evaluating the integral of the distance
$R(t)\chi$ as
\begin{equation}
R(t) \chi \equiv \int_0^{\sigma} ds = R(t)\int_0^\sigma \frac{d\sigma'}{\sqrt{1-k\sigma'^2}}
\end{equation}
\noindent
which gives
\begin{equation}
\label{eqn:sigma_chi}
\chi(\sigma) = \left\{ \begin{array}{lll}\arcsin(\sigma) & \mathrm{for}~k=1\\ \sigma & \mathrm{for}~k=0 \\ \mathrm{arcsinh}(\sigma) = \ln(\sigma + \sqrt{1+\sigma^2}) & \mathrm{for}~k=-1\end{array}\right.
\end{equation}
\noindent
Everything is more complicated in a non-flat universe, but in a flat universe $\chi = \sigma$ and we can safely call $R(t_0) \sigma_e$ the comoving distance.

Moving at $c$, in a time $\Delta t$ light travels a distance $c \Delta t = R(t)\Delta \chi$.  We can then write the total distance
\begin{equation}
\chi_e = \int_{t_e}^{t_0}\frac{c dt}{R(t)} = \int_0^{z} \frac{c}{R(t_0)}\frac{dz'}{H(t)}
\end{equation}
\noindent
For $\Omega_\Lambda=0$, we have
\begin{equation}
\frac{R(t_0) H_0}{c} \chi_e = \left\{ \begin{array}{ll} \int_0^{z} \frac{dz'}{(1+z')^{3/2}}  = 2 \left( 1 - \frac{1}{\sqrt{1+z}}\right) & \mathrm{for}~\Omega_0=1 \\ \int_0^z \frac{dz'}{(1+z')} = \ln(1+z) & \mathrm{for}~\Omega_0 = 0 \end{array}\right.
\end{equation}
\noindent
And subbing these into Equation \ref{eqn:sigma_chi} gives the comoving distance
\begin{equation}
R(t_0) \sigma_e = \left\{ \begin{array}{ll} \frac{2c}{H_0} \left[1 - \frac{1}{\sqrt{1+z}} \right] & \mathrm{for}~\Omega_0 = 1 \\ \frac{c}{H_0}\frac{z(1+z/2)}{1+z} & \mathrm{for}~\Omega_0=0 \end{array}\right.
\end{equation}
\noindent
``An excessive quantity of algebra'' gives the {\it Mattig formula} (still for $\Omega_\Lambda=0$)
\begin{equation}
R(t_0) \sigma_e = \frac{c}{H_0} \frac{2}{\Omega_0^2 (1+z)}[\Omega_0 z + (\Omega_0 - 2)(\sqrt{1+\Omega_0 z}-1)]
\end{equation}
\noindent
In the limit $z\gg1$, $R(t_0)\sigma_e \to 2 c/(H_0\Omega_0)$.

In the limit $\Omega_0\ll1$, we have
\begin{equation}
R(t_0)\sigma_e = \frac{c}{H_0} \frac{z}{1+z} \frac{1+\sqrt{1+\Omega_0 z}+z}{1+ \sqrt{1+\Omega_0 z} + \Omega_0 z/2}.
\end{equation}

For a flat universe with $\Omega_\Lambda\ne0$, there is no useful exact 
result.  But for $\Omega_m \gtrsim 0.1$ and in the limit $z\to\infty$, we have 
\begin{equation}
R(t_0)\sigma_e \approx \frac{2c}{(H_0 z \Omega_m^{0.4})}.
\end{equation}

One of the seemingly crazy things is that the angular diameter distance can have
a {\it maximum}, such that objects at high redshift actually can start to 
appear larger again.

\section{Flux Density}

If we take a bandpass limited observation of a galaxy, the total
flux that we measure from an object with the same intrinsic
spectrum will change with redshift as the spectrum stretches in
wavelength owing to cosmological redshifting.  We then have that
\begin{eqnarray}
F_{BP} &=& \frac{1}{4\pi d_L^2} \int_{\lambda_1(1+z)}^{\lambda_2(1+z)} L_{\lambda}(\lambda,t_e) d\lambda \\
&=& \frac{1}{4\pi d_L^2}\frac{1}{1+z} \int_{\lambda_1}^{\lambda_2} L_{\lambda}[\lambda/(1+z),t_e]d \lambda.
\end{eqnarray}
\noindent
At cosmological distances, the apparent magnitude is then
\begin{equation}
m_{BP} = M_{BP} + 5 \log_10 \left(\frac{d_L}{10\pc}\right) + k_{BP}(z) + e_{BP}(z)
\end{equation}
\noindent
where $k(z)$ is the {\it k correction} and the {\it evolutionary term} $e(z)$
allows for the galaxy to evole during the travel time of its light.  The
{\it k} correction is given by
\begin{equation}
k_{BP}(z) \equiv 2.5 \log_{10} (1+z) - 2.5 \log_{10} \left\{ \frac{\int_{\lambda_1}^{\lambda_2} L_{\lambda}[\lambda/(1+z),t_0]d\lambda}{\int_{\lambda_1}^{\lambda_2} L_{\lambda}(\lambda,t_0)d\lambda   } \right\}.
\end{equation}


\section{Galaxy Space Densities}

The volume of the sky area $A = 4\pi R^2(t_e)\sigma_e^2$ over the redshift interval
$z$, $z+\Delta z$, corresponding to thickness $c |\Delta t_e|$, gives us the
relation
\begin{equation}
\frac{A c |\Delta t_e|}{\Delta z} \approx \frac{dV}{dz} = \frac{4\pi c R^2(t_0)\sigma_e^2}{H(z)(1+z)^3}.
\end{equation}
The number density of objects per comoving volume element is then
\begin{equation}
\frac{d N}{dz} = n_0(1+z)^3 \frac{dV}{dz} = n_0 c \frac{4 \pi R^2(t_0)\sigma_e^2}{H(z)}.
\end{equation}


\section{Growth of Structure}

The initial perturbations seeded by cosmic inflation likely had a
power spectrum $P(k) \propto k$.  These perturbations were affected
by physical processes at later times to alter $P(k)$ on small scales.
We see evidence of this in the cosmic microwave background. As
photons try to move out of the gravitational potential wells of the
initial density perturbations, the photons experience a 
{\it gravitational redshift} $\Delta\Phi_g$ that changes the
temperature $T$ of the radiation by an amount $\Delta T$ according
to
\begin{equation}
\frac{\Delta T}{T}|_{\mathrm{grav}} \sim \Delta \Phi_g c^{-2}
\end{equation}
\noindent
The temperature is reduced where the potentially
is unusually deep since $\Delta \Phi_g$ is negative
there, and time runs more slowly such that $\Delta t/t =\Delta \Phi_g/c^2$
and we see the gas at an earlier time when it was hotter.  The temperature
declines by $T\propto 1/ a(t)$, so we have
\begin{equation}
\frac{\Delta T}{T}|_{\mathrm{time}} = -\frac{\Delta a}{a} = -\frac{2}{3}\frac{\Delta t}{t} =  -\frac{2}{3}\frac{\Delta Phi_g}{c^2}
\end{equation}
\noindent
where we've used $a\propto t^{2/3}$ for the matter-dominated era.
The net effect is that $\Delta T/T \sim \Delta \Phi_g / (3 c^2)$.
If the region has density $\rho = \bar{\rho}(1+\delta)$ and a
corresponding mass excess of $\Delta M = 4\pi \bar{\rho}R^3 \delta / 3$,
then 
\begin{equation}
3c^2\frac{\Delta T}{T} = \Delta \Phi_g \sim - \frac{2G\Delta M}{R} = - \frac{8\pi}{3} G \bar{\rho}R^2 \delta \approx - \delta(t)[\bar{H}(t)R]^2
\end{equation}
\noindent
such that radiation from dense regions is {\it colder}.

We can model the temperature map of the sky by expanding
the pattern of the CMB in spherical harmonics as
\begin{equation}
\Delta T(\theta, \phi) = \sum_{l>1} \sum_{-l\le m \le l} a_l^m Y_l^m(\theta, \phi).
\end{equation}
\noindent
The theoretical predictions are often expressed in terms of $C_l = \ave{|a_l^m|^2}$ averaged over $m$ since this does not depend on the direction of ``north''
in the map.  A commonly plotted quantity is $\Delta_T^2 = T^2 l(l+1)C_l/(2\pi)$.


Pressure forces can act to suppress structure, but only on the
sound horizon scale.  The sound horizon scale when the
gas becomes transparent to photons (recombination) is
\begin{equation}
R(t_{rec}) \sigma_H = 3 c t_{rec} = \frac{2c}{H(t_{rec})} \approx \frac{2c}{H_0\sqrt{\Omega_m}(1+z_{rec})^{3/2}}.
\end{equation}
\noindent
This works out to about $184/(h^2 \Omega_m)^{1/2}$~Mpc today.  The
angle $\theta_H$ subtended by the sound horizon depends on
the angular-size distance to $z_{rec}$.  When $\Omega_\Lambda\to0$
and $\Omega_0 z\gg1$, then $d_A\to 2c/(H_0 z \Omega_0)$. We then
have
\begin{equation}
\theta_H \approx \frac{R(t_{rec}\sigma_H)}{d_A(t_{rec})} \approx \sqrt{\frac{\Omega_0}{z_{rec}}} \approx 2^\circ \times \sqrt{\Omega_0},
\end{equation}
\noindent
and only points separated by this angle on the sky can communicate before $t_{rec}$.  For $\Omega_0=1$, $\Delta_T$ is largest on the scale of about a degree,
where the {\it first acoustic peak} of the CMB is found. A model
with $\Omega_0=0.3$ provides about half this angle. In a cosmology
close to what we think ours is, it turns out the 
first peak is at $l\approx 220$ that corresponds to a size scale of 
aobut $105~\Mpc$ today, which is a sphere that contains $2.5\times10^{16}M_{\odot}$.  The second peak is at $l\approx540$.


\end{document}
