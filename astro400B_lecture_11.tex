\documentclass[]{article}
\usepackage[margin=1.0in]{geometry}
\usepackage{amssymb}

%title material
\title{Astronomy 400B Lecture 11: Structure Formation and Cosmology}
\author{Brant Robertson}
\date{March, 2015}


%include latex definitions

%average
\newcommand{\ave}[1]{\langle#1\rangle}
%radian
\newcommand{\rad}{\mathrm{rad}}

%astronomical unit
\newcommand{\AU}{\mathrm{AU}}

%centimeter
\newcommand{\cm}{\mathrm{cm}}

%meter
\newcommand{\m}{\mathrm{m}}

%kilometer
\newcommand{\km}{\mathrm{km}}

%parsec
\newcommand{\pc}{\mathrm{pc}}

%kiloparsec
\newcommand{\kpc}{\mathrm{kpc}}

%megaparsec
\newcommand{\Mpc}{\mathrm{Mpc}}

%gigaparsec
\newcommand{\Gpc}{\mathrm{Gpc}}

%light year
\newcommand{\ly}{\mathrm{ly}}

%second
\newcommand{\s}{\mathrm{s}}
\newcommand{\yr}{\mathrm{yr}}
\newcommand{\Gyr}{\mathrm{Gyr}}

%solar mass
\newcommand{\Msun}{M_{\odot}}

%grams
\newcommand{\g}{\mathrm{g}}

%erg
\newcommand{\erg}{\mathrm{erg}}

%solar luminosity
\newcommand{\Lsun}{L_{\odot}}

%jansky 
\newcommand{\Jy}{\mathrm{Jy}}

%flux density
\newcommand{\Fnu}{F_{\nu}}
\newcommand{\Flambda}{F_{\lambda}}
\newcommand{\Lnu}{L_{\nu}}

%hertz 
\newcommand{\Hz}{\mathrm{Hz}}

%angstrom
\newcommand{\Ang}{\mathrm{\r{A}}}

%solar radius
\newcommand{\Rsun}{R_{\odot}}

%stefan-boltzmann
\newcommand{\sigmaSB}{\sigma_{\mathrm{SB}}}

%boltzmann
\newcommand{\kB}{k_{\mathrm{B}}}

%kelvin
\newcommand{\K}{\mathrm{K}}

%T_bandpass
\newcommand{\TBP}{T_{\mathrm{BP}}}

%magnitude
\newcommand{\mg}{\mathrm{mag}}


%arcsecond
\newcommand{\arcsec}{\mathrm{arcsec}}

%begin the document
\begin{document}

%make the title, goes after document begins
\maketitle

%first section
\section{Structure Formation}

Galaxies are not randomly distributed in the
Poisson or white-noise sense. They have a 
clustering strength that changes the probability
that two galaxies will lie near one another.
For instance, if the mean number density of
galaxies is $n$, then the expected number
of galaxies in a volume $\Delta V_1$ is $\sim n\Delta V_1$
as long as $n\Delta V_1\ll1$.  However, the joint
probability of having galaxies in two volumes
$V_1$ and $V_2$ depends on the radial separation $r_{12}$
of the two volumes as
\begin{equation}
\Delta P = n^2 [1 + \xi(r_{12})]\Delta V_1 \Delta V_2.
\end{equation}
\noindent
where $\xi(r)$ is called the {\it correlation function}.
If $\xi(r)>0$, then galaxies are clustered (biased) and
if $\xi(r)<0$ then galaxies are anti-biased.

The form of the correlation function is very roughly
a power-law of the form
\begin{equation}
\xi(r) \approx (r/r_0)^{-\gamma}
\end{equation}
\noindent
where $r_0$ is called the {\it correlation length} and
$\gamma<0$.  This holds on scales $r\lesssim10 h^{-1}\mathrm{Mpc}$.
When $r<r_0$ then the probability of
finding galaxies proximate to one another is much
larger than for a Poisson distribution.

The correlation length for all galaxies is about $r_0\sim 5 h^{-1}\Mpc$, but
a bit larger for ellipticals.  The slope is $\gamma\sim1.7$.  At
$r\lesssim50h^{-1}\Mpc$, $\xi(r)\sim0$ and galaxies are roughly
uniformly randomly distributed.

\subsection{Power Spectrum}

The spatial power spectrum of galaxy counts is given by
\begin{equation}
P(\vk) \equiv \int \xi(\vr) \exp(i\vk\cdot\vr)d^3\vr = 4\pi \int_0^\infty \xi(r) \frac{\sin(kr)}{kr} r^2 dr,
\end{equation}
\noindent
such that the galaxy power spectrum is the Fourier transform of the correlation function.
Remember, small $k$ corresponds to large $r$.  The correlation function has
no units, so the power spectrum has dimensions of volume.

\subsection{Galaxy Number Density}

We can write the galaxy density at some location $\vx$ as
\begin{equation}
\rho(\vx) = \bar{\rho}[1+\delta(\vx)]
\end{equation}
\noindent
where $\bar{\rho}$ is the mean density and $\delta$ is the overdensity
\begin{equation}
\delta(\vx) = \frac{\rho(\vx) - \bar{\rho}}{\bar{\rho}}.
\end{equation}
We can calculate the statistics of the overdensity most readily
by averaging the over all spherical volumes of radius $R$.
We write this quantity $\delta_R$, and the variance of 
this quantity $\sigma_R^2 = \ave{\delta_R^2}$.  Defining
the dimensionless quanity $\Delta_k^2$ as the typical
fluctuation in over density in a volume $k^{-1}~h^{-1}\Mpc$
in radius as
\begin{equation}
\Delta_k^2 \equiv \frac{k^3 P(k)}{2\pi^2},
\end{equation}
\noindent
we then have $\sigma_R^2 \approx \Delta_k^2$ and if
$P(k)\propto k^n$, then $\sigma_R \propto R^{-(n+3)/2}$.
Then if $n>-3$, the universe is lumpiest on small scales.
We often characterize the power spectrum by the 
quantity $\sigma_8\equiv\sigma(R=8~h^{-1}\Mpc)$.  Today, we think 
$\sigma_8\approx0.8$.

Instead of taking the variance in a volume corresponding to
a radius $R$, we can take the variance $\sigma(M)$ on a mass scale
$M \approx 4\pi R^3  \bar{\rho}/3$.  Since $M \propto R^3$, we
have that $\sigma(M) \propto M^{-(n+3)/6}$.

\section{Expansion of a Homogenous Universe}

We know from the CMB that the universe is homogeneous on
large scales.  We can describe the distance between two
points in such a universe by the relation
\begin{equation}
ds^2 = R^2 \left( \frac{d\sigma^2}{1-k\sigma^2} + \sigma^2 d\theta^2 + \sigma^2 \sin^2 \theta d\phi^2\right)
\end{equation}
\noindent
where $\sigma, \theta, \phi$ are spherical polar coordinates in curved space.
The constant $k$ specifies the curvature of space.  For $k=1$ the universe
is {\it closed} with positive curvature.  If $k=-1$ the universe is
{\it open} with negative curvature.  If $k=0$ we have normal Euclidean
flat space.

We can include cosmic expansion just by setting $R = R(t)$. Galaxies
remain at points with fixed values of $\sigma, \theta, \phi$ called
the comoving coordinates.  Two systems a distance $d$ in this coordinate system
move apart from each other at the rate
\begin{equation}
V = \dot{d} = \frac{\dot{R}(t)}{R(t)}d \equiv H(t) d
\end{equation}
\noindent
where $H(t)$ is the Hubble parameter that has $H(t_0) = H_0$
today.

\section{Time in an Expanding Universe}
Owing to the finite speed of light, General Relativity
informs us that the distance between two events happening
at different times in different places depends on  the
motion of an observer.  But all observers will measure
the same proper time $\tau$ along a path through space
and time connecting the events.  The proper time
is found by integrating the relation
\begin{equation}
c^2 d\tau^2 = c^2 dt^2 - ds^2.
\end{equation}
\noindent
Light rays travel paths with $d\tau^2 = 0$, called
null geodesics.

Place yourself at the origin.  Then the light
received from a galaxy at comoving distance
$\sigma_e$ has followed the path
\begin{equation}
\frac{c dt}{R(t)} = -\frac{d\sigma}{\sqrt{1-k\sigma^2}}.
\end{equation}
\noindent
As the universe grows, light covers less and less 
comoving distance per unit time.

\section{Redshift}

Consider a light wave that leaves at time $t_e$
from a distance $\sigma_e$ and arrives today $t_0$.
We have that
\begin{equation}
c\int_{t_e}^{t_0} \frac{dt}{R(t)} = \int_0^{\sigma_e} \frac{d\sigma}{\sqrt{1-k\sigma^2}}.
\end{equation}
\noindent
Now consider another wave leaving at a later time $t_e + \Delta t_e$.
It arrives at $t + \Delta t$.  The comoving position hasn't changed
so we have that
\begin{equation}
\int_{t_e+\Delta t_e}^{t_0 + \Delta t} \frac{dt}{R(t)} = \int_{t_e}^{t_0} \frac{dt}{R(t)},~~\mathrm{so}~\frac{\Delta t_e}{R(t_e)} = \frac{\Delta t}{R(t_0)},
\end{equation}
\noindent
provided that $\Delta t \ll R(t)/\dot{R}(t)$ (e.g., a Hubble time).  So all processes appear to be
slowed by a factor $R(t_0)/R(t_e)$.

If the light was emitted with wavelength $\lambda_e = c \Delta t_e$, it is
received with wavelength $\lambda_{obs} = c \Delta t$.  The
redshift is then
\begin{equation}
1 + z = \frac{\lambda_{obs}}{\lambda_e} = \frac{R(t_0)}{R(t_e)}.
\end{equation}

\section{Expansion History}

The rate of expansion is determined by the gravitational effects
of the energy densities it contains.  We can model the expansion
using Newtonian physics first, and then use general relativity
to revise for a more correct answer.

Consider a sphere of radius $r$ at time $t$ when the 
universe has a typical density $\rho(t)$, and assume
$r$ is much less than any curvature radius in the 
universe.  Assuming symmetry about $r=0$, the
gravitational force at radius $r$ is just supplied by the
mass within the sphere.  If the sphere
is large enough that pressure forces are small, then
the force on a object of mass $m$ at radius $r$ is
\begin{equation}
m \frac{d^2 r}{dt^2} - \frac{G m M(<r)}{r^2} = - \frac{4\pi Gm}{3}\rho(t)r.
\end{equation}
\noindent
The radius of the sphere of matter is expanding with the
rest of the universe, so $r\propto R(t)$.  The mass of the
cloud cancels, such that
\begin{equation}
\label{eqn:friedman_A}
\ddot{R}(t) = - \frac{4\pi G}{3} \rho(t) R(t).
\end{equation}
\noindent
The mass doesn't change, so $\rho(t) R(t)$ is a constant.
We can multiply by $\dot{R}(t)$ to find
\begin{equation}
\frac{1}{2} \frac{d}{dt}[\dot{R}^2(t)] = - \frac{4 \pi G}{3} \frac{\rho(t_0)R^3(t_0)}{R^2(t)} \dot{R}(t)
\end{equation}
\noindent
where $t_0$ is the present day.  We then integrate to find
\begin{equation}
\label{eqn:friedman_B}
\dot{R}^2(t) = \frac{8\pi G}{3}\rho(t) R^2(r) - kc^2
\end{equation}
\noindent
where $k$ is a constant of integration.  Turns out Equation
\ref{eqn:friedman_B} is valid in GR, and tells us
that $k$ is the same constant as in the metric.

We appeal to thermodynamics to tell us that
as heat $\Delta Q$ flows into a volume $V$ its
internal energy $E$ must increase, or it expands
and does work against pressure
\begin{equation}
\Delta Q = \Delta E + p \Delta V = V \Delta(\rho c^2) + (\rho c^2 + p)\Delta V
\end{equation}
\noindent
where $\rho$ includes all the forms of matter and energy.
But no volume $V$ gains heat at the expense of another and
\begin{equation}
\Delta Q = 0 = \Delta \rho + \left(\rho + \frac{p}{c^2}\right)\frac{\Delta V}{V}
\end{equation}
\noindent
or
\begin{equation}
\frac{d\rho}{dt} = - c \frac{\dot{R}(t)}{R(t)}\left( \rho + \frac{p}{c^2}\right)
\end{equation}
\noindent
Differentiating Equation \ref{eqn:friedman_B} and substituting for $d\rho/dt$
yields
\begin{equation}
\label{eqn:friedman_C}
\ddot{R}(t) = - \frac{4\pi G}{3} R(t) \left[ \rho(t) + \frac{3p(t)}{c^2}\right]
\end{equation}
\noindent
So in GR, the pressure $p$ adds to the gravitational attraction. 
Equations \ref{eqn:friedman_B} and \ref{eqn:friedman_C} are
called the {\it Friedmann equations}.  For cool matter
the pressure $p \sim \rho c_s^2$ where the sound speed $c_s \ll c$,
and the pressure term in equation \ref{eqn:friedman_C} is 0.
Then $\rho(t)\propto R^{-3}$.  For radiation and particles,
$p\approx pc^2/3$ and $\rho(t) \propto R^{-4}(t)$.
For matter and radiation $\rho + 3p/c^2$ is positive, and the
universe decelerates. 

The quantity $\rho(t)R^2(t)$ decreases as $R(t)$ grows, so the
right hand side of eqn \ref{eqn:friedman_B} becomes negative
for large $R$ if $k=1$.  Since $\dot{R}^2$ cannot be negative,
$R$ will reach a maximum and turn around.  For $k\le0$, the
expansion never ends.

GR allows for a {\it vacuum energy} with constant density $\rho_{\Lambda} = \Lambda/(8\pi G)$.  Since $\rho_{\Lambda}$ is a constant, the RHS of the $\Delta Q$ equation must
be zero since $\rho_{\Lambda}$
cannot change and the pressure is $p_{\Lambda} = -\Lambda c^2/(8\pi G)$,
which is more like a tension pulling out than a pressure pushing in.
Once the universe expands so $\rho_{\Lambda}$ is larger than the other
components, the universe expands exponentially.  This kind of 
expansion may have happened in the early universe (inflation), which
enables us to explain away several cosmological puzzles.

\section{Flat Cosmologies}

In the case where $k=0$, we need that the
density is equal to the {\it critical density}, such that
\begin{equation}
\rho(t) = \rhoc(t) \equiv \frac{3 H^2(t)}{8 \pi G}.
\end{equation}
\noindent
Today, $\rhoc(t) = 3.3\times 10^{11} h^2~\Msun~\Mpc^{-3}$.

We can express the abundance of various mass-energy components
relative to the critical density as
\begin{equation}
\Omega(t) \equiv \rho(t)/\rhoc(t).
\end{equation}
\noindent
We typically write $\Omega_0$ as the present value today.
We can re-write equation \ref{eqn:friedman_B} as
\begin{equation}
H^2(t)[1 - \Omega(t)] = -kc^2/R^2(t).
\end{equation}
\noindent
If $k=1$, then $\Omega(t)>1$ and the density exceeds the
critical density always.  If $k=-1$ then $\Omega(t)<1$
and the density is lower than $\rhoc$. For $k=0$,
we have that the density always equals the critical
density.

{\bf See Figure 8.7 of SG.}

We can measure the cosmic expansion relative to the present
through the {\it scale factor} $a\equiv R(t)/R(t_0)$.
The first Friedmann equation becomes
\begin{equation}
\frac{kc^2}{R^2(t_0)} = H_0^2(1 - \Omega) = a^2(t)\left[ H^{2}(t) - \frac{8\pi G}{3}\rho(t)\right].
\end{equation}
\noindent
We can then write an important equation for the Hubble parameter
\begin{equation}
H^2(t) = H_0^2[\Omega_r (1+z)^4 + \Omega_m(1+z)^3 + \Omega_k(1+z)^2 + \Omega_{\Lambda}]
\end{equation}
\noindent
where $\Omega_k = (1-\Omega)$ is the fractional density contributed by curvature with $k\ne0$.

\section{Lookback Time}
Since the universe is redshifting and time can be clocked by
successive troughs in a wave, the rate at which universal
time changes with redshift is
\begin{equation}
\frac{dt}{dz} = - \frac{1}{H(z)}\frac{1}{1+z}
\end{equation}
\noindent
We can integrate this to find the {\it lookback time}
\begin{equation}
t_0 - t_e = \int_{0}^{z} \frac{1}{H(z')}\frac{dz'}{1+z'}.
\end{equation}
\noindent
The appearance of $H(z)$ in the denominator complicates things,
but if $\Omega_m + \Omega_\Lambda = 1$ we can find
\begin{eqnarray}
t_e &=& \int_z^{\infty}\frac{1}{H(z')}\frac{dz'}{1+z'} \nonumber\\
&=&\int_z^{\infty} \frac{dz'}{H_0(1+z')[\Omega_m(1+z')^3 + \Omega_\Lambda]^{1/2}}\nonumber\\
&=&\frac{2}{3 H_0\sqrt{\Omega_\Lambda}}\ln\left(\frac{1+\cos\theta}{\sin\theta}\right)\nonumber\\
\mathrm{where}~\tan^{2}\theta&\equiv&\frac{(1-\Omega_\Lambda)}{\Omega_\Lambda}(1+z)^3
\end{eqnarray}
\noindent
which is accurate to a few percent.

For $\Lambda=0$, we have
\begin{equation}
t_e \int_z^\infty \frac{1}{H_0\sqrt{1+\Omega z}}\frac{dz'}{(1+z')^2}
\end{equation}
\noindent
Which we can solve for $\Omega=1$ or $\Omega=0$, giving
\begin{equation}
t_0 - t_e = \frac{1}{H_0}\int_0^z \frac{dz'}{(1+z')^{5/2}} = \frac{2}{3H_0}\left[1-\frac{1}{(1+z)^{3/2}}\right]~\mathrm{for}~\Omega = 1
\end{equation}
\noindent
and
\begin{equation}
t_0 - t_e = \frac{1}{H_0}\int_0^z \frac{dz'}{(1+z')^{2}} = \frac{1}{H_0}\frac{z}{(1+z)}~\mathrm{for}~\Omega = 0.
\end{equation}
\noindent
In the matter dominated era, we have
\begin{equation}
t_e \approx \frac{2}{3H_0\sqrt{\Omega_m}}\frac{1}{(1+z)^{3/2}}~\mathrm{when}~\frac{\Omega_m}{\Omega_r}\gg1+z \gg \left(\frac{\Omega_{\Lambda}}{\Omega_m}\right)^{1/3}~\mathrm{and}~\frac{1}{\Omega_m}.
\end{equation}

\end{document}