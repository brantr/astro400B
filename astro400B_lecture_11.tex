\documentclass[]{article}
\usepackage[margin=1.0in]{geometry}
\usepackage{amssymb}

%title material
\title{Astronomy 400B Lecture 11: Structure Formation and Cosmology}
\author{Brant Robertson}
\date{March, 2015}


%include latex definitions

%average
\newcommand{\ave}[1]{\langle#1\rangle}
%radian
\newcommand{\rad}{\mathrm{rad}}

%astronomical unit
\newcommand{\AU}{\mathrm{AU}}

%micron
\newcommand{\mum}{\mu\mathrm{m}}

%millimeter
\newcommand{\mm}{\mathrm{mm}}

%centimeter
\newcommand{\cm}{\mathrm{cm}}

%meter
\newcommand{\m}{\mathrm{m}}

%kilometer
\newcommand{\km}{\mathrm{km}}

%parsec
\newcommand{\pc}{\mathrm{pc}}

%kiloparsec
\newcommand{\kpc}{\mathrm{kpc}}

%megaparsec
\newcommand{\Mpc}{\mathrm{Mpc}}

%gigaparsec
\newcommand{\Gpc}{\mathrm{Gpc}}

%light year
\newcommand{\ly}{\mathrm{ly}}

%second
\newcommand{\s}{\mathrm{s}}
\newcommand{\yr}{\mathrm{yr}}
\newcommand{\Gyr}{\mathrm{Gyr}}

%solar mass
\newcommand{\Msun}{M_{\odot}}

%grams
\newcommand{\g}{\mathrm{g}}

%erg
\newcommand{\erg}{\mathrm{erg}}

%solar luminosity
\newcommand{\Lsun}{L_{\odot}}

%jansky 
\newcommand{\Jy}{\mathrm{Jy}}

%flux density
\newcommand{\Fnu}{F_{\nu}}
\newcommand{\Flambda}{F_{\lambda}}
\newcommand{\Lnu}{L_{\nu}}

%hertz 
\newcommand{\Hz}{\mathrm{Hz}}

%angstrom
\newcommand{\Ang}{\mathrm{\r{A}}}

%solar radius
\newcommand{\Rsun}{R_{\odot}}

%stefan-boltzmann
\newcommand{\sigmaSB}{\sigma_{\mathrm{SB}}}

%boltzmann
\newcommand{\kB}{k_{\mathrm{B}}}

%kelvin
\newcommand{\K}{\mathrm{K}}

%T_bandpass
\newcommand{\TBP}{T_{\mathrm{BP}}}

%magnitude
\newcommand{\mg}{\mathrm{mag}}


%arcsecond
\newcommand{\arcsec}{\mathrm{arcsec}}

%critical density
\newcommand{\rhoc}{\rho_{\mathrm{crit}}}

%proton mass
\newcommand{\mproton}{m_{\mathrm{p}}}

%electron volt
\newcommand{\eV}{\mathrm{eV}}

%kiloelectron volt
\newcommand{\keV}{\mathrm{keV}}

%megaelectron volt
\newcommand{\MeV}{\mathrm{MeV}}

%gigaelectron volt
\newcommand{\GeV}{\mathrm{GeV}}

%vector velocity
\newcommand{\vv}{\mathbf{v}}

%vector radius
\newcommand{\vr}{\mathbf{r}}

%vector position
\newcommand{\vx}{\mathbf{x}}

%vector force
\newcommand{\vF}{\mathbf{F}}

%vector surface
\newcommand{\vS}{\mathbf{S}}

%vector angular momentum
\newcommand{\vL}{\mathbf{L}}

%script I -- integral of motion
\newcommand{\cI}{\mathcal{I}}

%effective potential
\newcommand{\Phieff}{\Phi_\mathrm{eff}}


%begin the document
\begin{document}

%make the title, goes after document begins
\maketitle

%first section
\section{Structure Formation}

Galaxies are not randomly distributed in the
Poisson or white-noise sense. They have a 
clustering strength that changes the probability
that two galaxies will lie near one another.
For instance, if the mean number density of
galaxies is $n$, then the expected number
of galaxies in a volume $\Delta V_1$ is $\sim n\Delta V_1$
as long as $n\Delta V_1\ll1$.  However, the joint
probability of having galaxies in two volumes
$V_1$ and $V_2$ depends on the radial separation $r_{12}$
of the two volumes as
\begin{equation}
\Delta P = n^2 [1 + \xi(r_{12})]\Delta V_1 \Delta V_2.
\end{equation}
\noindent
where $\xi(r)$ is called the {\it correlation function}.
If $\xi(r)>0$, then galaxies are clustered (biased) and
if $\xi(r)<0$ then galaxies are anti-biased.

The form of the correlation function is very roughly
a power-law of the form
\begin{equation}
\xi(r) \approx (r/r_0)^{-\gamma}
\end{equation}
\noindent
where $r_0$ is called the {\it correlation length} and
$\gamma<0$.  This holds on scales $r\lesssim10 h^{-1}\mathrm{Mpc}$.
When $r<r_0$ then the probability of
finding galaxies proximate to one another is much
larger than for a Poisson distribution.

The correlation length for all galaxies is about $r_0\sim 5 h^{-1}\Mpc$, but
a bit larger for ellipticals.  The slope is $\gamma\sim1.7$.  At
$r\lesssim50h^{-1}\Mpc$, $\xi(r)\sim0$ and galaxies are roughly
uniformly randomly distributed.

\subsection{Power Spectrum}

The spatial power spectrum of galaxy counts is given by
\begin{equation}
P(\vk) \equiv \int \xi(\vr) \exp(i\vk\cdot\vr)d^3\vr = 4\pi \int_0^\infty \xi(r) \frac{\sin(kr)}{kr} r^2 dr,
\end{equation}
\noindent
such that the galaxy power spectrum is the Fourier transform of the correlation function.
Remember, small $k$ corresponds to large $r$.  The correlation function has
no units, so the power spectrum has dimensions of volume.

\subsection{Galaxy Number Density}

We can write the galaxy density at some location $\vx$ as
\begin{equation}
\rho(\vx) = \bar{\rho}[1+\delta(\vx)]
\end{equation}
\noindent
where $\bar{\rho}$ is the mean density and $\delta$ is the overdensity
\begin{equation}
\delta(\vx) = \frac{\rho(\vx) - \bar{\rho}}{\bar{\rho}}.
\end{equation}
We can calculate the statistics of the overdensity most readily
by averaging the over all spherical volumes of radius $R$.
We write this quantity $\delta_R$, and the variance of 
this quantity $\sigma_R^2 = \ave{\delta_R^2}$.  Defining
the dimensionless quanity $\Delta_k^2$ as the typical
fluctuation in over density in a volume $k^{-1}~h^{-1}\Mpc$
in radius as
\begin{equation}
\Delta_k^2 \equiv \frac{k^3 P(k)}{2\pi^2},
\end{equation}
\noindent
we then have $\sigma_R^2 \approx \Delta_k^2$ and if
$P(k)\propto k^n$, then $\sigma_R \propto R^{-(n+3)/2}$.
Then if $n>-3$, the universe is lumpiest on small scales.
We often characterize the power spectrum by the 
quantity $\sigma_8\equiv\sigma(R=8~h^{-1}\Mpc)$.  Today, we think 
$\sigma_8\approx0.8$.

Instead of taking the variance in a volume corresponding to
a radius $R$, we can take the variance $\sigma(M)$ on a mass scale
$M \approx 4\pi R^3  \bar{\rho}/3$.  Since $M \propto R^3$, we
have that $\sigma(M) \propto M^{-(n+3)/6}$.

\section{Expansion of a Homogenous Universe}

We know from the CMB that the universe is homogeneous on
large scales.  We can describe the distance between two
points in such a universe by the relation
\begin{equation}
ds^2 = R^2 \left( \frac{d\sigma^2}{1-k\sigma^2} + \sigma^2 d\theta^2 + \sigma^2 \sin^2 \theta d\phi^2\right)
\end{equation}
\noindent
where $\sigma, \theta, \phi$ are spherical polar coordinates in curved space.
The constant $k$ specifies the curvature of space.  For $k=1$ the universe
is {\it closed} with positive curvature.  If $k=-1$ the universe is
{\it open} with negative curvature.  If $k=0$ we have normal Euclidean
flat space.

We can include cosmic expansion just by setting $R = R(t)$.


\end{document}