\documentclass[]{article}
\usepackage[margin=1.0in]{geometry}
\usepackage{amssymb}

%title material
\title{Astronomy 400B Lecture 8: Spirals and Starlight}
\author{Brant Robertson}
\date{March, 2015}


%include latex definitions

%average
\newcommand{\ave}[1]{\langle#1\rangle}
%radian
\newcommand{\rad}{\mathrm{rad}}

%astronomical unit
\newcommand{\AU}{\mathrm{AU}}

%centimeter
\newcommand{\cm}{\mathrm{cm}}

%meter
\newcommand{\m}{\mathrm{m}}

%kilometer
\newcommand{\km}{\mathrm{km}}

%parsec
\newcommand{\pc}{\mathrm{pc}}

%kiloparsec
\newcommand{\kpc}{\mathrm{kpc}}

%megaparsec
\newcommand{\Mpc}{\mathrm{Mpc}}

%gigaparsec
\newcommand{\Gpc}{\mathrm{Gpc}}

%light year
\newcommand{\ly}{\mathrm{ly}}

%second
\newcommand{\s}{\mathrm{s}}
\newcommand{\yr}{\mathrm{yr}}
\newcommand{\Gyr}{\mathrm{Gyr}}

%solar mass
\newcommand{\Msun}{M_{\odot}}

%grams
\newcommand{\g}{\mathrm{g}}

%erg
\newcommand{\erg}{\mathrm{erg}}

%solar luminosity
\newcommand{\Lsun}{L_{\odot}}

%jansky 
\newcommand{\Jy}{\mathrm{Jy}}

%flux density
\newcommand{\Fnu}{F_{\nu}}
\newcommand{\Flambda}{F_{\lambda}}
\newcommand{\Lnu}{L_{\nu}}

%hertz 
\newcommand{\Hz}{\mathrm{Hz}}

%angstrom
\newcommand{\Ang}{\mathrm{\r{A}}}

%solar radius
\newcommand{\Rsun}{R_{\odot}}

%stefan-boltzmann
\newcommand{\sigmaSB}{\sigma_{\mathrm{SB}}}

%boltzmann
\newcommand{\kB}{k_{\mathrm{B}}}

%kelvin
\newcommand{\K}{\mathrm{K}}

%T_bandpass
\newcommand{\TBP}{T_{\mathrm{BP}}}

%magnitude
\newcommand{\mg}{\mathrm{mag}}


%arcsecond
\newcommand{\arcsec}{\mathrm{arcsec}}

%begin the document
\begin{document}

%make the title, goes after document begins
\maketitle

%first section
\section{Measuring Starlight}

\subsection{CCD Signal-to-Noise Ratio}

The signal-to-noise ratio of CCD detectors
depends on a variety of factors. 

\begin{enumerate}

\item The signal $St$ measured by a CCD
is measured in some number of electrons.
The electrons measured from the
observations will be the
product of the photon count per unit
time $C$ times
the gain $g$ that describes the response
of photoelectrons per photon deposited into
a pixel, all times time $r$.  We can then write $S=gC$. The
noise $N_S$ associated with this signal will follow
the Poisson distribution in the counts $Ct$
or $N_S\approx\sqrt{St}=\sqrt{gCt}$.

\item The
next issue is the {\it read noise} of the
detector -- the excess electronic noise
associated with the fact that charge in the
CCDs must be moved and then read out. There
is usually a constant noise associated with
the {\it read noise} such that every
read of every pixel in our aperture $n_{pix}$ 
produces on average some excess number
$R$ of electrons, but we add the noise of
each pixel in quadrature. We will therefore write that
the read noise $N_R \equiv \sqrt{R^2 \times n_{pix}}$.

\item {\it Dark current} is the noise associated
with the operation of the detector, and corresponds
to the signal produced by each pixel when it is
``dark'' (e.g., the shutter is closed).  The
dark current is a rate, a number of electrons per
second per pixel, so the total dark current noise depends on the
integration time across the aperture as $\sqrt{Dn_{pix}t}$.

\item {\it Sky background} $s_{sky}$ contributes some number of photoelectrons per pixel per
unit time. The amount of sky you include in your signal-to-noise
estimate depends on the aperture over which you integrate your
signal as $\sqrt{s_{sky} n_{pix} t}$.

\end{enumerate}


The total signal to noise is then

\begin{equation}
SNR = \frac{St}{\left[St + s_{sky}n_{pix}t + R^2 n_{pix} + Dn_{pix}t\right]^{\frac{1}{2}}}
\end{equation}

The limiting cases are:

\begin{enumerate}
\item Bright sources: $SNR\propto\sqrt{t}$.
\item Sky limited: $SNR\propto\sqrt{t}$.
\item Read noise limited: $SNR\propto t \propto\mathrm{\#~of~exposures}$.
\end{enumerate}

\subsection{Exponential Disk Surface Brightness Profiles}

The surface brightness profile of an edge-on exponential disk will be approximately
\begin{equation}
I(R,z) = I_0 \exp[-R/h_R] \exp[-|z|/h_z]
\end{equation}
\noindent
where $h_R$ and $h_z$ are the radial scalelength and the vertical scale height,
respectively. The central surface brightness $I_0$ will depend on the orientation.
For a face-on exponential disk, $I(R,z)\approx I_0 \exp[-R/h_R]$ but with a different $I_0$.

\subsection{Sersic Bulge Profile}

The {\it Sersic} profile for a bulge is given by
\begin{equation}
I(R) = I_0 \exp[-(R/R_0)^{1/n}]
\end{equation}
\noindent
where $R_0$ is a scale radius and $I_0$ is a central
brightnes. For $n=1$ you have an exponential, while for $n=4$ you
have a {\it de Vaucouleurs} profile.

\section{Measuring Gas}

Cosmic gas is typically directly observed in the radio where
either HI or CO emits. The technology we have for detecting
radio emission uses mirrors in the form of dishes and
receivers instead of CCDs. The long wavelength of radio
emission means that single dishes must be extremely large
if the resolution $\theta \approx \lambda/D$ is to be
interestingly small. In fact, single dishes must be $\sim100\m$
across if HI maps are to have resolution better than $10'$.
To improve resolution in the radio, we can instead use
interferometer arrays of dishes to improve our resolution.

Consider two dishes $1$ and $2$ a distance $d$ apart.
The electric field associated with a distant source will
have a time dependence of $E\propto\cos(2\pi ct/\lambda)$ where
$\lambda$ is the wavelength of the emission and $c$ is the speed
of light. For sources at an elevation $\phi$ on the sky, the
light must travel an extra distance and be delayed by a time
$d\cos(\phi)/c$ at position $2$ relative to position $1$.
The voltages $V_1$ and $V_2$ at the two feeds will be
\begin{equation}
V_1 \propto \cos\left(\frac{2\pi ct}{\lambda}\right)~~\mathrm{and}~~V_2\propto\cos\left(\frac{2\pi(ct-d\cos\phi)}{\lambda}\right)
\end{equation}
\noindent
If we multiply the voltages and filter high-frequency oscillations, we get a signal
\begin{equation}
S\propto \cos(2\pi d \cos(\phi)/\lambda)
\end{equation}
\noindent
which can tell the difference between $\Delta\phi = \lambda/2d$, and can
distinguish sources separated by $\lambda/d\sin\phi$. The effective resolution
of the interferometer is then $d\sin\phi$.


\subsection{Gas Mass}

The HI line in most galaxies is optically thin.  The mass of gas providing the
emission is then just related to the integral of the emission.  If a galaxy 
along the line of sight is rotating at velocity $V_r$ in $\km~\s^{-1}$, 
the total mass we measure
is
\begin{equation}
M(HI) = 2.36\times10^5 \Msun \times \left(\frac{d}{\Mpc}\right)^2 \int F_{\nu}\left[1421~\mathrm{MHz}~\times\left(1-\frac{V_r}{c}\right)\right] dV_r.
\end{equation}

\subsection{Rotation Curve}

For a galaxy at a systemic velocity $V_{sys}$ inclined at an angle $i$ to face-on, the rotation curve we measure as a function of radial distance $R$ and azimuth $\phi$ is
\begin{equation}
V_r(R,i) = V_{sys} + V(R)\sin i \cos \phi
\end{equation}
\noindent
so to determine $V(R)$ we must determine the inclination and perhaps determine an angular average over azimuth.

In determining the rotation curve, we are often trying to weigh the galaxy.  This is straightforward to do in
some limiting cases.

For a thin exponential disk that supplies its own gravity, the rotation curve can be written in terms of Bessel
functions as
\begin{equation}
V^2(R) = 4\pi G \Sigma_0 h_R y^2 [I_0(y) K_0(y) - I_1(y) K_1(y)]
\end{equation}
\noindent
where $\Sigma_0$ is the central mass surface density, $y\equiv R/2h_R$,
 and $I$ and $K$ are modified Bessel functions (mind the subscript!). We can
relate this to the total disk mass $M_d = 2\pi\Sigma_0 h_R^2$, such that
\begin{equation}
V^2(R) = \frac{2GM_d}{h_R}f(y).
\end{equation}

Otherwise, if we examine the regions of a disk where the dark halo dominates the potential, than we can just use
the radial force equation
\begin{equation}
\frac{V^2(R)}{R} = \frac{GM(<R)}{R^2}
\end{equation}
\noindent 
to estimate the interior mass from the rotation curve.

\subsection{Tully-Fisher Relation}

In 1977, Tully and Fisher showed that the luminosity of a disk galaxy was correlated with a
power of the maximum rotational velocity.  They found that
\begin{equation}
\frac{L_I}{4\times10^{10} L_{I,\odot}} \approx \left(\frac{V_{max}}{200~\km~\s^{-1}}\right)^4
\end{equation}
\noindent
It turns out that this relationship is perplexing if the circular velocity is dominated
by the dark halo, and the disk and halo must have some connection to force $L\propto V^4$.


\subsection{Spiral Arms}

The spiral pattern of disks can be described mathematically.  In polar
coordinates $(R,\phi)$ an $m$-armed spiral
can be described by the equation
\begin{equation}
\cos\{m[\phi + f(R,t)]\}=1
\end{equation}
\noindent
where $f(R,t)$ describes how tightly the spiral is wound, and is connected
to the {\it pitch angle}
\begin{equation}
\frac{1}{\tan i} = \left|R\frac{\partial\phi}{\partial R}\right| = \left|R\frac{\partial f}{\partial R}\right|.
\end{equation}
\noindent
For Sa spirals $i\approx5$ degrees, while for Sc $10\lesssim i \lesssim 30$.

Are spirals long lived, permanent structures?  Or density waves that may come and go?
Well, most spirals are trailing spirals, meaning that the ends of the spirals lag behind
the front of the spiral in its rotation.  If spirals lived a long time, then differential
rotation would wind them into a tight curl.  Consider stars stretched out along an
initial ray in radius at angle $\phi = \phi_0$.  The angular rate of rotation is
$\Omega(R) = V(R)/R$, so the curve spreads out on a spiral as $\phi = \phi_0 + \Omega(R)t$.
In other words, $f(R,t) = -\phi_0 - \Omega(R)t$.  At the Galactic radius $R\sim8~\kpc$
where $V\sim200~\km~\s^{-1}$ we have
\begin{equation}
\cot i = R \left| \frac{d\Omega(R)}{dR}\right| r \approx \frac{200}{8}\left(\frac{t}{1~\Gyr}\right)
\end{equation}
\noindent
which gives
\begin{equation}
i\approx2^\circ \times \left(\frac{1~\Gyr}{t}\right)
\end{equation}
\noindent
so after a Gyr the pitch angle would be much tighter than is observed in Sc galaxies like the MW.

So spirals may not be permanent structures, but if their orbits are elliptical they can 
reinforce the spiral pattern.  Consider a star on an eplicyclical orbit about guiding
center $R_g$.  The azimuthal position of the guiding center is $\phi_{gc} = \Omega(R_g)t$.
the radius of the star is
\begin{equation}
R = R_g + x = R_g + X \cos(\kappa t + \psi)
\end{equation}
\noindent
were $X$ is the amplitude, $\kappa$ is the eplicyclic frequency and $\psi$ determines the initial
radius $R$.  Now take several stars, put them on the circle $R_g$ but then set $\psi = 2\phi_{gc}(0)$.
They will then lie on an oval with its long axis pointing along $\phi = 0$. At a later time $t$
the guiding centers will be at
\begin{equation}
\phi_{gc} = \phi_{gc}(0) + \Omega t
\end{equation}
\noindent
and the stars will lie at
\begin{equation}
x = X\cos\{\kappa t + 2[\phi_{gc}(t) - \Omega t]\} = X \cos](2\Omega-\kappa)t - 2\phi_{gc}(t)].
\end{equation}
The long axis of the oval will now point along the direction where
\begin{equation}
(2\Omega-\kappa)t - 2\phi = 0,~~\mathrm{or}~~\phi=(\Omega-\kappa/2) t \equiv \Omega_p t
\end{equation}
\noindent
where $\Omega_p$ is the {\it pattern speed} of the spiral that will return to its original state after a time $T = 2\pi/\Omega_p$.
This will produce a two-arm spiral.  We can make an $m-$arm spiral by setting $\psi = 4\phi_{gc}$ and 
then the pattern speed would be $\Omega_p = \Omega - \kappa/m$.

Once stars do collect in spiral arms their gravity
can affect the spiral structure in a non linear way.  We can use the above calculation
to determine an analytical calculation for
tightly wound spirals that suggests that stars reinforce the spiral structure only if
\begin{equation}
m|\Omega_p - \Omega(R)| < \kappa(R).
\end{equation}
\noindent
Hence, spiral waves can only propagate in the region between the inner Lindblad resonance
where $\Omega_p = \Omega - \kappa/m$ and the outer Lindblad resonance where $\Omega_p = \Omega + \kappa/m$.

{\bf See Figures 5.29 and 5.30 of SG.}

There is a further condition on whether stars can help strengthen spiral structure.  The radial
velocity dispersion in the disk must be low (cold) relative to the disk surface gravity.  The condition
is
\begin{equation}
Q \equiv \frac{\kappa \sigma_R}{3.36 G \Sigma} \lesssim 1
\end{equation}
\noindent
where $Q$ is called the {\it Toomre Q}.  In the Milky Way, $\Sigma \sim 50 \Msun \pc^{-2}$
and $\kappa \approx 36~\km~\s^{-1}~\kpc^{-1}$.  We then find $Q\approx 1.4$, which is
large enough relative to one that we think the radial dispersion in the Milky Way is 
too large for stars to help strengthen the MW spirals. Confusing!

\end{document}