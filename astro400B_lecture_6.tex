\documentclass[]{article}
\usepackage[margin=1.0in]{geometry}
\usepackage{amssymb}

%title material
\title{Astronomy 400B Lecture 6: Collisionless Boltzmann Equation}
\author{Brant Robertson}
\date{February, 2015}


%include latex definitions

%average
\newcommand{\ave}[1]{\langle#1\rangle}
%radian
\newcommand{\rad}{\mathrm{rad}}

%astronomical unit
\newcommand{\AU}{\mathrm{AU}}

%micron
\newcommand{\mum}{\mu\mathrm{m}}

%millimeter
\newcommand{\mm}{\mathrm{mm}}

%centimeter
\newcommand{\cm}{\mathrm{cm}}

%meter
\newcommand{\m}{\mathrm{m}}

%kilometer
\newcommand{\km}{\mathrm{km}}

%parsec
\newcommand{\pc}{\mathrm{pc}}

%kiloparsec
\newcommand{\kpc}{\mathrm{kpc}}

%megaparsec
\newcommand{\Mpc}{\mathrm{Mpc}}

%gigaparsec
\newcommand{\Gpc}{\mathrm{Gpc}}

%light year
\newcommand{\ly}{\mathrm{ly}}

%second
\newcommand{\s}{\mathrm{s}}
\newcommand{\yr}{\mathrm{yr}}
\newcommand{\Gyr}{\mathrm{Gyr}}

%solar mass
\newcommand{\Msun}{M_{\odot}}

%grams
\newcommand{\g}{\mathrm{g}}

%erg
\newcommand{\erg}{\mathrm{erg}}

%solar luminosity
\newcommand{\Lsun}{L_{\odot}}

%jansky 
\newcommand{\Jy}{\mathrm{Jy}}

%flux density
\newcommand{\Fnu}{F_{\nu}}
\newcommand{\Flambda}{F_{\lambda}}
\newcommand{\Lnu}{L_{\nu}}

%hertz 
\newcommand{\Hz}{\mathrm{Hz}}

%angstrom
\newcommand{\Ang}{\mathrm{\r{A}}}

%solar radius
\newcommand{\Rsun}{R_{\odot}}

%stefan-boltzmann
\newcommand{\sigmaSB}{\sigma_{\mathrm{SB}}}

%boltzmann
\newcommand{\kB}{k_{\mathrm{B}}}

%kelvin
\newcommand{\K}{\mathrm{K}}

%T_bandpass
\newcommand{\TBP}{T_{\mathrm{BP}}}

%magnitude
\newcommand{\mg}{\mathrm{mag}}


%arcsecond
\newcommand{\arcsec}{\mathrm{arcsec}}

%critical density
\newcommand{\rhoc}{\rho_{\mathrm{crit}}}

%proton mass
\newcommand{\mproton}{m_{\mathrm{p}}}

%electron volt
\newcommand{\eV}{\mathrm{eV}}

%kiloelectron volt
\newcommand{\keV}{\mathrm{keV}}

%megaelectron volt
\newcommand{\MeV}{\mathrm{MeV}}

%gigaelectron volt
\newcommand{\GeV}{\mathrm{GeV}}

%vector velocity
\newcommand{\vv}{\mathbf{v}}

%vector radius
\newcommand{\vr}{\mathbf{r}}

%vector position
\newcommand{\vx}{\mathbf{x}}

%vector force
\newcommand{\vF}{\mathbf{F}}

%vector surface
\newcommand{\vS}{\mathbf{S}}

%vector angular momentum
\newcommand{\vL}{\mathbf{L}}

%script I -- integral of motion
\newcommand{\cI}{\mathcal{I}}

%effective potential
\newcommand{\Phieff}{\Phi_\mathrm{eff}}


%begin the document
\begin{document}

%make the title, goes after document begins
\maketitle

%first section
\section{Distribution Function}

The {\it distribution function} $f(\vx,\vv,t)$ gives the probability density
in six-dimensional {\it phase space} $(\vx,\vv)$ of having an object
in the volume $d\vx d\vv$.  The number density $n(\vx,t)$ of objects is the
volume integral of the distribution function
\begin{equation}
n(\vx,t) = \int_{-\infty}^{\infty}\int_{-\infty}^{\infty}\int_{-\infty}^{\infty} f(\vx,\vv,t) dv_{x}dv_y,dv_z.
\end{equation}
\noindent
We can use this expression to define moments of the velocity distribution, such as
the average velocity 
\begin{equation}
\ave{\vv(\vx,t)} = \frac{1}{n(\vx,t)} \int_{-\infty}^{\infty}\int_{-\infty}^{\infty}\int_{-\infty}^{\infty} \vv f(\vx,\vv,t) dv_{x}dv_y,dv_z.
\end{equation}

For a collisionless system where objects cannot be created or destroyed, the number density of objects
in a given volume will follow the continuity equation
\begin{equation}
\label{eqn:continuity}
\frac{\partial n}{\partial t} + \frac{\partial (nv)}{\partial x} = 0.
\end{equation}
\noindent
This equation simply describes the mass conservation of objects, such that
the time rate of change of the number density is balanced by the advection of
spatial gradients in the number density.

The {\it collisionless Boltzmann equation} that describes the probability density 
of objects in phase space is more complicated because it must describe a time variation
in the velocity as well as the spatial coordinates.  In one dimension, we have that
\begin{equation}
\frac{\partial f}{\partial t} + v\frac{\partial f}{\partial x} + \frac{dv}{dt}(x,v,t) \cdot \frac{\partial f}{\partial v} = 0
\end{equation}
\noindent
The acceleration of an object will only depend on position in a background potential, so
we have that $dv/dt = -\partial \Phi/\partial x$, and
\begin{equation}
\label{eqn:boltzmann}
\frac{\partial f}{\partial t} + v \frac{\partial f}{\partial x} - \frac{\partial \Phi}{\partial x}(x,t)\cdot \frac{\partial f}{\partial v} = 0.
\end{equation}
\noindent
In three dimensions, we can write
\begin{equation}
\label{eqn:bvec}
\frac{\partial f(\vx, \vv, t)}{\partial t} + \vv \cdot \nabla f - \nabla \Phi \cdot \frac{\partial f}{\partial \vv} = 0.
\end{equation}
\noindent
We usually don't deal with this equation in this form, but often will take moments with respect to
e.g., velocity
\begin{equation}
\frac{\partial n(x,t)}{\partial t} + \frac{\partial}{\partial x}[n(x,t)\ave{v(x,t)}] - \frac{\partial \Phi}{\partial x}(x,t)[f]_{-\infty}^{\infty} = 0
\end{equation}
\noindent
The last term will be zero if $f$ is well behaved, and we get back Equation \ref{eqn:continuity}.

If we integrate Equation \ref{eqn:boltzmann} multiplied by $v$, we instead find
\begin{equation}
\frac{\partial}{\partial t}[n(x,t)\ave{v(x,t)}] + \frac{\partial}{\partial x}[n(x,t)\ave{v^2(x,t)}] = - n(x,t) \frac{\partial \Phi}{\partial x}
\end{equation}
\noindent
If we define the velocity dispersion as
\begin{equation}
\ave{v^2(x,t)} = \ave{v(x,t)}^2 + \sigma^2,
\end{equation}
\noindent
apply this definition, and then divide by $n(x,t)$, we have
\begin{equation}
\label{eqn:v_ave}
\frac{d\ave{v}}{dt} + \ave{v}\frac{\partial \ave{v}}{\partial x} = -\frac{\partial \Phi}{\partial x} - \frac{1}{n}\frac{\partial}{\partial x}[n\sigma^2(x,t)].
\end{equation}

\section{Disk Mass and Jeans Equation}

Consider Equation \ref{eqn:v_ave} applied to the Galactic disk.  Well above the disk
plane, $\ave{v_z}=0$ and, if we assume the disk is stable, $\partial\ave{v_z}/\partial t=0$.
We can then write
\begin{equation}
\frac{\partial}{\partial z}[n(z)\sigma_z^2] = -n(z)\frac{\partial \Phi}{\partial z}
\end{equation}
\noindent
This equation is sometimes called the {\it Jeans Equation}.
To proceed, we need to find some way of expressing the disk potential. We can use Poisson's 
Equation in cylindrical polar coordinates assuming polar symmetry, and we find that
\begin{equation}
4\pi G \rho(R,z) = \nabla^{2} \Phi(R,z) = \frac{\partial^2\Phi}{\partial z^2} + \frac{1}{R}\frac{\partial}{\partial R}\left(R \frac{\partial \Phi}{\partial R}\right).
\end{equation}
\noindent
We can substitute $\partial \Phi  / \partial R = V^2(R)/R$, which gives
\begin{equation}
4 \pi G \rho(R,z) = \frac{d}{dz}\left\{ - \frac{1}{n(z)}\frac{d}{dz}[n(z)\sigma_z^2]\right\} + \frac{1}{R}\frac{d}{dR}[V^{2}(R)].
\end{equation}
\noindent
The last term is small since the Galactic rotation curve is nearly flat. Integrating over $z$, we then
have
\begin{equation}
2\pi G \Sigma(<z) \equiv 2\pi G\int_{-z}^{z}\rho(z')dz' \approx -\frac{1}{n(z)}\frac{d}{dz}[n(z)\sigma_z^2]
\end{equation}
\noindent
By measuring the vertical distribution of stars in the disk and the variation of $\sigma_z$ with height
above the disk we find that $\Sigma(<z) \approx 50-60~\Msun~\pc^{2}$ near the solar circle.  The mass in
gas and stars is about $\Sigma_{gs}(<z)\approx40-55~\Msun~\pc^{2}$.

\section{Integrals of Motion}

If the potential does not change with time, then the distribution
function is also constant. In this case, we can define {\it integrals of motion} that
remain constant along an orbit.  Examples are the total energy $E(\vx,v) = \vv^2/2 + \Phi(\vx)$,
or the total angular momentum $\vL$ in a spherical potential, or
the angular momentum along the symmetry axis $l_z$ in an axisymmetric potential.

For any quantity $\cI$ that is constant along the orbit, we have
\begin{equation}
\label{eqn:eom}
\frac{d}{dt}\cI(\vx,\vv) \equiv \frac{\partial \vx}{\partial t} \cdot \nabla \cI + \frac{\partial \vv}{\partial t} \cdot \frac{\partial \cI}{\partial \vv} = 0
\end{equation}
\noindent
or\begin{equation}
\vv \cdot \nabla \cI - \nabla \Phi \cdot \frac{\partial \cI}{\partial \vv} = 0.
\end{equation}

Note that Equation \ref{eqn:eom} looks like Equation \ref{eqn:bvec}, with the substitution
$f\to\cI$. So the phase space density around a moving object is also a constant.
So what can we do now?

Well, remember from the discussion of epicyclical frequency that for nearly circular orbits
\begin{equation}
\ddot{z} \approx - \nu^2(R_g)z,
\end{equation}
\noindent
such that motion in $z$ is basically independent of motion in $(R,\phi)$. This result
means that the energy in the vertical motion
\begin{equation}
E_z = \Phi(R_0,z) + v_z^2/2
\end{equation}
\noindent
is an integral of motion. We therefore have that
\begin{equation}
f(z,v_z) = f(E_z) = f\left(\Phi(R_0,z) + \frac{1}{2}v_z^2\right)
\end{equation}
\noindent
If we guess at the form of $f(E_z)$, we can integrate to find $n(z)$ and $\sigma_z$.
For instance, assume
\begin{equation}
f(E_z) = \frac{n_0}{\sqrt{2\pi\sigma^2}}\exp(-E_z/\sigma^2)~\mathrm{for}~E_z<0
\end{equation}
(if $E_z>0$, the stars would be unbound).  If we integrate over $v_z$, we have
\begin{equation}
\label{eqn:isothermal}
n(z) = n_0 \exp\left[-\Phi(R_0,z)/\sigma^2\right];~\sigma_z = \sigma.
\end{equation}

For a spherical potential $\Phi(r)$, any $f(E,\vL)$ that doesn't have unbound stars will
describe some density distribution that generates the potential.  If we adopt $f(E)$ without
an $\vL$ dependence, the velocity distribution will be isotropic.

\subsection{Isothermal Distribution}

Consider the isothermal distribution
\begin{equation}
f_{I}(E) = \frac{n_0}{(2\pi\sigma^2)^{3/2}}\exp\left\{-\left[\Phi(r) + \frac{v^2}{2}\right]\middle/\sigma^2\right\}
\end{equation}
\noindent
The spherical and vertical disk isothermal solutions for the number density are the same, e.g.,
\begin{equation}
n(r) = n_0 \exp\left[-\Phi(r)/\sigma^2\right]
\end{equation}
\noindent
and we have
\begin{equation}
4\pi G \rho(r) = \frac{1}{r^2} \frac{d}{d r}\left(r^2\frac{d\Phi}{dr}\right) = 4\pi G m n_0 \exp\left[-\frac{\Phi(r)}{\sigma^2}\right]
\end{equation}
\noindent
The calculation of $\Phi(r)$ necessarily requires a radial integral from $r=0$.  At the center,
if the potential is smooth then the radial force has to be zero and $d\Phi(r=0)/dr =0$.  But
no matter how $\Phi(r=0)$ is chosen, the mass will be infinite.  If the mass had been
finite, then the escape speed would drop below $\sigma$ at some radius and not all the stars 
would be bound.

\subsection{King Model}

To avoid these problems near the escape speed for finite mass, we can take the King model,
also known as the lowered isothermal model, which is given by
\begin{equation}
\label{eqn:king}
f_K(E) = \frac{n_0}{(2\pi\sigma)^{3/2}}\exp\left[-\left(\Phi(r) + \frac{v^2}{2}\right)\middle/\sigma^{2} - 1\right].
\end{equation}
\noindent
This model has a truncation radius beyond which the number density declines dramatically, but the interior is
nearly isothermal. This model provides a good representation of globular clusters.


\subsection{Angular Momentum}

We can modify Equation \ref{eqn:king} to give
\begin{equation}
f_A(E,L) = f_K(E)\exp\left[-\vL^2/(2\sigma^2 r_a^2)\right]
\end{equation}
\noindent
where $r_a$ is the {\it anisotropy radius}.  Outside of $r_a$ the
stars have radial orbits, as the large angular momentum orbits (e.g.,
circular) are
exponentially suppressed.


Flattened systems can be described by distribution functions that
depend on $L_z$, e.g., $f(E,L_z)$.  For example, if
\begin{equation}
f(E,L_z) = \tilde{f}(E)L_z^2~~\mathrm{for}~E>0
\end{equation}
\noindent
for some function $\tilde{f}$, then few stars will have
orbits that approach the $z$ axis (which requires $L_z\approx0$)
while many will have nearly circular orbits with large $L_z$.
Note that this requires
\begin{equation}
n(\vx)\ave{v_z^2} \equiv \int f\left[\Phi(\vx) + \frac{\vv^2}{2},R v_{\phi}\right] dv_R dv_\phi dv_z = n(\vx)\ave{v_R^2}.
\end{equation}
\noindent
But for the Galactic disk, we saw that $\ave{v_R^2}>\ave{v_z^2}$. So for the Galactic disk
the distribution function cannot just be $f(E,L_z)$ but must instead rely on a {\it third integral of motion}.
One can show that for an axisymmetric potential $\Phi(R,z)$ there is no other function
of position and velocity other than $E$ and $L_z$ that is conserved along an orbit, which is perplexing!






\end{document}