\documentclass[]{article}
\usepackage[margin=1.0in]{geometry}
\usepackage{amssymb}

%title material
\title{Astronomy 400B Homework \#4}
\author{Please Show Your Work for Full Credit}
\date{Due April 9, 2015 by 9:35am}


%include latex definitions

%average
\newcommand{\ave}[1]{\langle#1\rangle}
%radian
\newcommand{\rad}{\mathrm{rad}}

%astronomical unit
\newcommand{\AU}{\mathrm{AU}}

%centimeter
\newcommand{\cm}{\mathrm{cm}}

%meter
\newcommand{\m}{\mathrm{m}}

%kilometer
\newcommand{\km}{\mathrm{km}}

%parsec
\newcommand{\pc}{\mathrm{pc}}

%kiloparsec
\newcommand{\kpc}{\mathrm{kpc}}

%megaparsec
\newcommand{\Mpc}{\mathrm{Mpc}}

%gigaparsec
\newcommand{\Gpc}{\mathrm{Gpc}}

%light year
\newcommand{\ly}{\mathrm{ly}}

%second
\newcommand{\s}{\mathrm{s}}
\newcommand{\yr}{\mathrm{yr}}
\newcommand{\Gyr}{\mathrm{Gyr}}

%solar mass
\newcommand{\Msun}{M_{\odot}}

%grams
\newcommand{\g}{\mathrm{g}}

%erg
\newcommand{\erg}{\mathrm{erg}}

%solar luminosity
\newcommand{\Lsun}{L_{\odot}}

%jansky 
\newcommand{\Jy}{\mathrm{Jy}}

%flux density
\newcommand{\Fnu}{F_{\nu}}
\newcommand{\Flambda}{F_{\lambda}}
\newcommand{\Lnu}{L_{\nu}}

%hertz 
\newcommand{\Hz}{\mathrm{Hz}}

%angstrom
\newcommand{\Ang}{\mathrm{\r{A}}}

%solar radius
\newcommand{\Rsun}{R_{\odot}}

%stefan-boltzmann
\newcommand{\sigmaSB}{\sigma_{\mathrm{SB}}}

%boltzmann
\newcommand{\kB}{k_{\mathrm{B}}}

%kelvin
\newcommand{\K}{\mathrm{K}}

%T_bandpass
\newcommand{\TBP}{T_{\mathrm{BP}}}

%magnitude
\newcommand{\mg}{\mathrm{mag}}


%arcsecond
\newcommand{\arcsec}{\mathrm{arcsec}}

%begin the document
\begin{document}

%make the title, goes after document begins
\maketitle

\section{Sparke \& Gallagher Problem 7.1}

Suppose that gas atoms and galaxies in a group move at the same average random speed $\sigma$ along each direction. At temperature $T$, the average energy of gas particle is $3k_B T/2$, where $k_B$ is Boltzmann's constant.  If the gas is mainly ionized hydrogen, these particles are protons and electrons; show that, if the atom's kinetic energy $(3 m_p/2)\sigma^2$ is shared equally
between them, then
\begin{equation}
T \approx \frac{(m_p/2)\sigma^2}{k_B} \approx 5 \times 10^6 \left( \frac{\sigma}{300~\km~\s^{-1}}\right)^2~\K.
\end{equation}
\noindent
Hot gas in a group or cluster is usually close to this {\it virial temperature}.

\section{Sparke \& Gallagher Problem 7.13}

If the lens $L$ is an object of mass $M_{\odot}$ at a distance $d_{\mathrm{Lens}}$ from
us, show that the Einstein radius for a star at distance $d_S = 2 d_{\mathrm{Lens}}$ is
\begin{equation}
\theta_{E} = \sqrt{\frac{R_s}{d_{\mathrm{Lens}}}} \approx 2 \times 10^{-3} \sqrt{\frac{1~\kpc}{d_{\mathrm{Lens}}}}~\mathrm{arcsec}.
\end{equation}

\section{Sparke \& Gallagher Problem 7.17}

If a lens at distance $d_{\mathrm{Lens}}$ bends the light of a much more
distant galaxy, so that $d_S$ and $d_{LS}\gg d_{\mathrm{Lens}}$, show that the
critical density is
\begin{equation}
\Sigma_{\mathrm{crit}} \approx 2 \times 10^4 \left( \frac{100~\Mpc}{d_{\mathrm{Lens}}}\right)~M_{\odot}~\pc^{-2},
\end{equation}
\noindent
and that the mass projected within angle $\theta_E$ of the center is
\begin{equation}
M(<\theta_E) \approx \left(\frac{d_{\mathrm{Lens}}}{100~\Mpc}\right)\left(\frac{\theta_E}{1~\mathrm{arcsec}}\right)^2 ~ 10^{10}~M_{\odot}.
\end{equation}

\end{document}