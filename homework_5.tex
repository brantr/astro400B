\documentclass[]{article}
\usepackage[margin=1.0in]{geometry}
\usepackage{amssymb}

%title material
\title{Astronomy 400B Homework \#5}
\author{Please Show Your Work for Full Credit}
\date{Due April 30, 2015 by 9:35am}


%include latex definitions

%average
\newcommand{\ave}[1]{\langle#1\rangle}
%radian
\newcommand{\rad}{\mathrm{rad}}

%astronomical unit
\newcommand{\AU}{\mathrm{AU}}

%centimeter
\newcommand{\cm}{\mathrm{cm}}

%meter
\newcommand{\m}{\mathrm{m}}

%kilometer
\newcommand{\km}{\mathrm{km}}

%parsec
\newcommand{\pc}{\mathrm{pc}}

%kiloparsec
\newcommand{\kpc}{\mathrm{kpc}}

%megaparsec
\newcommand{\Mpc}{\mathrm{Mpc}}

%gigaparsec
\newcommand{\Gpc}{\mathrm{Gpc}}

%light year
\newcommand{\ly}{\mathrm{ly}}

%second
\newcommand{\s}{\mathrm{s}}
\newcommand{\yr}{\mathrm{yr}}
\newcommand{\Gyr}{\mathrm{Gyr}}

%solar mass
\newcommand{\Msun}{M_{\odot}}

%grams
\newcommand{\g}{\mathrm{g}}

%erg
\newcommand{\erg}{\mathrm{erg}}

%solar luminosity
\newcommand{\Lsun}{L_{\odot}}

%jansky 
\newcommand{\Jy}{\mathrm{Jy}}

%flux density
\newcommand{\Fnu}{F_{\nu}}
\newcommand{\Flambda}{F_{\lambda}}
\newcommand{\Lnu}{L_{\nu}}

%hertz 
\newcommand{\Hz}{\mathrm{Hz}}

%angstrom
\newcommand{\Ang}{\mathrm{\r{A}}}

%solar radius
\newcommand{\Rsun}{R_{\odot}}

%stefan-boltzmann
\newcommand{\sigmaSB}{\sigma_{\mathrm{SB}}}

%boltzmann
\newcommand{\kB}{k_{\mathrm{B}}}

%kelvin
\newcommand{\K}{\mathrm{K}}

%T_bandpass
\newcommand{\TBP}{T_{\mathrm{BP}}}

%magnitude
\newcommand{\mg}{\mathrm{mag}}


%arcsecond
\newcommand{\arcsec}{\mathrm{arcsec}}

%begin the document
\begin{document}

%make the title, goes after document begins
\maketitle

\section{Sparke \& Gallagher Problem 8.7}

The quantity $\ave{\Delta_k^2}^{1/2}$ gives the expected fractional deviation
$|\delta(\vx)|$ from the mean density in an overdense of diffuse region of
size $1/k$.  Write $\delta(\vx)$ and $\Phi(\vx)$ as Fourier transforms and
use Poisson's equation
\begin{equation}
\nabla^2 \Phi(\vx) = 4\pi G \rho(\vx)
\end{equation}
\noindent
to show that these lumps and voids cause fluctuations $\Delta\Phi_k$ in the
gravitational potential, where $k^2|\Delta\Phi_k|\sim4\pi G\bar{\rho}\ave{\Delta_k^2}^{1/2}$.
Show that, when $P(k)\propto k$, the {\it Harrison-Zel'dovich} spectrum, $|\Delta\Phi_k|$ does
not depend on $k$: the potential is equally `rippled' on all spatial scales.

\section{Sparke \& Gallagher Problem 8.9}

By substituting into the equation
\begin{equation}
\ddot{R}(t) = -\frac{4\pi G}{3} R(t)\left[ \rho(t) + \frac{3p(t)}{c^2}\right]
\end{equation}
\noindent
show that, when vacuum energy dominates the expansion, we have $R(t) \propto \exp(t\sqrt{\Lambda/3})$.

\section{Sparke \& Gallagher Problem 8.11}

Blackbody radiation and relativistic particles provide most of the
energy density at $t\ll t_{eq}$. Show that the equation
\begin{equation}
H^2(t) = H_0^2[\Omega_r(1+z)^4 + \Omega_m(1+z)^3 + (1 -\Omega_{tot})(1+z)^2 + \Omega_{\Lambda}]
\end{equation}
\noindent
then implies that $H(t)=1/(2t)$. Early on, the leftmost term of the
equation
\begin{equation}
\frac{kc^2}{R^2(t_0)} = H_0^2(1-\Omega_{tot}) = a^2(t)\left[H^2(t) - \frac{8\pi G}{3} \rho(t)\right]
\end{equation}
\noindent
is tiny, so $H^2(t)\approx8\pi G\rho(t)/3$: show that the temperature $T(t)$ is
given by
\begin{equation}
T = \left(\frac{3c^2}{32 \pi G a_B t^2}\right)^{1/4},
\end{equation}
where $a_B = 7.56\times 10^{-16}~\mathrm{J}~\mathrm{m}^{-3}~\K^{-4}$ is the blackbody constant.

\section{Sparke \& Gallagher Problem 8.13}

Show that, when cool matter accounts for most of the energy density, and the universe
is flat with $k=0$, we have
\begin{equation}
\dot{a}\propto a^{-1/2},~\mathrm{and}~a(t)\propto t^{2/3}.
\end{equation}
\end{document}