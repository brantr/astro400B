\documentclass[]{article}
\usepackage[margin=1.0in]{geometry}
\usepackage{amssymb}

%title material
\title{Astronomy 400B Lecture 7: Local Group }
\author{Brant Robertson}
\date{February, 2015}


%include latex definitions

%average
\newcommand{\ave}[1]{\langle#1\rangle}
%radian
\newcommand{\rad}{\mathrm{rad}}

%astronomical unit
\newcommand{\AU}{\mathrm{AU}}

%micron
\newcommand{\mum}{\mu\mathrm{m}}

%millimeter
\newcommand{\mm}{\mathrm{mm}}

%centimeter
\newcommand{\cm}{\mathrm{cm}}

%meter
\newcommand{\m}{\mathrm{m}}

%kilometer
\newcommand{\km}{\mathrm{km}}

%parsec
\newcommand{\pc}{\mathrm{pc}}

%kiloparsec
\newcommand{\kpc}{\mathrm{kpc}}

%megaparsec
\newcommand{\Mpc}{\mathrm{Mpc}}

%gigaparsec
\newcommand{\Gpc}{\mathrm{Gpc}}

%light year
\newcommand{\ly}{\mathrm{ly}}

%second
\newcommand{\s}{\mathrm{s}}
\newcommand{\yr}{\mathrm{yr}}
\newcommand{\Gyr}{\mathrm{Gyr}}

%solar mass
\newcommand{\Msun}{M_{\odot}}

%grams
\newcommand{\g}{\mathrm{g}}

%erg
\newcommand{\erg}{\mathrm{erg}}

%solar luminosity
\newcommand{\Lsun}{L_{\odot}}

%jansky 
\newcommand{\Jy}{\mathrm{Jy}}

%flux density
\newcommand{\Fnu}{F_{\nu}}
\newcommand{\Flambda}{F_{\lambda}}
\newcommand{\Lnu}{L_{\nu}}

%hertz 
\newcommand{\Hz}{\mathrm{Hz}}

%angstrom
\newcommand{\Ang}{\mathrm{\r{A}}}

%solar radius
\newcommand{\Rsun}{R_{\odot}}

%stefan-boltzmann
\newcommand{\sigmaSB}{\sigma_{\mathrm{SB}}}

%boltzmann
\newcommand{\kB}{k_{\mathrm{B}}}

%kelvin
\newcommand{\K}{\mathrm{K}}

%T_bandpass
\newcommand{\TBP}{T_{\mathrm{BP}}}

%magnitude
\newcommand{\mg}{\mathrm{mag}}


%arcsecond
\newcommand{\arcsec}{\mathrm{arcsec}}

%critical density
\newcommand{\rhoc}{\rho_{\mathrm{crit}}}

%proton mass
\newcommand{\mproton}{m_{\mathrm{p}}}

%electron volt
\newcommand{\eV}{\mathrm{eV}}

%kiloelectron volt
\newcommand{\keV}{\mathrm{keV}}

%megaelectron volt
\newcommand{\MeV}{\mathrm{MeV}}

%gigaelectron volt
\newcommand{\GeV}{\mathrm{GeV}}

%vector velocity
\newcommand{\vv}{\mathbf{v}}

%vector radius
\newcommand{\vr}{\mathbf{r}}

%vector position
\newcommand{\vx}{\mathbf{x}}

%vector force
\newcommand{\vF}{\mathbf{F}}

%vector surface
\newcommand{\vS}{\mathbf{S}}

%vector angular momentum
\newcommand{\vL}{\mathbf{L}}

%script I -- integral of motion
\newcommand{\cI}{\mathcal{I}}

%effective potential
\newcommand{\Phieff}{\Phi_\mathrm{eff}}


%begin the document
\begin{document}

%make the title, goes after document begins
\maketitle


\section{Satellite Dynamics: The Tidal Limit}

The orbits of stars in satellites can be very complicated.
While orbits in many-body systems are intractable analytically,
if a satellite follows a circular orbit and the
gravitational potential is constant in a frame that
uniformly rotates about the common center of mass, we
can define an effective potential and use it to 
describe the desired orbit of a star in the satellite.

Consider a vector $\vu$ that is constant in a 
non-rotating frame. In a rotating frame $\vu$
will change at the rate $d\vu/dt' = -\vOmega \times \vu$
where $d/dt'$ is the derivative measured by
the rotating observer and $\vOmega = \Omega\hat{\mathbf{z}}$ is the
vector angular frequency of the rotation about the $\hat{\mathbf{z}}$
direction.

If a star has velocity $\vv$ and position $\vx$ relative to an inertial
frame, then if a rotating observer chooses their coordinates such that
$\vx'\equiv\vx$ the velocity they observe would be
\begin{equation}
\vv' \equiv \frac{d\vx'}{dt'} = \vv - \vOmega \times \vx.
\end{equation}
\noindent
The velocity changes at the rate
\begin{eqnarray}
\frac{d\vv'}{dt'} &=& \frac{d\vv'}{dt} - \vOmega \times \vv' = \frac{d\vv}{dt} - \vOmega \times \vv - \vOmega \times \vv'\nonumber\\
&=& - \nabla \Phi - 2 \vOmega \times \vv' - \vOmega\times(\vOmega \times \vx)
\end{eqnarray}
\noindent
If you take the scalar product of $\vv'$ with the last term in the RHS, you find
\begin{equation}
-\vv'\cdot\vOmega\times(\vOmega \times \vx) = \Omega^2(\vx\cdot\vv') - (\vv'\cdot\vOmega)(\vOmega\cdot\vx) = \frac{1}{2}\frac{d}{dt'}[(\vOmega\times\vx)^2]
\end{equation}
\noindent
But $\vv'\cdot(\vOmega\times\vv')=0$ and $\vx'=\vx$.  Taking another scalar product, we have
\begin{equation}
\label{eqn:KE_mod}
\frac{1}{2}\frac{d}{dt'}[\vv'^2 - (\vOmega \times \vx')^2] = -\vv' \cdot \nabla\Phi(\vx')
\end{equation}
If $\vOmega$ follows a satellite in its orbit, then in the rotating frame the gravitational
potential does not change with time. The potential at the star's position
changes at the rate $d\Phi/dt' = \vv'\cdot\nabla\Phi$.

We now define the {\it Jacobi constant} $E_J$ by
\begin{equation}
E_J = \frac{1}{2}\vv'^2 + \Phieff(\vx'),~~\mathrm{where}~\Phieff(\vx')\equiv\Phi(\vx') - \frac{1}{2}(\vOmega\times\vx')^2
\end{equation}
\noindent
and we see that Equation \ref{eqn:KE_mod} suggest that $E_J$ does not change along the star's orbit.
We can re-write $E_J$ in terms of the star's energy and angular momentum in the inertial frame as
\begin{equation}
E_J = \frac{1}{2}(\vv-\vOmega\times\vx)^2 + \Phieff = \frac{1}{2}\vv^2 + \Phi(\vx,t) - \Omega\cdot(\vx\times\vv) = E - \vOmega\cdot\vL
\end{equation}

\subsection{A Simple Example of the Tidal Limit}

Consider the satellite to have mass $m$ and the main galaxy to have mass $M$, separated by distance $D$,
orbiting the center of mass $C$ with angular speed $\Omega$.  Take $x$ as the coordinate
along the ray between $m$ toward $M$.  The center of mass is then at $x=DM/(M+m)$.  The
effective potential is then
\begin{equation}
\Phieff(x) = -\frac{GM}{|D-x|} -\frac{Gm}{|x|} - \frac{\Omega}{2}\left(x-\frac{DM}{M+m}\right)^2.
\end{equation}

The effective potential has three maxima known as the {\it Lagrange points}, [$L_1$, $L_2$, $L_3$].  Let's find them by setting
the derivative of the effective potential to zero.
\begin{equation}
\label{eqn:find_phieff_max}
\frac{d\Phieff}{dx} = 0 = -\frac{GM}{(D-x)^2} \pm \frac{Gm}{x^2} - \Omega^2\left(x-\frac{DM}{M+m}\right)
\end{equation}
\noindent
The acceleration $\Omega^2DM/(M+m)$ of $m$ as it circles $C$ owes to the gravitational
attraction of $M$. We can then write
\begin{equation}
\Omega\frac{DM}{M+m}=\frac{GM}{D^2},~~\mathrm{so}~\Omega^2=\frac{G(M+m)}{D^3}
\end{equation}
\noindent
If the satellite is much less massive than the main galaxy, $L_1$ and $L_2$ will lie close to $m$.
Substitute $\Omega^2$ into Equation \ref{eqn:find_phieff_max}, and expand in powers of $x/D$ to find
\begin{equation}
0 \approx -\frac{GM}{D^2} - 2\frac{GM}{D^3}x \pm \frac{Gm}{x^2} - \frac{G(M+m)}{D^3}\left(x-\frac{DM}{M+m}\right)
\end{equation}
\noindent
At the Lagrange points $L_1$ and $L_2$ we have
\begin{equation}
x = \pm r_J, ~~\mathrm{where}~r_J = D\left(\frac{m}{3M+m}\right)^{1/3}
\end{equation}
\noindent
Stars that cannot stray further from the satellite than $r_J$, the {\it Jacobi radius} or {\it Roche limit},
will remain bound to it.  The radius $L_1$ is not where the gravitational force from the satellite and 
main galaxy are the same, but where the effective potential has a minimum and lies further from the satellite.
This radius is also where expanding stars lose mass to their companion.

When $M\gg m$, then the mean density within $r_J$ is three times the mean density within $D$ of the main galaxy.
A star orbiting the satellite near $r_J$ will have an orbital period comparable to the orbit period of the
satellite around the main galaxy.

When satellites are not on circular orbits, the relevant $r_J$ is determined at the pericenter. If the
satellite is orbiting within the dark matter halo of the main galaxy, the relevant radius is
\begin{equation}
r_J = D\left[\frac{m}{2M(<D)}\right]^{1/3}
\end{equation}
where $M(<D)$ is the mass within D.

What's the Jacobi radius of the LMC-Milky Way system? The LMC is at a distance of $\sim50\kpc$ where
the speed of a circular orbit is about the same as it is at the solar circle, or about $\sim200\km~\s^{-1}$.
The mass of the Milky Way within the LMC's orbit is about $5\times10^{11}\Msun$.  The LMC mass is about
$10^{10}\Msun$, so we have
\begin{equation}
r_J\approx 50\kpc~\times~\left(\frac{10^{10}\Msun}{2\times5\times10^{11}\Msun}\right)^{1/3} \approx 11\kpc.
\end{equation}
The LMC disk lies within this radius, but the SMC is too far away to stay bound to the LMC.


\end{document}