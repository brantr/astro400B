\documentclass[]{article}
\usepackage[margin=1.0in]{geometry}
\usepackage{amssymb}

%title material
\title{Astronomy 400B Lecture 7: Local Group }
\author{Brant Robertson}
\date{February, 2015}


%include latex definitions

%average
\newcommand{\ave}[1]{\langle#1\rangle}
%radian
\newcommand{\rad}{\mathrm{rad}}

%astronomical unit
\newcommand{\AU}{\mathrm{AU}}

%micron
\newcommand{\mum}{\mu\mathrm{m}}

%millimeter
\newcommand{\mm}{\mathrm{mm}}

%centimeter
\newcommand{\cm}{\mathrm{cm}}

%meter
\newcommand{\m}{\mathrm{m}}

%kilometer
\newcommand{\km}{\mathrm{km}}

%parsec
\newcommand{\pc}{\mathrm{pc}}

%kiloparsec
\newcommand{\kpc}{\mathrm{kpc}}

%megaparsec
\newcommand{\Mpc}{\mathrm{Mpc}}

%gigaparsec
\newcommand{\Gpc}{\mathrm{Gpc}}

%light year
\newcommand{\ly}{\mathrm{ly}}

%second
\newcommand{\s}{\mathrm{s}}
\newcommand{\yr}{\mathrm{yr}}
\newcommand{\Gyr}{\mathrm{Gyr}}

%solar mass
\newcommand{\Msun}{M_{\odot}}

%grams
\newcommand{\g}{\mathrm{g}}

%erg
\newcommand{\erg}{\mathrm{erg}}

%solar luminosity
\newcommand{\Lsun}{L_{\odot}}

%jansky 
\newcommand{\Jy}{\mathrm{Jy}}

%flux density
\newcommand{\Fnu}{F_{\nu}}
\newcommand{\Flambda}{F_{\lambda}}
\newcommand{\Lnu}{L_{\nu}}

%hertz 
\newcommand{\Hz}{\mathrm{Hz}}

%angstrom
\newcommand{\Ang}{\mathrm{\r{A}}}

%solar radius
\newcommand{\Rsun}{R_{\odot}}

%stefan-boltzmann
\newcommand{\sigmaSB}{\sigma_{\mathrm{SB}}}

%boltzmann
\newcommand{\kB}{k_{\mathrm{B}}}

%kelvin
\newcommand{\K}{\mathrm{K}}

%T_bandpass
\newcommand{\TBP}{T_{\mathrm{BP}}}

%magnitude
\newcommand{\mg}{\mathrm{mag}}


%arcsecond
\newcommand{\arcsec}{\mathrm{arcsec}}

%critical density
\newcommand{\rhoc}{\rho_{\mathrm{crit}}}

%proton mass
\newcommand{\mproton}{m_{\mathrm{p}}}

%electron volt
\newcommand{\eV}{\mathrm{eV}}

%kiloelectron volt
\newcommand{\keV}{\mathrm{keV}}

%megaelectron volt
\newcommand{\MeV}{\mathrm{MeV}}

%gigaelectron volt
\newcommand{\GeV}{\mathrm{GeV}}

%vector velocity
\newcommand{\vv}{\mathbf{v}}

%vector radius
\newcommand{\vr}{\mathbf{r}}

%vector position
\newcommand{\vx}{\mathbf{x}}

%vector force
\newcommand{\vF}{\mathbf{F}}

%vector surface
\newcommand{\vS}{\mathbf{S}}

%vector angular momentum
\newcommand{\vL}{\mathbf{L}}

%script I -- integral of motion
\newcommand{\cI}{\mathcal{I}}

%effective potential
\newcommand{\Phieff}{\Phi_\mathrm{eff}}


%begin the document
\begin{document}

%make the title, goes after document begins
\maketitle

\section{Galaxies in the Local Group}

There are more than 60 known galaxies in the
Local Group. Of these, the dominate
galaxies are M31 (Sb; $L_V=2.7\times10^{10}\Lsun$), 
the Milky Way (Sbc; $L_V=1.5\times10^{10}\Lsun$), 
M33 (Sc; $5.5\times10^9\Lsun$),
the LMC (SBm; $L_V=2.2\times10^9\Lsun$), 
the SMC (Irr; $L_V=5.5\times10^8\Lsun$), 
NGC205 (dE; $L_V=4\times10^8\Lsun$), and M32 (NGC221, E2; $L_V = 4\times10^8\Lsun$).

There are at least 35 satellite galaxies around M31, and some are
very very small.
The smallest known Local Group satellites are less than $10^4\Lsun$.

{\bf See Table 4.1 and Fig. 4.1 of Sparke and Gallagher.}

The most important Milky Way satellite galaxies are the
Magellanic Clouds.  The LMC is about $14\kpc$ across,
while the SMC is about $8\kpc$ across.  The LMC has
10\% the Milky Way brightness, while the SMC is about
100\% the MW luminosity.  The rotation speed of 
HI in the LMC reaches about $80\km~\s^{-1}$.

{\bf See Figure 4.4 of Sparke and Gallagher.}

A bridge of gas connects the LMC and the SMC, while
the Magellanic Stream trails behind them and wraps
about a third around the sky.

{\bf See Figure 4.6 of Sparke and Gallagher.}

There are some very faint dwarf galaxies, some of 
which are dSph.  Note that the smallest dSph have
velocity dispersions that are comparable to globular
clusters.  However, we think that dwarf spheroidals
have dark matter halos.

{\bf See Table 4.2 of Sparke and Gallagher.}

\subsection{M31}

Key facts about Andromeda (M31) 
\begin{itemize}
\item Luminosity is 50\% larger than MW
\item Disk scale length is 2$\times$ longer
\item Disk circular speed is $\sim260\km~\s^{-1}$
\item It has 300 known globular clusters, about $2\times$ as many as the Milky Way
\item M31 satellites include M32
\item The M31 bulge is proportionally larger than the MW's.
\item $4-6\times10^9\Msun$ of HI, about 50\% more than MW
\end{itemize}

\subsection{M33}

Key facts about the Triangulum (M33)

\begin{itemize}
\item Sc or Scd, about 1/3 luminosity of the MW
\item Disk scale length is $h\sim1.7\kpc$.
\item Rotational velocity is $\sim120\km~\s^{-1}$.
\item Gas rich ($1.5\times10^9\Msun$ of HI), but little CO.
\item Nuclear star cluster with $L\sim2.5\times10^6\Lsun$.
\end{itemize}

\section{Satellite Dynamics: The Tidal Limit}

The orbits of stars in satellites can be very complicated.
While orbits in many-body systems are intractable analytically,
if a satellite follows a circular orbit and the
gravitational potential is constant in a frame that
uniformly rotates about the common center of mass, we
can define an effective potential and use it to 
describe the desired orbit of a star in the satellite.

Consider a vector $\vu$ that is constant in a 
non-rotating frame. In a rotating frame $\vu$
will change at the rate $d\vu/dt' = -\vOmega \times \vu$
where $d/dt'$ is the derivative measured by
the rotating observer and $\vOmega = \Omega\hat{\mathbf{z}}$ is the
vector angular frequency of the rotation about the $\hat{\mathbf{z}}$
direction.

If a star has velocity $\vv$ and position $\vx$ relative to an inertial
frame, then if a rotating observer chooses their coordinates such that
$\vx'\equiv\vx$ the velocity they observe would be
\begin{equation}
\vv' \equiv \frac{d\vx'}{dt'} = \vv - \vOmega \times \vx.
\end{equation}
\noindent
The velocity changes at the rate
\begin{eqnarray}
\frac{d\vv'}{dt'} &=& \frac{d\vv'}{dt} - \vOmega \times \vv' = \frac{d\vv}{dt} - \vOmega \times \vv - \vOmega \times \vv'\nonumber\\
&=& - \nabla \Phi - 2 \vOmega \times \vv' - \vOmega\times(\vOmega \times \vx)
\end{eqnarray}
\noindent
If you take the scalar product of $\vv'$ with the last term in the RHS, you find
\begin{equation}
-\vv'\cdot\vOmega\times(\vOmega \times \vx) = \Omega^2(\vx\cdot\vv') - (\vv'\cdot\vOmega)(\vOmega\cdot\vx) = \frac{1}{2}\frac{d}{dt'}[(\vOmega\times\vx)^2]
\end{equation}
\noindent
But $\vv'\cdot(\vOmega\times\vv')=0$ and $\vx'=\vx$.  Taking another scalar product, we have
\begin{equation}
\label{eqn:KE_mod}
\frac{1}{2}\frac{d}{dt'}[\vv'^2 - (\vOmega \times \vx')^2] = -\vv' \cdot \nabla\Phi(\vx')
\end{equation}
If $\vOmega$ follows a satellite in its orbit, then in the rotating frame the gravitational
potential does not change with time. The potential at the star's position
changes at the rate $d\Phi/dt' = \vv'\cdot\nabla\Phi$.

We now define the {\it Jacobi constant} $E_J$ by
\begin{equation}
E_J = \frac{1}{2}\vv'^2 + \Phieff(\vx'),~~\mathrm{where}~\Phieff(\vx')\equiv\Phi(\vx') - \frac{1}{2}(\vOmega\times\vx')^2
\end{equation}
\noindent
and we see that Equation \ref{eqn:KE_mod} suggest that $E_J$ does not change along the star's orbit.
We can re-write $E_J$ in terms of the star's energy and angular momentum in the inertial frame as
\begin{equation}
E_J = \frac{1}{2}(\vv-\vOmega\times\vx)^2 + \Phieff = \frac{1}{2}\vv^2 + \Phi(\vx,t) - \Omega\cdot(\vx\times\vv) = E - \vOmega\cdot\vL
\end{equation}

\subsection{A Simple Example of the Tidal Limit}

Consider the satellite to have mass $m$ and the main galaxy to have mass $M$, separated by distance $D$,
orbiting the center of mass $C$ with angular speed $\Omega$.  Take $x$ as the coordinate
along the ray between $m$ toward $M$.  The center of mass is then at $x=DM/(M+m)$.  The
effective potential is then
\begin{equation}
\Phieff(x) = -\frac{GM}{|D-x|} -\frac{Gm}{|x|} - \frac{\Omega^2}{2}\left(x-\frac{DM}{M+m}\right)^2.
\end{equation}

The effective potential has three maxima known as the {\it Lagrange points}, [$L_1$, $L_2$, $L_3$].  Let's find them by setting
the derivative of the effective potential to zero.
\begin{equation}
\label{eqn:find_phieff_max}
\frac{\partial\Phieff}{\partial x} = 0 = -\frac{GM}{(D-x)^2} \pm \frac{Gm}{x^2} - \Omega^2\left(x-\frac{DM}{M+m}\right)
\end{equation}
\noindent
The acceleration $\Omega^2DM/(M+m)$ of $m$ as it circles $C$ owes to the gravitational
attraction of $M$. We can then write
\begin{equation}
\Omega\frac{DM}{M+m}=\frac{GM}{D^2},~~\mathrm{so}~\Omega^2=\frac{G(M+m)}{D^3}
\end{equation}
\noindent
If the satellite is much less massive than the main galaxy, $L_1$ and $L_2$ will lie close to $m$.
Substitute $\Omega^2$ into Equation \ref{eqn:find_phieff_max}, and expand in powers of $x/D$ to find
\begin{equation}
0 \approx -\frac{GM}{D^2} - 2\frac{GM}{D^3}x \pm \frac{Gm}{x^2} - \frac{G(M+m)}{D^3}\left(x-\frac{DM}{M+m}\right)
\end{equation}
\noindent
At the Lagrange points $L_1$ and $L_2$ we have
\begin{equation}
x = \pm r_J, ~~\mathrm{where}~r_J = D\left(\frac{m}{3M+m}\right)^{1/3}
\end{equation}
\noindent
Stars that cannot stray further from the satellite than $r_J$, the {\it Jacobi radius} or {\it Roche limit},
will remain bound to it.  The radius $L_1$ is not where the gravitational force from the satellite and 
main galaxy are the same, but where the effective potential has a minimum and lies further from the satellite.
This radius is also where expanding stars lose mass to their companion.

When $M\gg m$, then the mean density within $r_J$ is three times the mean density within $D$ of the main galaxy.
A star orbiting the satellite near $r_J$ will have an orbital period comparable to the orbit period of the
satellite around the main galaxy.

When satellites are not on circular orbits, the relevant $r_J$ is determined at the pericenter. If the
satellite is orbiting within the dark matter halo of the main galaxy, the relevant radius is
\begin{equation}
r_J = D\left[\frac{m}{2M(<D)}\right]^{1/3}
\end{equation}
where $M(<D)$ is the mass within D.

What's the Jacobi radius of the LMC-Milky Way system? The LMC is at a distance of $\sim50\kpc$ where
the speed of a circular orbit is about the same as it is at the solar circle, or about $\sim200\km~\s^{-1}$.
The mass of the Milky Way within the LMC's orbit is about $5\times10^{11}\Msun$.  The LMC mass is about
$10^{10}\Msun$, so we have
\begin{equation}
r_J\approx 50\kpc~\times~\left(\frac{10^{10}\Msun}{2\times5\times10^{11}\Msun}\right)^{1/3} \approx 11\kpc.
\end{equation}
The LMC disk lies within this radius, but the SMC is too far away to stay bound to the LMC.

\section{Galactic Chemical Evolution}

An important clue in the formation and evolution of a galaxy is its metallicity content.
Since stars provide the majority of the metals to be found in a galaxy, we can try to connect the
metallicity of gas and stars in a galaxy with its star formation rate over time.

The simplest model we can devise assumes that the gas in a galaxy is instantaneously 
mixed and no gas flows in or out.  This kind of model is called a {\it one-zone},
{\it instantaneous recycling}, {\it closed-box} model.

We make the following definitions:

\begin{itemize}
\item $M_g(t)$ is the mass in gas at time $t$.
\item $M_\star(t)$ is the mass in low-mass stars, white dwarfs, neutron
stars or black holes
\item $M_Z(t)$ is the total mass in heavy elements in the galactic gas
\item The metallicity is $Z = M_Z/M_g$.
\end{itemize}

At some time $t$ a mass $\Delta M_\star'$ is formed. After the massive
stars evolve (quickly), only $\Delta M_\star$ of stars is left.

These stars return gas to the ISM, including $p\Delta M_\star$ of
heavy elements.  The quantity $p$ is called the yield, and
represents an average over the initial mass function (lots of
other things can affect the yield!).

The change in the metal content of the galaxy from the population
of $\Delta M_\star'$ is
\begin{equation}
\Delta M_Z = p \Delta M_\star - Z \Delta M_\star = (p-Z)\Delta M_\star
\end{equation}
\noindent
so the metallicity of the gas increases by
\begin{equation}
\label{eqn:metallicity_change}
\Delta Z \equiv \Delta \left(\frac{M_z}{M_g}\right) = \frac{p\Delta M_\star - Z[\Delta M_\star + \Delta M_g]}{M_g}
\end{equation}
\noindent
but since no gas leaves are arrives, we have that
\begin{equation}
\Delta M_\star + \Delta M_g = 0
\end{equation}
\noindent
Let's assume the yield does not depend on the current metallicity, such that $p$ is independent of $Z$.
We can integrate Equation \ref{eqn:metallicity_change} to find the metallicity with time as
\begin{equation}
\label{eqn:metallicity}
Z(t) = Z(t=0) + p \ln\left[\frac{M_g(t=0)}{M_g(t)}\right]
\end{equation}
The metallicity grows with time as the gas is consumed.  The mass in stars that have
formed $M_\star(t) = M_g(t=0) - M_g(t)$, so the mass in stars with metallicity
less that $Z(t)$ is
\begin{equation}
M_\star(<Z) = M_g(0)\left[1 - \exp\left\{-\frac{[Z-Z(0)]}{p}\right\}\right]
\end{equation}
\noindent
This model can therefore explain why metallicity is low where there is a lot of gas.

Once all the gas is gone, the mass of stars with metallicity between $Z$ and $\Delta Z$
should be
\begin{equation}
\frac{dM_\star(<Z)}{dZ} \Delta Z \propto \exp\left\{-\frac{[Z(t)-Z(0)]}{p}\right\}\Delta Z.
\end{equation}
\noindent
If the initial metallicity is $Z(0)=0$, then the final metallicity when the gas
is all used up is $Z_f = p$.

Using Equation \ref{eqn:metallicity}, we can estimate the yield in the
local Solar region of the MW.  The current stellar mass density is $\Sigma_\star\approx(30-40)\Msun~\pc^{-2}$.
The current gas density is $\Sigma_g \approx 13\Msun~\pc^{-2}$.  By Equation \ref{eqn:metallicity}, and
assuming $Z(0) \approx 0$, then we have
\begin{equation}
Z(\mathrm{today}) \approx Z_{\odot} \approx p \ln (50/13),~~\mathrm{so}~p\approx0.74Z_{\odot}.
\end{equation}
\noindent
However, in this model
the fraction with stars that have less than 1/4 the solar metallicity should be
\begin{equation}
\frac{M_\star(Z<Z_{\odot}/4)}{M_\star(<Z_\odot)} = \frac{1 - \exp[-Z_\odot/(4p)]}{1-\exp(-Z_\odot/p)}\approx 0.4
\end{equation}
\noindent
but only 25\% of observed stars in the solar neighborhood have $<1/4Z_{\odot}$.  This discrepancy
is called the {\it G-dwarf problem}. One solution is for pre-enrichment of $Z(0)\approx0.15Z_{\odot}$
before the stars in the solar neighborhood started forming.  Another option is that
the stellar distribution formed from a smaller amount of the initial gas content, and then later
some gas was added.  Further, gas returned from stars could have low metallicity or gas
lost by the Milky Way could have been preferentially metal rich.


\end{document}