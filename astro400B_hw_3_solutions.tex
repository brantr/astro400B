\documentclass[]{article}
\usepackage[margin=1.0in]{geometry}
\usepackage{amssymb}

%title material
\title{Astronomy 400B Lecture X: YYY}
\author{Brant Robertson}
\date{March, 2015}


%include latex definitions

%average
\newcommand{\ave}[1]{\langle#1\rangle}
%radian
\newcommand{\rad}{\mathrm{rad}}

%astronomical unit
\newcommand{\AU}{\mathrm{AU}}

%centimeter
\newcommand{\cm}{\mathrm{cm}}

%meter
\newcommand{\m}{\mathrm{m}}

%kilometer
\newcommand{\km}{\mathrm{km}}

%parsec
\newcommand{\pc}{\mathrm{pc}}

%kiloparsec
\newcommand{\kpc}{\mathrm{kpc}}

%megaparsec
\newcommand{\Mpc}{\mathrm{Mpc}}

%gigaparsec
\newcommand{\Gpc}{\mathrm{Gpc}}

%light year
\newcommand{\ly}{\mathrm{ly}}

%second
\newcommand{\s}{\mathrm{s}}
\newcommand{\yr}{\mathrm{yr}}
\newcommand{\Gyr}{\mathrm{Gyr}}

%solar mass
\newcommand{\Msun}{M_{\odot}}

%grams
\newcommand{\g}{\mathrm{g}}

%erg
\newcommand{\erg}{\mathrm{erg}}

%solar luminosity
\newcommand{\Lsun}{L_{\odot}}

%jansky 
\newcommand{\Jy}{\mathrm{Jy}}

%flux density
\newcommand{\Fnu}{F_{\nu}}
\newcommand{\Flambda}{F_{\lambda}}
\newcommand{\Lnu}{L_{\nu}}

%hertz 
\newcommand{\Hz}{\mathrm{Hz}}

%angstrom
\newcommand{\Ang}{\mathrm{\r{A}}}

%solar radius
\newcommand{\Rsun}{R_{\odot}}

%stefan-boltzmann
\newcommand{\sigmaSB}{\sigma_{\mathrm{SB}}}

%boltzmann
\newcommand{\kB}{k_{\mathrm{B}}}

%kelvin
\newcommand{\K}{\mathrm{K}}

%T_bandpass
\newcommand{\TBP}{T_{\mathrm{BP}}}

%magnitude
\newcommand{\mg}{\mathrm{mag}}


%arcsecond
\newcommand{\arcsec}{\mathrm{arcsec}}

%begin the document
\begin{document}

%make the title, goes after document begins
\maketitle

%Problem
\section{Sparke \& Gallagher Problem 3.20}

\subsection{Show that there are no circular orbits at $r<3GM_{BH}/c^2$}

Our effective potential is
\begin{equation}
\Phi_{eff}(r,L) = \frac{1}{2}\left(c^2-\frac{2GM_{BH}}{r}\right)\left(1+ \frac{L^2}{c^2r^2}\right).
\end{equation}
\noindent
Circular orbits exist only where $\partial \Phi_{eff}/\partial r = 0$.  We have
\begin{equation}
\frac{\partial \Phi_{eff}}{\partial r} = 3GM_{BH}L^2 - \frac{c^2L^2}{r} + GM_{BH}c^2r^2 =0
\end{equation}
Solving for $r$, we have
\begin{equation}
\label{r:dphidr}
r = \frac{c^2L^2 \pm c^2 L^2 \sqrt{1-12\frac{G^2M_{BH}^2}{c^2L^2}} }{2GM_{BH}c^2}
\end{equation}
\noindent
Taking the limit of large $L$, we have
\begin{equation}
r = \frac{c^2L^2 \pm c^2 L^2 \left( 1-6 \frac{G^2M_{BH}^2}{c^2L^2}\right) }{2GM_{BH}c^2} = \left[\frac{L^2}{GM_{BH}}, \frac{3GM_{BH}}{c}\right]
\end{equation}
\noindent
so, for large $L$ there is a minimum radius $r = 3GM_{BH}/c^2$ where $\frac{\partial \Phi_{eff}}{\partial r}=0$.

\subsection{Show that the stable circular orbits lie at $r>6GM_{BH}/c^2$ with $L>2\sqrt{3}GM_{BH}/c$}

There are at least two ways to solve this problem.  First, for Equation \ref{r:dphidr} 
we can see that the discriminant is zero when $L = \sqrt{12}GM_{BH}/c = 2 \sqrt{3} G M_{BH}/c$.
The radius of this circular orbit (that must be larger than $r = 3 GM_{BH}/c^2$) is then $r = c^2 L^2 / 2 GM_{BH} c^2 = 6 G M_{BH} / c^2$.

The second method is just to assume $L = 2 \sqrt{3} G M_{BH}/c$ and solve for where $\frac{\partial^{2}\Phi_{eff}(r)}{\partial r^2} = 0$.
Our effective potential is
\begin{equation}
\Phi_{eff}(r,L) = \frac{1}{2}\left(c^2-\frac{2GM_{BH}}{r}\right)\left(1+ \frac{L^2}{c^2r^2}\right).
\end{equation}
\noindent
If we plug in $L=2\sqrt{3}GM_{BH}/c$, we have
\begin{equation}
\Phi_{eff}(r) = \frac{1}{2}\left(1+\frac{12G^2M^2}{c^4 r^2}\right)\left(c^2 - \frac{2GM}{r}\right)
\end{equation}
We need to find where
\begin{equation}
\kappa^2 = \frac{\partial^{2}\Phi_{eff}(r)}{\partial r^2} = 0.
\end{equation}
\noindent
We have that
\begin{equation}
\frac{\partial^{2}\Phi_{eff}(r)}{\partial r^2} = -\frac{48G^3 M^3}{c^4 r^5} + \frac{36 G^2 M^2 (c^2-\frac{2GM}{r})}{c^4 r^4} - \frac{2GM(1+\frac{12G^2M^2}{c^4r^2})}{r^3} = 0
\end{equation}
\noindent
Multiplying by $r^5$ and solving for $r$ gives $r = 6GM/c^2$.



\end{document}