\documentclass[]{article}
\usepackage[margin=1.0in]{geometry}
\usepackage{amssymb}

%title material
\title{Astronomy 400B Homework \#3 Solutions}
\author{Brant Robertson}
\date{March, 2015}


%include latex definitions

%average
\newcommand{\ave}[1]{\langle#1\rangle}
%radian
\newcommand{\rad}{\mathrm{rad}}

%astronomical unit
\newcommand{\AU}{\mathrm{AU}}

%centimeter
\newcommand{\cm}{\mathrm{cm}}

%meter
\newcommand{\m}{\mathrm{m}}

%kilometer
\newcommand{\km}{\mathrm{km}}

%parsec
\newcommand{\pc}{\mathrm{pc}}

%kiloparsec
\newcommand{\kpc}{\mathrm{kpc}}

%megaparsec
\newcommand{\Mpc}{\mathrm{Mpc}}

%gigaparsec
\newcommand{\Gpc}{\mathrm{Gpc}}

%light year
\newcommand{\ly}{\mathrm{ly}}

%second
\newcommand{\s}{\mathrm{s}}
\newcommand{\yr}{\mathrm{yr}}
\newcommand{\Gyr}{\mathrm{Gyr}}

%solar mass
\newcommand{\Msun}{M_{\odot}}

%grams
\newcommand{\g}{\mathrm{g}}

%erg
\newcommand{\erg}{\mathrm{erg}}

%solar luminosity
\newcommand{\Lsun}{L_{\odot}}

%jansky 
\newcommand{\Jy}{\mathrm{Jy}}

%flux density
\newcommand{\Fnu}{F_{\nu}}
\newcommand{\Flambda}{F_{\lambda}}
\newcommand{\Lnu}{L_{\nu}}

%hertz 
\newcommand{\Hz}{\mathrm{Hz}}

%angstrom
\newcommand{\Ang}{\mathrm{\r{A}}}

%solar radius
\newcommand{\Rsun}{R_{\odot}}

%stefan-boltzmann
\newcommand{\sigmaSB}{\sigma_{\mathrm{SB}}}

%boltzmann
\newcommand{\kB}{k_{\mathrm{B}}}

%kelvin
\newcommand{\K}{\mathrm{K}}

%T_bandpass
\newcommand{\TBP}{T_{\mathrm{BP}}}

%magnitude
\newcommand{\mg}{\mathrm{mag}}


%arcsecond
\newcommand{\arcsec}{\mathrm{arcsec}}

%begin the document
\begin{document}

%make the title, goes after document begins
\maketitle

%Problem
\section{Sparke \& Gallagher Problem 3.20}

\subsection{Show that there are no circular orbits at $r<3GM_{BH}/c^2$}

Our effective potential is
\begin{equation}
\Phi_{eff}(r,L) = \frac{1}{2}\left(c^2-\frac{2GM_{BH}}{r}\right)\left(1+ \frac{L^2}{c^2r^2}\right).
\end{equation}
\noindent
Circular orbits exist only where $\partial \Phi_{eff}/\partial r = 0$.  We have
\begin{equation}
\frac{\partial \Phi_{eff}}{\partial r} = 3GM_{BH}L^2 - \frac{c^2L^2}{r} + GM_{BH}c^2r^2 =0
\end{equation}
Solving for $r$, we have
\begin{equation}
\label{r:dphidr}
r = \frac{c^2L^2 \pm c^2 L^2 \sqrt{1-12\frac{G^2M_{BH}^2}{c^2L^2}} }{2GM_{BH}c^2}
\end{equation}
\noindent
Taking the limit of large $L$, we have
\begin{equation}
r = \frac{c^2L^2 \pm c^2 L^2 \left( 1-6 \frac{G^2M_{BH}^2}{c^2L^2}\right) }{2GM_{BH}c^2} = \left[\frac{L^2}{GM_{BH}}, \frac{3GM_{BH}}{c}\right]
\end{equation}
\noindent
so, for large $L$ there is a minimum radius $r = 3GM_{BH}/c^2$ where $\frac{\partial \Phi_{eff}}{\partial r}=0$.

\subsection{Show that the stable circular orbits lie at $r>6GM_{BH}/c^2$ with $L>2\sqrt{3}GM_{BH}/c$}

There are at least two ways to solve this problem.  First, for Equation \ref{r:dphidr} 
we can see that the discriminant is zero when $L = \sqrt{12}GM_{BH}/c = 2 \sqrt{3} G M_{BH}/c$.
The radius of this circular orbit (that must be larger than $r = 3 GM_{BH}/c^2$) is then $r = c^2 L^2 / 2 GM_{BH} c^2 = 6 G M_{BH} / c^2$.

The second method is just to assume $L = 2 \sqrt{3} G M_{BH}/c$ and solve for where $\frac{\partial^{2}\Phi_{eff}(r)}{\partial r^2} = 0$.
Our effective potential is
\begin{equation}
\Phi_{eff}(r,L) = \frac{1}{2}\left(c^2-\frac{2GM_{BH}}{r}\right)\left(1+ \frac{L^2}{c^2r^2}\right).
\end{equation}
\noindent
If we plug in $L=2\sqrt{3}GM_{BH}/c$, we have
\begin{equation}
\Phi_{eff}(r) = \frac{1}{2}\left(1+\frac{12G^2M^2}{c^4 r^2}\right)\left(c^2 - \frac{2GM}{r}\right)
\end{equation}
We need to find where
\begin{equation}
\kappa^2 = \frac{\partial^{2}\Phi_{eff}(r)}{\partial r^2} = 0.
\end{equation}
\noindent
We have that
\begin{equation}
\frac{\partial^{2}\Phi_{eff}(r)}{\partial r^2} = -\frac{48G^3 M^3}{c^4 r^5} + \frac{36 G^2 M^2 (c^2-\frac{2GM}{r})}{c^4 r^4} - \frac{2GM(1+\frac{12G^2M^2}{c^4r^2})}{r^3} = 0
\end{equation}
\noindent
Multiplying by $r^5$ and solving for $r$ gives $r = 6GM/c^2$.


\section{Sparke \& Gallagher Problem 3.24}

The divergence theorem states
\begin{equation}
\int \nabla^2 \Phi dV = \oint \nabla \Phi \cdot d\vS
\end{equation}
\noindent
but we know that $\nabla^2 \Phi = 4\pi G \rho$.
For a thin disk, integrated over some area $A = \int dS$ we have
\begin{equation}
4\pi G \int \rho dV = 4\pi G \int \Sigma dS = 4\pi G \Sigma A
\end{equation}
\noindent
The surface integral over the same area $A$ will have two
components above and below the disk.  The dot product
$\nabla \Phi \cdot d\vS>0$ on both sides, so we have
\begin{equation}
\oint \nabla \Phi \cdot d\vS = 2 A \nabla \Phi %\cdot \frac{\vz}{|z|}
\end{equation}
\noindent
We then have
\begin{eqnarray}
\nabla \Phi = \frac{1}{2A} 4 \pi G \Sigma A = 2\pi G \Sigma
\end{eqnarray}
\noindent
Above the disk, we have
\begin{equation}
\int_0^{z} \nabla \Phi dz' = \Phi(z) = \int_{0}^{z} 2\pi G \Sigma dz' = 2\pi G \Sigma z~~~\mathrm{for}~~z>0.
\end{equation}
Similarly, below the disk we have
\begin{equation}
\int_{z}^{0} \nabla \Phi dz' = \Phi = \int_{z}^{0} 2\pi G \Sigma dz' = 2\pi G \Sigma (-z) =2\pi G \Sigma|z|  ~~~\mathrm{for}~~z<0.
\end{equation}
\noindent
So, connecting above and below the disk we can write
\begin{equation}
\Phi = 2\pi G \Sigma|z| 
\end{equation}
\noindent
The vertical force is
\begin{equation}
-\nabla \Phi = -\nabla 2\pi G \Sigma|z| = -2 \pi G \Sigma \frac{z}{|z|},
\end{equation}
\noindent
or
\begin{eqnarray}
-\nabla \Phi &=& -2\pi G \Sigma~~\mathrm{for}~z>0\\
&=& 2\pi G \Sigma~~\mathrm{for}~z<0\\
\end{eqnarray}
\noindent
Which is independent of $z$.
\noindent
We can check that for $z\ne0$
\begin{equation}
\nabla^2 \Phi = 2\pi G \Sigma \frac{\partial^2}{\partial z^2}|z| = 2\pi G \Sigma \frac{\partial}{\partial z}\frac{z}{|z|} = 2\pi G \Sigma \left(\frac{1}{|z|} - \frac{z^2}{|z|^3} \right) =  2\pi G \Sigma \left(\frac{1}{|z|} - \frac{1}{|z|} \right) = 0.
\end{equation}
\noindent
Now consider the equation
\begin{equation}
\frac{d}{dz}[n(z)\sigma_z^2] = - \frac{\partial \Phi}{\partial z} n(z)
\end{equation}
\noindent
Using the previous information, above the disk plane we have
\begin{eqnarray}
\sigma_z^2 \frac{dn}{dz} &=& -2\pi G \Sigma \frac{z}{|z|} n(z) = -2\pi G \Sigma n(z)\\
\frac{dn}{n} &=& -\frac{2\pi G\Sigma}{\sigma_z^2}\\
\ln n &=& -\frac{2\pi G\Sigma}{\sigma_z^2} z\\
n(z) &=& \exp\left(-\frac{2\pi G\Sigma}{\sigma_z^2} z \right) = \exp\left(- \frac{z}{h_z}\right)
\end{eqnarray}
\noindent
where
\begin{equation}
h_z = \sigma_z^2 / (2\pi G \Sigma)
\end{equation}
\noindent
If $\sigma_z = 20~\km~\s^{-1}$, then taking $\Sigma = 50~\Msun~\pc^{-2}$ we have
\begin{equation}
h_z = \frac{(20~\km~\s^{-1})^2}{2\pi(4.301\times10^{-6}~\km^2~\s^{-2}~\Msun^{-1}~\kpc)(50~\Msun~\pc^{-2})(1000~\pc/\kpc)}
\end{equation}
\noindent
which gives $h_z \approx 296~\pc$.


\section{Sparke \& Gallagher Problem 3.25}
The distribution function is
\begin{equation}
f(E_z) = \frac{n_0}{\sqrt{2\pi\sigma^2}}\exp\left[-\left(E_z\right)/\sigma^2\right].
\end{equation}
\noindent
%Let's define the relative potential $\Psi\equiv -\Phi$ and the
%relative energy $\mathcal{E} \equiv -E_z$.

We find the number density as
\begin{eqnarray}
n(z) &=& \int_{-\infty}^{\infty} f(z,v_z) dv_z\\
n(z) &=& \int_{-\infty}^{\infty} \frac{n_0}{\sqrt{2\pi\sigma^2}}\exp\left[-\left(E_z(z,v_z)\right)/\sigma^2\right] dv_z\\
n(z)&=&\frac{n_0}{\sqrt{2\pi\sigma^2}}\int_{-\infty}^{\infty}  \exp(-\Phi(z)/\sigma^2)\exp\left[-\left(\frac{1}{2}v_z^2\right)/\sigma^2\right]dv_z\\
n(z)&=&n_0\exp(-\Phi(z)/\sigma^2)\int_{-\infty}^{\infty} \frac{1}{\sqrt{2\pi\sigma^2}} \exp\left[-\left(\frac{1}{2}v_z^2\right)/\sigma^2\right]dv_z\\
n(z)&=&n_0\exp(-\Phi(z)/\sigma^2)
\end{eqnarray}
For $z=0$, we have $\Phi(z) = 0$ and
%\begin{eqnarray}
%n(0)&=&\frac{n_0}{\sqrt{2\pi\sigma^2}}\int_{-\infty}^{\infty}  \exp\left[-\frac{1}{2}(v_z^2/\sigma^2)\right]dv_z\\
%\end{eqnarray}
%\noindent
so $n(0) = n_0$.

\subsection{A Self-consistent model}
Let
\begin{equation}
\Phi(z) = \sigma^2\phi(z).
\end{equation}
\noindent
Poisson's equation is
\begin{eqnarray}
\frac{\partial^2\Phi}{\partial z^2} = 4\pi G \rho &=& 4\pi G m n_0 \exp(-\Phi(z)/\sigma^2)\\
\frac{\partial^2\Phi}{\partial z^2}  &=& 4\pi G m n_0 \exp[-\phi(z)]
\end{eqnarray}
\noindent
where $\phi(z)=\Phi(z)/\sigma^2$.
Defining $y \equiv z/z_0$ with $z_0^2 = \sigma^2/(8\pi G m n_0)$, we have
that $dy = dz/z_0$ or $dz = z_0 dy$ and
\begin{eqnarray}
\frac{\partial^2\Phi}{\partial z^2} = \frac{\sigma^2}{z_0^2}\frac{\partial^2\phi}{\partial y^2}  &=& 4\pi G m n_0 \exp[-\phi(z)]\\
\frac{\partial^2\phi}{\partial y^2}  &=& \frac{4\pi G m n_0 z_0^2}{\sigma^2} \exp[-\phi(z)] = \frac{1}{2}\exp[-\phi(z)]
\end{eqnarray}
\noindent
or
\begin{equation}
2 \frac{d^2 \phi}{d y^2} = e^{-\phi}
\end{equation}
\noindent
Note that $2d^2\phi/dy^2 \cdot d\phi/dy = d(d\phi/dy)^2/dy$.  So we have
\begin{eqnarray}
2d^2\phi/dy^2 \cdot d\phi/dy = d(d\phi/dy)^2/dy = e^{-\phi} d\phi/dy\\
\int_{0}^{y} \frac{d}{dy'}\left[\left(\frac{d\phi}{dy}\right)^2\right] dy' = \int_{0}^{\phi} e^{-\phi'} \frac{d\phi'}{dy}dy\\
\left[\left(\frac{d\phi}{dy}\right)^2\right] |_{0}^{y} = -e^{-\phi} |_{0}^{\phi}\\
\frac{d\phi}{dy} = \sqrt{1-e^{-\phi}}\\
y(\phi) = \int_{0}^{\phi}\frac{d\phi'}{\sqrt{1-e^{-\phi'}}}\\
u = \exp[-\phi/2]\\
du = -\exp[-\phi/2] d\phi/2 \\
d\phi = -2 u^{-1} du\\
y(\phi) = -\int \frac{2 u^{-1} du}{\sqrt{1-u^2}}\\ 
y(\phi) = -2[ \ln u - \ln[1 + \sqrt{1-u^2}]]\\
y(\phi) = -2\left[ \ln \frac{u}{1+\sqrt{1-u^2}}\right]= -2\left[ \ln \frac{e^{-\phi/2}}{1+\sqrt{1-e^{-\phi}}}\right]\\
\phi(y) = \ln\left[\frac{1}{4}e^{-y}(1+e^y)^{2}\right]\\
n(z) = n_0 \exp[-\phi(y)] = n_0 \exp  \ln\left[\frac{1}{4}e^{-y}(1+e^y)^{2}\right]^{-1} = n_0 \frac{4 e^y}{(1+e^y)^2} = n_0 \left[\frac{2 e^{y/2}}{1+e^{y}}\right]^2\\
\end{eqnarray}
\noindent
But
\begin{equation}
\mathrm{sech}~x= \frac{2e^x}{1+e^{2x}}
\end{equation}
\noindent
So, we have
\begin{equation}
n(z) = n_0 \left[\frac{2 e^{y/2}}{1+e^{y}}\right]^2 = n_0 \mathrm{sech}^{2}~\frac{y}{2} = n_0 \mathrm{sech}^2~\frac{z}{2 z_0}
\end{equation}
%\phi(y) = \ln \left[\frac{1}{e^{y/2} - \sqrt{e^{y}-1}}\right]^2\\
%n(z) = n_0\exp(-\phi(y)) =  n_0\left[e^{z/2z_0} - \sqrt{e^{z/z_0}-1}\right]^2
%u = \exp[-\phi/2]\\
%du = -\exp[-\phi/2] d\phi/2 \\
%d\phi = -2 u^{-1} du
%y = 2 \ln \left[ e^{-\phi/2} + \sqrt{e^{\phi}-1}\right]\\
%\exp(y/2) =  e^{-\phi/2} + \sqrt{e^{\phi}-1}


\end{document}