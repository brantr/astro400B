\documentclass[]{article}
\usepackage[margin=1.0in]{geometry}
\usepackage{amssymb}

%title material
\title{Astronomy 400B Lecture 5: Stellar Orbits}
\author{Brant Robertson}
\date{February, 2015}


%include latex definitions

%average
\newcommand{\ave}[1]{\langle#1\rangle}
%radian
\newcommand{\rad}{\mathrm{rad}}

%astronomical unit
\newcommand{\AU}{\mathrm{AU}}

%centimeter
\newcommand{\cm}{\mathrm{cm}}

%meter
\newcommand{\m}{\mathrm{m}}

%kilometer
\newcommand{\km}{\mathrm{km}}

%parsec
\newcommand{\pc}{\mathrm{pc}}

%kiloparsec
\newcommand{\kpc}{\mathrm{kpc}}

%megaparsec
\newcommand{\Mpc}{\mathrm{Mpc}}

%gigaparsec
\newcommand{\Gpc}{\mathrm{Gpc}}

%light year
\newcommand{\ly}{\mathrm{ly}}

%second
\newcommand{\s}{\mathrm{s}}
\newcommand{\yr}{\mathrm{yr}}
\newcommand{\Gyr}{\mathrm{Gyr}}

%solar mass
\newcommand{\Msun}{M_{\odot}}

%grams
\newcommand{\g}{\mathrm{g}}

%erg
\newcommand{\erg}{\mathrm{erg}}

%solar luminosity
\newcommand{\Lsun}{L_{\odot}}

%jansky 
\newcommand{\Jy}{\mathrm{Jy}}

%flux density
\newcommand{\Fnu}{F_{\nu}}
\newcommand{\Flambda}{F_{\lambda}}
\newcommand{\Lnu}{L_{\nu}}

%hertz 
\newcommand{\Hz}{\mathrm{Hz}}

%angstrom
\newcommand{\Ang}{\mathrm{\r{A}}}

%solar radius
\newcommand{\Rsun}{R_{\odot}}

%stefan-boltzmann
\newcommand{\sigmaSB}{\sigma_{\mathrm{SB}}}

%boltzmann
\newcommand{\kB}{k_{\mathrm{B}}}

%kelvin
\newcommand{\K}{\mathrm{K}}

%T_bandpass
\newcommand{\TBP}{T_{\mathrm{BP}}}

%magnitude
\newcommand{\mg}{\mathrm{mag}}


%arcsecond
\newcommand{\arcsec}{\mathrm{arcsec}}

%begin the document
\begin{document}

%make the title, goes after document begins
\maketitle

%first section
\section{Motion Under Gravity}

Newton's Law of Gravity point mass $M$ attracts another mass $m$ separated by 
distance $\vr$, causing a change in momentum $m\vv$ of:

\begin{equation}
\frac{d}{dt}(m\vv) = -\frac{GmM}{r^{3}} \vr
\end{equation}
\noindent
where $G$ is Newton's gravitational constant. For an $N$-body system, we have
\begin{equation}
\frac{d}{dt}(m{i}\vv_{i}) = -\sum_{j\ne i}\frac{G m_{i} m_{j}}{|\vx_i - \vx_j|^{3}} (\vx_i - \vx_j)
\end{equation}
\noindent
This equation can be re-written as
\begin{equation}
\frac{d}{dt}(m \vv_i) = - m \nabla\Phi(\vx_i)
\end{equation}
\noindent
where
\begin{equation}
\Phi(\vx_i) = - \sum_{i\ne j} \frac{G m_j}{|\vx_i - \vx_j|}
\end{equation}
\noindent
is the gravitational potential supplied by the point mass distribution at positions $\vx_i$.
Note we have chosen to define the potential such that $\Phi(x)\to0$ as $x\to\infty$ but
this is arbitrary. Note that
\begin{equation}
\nabla = \left[\frac{\partial}{\partial x},\frac{\partial}{\partial y},\frac{\partial}{\partial x}\right]
\end{equation}

\subsection{Continuous Matter Distributions}
Now consider a continuous distribution of matter density $\rho(\vx)$.  The
potential generated by $\rho(\vx)$ is given by
\begin{equation}
\label{eqn:potential}
\Phi(\vx) = - \int \frac{G\rho(\vx')}{|\vx - \vx'|}d^{3}\vx'
\end{equation}
\noindent
Note that the integral is performed over $\vx'$.  The force $\vF$ per unit mass
is
\begin{equation}
\vF(\vx) = - \nabla \Phi(\vx) = - \int \frac{G\rho(\vx')(\vx-\vx')}{|\vx-\vx'|^{3}}d^{3}\vx'
\end{equation}

\subsection{Poisson's Equation}

Take Equation \ref{eqn:potential} and apply the Laplacian operator 
\begin{equation}
\nabla^{2} \equiv \nabla \cdot \nabla = \left[\frac{\partial^{2}}{\partial x^{2}}+\frac{\partial^{2}}{\partial y^{2}}+\frac{\partial^{2}}{\partial z^{2}}\right]
\end{equation}
\noindent
to both sides.  Remembering that the operator acts on $\vx$ and not $\vx'$, we have
\begin{equation}
\label{eqn:pois_init}
\nabla^{2} \Phi(\vx) = - \int G \rho(\vx') \nabla^{2} \left(\frac{1}{|\vx-\vx'|}\right)d^{3}\vx'.
\end{equation}
\noindent
We can evaluate this by noting that
\begin{equation}
\nabla\left(\frac{1}{|\vx - \vx'|}\right), \nabla^{2}\left(\frac{1}{|\vx - \vx'|}\right) = 0.
\end{equation}
\noindent
So we conclude that outside of a very small region around $\vx$, $\nabla^{2}\Phi(\vx)=0$.
Let's take a spherical region $S(\epsilon)$ of radius $\epsilon$ centered on $\vx$. In proceeding, let's 
note that 
\begin{equation}
\nabla^{2} f(|\vx-\vx'|) = \nabla_{\vx'}^{2} f(|\vx-\vx'|)
\end{equation}
\noindent
for any function $f(|\vx - \vx'|)$. If we take $\epsilon$ to be small enough such 
that $\rho(\vx)\approx$ a constant, then we can write
\begin{eqnarray}
\label{eqn:pois_med}
\nabla^{2}\Phi(\vx) &\approx& - G\rho(\vx) \int_{S(\epsilon)} \nabla^2 \left( \frac{1}{|\vx - \vx'|} \right)d^{3}\vx' \nonumber \\
&=& - G \rho(\vx)  \int_{S(\epsilon)} \nabla_{\vx}^2 \left( \frac{1}{|\vx - \vx'|} \right)d V'.
\end{eqnarray}
\noindent
Now we get to use the {\it divergence} theorem
\begin{equation}
\int \nabla^{2} f dV = \oint \nabla f \cdot dS,
\end{equation}
\noindent
which allows us to write Equation \ref{eqn:pois_med} as
\begin{equation}
- G \rho(\vx)  \int_{S(\epsilon)} \nabla_{\vx}^2 \left( \frac{1}{|\vx - \vx'|} \right)d V' = - G \rho(\vx) \oint_S(\epsilon) \nabla\left(\frac{1}{|\vx-\vx'|}\right)
\end{equation}











\end{document}