\documentclass[]{article}
\usepackage[margin=1.0in]{geometry}
\usepackage{amssymb}

%title material
\title{Astronomy 400B Lecture 5: Stellar Orbits}
\author{Brant Robertson}
\date{February, 2015}


%include latex definitions

%average
\newcommand{\ave}[1]{\langle#1\rangle}
%radian
\newcommand{\rad}{\mathrm{rad}}

%astronomical unit
\newcommand{\AU}{\mathrm{AU}}

%centimeter
\newcommand{\cm}{\mathrm{cm}}

%meter
\newcommand{\m}{\mathrm{m}}

%kilometer
\newcommand{\km}{\mathrm{km}}

%parsec
\newcommand{\pc}{\mathrm{pc}}

%kiloparsec
\newcommand{\kpc}{\mathrm{kpc}}

%megaparsec
\newcommand{\Mpc}{\mathrm{Mpc}}

%gigaparsec
\newcommand{\Gpc}{\mathrm{Gpc}}

%light year
\newcommand{\ly}{\mathrm{ly}}

%second
\newcommand{\s}{\mathrm{s}}
\newcommand{\yr}{\mathrm{yr}}
\newcommand{\Gyr}{\mathrm{Gyr}}

%solar mass
\newcommand{\Msun}{M_{\odot}}

%grams
\newcommand{\g}{\mathrm{g}}

%erg
\newcommand{\erg}{\mathrm{erg}}

%solar luminosity
\newcommand{\Lsun}{L_{\odot}}

%jansky 
\newcommand{\Jy}{\mathrm{Jy}}

%flux density
\newcommand{\Fnu}{F_{\nu}}
\newcommand{\Flambda}{F_{\lambda}}
\newcommand{\Lnu}{L_{\nu}}

%hertz 
\newcommand{\Hz}{\mathrm{Hz}}

%angstrom
\newcommand{\Ang}{\mathrm{\r{A}}}

%solar radius
\newcommand{\Rsun}{R_{\odot}}

%stefan-boltzmann
\newcommand{\sigmaSB}{\sigma_{\mathrm{SB}}}

%boltzmann
\newcommand{\kB}{k_{\mathrm{B}}}

%kelvin
\newcommand{\K}{\mathrm{K}}

%T_bandpass
\newcommand{\TBP}{T_{\mathrm{BP}}}

%magnitude
\newcommand{\mg}{\mathrm{mag}}


%arcsecond
\newcommand{\arcsec}{\mathrm{arcsec}}

%begin the document
\begin{document}

%make the title, goes after document begins
\maketitle

%first section
\section{Motion Under Gravity}

Newton's Law of Gravity point mass $M$ attracts another mass $m$ separated by 
distance $\vr$, causing a change in momentum $m\vv$ of:

\begin{equation}
\frac{d}{dt}(m\vv) = -\frac{GmM}{r^{3}} \vr
\end{equation}
\noindent
where $G$ is Newton's gravitational constant. For an $N$-body system, we have
\begin{equation}
\frac{d}{dt}(m{i}\vv_{i}) = -\sum_{j\ne i}\frac{G m_{i} m_{j}}{|\vx_i - \vx_j|^{3}} (\vx_i - \vx_j)
\end{equation}
\noindent
This equation can be re-written as
\begin{equation}
\frac{d}{dt}(m \vv_i) = - m \nabla\Phi(\vx_i)
\end{equation}
\noindent
where
\begin{equation}
\Phi(\vx_i) = - \sum_{i\ne j} \frac{G m_j}{|\vx_i - \vx_j|}
\end{equation}
\noindent
is the gravitational potential supplied by the point mass distribution at positions $\vx_i$.
Note we have chosen to define the potential such that $\Phi(x)\to0$ as $x\to\infty$ but
this is arbitrary. Note that
\begin{equation}
\nabla = \left[\frac{\partial}{\partial x},\frac{\partial}{\partial y},\frac{\partial}{\partial x}\right]
\end{equation}

\subsection{Continuous Matter Distributions}
Now consider a continuous distribution of matter density $\rho(\vx)$.  The
potential generated by $\rho(\vx)$ is given by
\begin{equation}
\label{eqn:potential}
\Phi(\vx) = - \int \frac{G\rho(\vx')}{|\vx - \vx'|}d^{3}\vx'
\end{equation}
\noindent
Note that the integral is performed over $\vx'$.  The force $\vF$ per unit mass
is
\begin{equation}
\vF(\vx) = - \nabla \Phi(\vx) = - \int \frac{G\rho(\vx')(\vx-\vx')}{|\vx-\vx'|^{3}}d^{3}\vx'
\end{equation}

\subsection{Poisson's Equation}

Take Equation \ref{eqn:potential} and apply the Laplacian operator 
\begin{equation}
\nabla^{2} \equiv \nabla \cdot \nabla = \left[\frac{\partial^{2}}{\partial x^{2}}+\frac{\partial^{2}}{\partial y^{2}}+\frac{\partial^{2}}{\partial z^{2}}\right]
\end{equation}
\noindent
to both sides.  Remembering that the operator acts on $\vx$ and not $\vx'$, we have
\begin{equation}
\label{eqn:pois_init}
\nabla^{2} \Phi(\vx) = - \int G \rho(\vx') \nabla^{2} \left(\frac{1}{|\vx-\vx'|}\right)d^{3}\vx'.
\end{equation}
\noindent
We can evaluate this by noting that
\begin{equation}
\label{eqn:nabla}
\nabla\left(\frac{1}{|\vx - \vx'|}\right), \nabla^{2}\left(\frac{1}{|\vx - \vx'|}\right) = 0.
\end{equation}
\noindent
So we conclude that outside of a very small region around $\vx$, $\nabla^{2}\Phi(\vx)=0$.
Let's take a spherical region $S(\epsilon)$ of radius $\epsilon$ centered on $\vx$. In proceeding, let's 
note that 
\begin{equation}
\nabla^{2} f(|\vx-\vx'|) = \nabla_{\vx'}^{2} f(|\vx-\vx'|)
\end{equation}
\noindent
for any function $f(|\vx - \vx'|)$. If we take $\epsilon$ to be small enough such 
that $\rho(\vx)\approx$ a constant, then we can write
\begin{eqnarray}
\label{eqn:pois_med}
\nabla^{2}\Phi(\vx) &\approx& - G\rho(\vx) \int_{S(\epsilon)} \nabla^2 \left( \frac{1}{|\vx - \vx'|} \right)d^{3}\vx' \nonumber \\
&=& - G \rho(\vx)  \int_{S(\epsilon)} \nabla_{\vx'}^2 \left( \frac{1}{|\vx - \vx'|} \right)d V'.
\end{eqnarray}
\noindent
Now we get to use the {\it divergence} theorem
\begin{equation}
\int \nabla^{2} f dV = \oint \nabla f \cdot dS,
\end{equation}
\noindent
which allows us to write Equation \ref{eqn:pois_med} as
\begin{equation}
- G \rho(\vx)  \int_{S(\epsilon)} \nabla_{\vx'}^2 \left( \frac{1}{|\vx - \vx'|} \right)d V' = - G \rho(\vx) \oint_{S(\epsilon)} \nabla_{\vx'}\left(\frac{1}{|\vx-\vx'|}\right) \cdot d\vS'
\end{equation}
\noindent
By applying Equation \ref{eqn:nabla} and the identity $\nabla_{\vx'} f = -\nabla f$, we have
\begin{eqnarray}
- G \rho(\vx) \oint_{S(\epsilon)} \nabla_{\vx'}\left(\frac{1}{|\vx-\vx'|}\right) \cdot d\vS' &=&
-G \rho(\vx) \oint_{S(\epsilon)} \left(\frac{\vx-\vx'}{|\vx-\vx'|^{3}}\right) \cdot d\vS' \nonumber \\
&=& 4 \pi G \rho(\vx)
\end{eqnarray}


\subsection{Inside a Uniform Shell}

The gravitational force inside a spherical shell of uniform density is zero.  The potential is a
constant.

{\bf See Figure 3.1 of Sparke and Gallagher.}

The opening angle OA is the same as OB, so the ratio of the enclosed mass is (SA/SB)$^{2}$.  Since
the ratio of the forces scale like the inverse of this ratio (from the inverse square law), the
force contributions of the A and B patches are equal and opposite.

\subsection{Gravitational Potential Outside a Uniform Spherical Shell}

{\bf See Figure 3.2 of Sparke and Gallagher}

We are calculating the potential a uniform spherical shell of mass $M$ and radius $a$.  
Consider a point $P$ a distance $r$. The contribution of a narrow cone of opening
solid angle $\Delta\Omega$ around another point $Q'$ is
\begin{equation}
\Delta\Phi[\vx(P)] = -\frac{GM}{|\vx(P) - \vx(Q')|} \frac{\Delta\Omega}{4\pi}
\end{equation}
\noindent
Now consider the potential $\Phi'$ at point $P'$ at a radius $a$ away from the
center of a shell of the same mass $M$ but with a radius $r$.
The contribution $\Delta\Phi'$ from the material in the same cone of solid
angle $\Delta\Omega$ but at point $Q$ a distance $r$ away is
\begin{equation}
\Delta\Phi'[\vx(P')] = - \frac{GM}{|\vx(P') - \vx(Q)|}\frac{\Delta\Omega}{4\pi}
\end{equation}
\noindent
but since $PQ'=P'Q$, $\Delta\Phi[\vx(P)] = \Delta\Phi'[\vx(P')]$.
When we integrate over $4\pi$, we have
\begin{equation}
\Phi[\vx(P)] = \Phi'[\vx(P')]= \Phi'[\vx=0] = -\frac{GM}{r}
\end{equation}
\noindent
The force associated with this spherical shell is just $F(r) = \nabla \Phi[\vx(P)] = -\frac{GMm}{r}$.
So the force outside the shell is the same as for a point mass at distance $r$.

Inside a spherical mass distribution $\rho(r)$, the 
centripetal acceleration that allows for a circular 
orbit must be the radial gravitational force inwards.
On a circular orbit, in terms of the circular velocity
$V$ this acceleration is just
\begin{equation}
a = \frac{V^{2}(r)}{r} = -F(r) = \frac{GM(<r)}{r^2}.
\end{equation}
\noindent
For a point mass, $V(r)\propto r^{-1/2}$.  No extended
distribution can have a circular velocity curve that
declines more rapidly than $\propto r^{-1/2}$.

Note that the potential of a distributed mass density $\rho(\vx)$
is not the same as for a point mass.  Instead, we have
\begin{equation}
\Phi(r) = - \left[ \frac{GM(<r)}{r} 4 \pi G \int_{r}^{\infty} \rho(r') r' dr'\right].
\end{equation}
\noindent
But as long as the spherical mass distribution has a finite size, eventually we 
will have
\begin{equation}
\Phi(\vx) \to - \frac{G M_{\mathrm{tot}}}{|\vx|}
\end{equation}
\noindent
at large enough radius.


\subsection{Moving Through a Potential}

If we are moving through a background potential $\Phi(\vx)$ with
velocity $\vx$, the potential we experience changes
with time according to $d\Phi/dt = \vx\cdot\nabla\Phi(\vx)$.
We can re-write Newton's equation as
\begin{equation}
\vx \cdot \frac{d}{dt} ( m\vv) + m \vv\cdot\nabla \Phi(\vx) = 0 = \frac{d}{dt}\left[\frac{1}{2}m\vv^{2} + m \Phi(\vx)\right]
\end{equation}
\noindent
Therefore, the total energy
\begin{equation}
\label{eqn:total_energy}
E \equiv \frac{1}{2} m\vv^{2} + m\Phi(\vx) =~\mathrm{const}
\end{equation}
\noindent.
We can write of course that $E = KE + PE$, where $KE = \frac{1}{2}m\vv^{2}$ and $PE = m\Phi(\vx)$.
The kinetic energy cannot be negative, and we adopt $\Phi(\vx)\to0$ as $\vx\to\infty$.  At position
$\vx$, an orbit is unboutnd only if the total energy $E>0$. The speed at this place in the
orbit must exceed the escape speed, which is found by setting Equation \ref{eqn:total_energy} to zero.
We then have
\begin{equation}
v_{e}^{2} = - 2 \Phi(\vx).
\end{equation}

\subsection{Angular Momentum}

The angular momentum of an orbit is $L = \vx \times m\vv$. The time rate of change is
\begin{equation}
\frac{dL}{dt} = \vx \times \frac{d}{dt} (m \vv) = -m\vx \times \nabla \Phi.
\end{equation}
\noindent
For a spherically symmetric distribution, the force is central and $dL/dt = 0$ (angular
momentum is conserved).
In an axisymmetric distribution, on the component of $L$ parallel to the symmetry
axis is conserved.

\subsection{Total Energy is Not Conserved in a Time-Dependent Potential}

In a many-body system, the total energy of each star is not individually conserved.
The time derivative of the kinetic energy of star $i$ is
\begin{equation}
\sum_{i} \vv_i \cdot \frac{d}{dt} (m_i \vv_i) = \frac{d}{dt} KE = -\sum_{i,j;i\ne j} \frac{Gm_i m_j}{|\vx_i - \vx_j|^{3}}(\vx_i - \vx_j) \cdot \vv_i
\end{equation}
\noindent
Doing the same calculation on star $j$ and taking the dot product with $\vx_j$ gives
\begin{equation}
\frac{1}{2}\sum_{j} (m_j \vv_j \cdot \vv_j) = - \sum_{i,j; i\ne j}\frac{Gm_i m_j}{|\vx_i - \vx_j|^{3}}(\vx_j - \vx_i)\cdot \vx_j
\end{equation}
\noindent
Adding the RHS of these two equations gives
\begin{equation}
- \sum_{i,j; i\ne j}\frac{Gm_i m_j}{|\vx_i - \vx_j|^{3}}(\vx_j - \vx_i)\cdot(\vv_i - \vv_j) = \sum_{i,j;i\ne j}\left(\frac{Gm_i m_j}{|\vx_i - \vx_j|}\right).
\end{equation}
The potential energy $PE$ is a sum of pairs of potentials from individual objects
\begin{equation}
PE = -\frac{1}{2} \sum_{i,j;i\ne j}\frac{Gm_i m_j}{|\vx_i - \vx_j|} = \frac{1}{2} \sum_i m_i \Phi(\vx_i) = \frac{1}{2} \int \rho(\vx)\Phi(\vx)dV.
\end{equation}
\noindent
We divided by two so every object contributes only once.

We can now see that, for the whole collection of objects 
\begin{equation}
2 \frac{d}{dt} \left[ KE - \frac{1}{2} \sum_{i,j;i\ne j} \frac{G m_i m_j}{|\vx_i - \vx_j|}\right] = 0.
\end{equation}
\noindent
This means the total energy of the system is conserved.

\subsection{External Forces}

Consider the total force on an object $i$ in 
a many body system under the influence of an external force $\vF_{\mathrm{ext}}$.
\begin{equation}
\sum_i \frac{d}{dt} (m_i \vv_i) \cdot \vx_i = -\sum_{i,j; i\ne j} \frac{Gm_i m_j}{|\vx_i - \vx_j|^{3}} (\vx_i - \vx_j) \cdot \vx_i + \sum_i \vF_{\mathrm{ext}}^{i} \cdot \vx_i.
\end{equation}
\noindent
The force on the $j$th object is
\begin{equation}
\sum_j \frac{d}{dt} (m_j \vv_j) \cdot \vx_j = - \sum_{i,j;i\ne j }\frac{G m_i m_j}{|\vx_i - \vx_j|^{3}}(\vx_j - \vx_i)\cdot\vx_j + \sum_j \vF_{\mathrm{ext}}^{j} \cdot \vx_j
\end{equation}
\noindent
The left hand sides of these equations are equal, and are equal to
\begin{equation}
\frac{1}{2} \sum_i \frac{d^2}{dt^2}(m_i\vx_i \cdot \vx_i) - \sum_i m_i \vv_i \cdot \vv_i = \frac{1}{2} \frac{d^2I}{dt^2} -2 KE
\end{equation}
\noindent
where the moment of inertia $I$ is
\begin{equation}
I \equiv \sum_i m_i \vx_i \cdot \vx_i.
\end{equation}
\noindent
By averaging the force on $i$ and $j$, we find
\begin{equation}
\frac{1}{2} \frac{d^2 I}{dt^2} - 2 KE = PE + \sum_i \vF_{\mathrm{ext}}^{i} \cdot \vx_i
\end{equation}
\noindent
and averaging this over a short time interval $0<t<\tau$ gives
\begin{equation}
\frac{1}{2\tau}\left[ \frac{dI}{dt(\tau)} - \frac{dI}{dt}(0)\right] = 2 \ave{KE} + \ave{PE} + \sum_i \ave{\vF_{\mathrm{ext}}^{i} \cdot \vx_i}
\end{equation}
\noindent
If all objects in the system are bound, then $|\vx_i \cdot \vv_i|$ and $dI/dt$ will be finite.  As $\tau\to\infty$, the LHS goes to zero.  Then we have
\begin{equation}
2\ave{KE} + \ave{PE} + \sum_i \ave{\vF_{\mathrm{ext}}^{i} \cdot \vx_i} = 0
\end{equation}

\end{document}