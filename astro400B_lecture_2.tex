\documentclass[12pt]{article}
\usepackage[margin=1.0in]{geometry}
\usepackage{amssymb}

%title material
\title{Astronomy 400B Lecture 2: Magnitudes, the Milky Way, and Other Galaxies}
\author{Brant Robertson}
\date{January 5, 2015}


%include latex definitions

%average
\newcommand{\ave}[1]{\langle#1\rangle}
%radian
\newcommand{\rad}{\mathrm{rad}}

%astronomical unit
\newcommand{\AU}{\mathrm{AU}}

%centimeter
\newcommand{\cm}{\mathrm{cm}}

%meter
\newcommand{\m}{\mathrm{m}}

%kilometer
\newcommand{\km}{\mathrm{km}}

%parsec
\newcommand{\pc}{\mathrm{pc}}

%kiloparsec
\newcommand{\kpc}{\mathrm{kpc}}

%megaparsec
\newcommand{\Mpc}{\mathrm{Mpc}}

%gigaparsec
\newcommand{\Gpc}{\mathrm{Gpc}}

%light year
\newcommand{\ly}{\mathrm{ly}}

%second
\newcommand{\s}{\mathrm{s}}
\newcommand{\yr}{\mathrm{yr}}
\newcommand{\Gyr}{\mathrm{Gyr}}

%solar mass
\newcommand{\Msun}{M_{\odot}}

%grams
\newcommand{\g}{\mathrm{g}}

%erg
\newcommand{\erg}{\mathrm{erg}}

%solar luminosity
\newcommand{\Lsun}{L_{\odot}}

%jansky 
\newcommand{\Jy}{\mathrm{Jy}}

%flux density
\newcommand{\Fnu}{F_{\nu}}
\newcommand{\Flambda}{F_{\lambda}}
\newcommand{\Lnu}{L_{\nu}}

%hertz 
\newcommand{\Hz}{\mathrm{Hz}}

%angstrom
\newcommand{\Ang}{\mathrm{\r{A}}}

%solar radius
\newcommand{\Rsun}{R_{\odot}}

%stefan-boltzmann
\newcommand{\sigmaSB}{\sigma_{\mathrm{SB}}}

%boltzmann
\newcommand{\kB}{k_{\mathrm{B}}}

%kelvin
\newcommand{\K}{\mathrm{K}}

%T_bandpass
\newcommand{\TBP}{T_{\mathrm{BP}}}

%magnitude
\newcommand{\mg}{\mathrm{mag}}


%arcsecond
\newcommand{\arcsec}{\mathrm{arcsec}}

%begin the document
\begin{document}

%make the title, goes after document begins
\maketitle

%magnitudes
\section{Magnitudes}
\subsection{Vega Magnitudes}

Vega magnitudes are both the scourge and pride of astronomy, and enable astronomers to
engage in discussions no other scientists understand. The
{\it apparent magnitude}
difference $m_1 - m_2$ 
between two stars is related to the ratio of their fluxes
$F_1$ and $F_2$ 
via the relation
\begin{equation}
\label{eqn:apparent_magnitude_difference}
m_{1} - m_{2} = -2.5 \log (F_1 / F_2).
\end{equation}
\noindent
Note that I have used fluxes here rather than flux densities,
but in fact we can use Equation \ref{eqn:apparent_magnitude_difference}
with flux densities as well (see below).

Typically, when we measure a magnitude we are band-limited by a
filter with a bandpass function $\TBP(\lambda)$. The total flux in
a given filter can be written as
\begin{equation}
F_{\mathrm{PB}} = \int_{0}^{\infty} \TBP(\lambda)\Flambda(\lambda)d\lambda \approx \Flambda(\lambda_{\mathrm{eff}})\Delta\lambda
\end{equation}
\noindent
where $\Delta \lambda$
is the wavelength-width of the filter and $\lambda_{\mathrm{eff}}$ is the effective wavelength of the filter 
\begin{equation}
\lambda_{\mathrm{eff}} = \left. \int_{0}^{\infty} \lambda \TBP(\lambda) \Flambda(\lambda) d\lambda \middle/ \int_{0}^{\infty} \TBP(\lambda) \Flambda(\lambda) d\lambda \right. .
\end{equation} 
\noindent
The apparent magnitude difference between two objects measured in a given filter is
\begin{equation}
m_{1,\mathrm{BP}} - m_{2,\mathrm{BP}} = -2.5 \log \left\{ \int_{0}^{\infty} \TBP(\lambda) F_{1,\lambda}(\lambda) d \lambda \middle/ \int_{0}^{\infty} \TBP(\lambda) F_{2,\lambda}(\lambda) d \lambda \right\}.
\end{equation}

For some bands, such as in the UV, there were no well-measured stellar spectra.  In this case,
astronomers would define the average flux density
\begin{equation}
\label{eqn:average_flux_density_lambda}
\ave{F_{\mathrm{BP}}} = \frac{\int \TBP(\lambda) \Flambda(\lambda) d\lambda}{\int \TBP(\lambda) d\lambda},
\end{equation}
\noindent
from which we could define
\begin{equation}
m_{\mathrm{BP}} = -2.5 \log \left( \frac{\ave{F_{\mathrm{BP}}}}{\ave{F_{V,0}}}\right).
\end{equation}
\noindent
The quantity $\ave{F_{V,0}} \approx 3.63 \times 10^{-9} \erg~\cm^{-2}~\s^{-1}~\Ang^{-1}$ 
is the average flux density of a star with apparent(!) magnitude $m_{V}=0$ (e.g., Vega).
A quick cheat is
\begin{equation}
m_{\mathrm{BP}} = -2.5 \log \ave{F_{\mathrm{BP}}} - 21.1
\end{equation}
\noindent
if $\ave{F_{\mathrm{BP}}}$ is also measured in units of $\erg~\cm^{-2}~\s^{-1}~\Ang^{-1}$.

Why are Vega magnitudes appalling? The issue is $\Flambda(\lambda)$ must be known for Vega
for all wavelengths. A corollary is that the differences between the apparent magnitudes
of different stars in different filters depend on the spectral shape of Vega. It's a bit
insane.

\section{AB Magnitudes}

Eventually, astronomers came to their senses and started using ``AB'' magnitudes that are measured 
relative to a source that is flat in $\Fnu$. There are two main advantages:

\begin{enumerate}
\item Magnitudes depend on the spectral shape of the measured object but not a reference star.
\item Magnitudes can be defined as a flux density easily, e.g., AB magnitudes $\equiv$ Janskys.
\item Absolute magnitudes are easily understood interms of flux density (including luminosity and distance).
\end{enumerate}

Let's redefine the average flux density in some filter as
\begin{equation}
\label{eqn:average_flux_density}
\ave{F_{\mathrm{BP}}} = \frac{\int \TBP(\nu) \Fnu(\nu) d\nu}{\int \TBP(\nu) d\nu}.
\end{equation}
\noindent
This quantity has units $\erg~\cm^{-2}~\s^{-1}~\Hz^{-1}$, which is just a multiple
of the Jansky.

For distant extragalactic work, the micro- ($\mu\Jy$) and nanojansky (nJy) are most convenient.
We can easily convert between the AB magnitude and nanojansky as
\begin{equation}
m_{\mathrm{AB}} = 31.4 - 2.5 \log \left(\frac{\ave{\Fnu}}{1~\mathrm{nJy}}\right).
\end{equation}
\noindent
Note that this sensibly has nothing to do with Vega.

\subsection{Absolute Magnitude}
The absolute magnitude of an object is the apparent magnitude of
an object if it were placed at a distance of 10$\pc$
\begin{equation}
\label{eqn:absolute_magnitude}
M = m - 5\log(d/10\pc).
\end{equation}
\noindent
For instance, the absolute magnitude of the Sun depends on the
filter used to measure the Sun's flux density (and for Vega
magnitudes the spectral shape of Vega). It should be clear that
the Vega and AB absolute magnitudes for the Sun (or any other star)
are not the same. For Vega magnitudes, the absolute filter magnitudes of
the Sun are
\begin{equation}
[U,B,V,R,I,J,H,K]=[5.61,5.48,4.83,4.42,4.08,3.64,3.32,3.28].
\end{equation}
\noindent
In AB magnitudes, the absolute filter magnitudes of the
Sun are 
\begin{equation}
[U,B,V,R,I,J,H,K] = [6.36,5.36,4.82,4.65,4.55,4.57,4.71,5.19].
\end{equation}

\subsection{Bolometric Correction}

The absolute magnitude of a star (nominally) measures the bandpass
averaged flux of an object, referenced to a distance of 10 parsecs.
This quantity involves the luminosity (density) of the object in a given 
filter, and differs from the total luminosity of an object. The {\it bolometric
magnitude} of an object is related to its total luminosity, and we can define
the {\it bolometric correction} to be the amount that needs to be subtracted
from the bandpass magnitude of an object to find its bolometric magnitude.
In an equation, we can write
\begin{equation}
M_{\mathrm{bol}} = M_V - \mathrm{BC}
\end{equation}
For the Sun in Vega magnitudes, $\mathrm{BC}\approx0.07$.

\section{The Milky Way}

The Milky Way is a fantastic galaxy, and not just because we live there.
It has a terrific richness of structure and complexity, and simply put we
currently do not know how such a galaxy forms.

The main structures of the Milky Way are ({\bf See Figure 1.8 of Sparke and Gallagher})

\begin{enumerate}


\item The dark matter halo, consisting of (presumably) subatomic particles that
participate only in weak (Weak, gravitational force) interactions. The mass of
the Milky Way halo is about $10^{12}\Msun$. The radius of the dark matter halo
is roughly $300\kpc$. The dark matter halo is thought to obey a roughly broken
power-law density profile, with $\rho\propto r^{-1}$ within about $30\kpc$ and
$\rho\propto r^{-3}$ in the exterior.

\item The stellar halo, consisting of old, metal-poor stars and globular clusters
(old clusters of $\sim10^{5}-10^{6}$ stars). The stellar halo is only about
$10^{9}\Msun$.

\item The central bulge of the galaxy is a pseudospheroidal distribution of
stars with a luminosity of $L\approx 5\times 10^{9}\Lsun$ 
and mass $2\times10^{10}\Msun$.

\item The supermassive black hole at the center of the Milky Way, which has a
mass of about $\approx4.1\times10^{6}\Msun$.

\item The stellar disk of the galaxy is roughly exponential, such that
the surface density scales as $\Sigma(r) \propto \exp(-r/h)$, 
with a scale length $h\approx3\kpc$. The total luminosity of the disk
is about $2\times10^{10}\Lsun$ and with a mass in stars of
about $6\times10^{10}\Msun$. The stellar disk has a {\it thin disk} 
component with
a scale height of about $300\pc$ containing $95\%$ of the mass, and a
{\it thick disk} containing about $5\%$ with a scale height of about $1\kpc$.

\item The gaseous disk of the galaxy is a thin layer about a $100\pc$ thick
consisting mostly of neutral hydrogen (HI) and molecular hydrogen (H$_2$) gas.
The gaseous disk also contains a warm ionized and hot ionized interstellar medium
(see below).
The gaseous disk is also very dusty!

\end{enumerate}

\section{Ionized Gas}

The Milky Way (and other galaxies) contain a variety of ions of various elements,
and the emission and absorption mechanisms of this gas enables us to learn a lot
about the galaxies observationally. The radiation emitted by the gas can be
in the form of spectral lines or continuum processes (like bound-free emission).

Atoms and ions produce line emission when electrons move between orbitals and
release the corresponding energy difference between orbitals as a photon.
Recombination radiation occurs when free electrons get captured into an orbital
and release the corresponding free energy difference as light. {\it Balmer} lines
from Hydrogen correspond to electrons entering the $n=2$ state, while the {\it Lyman}
series corresponds to transitions to the $n=1$ state. Transitions between $n=2$ and
$n=1$ correspond to the {\it Lyman} $\alpha$ line at $\lambda = 1216\Ang$ (10.2eV). 

\subsection{Photoionization}
We call the process of freeing electrons from ions via the absorption of a photon
{\it photoionization}. The energy of a photon required to photoionize an ion depends
on the electrical properties of the ion.
A photon of
energy $13.6$eV or more and wavelength $\lambda=912\Ang$ or shorter can fully liberate
an electron from the ground state of Hydrogen. Such photons are called {\it Lyman continuum}
photons.

\subsection{Collisional Ionization}

For hot gas, when the mean kinetic energy $\kB T$ is comparable to the energy
difference between levels, then {\it collisional ionization} may occur via the
collision of two ions. In this case, we will have the reaction
\begin{equation}
\mathrm{A}+\mathrm{B}\to \mathrm{A}^{*} + \mathrm{B}; \mathrm{A}^{*} \to \mathrm{A} + h\nu
\end{equation}
\noindent
where $h\nu \sim \kB T$ is the energy of photon emitted after the ion A is collisionally
excited. 

\subsection{Forbidden Lines}
The rate of emission depends on density, and some {\it forbidden lines} that forge 
less probable paths through the orbital ladder of an ion require very low densities since
their emission timescale can be of order $1\s$ (allowed lines have emission timescales that
are $\sim10^{-8}\s$). The {\it critical density} of a line corresponds to the density where
emission is the most likely compared with further excitation or collision. Forbidden lines 
therefore provide information about the density and temperature of their emitting gas. We 
typically indicate a forbidden line with the $[]$ notation, as the famous $[\mathrm{OIII}]$
line at $\lambda=5007\Ang$.

\subsection{Fine Structure Lines}

The {\it fine structure} lines of an atom correspond to level splittings that arise from
the coupling between an electron orbital's angular momentum and the electron spin. The
energy difference between typical orbital transitions and fine structure lines is a 
factor of $\sim\alpha^{-2}$, where $\alpha$ is the fine structure constant
\begin{equation}
\alpha = \frac{1}{2\epsilon_{0}}\frac{e^2}{hc}
\end{equation}
\noindent
where $\epsilon_{0}$ is the permitivity of free space, $e$ is the
charge of the electron, $h$ is Planck's constant, and $c$ is the speed of light.
The value of $\alpha^{-1} \approx 137.036$ is famous throughout physics.

This large factor means that the fine structure lines of metals typically lie
at wavelengths $>100\mu m$.

\end{document}




























