\documentclass[12pt]{article}
\usepackage[margin=1.0in]{geometry}
\usepackage{amssymb}

%title material
\title{Astronomy 400B Lecture 2: Magnitudes, the Milky Way, and Other Galaxies}
\author{Brant Robertson}
\date{January 5, 2015}


%include latex definitions

%average
\newcommand{\ave}[1]{\langle#1\rangle}
%radian
\newcommand{\rad}{\mathrm{rad}}

%astronomical unit
\newcommand{\AU}{\mathrm{AU}}

%micron
\newcommand{\mum}{\mu\mathrm{m}}

%millimeter
\newcommand{\mm}{\mathrm{mm}}

%centimeter
\newcommand{\cm}{\mathrm{cm}}

%meter
\newcommand{\m}{\mathrm{m}}

%kilometer
\newcommand{\km}{\mathrm{km}}

%parsec
\newcommand{\pc}{\mathrm{pc}}

%kiloparsec
\newcommand{\kpc}{\mathrm{kpc}}

%megaparsec
\newcommand{\Mpc}{\mathrm{Mpc}}

%gigaparsec
\newcommand{\Gpc}{\mathrm{Gpc}}

%light year
\newcommand{\ly}{\mathrm{ly}}

%second
\newcommand{\s}{\mathrm{s}}
\newcommand{\yr}{\mathrm{yr}}
\newcommand{\Gyr}{\mathrm{Gyr}}

%solar mass
\newcommand{\Msun}{M_{\odot}}

%grams
\newcommand{\g}{\mathrm{g}}

%erg
\newcommand{\erg}{\mathrm{erg}}

%solar luminosity
\newcommand{\Lsun}{L_{\odot}}

%jansky 
\newcommand{\Jy}{\mathrm{Jy}}

%flux density
\newcommand{\Fnu}{F_{\nu}}
\newcommand{\Flambda}{F_{\lambda}}
\newcommand{\Lnu}{L_{\nu}}

%hertz 
\newcommand{\Hz}{\mathrm{Hz}}

%angstrom
\newcommand{\Ang}{\mathrm{\r{A}}}

%solar radius
\newcommand{\Rsun}{R_{\odot}}

%stefan-boltzmann
\newcommand{\sigmaSB}{\sigma_{\mathrm{SB}}}

%boltzmann
\newcommand{\kB}{k_{\mathrm{B}}}

%kelvin
\newcommand{\K}{\mathrm{K}}

%T_bandpass
\newcommand{\TBP}{T_{\mathrm{BP}}}

%magnitude
\newcommand{\mg}{\mathrm{mag}}


%arcsecond
\newcommand{\arcsec}{\mathrm{arcsec}}

%critical density
\newcommand{\rhoc}{\rho_{\mathrm{crit}}}

%proton mass
\newcommand{\mproton}{m_{\mathrm{p}}}

%electron volt
\newcommand{\eV}{\mathrm{eV}}

%kiloelectron volt
\newcommand{\keV}{\mathrm{keV}}

%megaelectron volt
\newcommand{\MeV}{\mathrm{MeV}}

%gigaelectron volt
\newcommand{\GeV}{\mathrm{GeV}}

%vector velocity
\newcommand{\vv}{\mathbf{v}}

%vector radius
\newcommand{\vr}{\mathbf{r}}

%vector position
\newcommand{\vx}{\mathbf{x}}

%vector force
\newcommand{\vF}{\mathbf{F}}

%vector surface
\newcommand{\vS}{\mathbf{S}}

%vector angular momentum
\newcommand{\vL}{\mathbf{L}}

%script I -- integral of motion
\newcommand{\cI}{\mathcal{I}}

%effective potential
\newcommand{\Phieff}{\Phi_\mathrm{eff}}


%begin the document
\begin{document}

%make the title, goes after document begins
\maketitle

%magnitudes
\section{Magnitudes}
\subsection{Vega Magnitudes}

Vega magnitudes are both the scourge and pride of astronomy, and enable astronomers to
engage in discussions no other scientists understand. The
{\it apparent magnitude}
difference $m_1 - m_2$ 
between two stars is related to the ratio of their fluxes
$F_1$ and $F_2$ 
via the relation
\begin{equation}
\label{eqn:apparent_magnitude_difference}
m_{1} - m_{2} = -2.5 \log (F_1 / F_2).
\end{equation}
\noindent
Note that I have used fluxes here rather than flux densities,
but in fact we can use Equation \ref{eqn:apparent_magnitude_difference}
with flux densities as well (see below).

Typically, when we measure a magnitude we are band-limited by a
filter with a bandpass function $\TBP(\lambda)$. The total flux in
a given filter can be written as
\begin{equation}
F_{\mathrm{PB}} = \int_{0}^{\infty} \TBP(\lambda)\Flambda(\lambda)d\lambda \approx \Flambda(\lambda_{\mathrm{eff}})\Delta\lambda
\end{equation}
\noindent
where $\Delta \lambda$
is the wavelength-width of the filter and $\lambda_{\mathrm{eff}}$ is the effective wavelength of the filter 
\begin{equation}
\lambda_{\mathrm{eff}} = \left. \int_{0}^{\infty} \lambda \TBP(\lambda) \Flambda(\lambda) d\lambda \middle/ \int_{0}^{\infty} \TBP(\lambda) \Flambda(\lambda) d\lambda \right. .
\end{equation} 
\noindent
The apparent magnitude difference between two objects measured in a given filter is
\begin{equation}
m_{1,\mathrm{BP}} - m_{2,\mathrm{BP}} = -2.5 \log \left\{ \int_{0}^{\infty} \TBP(\lambda) F_{1,\lambda}(\lambda) d \lambda \middle/ \int_{0}^{\infty} \TBP(\lambda) F_{2,\lambda}(\lambda) d \lambda \right\}.
\end{equation}

For some bands, such as in the UV, there were no well-measured stellar spectra.  In this case,
astronomers would define the average flux density
\begin{equation}
\label{eqn:average_flux_density_lambda}
\ave{F_{\mathrm{BP}}} = \frac{\int \TBP(\lambda) \Flambda(\lambda) d\lambda}{\int \TBP(\lambda) d\lambda},
\end{equation}
\noindent
from which we could define
\begin{equation}
m_{\mathrm{BP}} = -2.5 \log \left( \frac{\ave{F_{\mathrm{BP}}}}{\ave{F_{V,0}}}\right).
\end{equation}
\noindent
The quantity $\ave{F_{V,0}} \approx 3.63 \times 10^{-9} \erg~\cm^{-2}~\s^{-1}~\Ang^{-1}$ 
is the average flux density of a star with apparent(!) magnitude $m_{V}=0$ (e.g., Vega).
A quick cheat is
\begin{equation}
m_{\mathrm{BP}} = -2.5 \log \ave{F_{\mathrm{BP}}} - 21.1
\end{equation}
\noindent
if $\ave{F_{\mathrm{BP}}}$ is also measured in units of $\erg~\cm^{-2}~\s^{-1}~\Ang^{-1}$.

Why are Vega magnitudes appalling? The issue is $\Flambda(\lambda)$ must be known for Vega
for all wavelengths. A corollary is that the differences between the apparent magnitudes
of different stars in different filters depend on the spectral shape of Vega. It's a bit
insane.

\section{AB Magnitudes}

Eventually, astronomers came to their senses and started using ``AB'' magnitudes that are measured 
relative to a source that is flat in $\Fnu$. There are two main advantages:

\begin{enumerate}
\item Magnitudes depend on the spectral shape of the measured object but not a reference star.
\item Magnitudes can be defined as a flux density easily, e.g., AB magnitudes $\equiv$ Janskys.
\item Absolute magnitudes are easily understood interms of flux density (including luminosity and distance).
\end{enumerate}

Let's redefine the average flux density in some filter as
\begin{equation}
\label{eqn:average_flux_density}
\ave{F_{\mathrm{BP}}} = \frac{\int \TBP(\nu) \Fnu(\nu) d\nu}{\int \TBP(\nu) d\nu}.
\end{equation}
\noindent
This quantity has units $\erg~\cm^{-2}~\s^{-1}~\Hz^{-1}$, which is just a multiple
of the Jansky.

For distant extragalactic work, the micro- ($\mu\Jy$) and nanojansky (nJy) are most convenient.
We can easily convert between the AB magnitude and nanojansky as
\begin{equation}
m_{\mathrm{AB}} = 31.4 - 2.5 \log \left(\frac{\ave{\Fnu}}{1~\mathrm{nJy}}\right).
\end{equation}
\noindent
Note that this sensibly has nothing to do with Vega.

\subsection{Absolute Magnitude}
The absolute magnitude of an object is the apparent magnitude of
an object if it were placed at a distance of 10$\pc$
\begin{equation}
\label{eqn:absolute_magnitude}
M = m - 5\log(d/10\pc).
\end{equation}
\noindent
For instance, the absolute magnitude of the Sun depends on the
filter used to measure the Sun's flux density (and for Vega
magnitudes the spectral shape of Vega). It should be clear that
the Vega and AB absolute magnitudes for the Sun (or any other star)
are not the same. For Vega magnitudes, the absolute filter magnitudes of
the Sun are
\begin{equation}
[U,B,V,R,I,J,H,K]=[5.61,5.48,4.83,4.42,4.08,3.64,3.32,3.28].
\end{equation}
\noindent
In AB magnitudes, the absolute filter magnitudes of the
Sun are 
\begin{equation}
[U,B,V,R,I,J,H,K] = [6.36,5.36,4.82,4.65,4.55,4.57,4.71,5.19].
\end{equation}

\subsection{Bolometric Correction}

The absolute magnitude of a star (nominally) measures the bandpass
averaged flux of an object, referenced to a distance of 10 parsecs.
This quantity involves the luminosity (density) of the object in a given 
filter, and differs from the total luminosity of an object. The {\it bolometric
magnitude} of an object is related to its total luminosity, and we can define
the {\it bolometric correction} to be the amount that needs to be subtracted
from the bandpass magnitude of an object to find its bolometric magnitude.
In an equation, we can write
\begin{equation}
M_{\mathrm{bol}} = M_V - \mathrm{BC}
\end{equation}
For the Sun in Vega magnitudes, $\mathrm{BC}\approx0.07$.

\end{document}