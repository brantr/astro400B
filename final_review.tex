\documentclass[]{article}
\usepackage[margin=1.0in]{geometry}
\usepackage{amssymb}

%title material
\title{Astronomy 400B Final Review}
\author{Brant Robertson}
\date{May, 2015}


%include latex definitions

%average
\newcommand{\ave}[1]{\langle#1\rangle}
%radian
\newcommand{\rad}{\mathrm{rad}}

%astronomical unit
\newcommand{\AU}{\mathrm{AU}}

%micron
\newcommand{\mum}{\mu\mathrm{m}}

%millimeter
\newcommand{\mm}{\mathrm{mm}}

%centimeter
\newcommand{\cm}{\mathrm{cm}}

%meter
\newcommand{\m}{\mathrm{m}}

%kilometer
\newcommand{\km}{\mathrm{km}}

%parsec
\newcommand{\pc}{\mathrm{pc}}

%kiloparsec
\newcommand{\kpc}{\mathrm{kpc}}

%megaparsec
\newcommand{\Mpc}{\mathrm{Mpc}}

%gigaparsec
\newcommand{\Gpc}{\mathrm{Gpc}}

%light year
\newcommand{\ly}{\mathrm{ly}}

%second
\newcommand{\s}{\mathrm{s}}
\newcommand{\yr}{\mathrm{yr}}
\newcommand{\Gyr}{\mathrm{Gyr}}

%solar mass
\newcommand{\Msun}{M_{\odot}}

%grams
\newcommand{\g}{\mathrm{g}}

%erg
\newcommand{\erg}{\mathrm{erg}}

%solar luminosity
\newcommand{\Lsun}{L_{\odot}}

%jansky 
\newcommand{\Jy}{\mathrm{Jy}}

%flux density
\newcommand{\Fnu}{F_{\nu}}
\newcommand{\Flambda}{F_{\lambda}}
\newcommand{\Lnu}{L_{\nu}}

%hertz 
\newcommand{\Hz}{\mathrm{Hz}}

%angstrom
\newcommand{\Ang}{\mathrm{\r{A}}}

%solar radius
\newcommand{\Rsun}{R_{\odot}}

%stefan-boltzmann
\newcommand{\sigmaSB}{\sigma_{\mathrm{SB}}}

%boltzmann
\newcommand{\kB}{k_{\mathrm{B}}}

%kelvin
\newcommand{\K}{\mathrm{K}}

%T_bandpass
\newcommand{\TBP}{T_{\mathrm{BP}}}

%magnitude
\newcommand{\mg}{\mathrm{mag}}


%arcsecond
\newcommand{\arcsec}{\mathrm{arcsec}}

%critical density
\newcommand{\rhoc}{\rho_{\mathrm{crit}}}

%proton mass
\newcommand{\mproton}{m_{\mathrm{p}}}

%electron volt
\newcommand{\eV}{\mathrm{eV}}

%kiloelectron volt
\newcommand{\keV}{\mathrm{keV}}

%megaelectron volt
\newcommand{\MeV}{\mathrm{MeV}}

%gigaelectron volt
\newcommand{\GeV}{\mathrm{GeV}}

%vector velocity
\newcommand{\vv}{\mathbf{v}}

%vector radius
\newcommand{\vr}{\mathbf{r}}

%vector position
\newcommand{\vx}{\mathbf{x}}

%vector force
\newcommand{\vF}{\mathbf{F}}

%vector surface
\newcommand{\vS}{\mathbf{S}}

%vector angular momentum
\newcommand{\vL}{\mathbf{L}}

%script I -- integral of motion
\newcommand{\cI}{\mathcal{I}}

%effective potential
\newcommand{\Phieff}{\Phi_\mathrm{eff}}


%begin the document
\begin{document}

%make the title, goes after document begins
\maketitle

%first section
\section{Letcure 1: Stellar Mass, Luminosity, Flux, Radius, and Temperature}
Stellar masses are often expressed in terms of a {\it solar mass} ($\Msun$; mass of the Sun)
\begin{equation}
1\Msun = 1.9891\times10^{33}\g
\end{equation}
\noindent
in grams ($\g$). The most massive stars are $\approx100\Msun$ while the
least massive stars are $\approx0.075\Msun$.

Stellar luminosities (ergs of energy emitted per second; $1\erg = 1~\g~\cm^{2}~\s^{-2}$) are also 
often expressed in terms of the total (bolometric) {\it solar luminosity} ($\Lsun$; luminosity of the Sun)
\begin{equation}
1\Lsun = 3.846\times10^{33} \erg~\s^{-1}
\end{equation}
\noindent
Stars range in luminosity from $10^{6}\Lsun$ to less than $10^{-4}\Lsun$.

The {\it flux} $F$ is the energy per unit second per unit area ($\erg~\s^{-1}~\cm^{-2}$)
that is received from an object a distance $d$ away, and is
given by the {\it inverse square law}
\begin{equation}
\label{eqn:inverse_square}
F = \frac{L}{4\pi d^{2}}.
\end{equation}
\noindent

The {\it flux density} ($\Fnu \equiv dF/d\nu$) of an object is best measured in {\it janskys}
\begin{equation}
1\Jy = 10^{-23} \erg~\s^{-1}~\cm^{-2}~\Hz^{-1}.
\end{equation}
\noindent
Note that the {\it hertz} ($1\Hz = 1 s^{-1}$) is the unit of frequency $\nu$.
Astronomers will also use a flux density ($\Flambda = dF/d\lambda$) 
defined relative to wavelength. Typically, the units of $\Flambda$
are in $\erg~\cm^{-2}~\s^{-1}~\Ang^{-1}$, where $\Ang$ is the {\it angstrom}
($1\Ang = 10^{-8}\cm$). The wavelength and frequency of light are related by
$c = \nu \lambda$, and the two flux densities are therefore 
related by $\Flambda = (c/\lambda^2)\Fnu$.
The total flux is related to the flux density by
\begin{equation}
F = \int \Fnu d \nu = \int \Flambda d\lambda.
\end{equation}
\noindent
If the distance of an object is known, astronomers will also use the term
{\it luminosity density} (e.g., $\Lnu = dL/d\nu$).

The luminosity, radius, and temperature of an object are related. For stellar
objects, we often use the {\it solar radius} ($\Rsun$) to express sizes
\begin{equation}
1\Rsun = 6.995 \times 10^{10} \cm
\end{equation}
\noindent
If the luminosity $L$ and radius $R$ of an object are known, we can define
the effective temperature $T$ in {\it Kelvin} ($K$)
of an object through the {\it Stefan-Boltzmann Law}
\begin{equation}
L = 4\pi R^{2}\sigmaSB T^{4}
\end{equation}
\noindent
where the {\it Stefan-Boltzmann constant} is
\begin{equation}
\sigmaSB = 5.670373\times10^{-5} \erg~\cm^{-2}~\s^{-1}~\K^{-4}.
\end{equation}
The Stefan-Boltzmann constant can be computed theoretically in
terms of the speed of light, the {\it Boltzmann constant}
\begin{equation}
\kB = 1.380658\times10^{-16} \erg~\K^{-1}
\end{equation}
\noindent
and the {\it Planck constant}
\begin{equation}
h = 6.6260755\times10^{-27}\erg~\s,
\end{equation}
\noindent
which gives the definition
\begin{equation}
\sigmaSB \equiv \frac{2\pi^5 \kB^4}{15 h^3 c^2}.
\end{equation}

The effective temperature is typically a measure of the temperature
of the star at its {\it photosphere}, or the radius where the optical
depth of the star's atmosphere is $\tau\approx1$ and it becomes opaque. 
For the Sun,
$T\approx5780\K$. The temperature is also related to the wavelength
at the peak of the black body curve via {\it Wien's Displacement Law}
\begin{equation}
\lambda_{\mathrm{max}} = \frac{2.897756\times10^{7} \Ang~\K}{T},
\end{equation}
which gives $\lambda_{\mathrm{max}}\approx5000\Ang$ (yellow)
for the Sun.


\section{Lecture 2: Extinction}

Dust absorbs and scatters optical light, and we can describe
the rate at which light is absorbed as it travels along the
$x$ direction through the differential relation
\begin{equation}
\frac{d\Flambda}{dx} = - \kappa_{\lambda} \Flambda
\end{equation}
\noindent
where $\kappa_{\lambda}$ is the {\it opacity}. This relation
has a simple solution
\begin{equation}
\Flambda(x) = \Flambda(x=0)\exp\left(-\int_{0}^{x} \kappa_{\lambda} dx'\right)
\end{equation}
\noindent
where we can define the optical depth
\begin{equation}
\tau = \int_{0}^{x} \kappa_{\lambda} dx'
\end{equation}
\noindent
In the limit of pure absorption, a system is {\it optically thick}
when $\tau = 1$ and $\Flambda = \Flambda(x=0)/e$. For
optical light, interstellar dust has approximately $\kappa_{\lambda}\propto1/\lambda$.

\section{Lecture 3: Galaxy Luminosity Function}

We can count the number density of galaxies as a function of their luminosity,
and we appropriately call this distribution the {\it luminosity function}. The
galaxy luminosity function has been found to have a shape close to a parameterized
form called the {\it Schechter} function (after Paul Schechter).  The Schechter
function provides the number density of galaxies in a differential luminosity bin
$dL$ as
\begin{equation}
\label{eqn:schechter_function_luminosity}
\Phi(L)dL = \phi_{\star} \left(\frac{L}{L_{\star}}\right)^{\alpha}\exp\left(-\frac{L}{L_{\star}}\right)\frac{dL}{L_{\star}}
\end{equation}
\noindent
where $L_{\star}$ is a characteristic luminosity of galaxies and $\phi_{\star}$ is 
a typical abundance.  Below $L_\star$, the luminosity function is a power law, and
above $L_\star$ the abundance of galaxies drops exponentially. Sometimes, astronomers
will use $1+\alpha$ as the power law exponent, so beware!

{\bf See Figure 1.16 of Sparke and Gallagher.}

It happens to be the case that $L_{\star}\approx2\times10^{10}\Lsun$, which is close
to the luminosity of the Milky Way. The typical abundance of galaxies is $\phi_\star\approx7\times10^{-3}\Mpc^{-3}$. The faint-end slope of the 2DF luminosity function is $\alpha=-0.46$.
Defined as in Equation \ref{eqn:schechter_function_luminosity}, the number of galaxies
diverges as $L\to0$ if $\alpha<-1$.

The total luminosity density provided by galaxies can be found by integrating 
Equation \ref{eqn:schechter_function_luminosity} as
\begin{equation}
\rho_{L} = \int_{0}^{\infty} \Phi(L) L dL = \phi_\star L_{\star} \Gamma(\alpha + 2)
\end{equation}
\noindent
where $\Gamma$ is the Gamma function, which for an integer $n$ is $\Gamma(n) = (n-1)!$.
It turns out that $\Gamma(1.5)\approx0.886227$, so we have that 
$\rho_{L}\approx1.25\times10^{8}\Lsun\Mpc^{-3}$.

For a partial integral, we have
\begin{equation}
\rho_{L}(L>L_{\mathrm{min}}) = \int_{L_{\mathrm{min}}}^{\infty} \Phi(L) L dL = \phi_\star L_{\star} \Gamma(\alpha + 2, L_{\mathrm{min}}/L_{\star}),
\end{equation}
\noindent
where $\Gamma(a,x)$ is the incomplete gamma function.

\section{Lecture 4: Recombination and the Ionization State of the ISM}

Ionized hydrogen (proton plus electron) can undergo a recombination
to form neutral hydrogen. The rate of this process depends on the
properties of the ionized gas.  We can write the recombination rate
$dn_e/dt$ that changes the number density $n_e$ of free electrons as
\begin{equation}
\frac{d n_{e}}{d t} = n_{e}^2 \alpha(T_e)
\end{equation}
\noindent
where $T_e$ is the temperature of the gas and $\alpha(T_e)$ is 
the ``recombination coefficient''
\begin{equation}
\alpha(T_e) \approx 2 \times 10^{-13} \left( \frac{T_e}{10^4~\K}\right)^{-3/4}~\cm^{3}~\s^{-1}.
\end{equation}
\noindent
There's a lot hidden in the coefficient!

We can also define the recombination time $t_{rec}$ that characterizes
the timescale over which the ionized gas will significantly increase
its neutrality.  The recombination time is given by
\begin{equation}
t_{rec} = \frac{n_e}{|dn_e/dt|} = \frac{1}{n_e \alpha(T_e)} \approx 1500~\yr \times \left(\frac{T_e}{10^{4}K}\right)^{3/4}\left(\frac{100~\cm^{-3}}{n_e}\right)
\end{equation}

In HII regions, $t_{rec}$ is a few thousand years.  In the diffuse ionized ISM, the recombination time is a few million years.

\subsection{Cooling Rate and Time}

Gas has an internal thermal energy of $E \propto nT$. If the gas radiates this thermal energy with an energy $L$, the
cooling timescale for the gas is $t_{cool} \propto nT/L$. For optically thin gas, we can write
\begin{equation}
L = n^{2} \Lambda(T)
\end{equation}
\noindent
and
\begin{equation}
t_{cool}  \propto T/[n\Lambda(T)].
\end{equation}
The quantity $\Lambda(T)$ is called the ``cooling function'' and depends only on temperature.

{\bf See Table 2.5 and Figure 2.25 of Sparke and Gallagher.}

At high temperatures $T>10^{7}~\K$, free-free cooling dominates such that $\Lambda(T) \propto T^{1/2}$
and $t_{cool} \propto \sqrt{T}/n$.



\section{Lecture 5: Poisson's Equation}

Take the potential and apply the Laplacian operator 
\begin{equation}
\nabla^{2} \equiv \nabla \cdot \nabla = \left[\frac{\partial^{2}}{\partial x^{2}}+\frac{\partial^{2}}{\partial y^{2}}+\frac{\partial^{2}}{\partial z^{2}}\right]
\end{equation}
\noindent
to both sides.  Remembering that the operator acts on $\vx$ and not $\vx'$, we have
\begin{equation}
\label{eqn:pois_init}
\nabla^{2} \Phi(\vx) = - \int G \rho(\vx') \nabla^{2} \left(\frac{1}{|\vx-\vx'|}\right)d^{3}\vx'.
\end{equation}
\noindent
We can evaluate this by noting that
\begin{equation}
\label{eqn:nabla}
\nabla\left(\frac{1}{|\vx - \vx'|}\right) = -\frac{\vx-\vx'}{|\vx-\vx'|^3}, \nabla^{2}\left(\frac{1}{|\vx - \vx'|}\right) = 0.
\end{equation}
\noindent
So we conclude that outside of a very small region around $\vx$, $\nabla^{2}\Phi(\vx)=0$.
Let's take a spherical region $S(\epsilon)$ of radius $\epsilon$ centered on $\vx$. In proceeding, let's 
note that 
\begin{equation}
\nabla^{2} f(|\vx-\vx'|) = \nabla_{\vx'}^{2} f(|\vx-\vx'|)
\end{equation}
\noindent
for any function $f(|\vx - \vx'|)$. If we take $\epsilon$ to be small enough such 
that $\rho(\vx)\approx$ a constant, then we can write
\begin{eqnarray}
\label{eqn:pois_med}
\nabla^{2}\Phi(\vx) &\approx& - G\rho(\vx) \int_{S(\epsilon)} \nabla^2 \left( \frac{1}{|\vx - \vx'|} \right)d^{3}\vx' \nonumber \\
&=& - G \rho(\vx)  \int_{S(\epsilon)} \nabla_{\vx'}^2 \left( \frac{1}{|\vx - \vx'|} \right)d V'.
\end{eqnarray}
\noindent
Now we get to use the {\it divergence} theorem
\begin{equation}
\int \nabla^{2} f dV = \oint \nabla f \cdot dS,
\end{equation}
\noindent
which allows us to write Equation \ref{eqn:pois_med} as
\begin{equation}
- G \rho(\vx)  \int_{S(\epsilon)} \nabla_{\vx'}^2 \left( \frac{1}{|\vx - \vx'|} \right)d V' = - G \rho(\vx) \oint_{S(\epsilon)} \nabla_{\vx'}\left(\frac{1}{|\vx-\vx'|}\right) \cdot d\vS'
\end{equation}
\noindent
By applying Equation \ref{eqn:nabla} and the identity $\nabla_{\vx'} f = -\nabla f$, we have
\begin{eqnarray}
- G \rho(\vx) \oint_{S(\epsilon)} \nabla_{\vx'}\left(\frac{1}{|\vx-\vx'|}\right) \cdot d\vS' &=&
-G \rho(\vx) \oint_{S(\epsilon)} \left(\frac{\vx-\vx'}{|\vx-\vx'|^{3}}\right) \cdot d\vS' \nonumber \\
&=& 4 \pi G \rho(\vx)
\end{eqnarray}






\section{Lecture 6: Distribution Function}

The {\it distribution function} $f(\vx,\vv,t)$ gives the probability density
in six-dimensional {\it phase space} $(\vx,\vv)$ of having an object
in the volume $d\vx d\vv$.  The number density $n(\vx,t)$ of objects is the
volume integral of the distribution function
\begin{equation}
n(\vx,t) = \int_{-\infty}^{\infty}\int_{-\infty}^{\infty}\int_{-\infty}^{\infty} f(\vx,\vv,t) dv_{x}dv_y,dv_z.
\end{equation}
\noindent
We can use this expression to define moments of the velocity distribution, such as
the average velocity 
\begin{equation}
\ave{\vv(\vx,t)} = \frac{1}{n(\vx,t)} \int_{-\infty}^{\infty}\int_{-\infty}^{\infty}\int_{-\infty}^{\infty} \vv f(\vx,\vv,t) dv_{x}dv_y,dv_z.
\end{equation}

For a collisionless system where objects cannot be created or destroyed, the number density of objects
in a given volume will follow the continuity equation
\begin{equation}
\label{eqn:continuity}
\frac{\partial n}{\partial t} + \frac{\partial (nv)}{\partial x} = 0.
\end{equation}
\noindent
This equation simply describes the mass conservation of objects, such that
the time rate of change of the number density is balanced by the advection of
spatial gradients in the number density.

The {\it collisionless Boltzmann equation} that describes the probability density 
of objects in phase space is more complicated because it must describe a time variation
in the velocity as well as the spatial coordinates.  In one dimension, we have that
\begin{equation}
\frac{\partial f}{\partial t} + v\frac{\partial f}{\partial x} + \frac{dv}{dt}(x,v,t) \cdot \frac{\partial f}{\partial v} = 0
\end{equation}
\noindent
The acceleration of an object will only depend on position in a background potential, so
we have that $dv/dt = -\partial \Phi/\partial x$, and
\begin{equation}
\label{eqn:boltzmann}
\frac{\partial f}{\partial t} + v \frac{\partial f}{\partial x} - \frac{\partial \Phi}{\partial x}(x,t)\cdot \frac{\partial f}{\partial v} = 0.
\end{equation}
\noindent
In three dimensions, we can write
\begin{equation}
\label{eqn:bvec}
\frac{\partial f(\vx, \vv, t)}{\partial t} + \vv \cdot \nabla f - \nabla \Phi \cdot \frac{\partial f}{\partial \vv} = 0.
\end{equation}
\noindent
We usually don't deal with this equation in this form, but often will take moments with respect to
e.g., velocity
\begin{equation}
\frac{\partial n(x,t)}{\partial t} + \frac{\partial}{\partial x}[n(x,t)\ave{v(x,t)}] - \frac{\partial \Phi}{\partial x}(x,t)[f]_{-\infty}^{\infty} = 0
\end{equation}
\noindent
The last term will be zero if $f$ is well behaved, and we get back Equation \ref{eqn:continuity}.

If we integrate Equation \ref{eqn:boltzmann} multiplied by $v$, we instead find
\begin{equation}
\frac{\partial}{\partial t}[n(x,t)\ave{v(x,t)}] + \frac{\partial}{\partial x}[n(x,t)\ave{v^2(x,t)}] = - n(x,t) \frac{\partial \Phi}{\partial x}
\end{equation}
\noindent
If we define the velocity dispersion as
\begin{equation}
\ave{v^2(x,t)} = \ave{v(x,t)}^2 + \sigma^2,
\end{equation}
\noindent
apply this definition, and then divide by $n(x,t)$, we have
\begin{equation}
\label{eqn:v_ave}
\frac{d\ave{v}}{dt} + \ave{v}\frac{\partial \ave{v}}{\partial x} = -\frac{\partial \Phi}{\partial x} - \frac{1}{n}\frac{\partial}{\partial x}[n\sigma^2(x,t)].
\end{equation}



\section{Lecture 7: A Simple Example of the Tidal Limit}

Consider the satellite to have mass $m$ and the main galaxy to have mass $M$, separated by distance $D$,
orbiting the center of mass $C$ with angular speed $\Omega$.  Take $x$ as the coordinate
along the ray between $m$ toward $M$.  The center of mass is then at $x=DM/(M+m)$.  The
effective potential is then
\begin{equation}
\Phieff(x) = -\frac{GM}{|D-x|} -\frac{Gm}{|x|} - \frac{\Omega^2}{2}\left(x-\frac{DM}{M+m}\right)^2.
\end{equation}

The effective potential has three maxima known as the {\it Lagrange points}, [$L_1$, $L_2$, $L_3$].  Let's find them by setting
the derivative of the effective potential to zero.
\begin{equation}
\label{eqn:find_phieff_max}
\frac{\partial\Phieff}{\partial x} = 0 = -\frac{GM}{(D-x)^2} \pm \frac{Gm}{x^2} - \Omega^2\left(x-\frac{DM}{M+m}\right)
\end{equation}
\noindent
The acceleration $\Omega^2DM/(M+m)$ of $m$ as it circles $C$ owes to the gravitational
attraction of $M$. We can then write
\begin{equation}
\Omega\frac{DM}{M+m}=\frac{GM}{D^2},~~\mathrm{so}~\Omega^2=\frac{G(M+m)}{D^3}
\end{equation}
\noindent
If the satellite is much less massive than the main galaxy, $L_1$ and $L_2$ will lie close to $m$.
Substitute $\Omega^2$ into Equation \ref{eqn:find_phieff_max}, and expand in powers of $x/D$ to find
\begin{equation}
0 \approx -\frac{GM}{D^2} - 2\frac{GM}{D^3}x \pm \frac{Gm}{x^2} - \frac{G(M+m)}{D^3}\left(x-\frac{DM}{M+m}\right)
\end{equation}
\noindent
At the Lagrange points $L_1$ and $L_2$ we have
\begin{equation}
x = \pm r_J, ~~\mathrm{where}~r_J = D\left(\frac{m}{3M+m}\right)^{1/3}
\end{equation}
\noindent
Stars that cannot stray further from the satellite than $r_J$, the {\it Jacobi radius} or {\it Roche limit},
will remain bound to it.  The radius $L_1$ is not where the gravitational force from the satellite and 
main galaxy are the same, but where the effective potential has a minimum and lies further from the satellite.
This radius is also where expanding stars lose mass to their companion.

When $M\gg m$, then the mean density within $r_J$ is three times the mean density within $D$ of the main galaxy.
A star orbiting the satellite near $r_J$ will have an orbital period comparable to the orbit period of the
satellite around the main galaxy.

When satellites are not on circular orbits, the relevant $r_J$ is determined at the pericenter. If the
satellite is orbiting within the dark matter halo of the main galaxy, the relevant radius is
\begin{equation}
r_J = D\left[\frac{m}{2M(<D)}\right]^{1/3}
\end{equation}
where $M(<D)$ is the mass within D.

What's the Jacobi radius of the LMC-Milky Way system? The LMC is at a distance of $\sim50\kpc$ where
the speed of a circular orbit is about the same as it is at the solar circle, or about $\sim200\km~\s^{-1}$.
The mass of the Milky Way within the LMC's orbit is about $5\times10^{11}\Msun$.  The LMC mass is about
$10^{10}\Msun$, so we have
\begin{equation}
r_J\approx 50\kpc~\times~\left(\frac{10^{10}\Msun}{2\times5\times10^{11}\Msun}\right)^{1/3} \approx 11\kpc.
\end{equation}
The LMC disk lies within this radius, but the SMC is too far away to stay bound to the LMC.


\section{Lecture 8: Rotation Curve}

For a galaxy at a systemic velocity $V_{sys}$ inclined at an angle $i$ to face-on, the rotation curve we measure as a function of radial distance $R$ and azimuth $\phi$ is
\begin{equation}
V_r(R,i) = V_{sys} + V(R)\sin i \cos \phi
\end{equation}
\noindent
so to determine $V(R)$ we must determine the inclination and perhaps determine an angular average over azimuth.

In determining the rotation curve, we are often trying to weigh the galaxy.  This is straightforward to do in
some limiting cases.

For a thin exponential disk that supplies its own gravity, the rotation curve can be written in terms of Bessel
functions as
\begin{equation}
V^2(R) = 4\pi G \Sigma_0 h_R y^2 [I_0(y) K_0(y) - I_1(y) K_1(y)]
\end{equation}
\noindent
where $\Sigma_0$ is the central mass surface density, $y\equiv R/2h_R$,
 and $I$ and $K$ are modified Bessel functions (mind the subscript!). We can
relate this to the total disk mass $M_d = 2\pi\Sigma_0 h_R^2$, such that
\begin{equation}
V^2(R) = \frac{2GM_d}{h_R}f(y).
\end{equation}

Otherwise, if we examine the regions of a disk where the dark halo dominates the potential, than we can just use
the radial force equation
\begin{equation}
\frac{V^2(R)}{R} = \frac{GM(<R)}{R^2}
\end{equation}
\noindent 
to estimate the interior mass from the rotation curve.

\section{Lecture 9: Fundamental Relations of Elliptical Galaxies}

There is a correlation between elliptical galaxy luminosity and
velocity dispersion called the Faber-Jackson relation
\begin{equation}
\frac{L_V}{2\times10^{10}\Lsun} \approx \left( \frac{\sigma}{200~\km~\s^{-1}}\right)^{4}.
\end{equation}

There is another relation called the {\it Fundamental Plane} that
connects the effective radius, surface brightness within the effective
radius, and velocity dispersion as
\begin{equation}
\Reff \propto \sigma^{1.2} I_{e}^{-0.8}.
\end{equation}
Note that expectations would suggest $\Reff \propto \sigma^2 I_{e}^{-1}$.

\section{Lecture 10: Dynamical Friction}

One of the most important phenomena in groups and clusters is
{\it dynamical friction}, which is the process by which the
galaxies in groups and clusters lose energy and angular momentum
to the surrounding sea of dark matter and stars in the intragroup
or intracluster medium. Dynamical friction causes galaxies to ``sink''
to the central regions of the group or cluster and thereby be
assimilated into the larger system.  So how does this process work?

As with the impluse approximation we studied before, consider a
{\it galaxy} of mass $M$ moving past a star (or clump of dark matter)
with a mass $m$ at an impact parameter $b$.  During the passage, the
galaxy has a change of velocity in the direction perpendicular to the
direction motion of
\begin{equation}
\Delta V_{\perp} = \frac{2Gm}{bV}.
\end{equation}
\noindent
This equation applies when the impact parameter $b$ is larger than the
size of the galaxy with mass $M$, and when the velocity $V$ is large
enough at separation $b$ such that $\Delta V_{\perp}\ll V$.  We then
require that
\begin{equation}
b \gg \frac{2G(M+m)}{V^2} \equiv 2 r_s.
\end{equation}
\noindent
Note this radius differs from the strong encounter radius we used
previously because $M\ne m$ in this case.

The perturber must obtain an equal and opposite momentum, so the
total kinetic energy in the perpendicular motion becomes
\begin{equation}
\Delta KE_{\perp} = \frac{M}{2}\left(\frac{2Gm}{bV}\right)^2 + \frac{m}{2}\left(\frac{2GM}{bV}\right)^2 = \frac{2G^2mM(M+m)}{b^2 V^2}.
\end{equation}
\noindent
The smaller object acquires most of the energy, which must be depleted from the motion by an amount $\Delta V_{\parallel}$ of the
larger object.  Once the galaxy and the perturber are far away (and before the encounter) the kinetic energies 
before and after the encounter must be the same.  We then have that
\begin{equation}
\frac{M}{2}V^2 = \Delta KE_{\perp} + \frac{M}{2}(V + \Delta V_{\parallel})^2 + \frac{m}{2}\left(\frac{M}{m} \Delta V_{\parallel} \right)^2
\end{equation}
\noindent
If we take $\Delta V_{\parallel} \ll V$, then the terms proportional to $\Delta V_{\parallel}^2$ are very small and
can be ignored.  We can then find the amount by which each perturber reduces the velocity of the galaxy with
mass $M$ as
\begin{equation}
- \Delta V_{\parallel} \approx \frac{\Delta KE_{\perp}}{M V} = \frac{2G^2 m(M+m)}{b^2 V^3}.
\end{equation}

OK, now what if there are a number density per cubic parsec $n$ perturbers of mass $m$?  
We then have to integrate through the cylinder of radius $b$ to find the total
effect on the galaxy of mass $M$.  We find that
\begin{equation}
-\frac{dV}{dt} = \int_{b_{\mathrm{min}}}^{b_{\mathrm{max}}} n V \frac{2G^2 m (M+m)}{b^2 V^3} 2\pi b db = \frac{4\pi G^2(M+m)}{V^2} n m \ln \Lambda
\end{equation}
\noindent
where $\Lambda\equiv b_{\mathrm{max}}/b_{\mathrm{min}}$.

Some interesting things to note
\begin{enumerate}
\item The slower the galaxy $M$ moves, the larger its deceleration.
\item If $V\ll\sigma$, where $\sigma$ is the velocity dispersion of perturbers, we find that $dV/dt\propto -V$.  This is the same as what happens to a parachutist.
\item The net effect is to lower the total kinetic motions of the galaxies over time.
Before the encounter, the kinetic energy of one of the galaxies is
\begin{equation}
E_0 = KE_0 + PE_0 = - KE_0
\end{equation}
\noindent
since the potential energy is $ PE_0 = - 2KE_0$ in virial equilibrium.
Dynamical friction increases the energy in random motions and the internal kinetic
energy by $\Delta KE$.  After the system again reaches virial equilibrium,
the kinetic energy is less than before
\begin{equation}
KE_1  = - (E_0 + \Delta KE) = KE_0 - \Delta KE
\end{equation}
\noindent
Stars that gain the most kinetic energy are ejected.
\item Groups have lower relative velocties, so the dynamical
friction should operate more efficiently.
\end{enumerate}



\section{Lecture 10: Gravitational Lensing}

Einstein predicted that light passing a distance $b$
past a mass $M$ would be deflected by an angle
\begin{equation}
\label{eqn:alpha_grav_lens}
\alpha \approx \frac{4GM}{bc^2} = \frac{2R_s}{b}
\end{equation}
\noindent
where $R_s = 2 GM/c^2$ is the {\it Schwarzschild radius}.
For the Sun, $R_s\sim3~\km$. The approximation
holds only for small deflections $\alpha\ll1$.
Note this is exactly twice the deflection determined by 
the impluse approximation.

{\bf Show figure 7.14 of SG.}

Without the lense $L$, the object would appear at an angle
$\beta = y/d_{S}$ as long as $d_{S} \gg y$.
The light is bent by $\alpha$, so the object instead appears
at an angle $\theta \approx x/d_S$ (for $d_S \gg x$).
When the bending is small $x-y = \alpha d_{LS}$.
The impact parameter in the lens plane is $b = \theta d_{L}$
as long as $d_S \gg b$.

If we divide Equation \ref{eqn:alpha_grav_lens} by $d_S$
we find
\begin{equation}
\theta - \beta = \frac{\alpha d_{LS}}{d_S} = \frac{1}{\theta} \frac{4 GM}{c^2} \frac{d_{LS}}{d_L d_S} \equiv \frac{1}{\theta} \theta_E^2
\end{equation}
\noindent
where $\theta_E$ is called the {\it Einstein radius}.
We have a quadratic relation between the angular distance $\theta$ between
$L$ and the object's position as
\begin{equation}
\theta^2 - \beta \theta - \theta_E^2 = 0,~\mathrm{so}~\theta_{\pm} = \frac{\beta \pm \sqrt{\beta^2 + 4 \theta_E^2}}{2}.
\end{equation}
\noindent
A star directly behind the lense with $\beta=0$ will be seen as a circle
on the sky with radius $\theta_E$.  When $\beta>0$, the image at $\theta_{+}$
is further away from the lense with $\theta_{+}>\beta$ and is outside
the Einstein radius with $\theta_{+}>\theta_E$ (these were the images
seen in the Sun's lensing).  The image at $\theta_{-}$ is inverted, on
the other side of the lens, and is within the Einstein radius.

Stars in the disk plane of the Milky Way will lense each other.  We
often cannot resolve the lense and the source separately, but we will
see the magnification of the star -- we call this microlensing.

Gravitational lensing leaves the surface brightness unchanged by increases
the area of an extended source on the sky.  The increase of the apparent
brightness is then just proportional to the increase in the area.

{\bf Show figure 7.15 of SG.}

Consider an annulus of width $S'$ centered on $L$ between
radius $y$ and $y+\Delta y$.  An image $I$ of
$S'$ occupies the same angle $\delta \Phi$ but the distance
from the center is expanded or contracted as $x/y = \theta/\beta$
while $\delta x/\delta y= d\theta/d\beta$.  The ratio of the areas
is 
\begin{equation}
\frac{A_{\pm}(image)}{A(source)} = \left| \frac{\theta}{\beta}\frac{d\theta}{d\beta}\right| = \frac{1}{4}\left( \frac{\beta}{\sqrt{\beta^2 + 4 \theta_E^2}} + \frac{\sqrt{\beta^2 + 4 \theta_E^2}}{\beta}\pm 2 \right)
\end{equation}
\noindent
The image at $\theta_{+}$ is always brighter than the source and is stretched in the
tangential direction.  The closer image is dimmer unless
\begin{equation}
\beta^2 < (3-2\sqrt{2})\theta_E^2 / \sqrt{2}
\end{equation}
\noindent
or
\begin{equation}
\beta \lesssim 0.348 \theta_E.
\end{equation}




\section{Lecture 11: Expansion History}

The rate of expansion is determined by the gravitational effects
of the energy densities it contains.  We can model the expansion
using Newtonian physics first, and then use general relativity
to revise for a more correct answer.

Consider a sphere of radius $r$ at time $t$ when the 
universe has a typical density $\rho(t)$, and assume
$r$ is much less than any curvature radius in the 
universe.  Assuming symmetry about $r=0$, the
gravitational force at radius $r$ is just supplied by the
mass within the sphere.  If the sphere
is large enough that pressure forces are small, then
the force on a object of mass $m$ at radius $r$ is
\begin{equation}
m \frac{d^2 r}{dt^2} = - \frac{G m M(<r)}{r^2} = - \frac{4\pi Gm}{3}\rho(t)r.
\end{equation}
\noindent
The radius of the sphere of matter is expanding with the
rest of the universe, so $r\propto R(t)$.  The mass of the
cloud cancels, such that
\begin{equation}
\label{eqn:friedman_A}
\ddot{R}(t) = - \frac{4\pi G}{3} \rho(t) R(t).
\end{equation}
\noindent
The mass doesn't change, so $\rho(t) R(t)$ is a constant.
We can multiply by $\dot{R}(t)$ to find
\begin{equation}
\frac{1}{2} \frac{d}{dt}[\dot{R}^2(t)] = - \frac{4 \pi G}{3} \frac{\rho(t_0)R^3(t_0)}{R^2(t)} \dot{R}(t)
\end{equation}
\noindent
where $t_0$ is the present day.  We then integrate to find
\begin{equation}
\label{eqn:friedman_B}
\dot{R}^2(t) = \frac{8\pi G}{3}\rho(t) R^2(t) - kc^2
\end{equation}
\noindent
where $k$ is a constant of integration.  Turns out Equation
\ref{eqn:friedman_B} is valid in GR, and tells us
that $k$ is the same constant as in the metric.

We appeal to thermodynamics to tell us that
as heat $\Delta Q$ flows into a volume $V$ its
internal energy $E$ must increase, or it expands
and does work against pressure
\begin{equation}
\Delta Q = \Delta E + p \Delta V = V \Delta(\rho c^2) + (\rho c^2 + p)\Delta V
\end{equation}
\noindent
where $\rho$ includes all the forms of matter and energy.
But no volume $V$ gains heat at the expense of another and
\begin{equation}
\Delta Q = 0 = \Delta \rho + \left(\rho + \frac{p}{c^2}\right)\frac{\Delta V}{V}
\end{equation}
\noindent
or
\begin{equation}
\frac{d\rho}{dt} = - c \frac{\dot{R}(t)}{R(t)}\left( \rho + \frac{p}{c^2}\right)
\end{equation}
\noindent
Differentiating Equation \ref{eqn:friedman_B} and substituting for $d\rho/dt$
yields
\begin{equation}
\label{eqn:friedman_C}
\ddot{R}(t) = - \frac{4\pi G}{3} R(t) \left[ \rho(t) + \frac{3p(t)}{c^2}\right]
\end{equation}
\noindent
So in GR, the pressure $p$ adds to the gravitational attraction. 
Equations \ref{eqn:friedman_B} and \ref{eqn:friedman_C} are
called the {\it Friedmann equations}.  For cool matter
the pressure $p \sim \rho c_s^2$ where the sound speed $c_s \ll c$,
and the pressure term in equation \ref{eqn:friedman_C} is 0.
Then $\rho(t)\propto R^{-3}$.  For radiation and particles,
$p\approx pc^2/3$ and $\rho(t) \propto R^{-4}(t)$.
For matter and radiation $\rho + 3p/c^2$ is positive, and the
universe decelerates. 

The quantity $\rho(t)R^2(t)$ decreases as $R(t)$ grows, so the
right hand side of eqn \ref{eqn:friedman_B} becomes negative
for large $R$ if $k=1$.  Since $\dot{R}^2$ cannot be negative,
$R$ will reach a maximum and turn around.  For $k\le0$, the
expansion never ends.

GR allows for a {\it vacuum energy} with constant density $\rho_{\Lambda} = \Lambda/(8\pi G)$.  Since $\rho_{\Lambda}$ is a constant, the RHS of the $\Delta Q$ equation must
be zero since $\rho_{\Lambda}$
cannot change and the pressure is $p_{\Lambda} = -\Lambda c^2/(8\pi G)$,
which is more like a tension pulling out than a pressure pushing in.
Once the universe expands so $\rho_{\Lambda}$ is larger than the other
components, the universe expands exponentially.  This kind of 
expansion may have happened in the early universe (inflation), which
enables us to explain away several cosmological puzzles.


\section{Lecture 11: Growth of Structure}

The initial perturbations seeded by cosmic inflation likely had a
power spectrum $P(k) \propto k$.  These perturbations were affected
by physical processes at later times to alter $P(k)$ on small scales.
We see evidence of this in the cosmic microwave background. As
photons try to move out of the gravitational potential wells of the
initial density perturbations, the photons experience a 
{\it gravitational redshift} $\Delta\Phi_g$ that changes the
temperature $T$ of the radiation by an amount $\Delta T$ according
to
\begin{equation}
\frac{\Delta T}{T}|_{\mathrm{grav}} \sim \Delta \Phi_g c^{-2}
\end{equation}
\noindent
The temperature is reduced where the potentially
is unusually deep since $\Delta \Phi_g$ is negative
there, and time runs more slowly such that $\Delta t/t =\Delta \Phi_g/c^2$
and we see the gas at an earlier time when it was hotter.  The temperature
declines by $T\propto 1/ a(t)$, so we have
\begin{equation}
\left.\frac{\Delta T}{T}\right|_{\mathrm{time}} = -\frac{\Delta a}{a} = -\frac{2}{3}\frac{\Delta t}{t} =  -\frac{2}{3}\frac{\Delta \Phi_g}{c^2}
\end{equation}
\noindent
where we've used $a\propto t^{2/3}$ for the matter-dominated era.
The net effect is that $\Delta T/T \sim \Delta \Phi_g / (3 c^2)$.
If the region has density $\rho = \bar{\rho}(1+\delta)$ and a
corresponding mass excess of $\Delta M = 4\pi \bar{\rho}R^3 \delta / 3$,
then 
\begin{equation}
3c^2\frac{\Delta T}{T} = \Delta \Phi_g \sim - \frac{2G\Delta M}{R} = - \frac{8\pi}{3} G \bar{\rho}R^2 \delta \approx - \delta(t)[\bar{H}(t)R]^2
\end{equation}
\noindent
such that radiation from dense regions is {\it colder}.

We can model the temperature map of the sky by expanding
the pattern of the CMB in spherical harmonics as
\begin{equation}
\Delta T(\theta, \phi) = \sum_{l>1} \sum_{-l\le m \le l} a_l^m Y_l^m(\theta, \phi).
\end{equation}
\noindent
The theoretical predictions are often expressed in terms of $C_l = \ave{|a_l^m|^2}$ averaged over $m$ since this does not depend on the direction of ``north''
in the map.  A commonly plotted quantity is $\Delta_T^2 = T^2 l(l+1)C_l/(2\pi)$.


Pressure forces can act to suppress structure, but only on the
sound horizon scale.  The sound horizon scale when the
gas becomes transparent to photons (recombination) is
\begin{equation}
R(t_{rec}) \sigma_H = 3 c t_{rec} = \frac{2c}{H(t_{rec})} \approx \frac{2c}{H_0\sqrt{\Omega_m}(1+z_{rec})^{3/2}}.
\end{equation}
\noindent
This works out to about $184/(h^2 \Omega_m)^{1/2}$~Mpc today.  The
angle $\theta_H$ subtended by the sound horizon depends on
the angular-size distance to $z_{rec}$.  When $\Omega_\Lambda\to0$
and $\Omega_0 z\gg1$, then $d_A\to 2c/(H_0 z \Omega_0)$. We then
have
\begin{equation}
\theta_H \approx \frac{R(t_{rec}\sigma_H)}{d_A(t_{rec})} \approx \sqrt{\frac{\Omega_0}{z_{rec}}} \approx 2^\circ \times \sqrt{\Omega_0},
\end{equation}
\noindent
and only points separated by this angle on the sky can communicate before $t_{rec}$.  For $\Omega_0=1$, $\Delta_T$ is largest on the scale of about a degree,
where the {\it first acoustic peak} of the CMB is found. A model
with $\Omega_0=0.3$ provides about half this angle. In a cosmology
close to what we think ours is, it turns out the 
first peak is at $l\approx 220$ that corresponds to a size scale of 
aobut $105~\Mpc$ today, which is a sphere that contains $2.5\times10^{16}M_{\odot}$.  The second peak is at $l\approx540$.


\section{Lecture 12: Tidal Torques}

Peculiar velocities grow as $\vv\propto t^{1/3}$ while distances
grow as $d\propto a(t) \propto t^{2/3}$. In the linear regime,
angular momentum then grows as $d \times v\propto t$. The
region stops accruing angular momentum once it turns around
and begins to collapse. Therefore, more overdense regions 
tend to have less time to spin up.  However, tidal torques
are also stronger in dense regions and so objects
acquire the same average angular momentum in relation to their
mass or energy.

A galaxy of radius $R$, mass $M$, and angular momentum $L$
will rotate an angular speed
\begin{equation}
\omega \sim L / (MR^2).
\end{equation}
\noindent
The angular speed of a circular orbit at radius $R$ is
\begin{equation}
\omega_c^2 R \sim G M/R^2.
\end{equation}
The energy is $E\sim - GM^2/R$.  We therefore have 
\begin{equation}
\frac{\omega}{\omega_c}\equiv\lambda = \frac{L}{MR^2} \times \frac{R^{3/2}}{\sqrt{GM}} = \frac{L|E|^{1/2}}{GM^{5/2}}.
\end{equation}
\noindent
From N-body simulations, we expect that $\lambda\sim$ a  few percent.
Ellipticals have about this spin, but the Milky Way has $\lambda \approx 0.5$.
Since the MW is a disk, energy dissipation can help amplify $\lambda$.

This also argues for a dark halo, as otherwise the disk would
not have time to form.  Without a halo, $L$ and $M$ remain
fixed as the disk moves in.  The radius must decrease by $100\times$
for $E$ to increase proportionally by the same amount.
Disk material near the Sun would need to originate $800~\kpc$
from the center, but $M(<R)$ would lie interior to the Sun's orbit.
The orbital period of the Sun would be $1000\times$ longer than
it's observed to be ($240~\Gyr$).  It would take many times the
age of the universe to make the disk.

Since the milky way has a large DM halo, the gas in the MW disk
originates from a radius closer by a factor $M_d/M_{DM}$. The
decrease in size is only a factor of 10.  Shrinking at
$200~\km~\s^{-1}$ from a radius $80~\kpc$, the disk could
have formed in $\lesssim2~\Gyr$.





\end{document}