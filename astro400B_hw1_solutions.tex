\documentclass[]{article}
\usepackage[margin=1.0in]{geometry}
\usepackage{amssymb}
\tiny

%title material
\title{Astronomy 400B Homework 1 Solutions}
\author{\vspace{-10ex}}

\date{\vspace{-10ex}}

%include latex definitions

%average
\newcommand{\ave}[1]{\langle#1\rangle}
%radian
\newcommand{\rad}{\mathrm{rad}}

%astronomical unit
\newcommand{\AU}{\mathrm{AU}}

%centimeter
\newcommand{\cm}{\mathrm{cm}}

%meter
\newcommand{\m}{\mathrm{m}}

%kilometer
\newcommand{\km}{\mathrm{km}}

%parsec
\newcommand{\pc}{\mathrm{pc}}

%kiloparsec
\newcommand{\kpc}{\mathrm{kpc}}

%megaparsec
\newcommand{\Mpc}{\mathrm{Mpc}}

%gigaparsec
\newcommand{\Gpc}{\mathrm{Gpc}}

%light year
\newcommand{\ly}{\mathrm{ly}}

%second
\newcommand{\s}{\mathrm{s}}
\newcommand{\yr}{\mathrm{yr}}
\newcommand{\Gyr}{\mathrm{Gyr}}

%solar mass
\newcommand{\Msun}{M_{\odot}}

%grams
\newcommand{\g}{\mathrm{g}}

%erg
\newcommand{\erg}{\mathrm{erg}}

%solar luminosity
\newcommand{\Lsun}{L_{\odot}}

%jansky 
\newcommand{\Jy}{\mathrm{Jy}}

%flux density
\newcommand{\Fnu}{F_{\nu}}
\newcommand{\Flambda}{F_{\lambda}}
\newcommand{\Lnu}{L_{\nu}}

%hertz 
\newcommand{\Hz}{\mathrm{Hz}}

%angstrom
\newcommand{\Ang}{\mathrm{\r{A}}}

%solar radius
\newcommand{\Rsun}{R_{\odot}}

%stefan-boltzmann
\newcommand{\sigmaSB}{\sigma_{\mathrm{SB}}}

%boltzmann
\newcommand{\kB}{k_{\mathrm{B}}}

%kelvin
\newcommand{\K}{\mathrm{K}}

%T_bandpass
\newcommand{\TBP}{T_{\mathrm{BP}}}

%magnitude
\newcommand{\mg}{\mathrm{mag}}


%arcsecond
\newcommand{\arcsec}{\mathrm{arcsec}}

%begin the document
\begin{document}


%make the title, goes after document begins
\maketitle

%first section
\section{Problem 1: Sparke \& Gallagher 1.3}

For Betelguese, $T\approx3500\K$, $\theta=0.045~\arcsec$ in diameter, and $d=140\pc$.

\begin{eqnarray}
R &=& \theta \times d\\
R &=& (0.045/2)~\arcsec \times (4.848137 \times 10^{-6} \rad / \arcsec) \times 140 \pc \nonumber \\ &\times& (3.0856\times10^{18}\cm/\pc) / (6.96\times10^{10} \cm/\Rsun) \\
R &=& 677\Rsun \approx 700\Rsun
\end{eqnarray}
\noindent
The luminosity is then
\begin{eqnarray}
L &=& 4\pi R^2 \sigma T^{4}\\
L &=& 4\pi (677\Rsun\times 6.96\times 10^{10}\cm/\Rsun)^{2} \times (5.67 \times10^{-5}~\erg~\s^{-1}~\cm^{-2}\K^{-4})\times(3500\K)^{4}\\
L &=& 2.37\times10^{38}~\erg~\s^{-1} / (3.839\times10^{33}~\erg~\s^{-1} / \Lsun) = 0.62\times10^{5}\Lsun\approx10^{5}\Lsun
\end{eqnarray}

\section{Problem 2: Sparke \& Gallagher 1.10}

Equation 1.10 from the book is
\begin{equation}
m_1 - m_2 = -2.5 \log_{10} (F_1/F_2)
\end{equation}
\noindent
If a source is dimmed by $\exp(-\tau_{\lambda})$, then
the attenuated magnitude $m_1$ relative to the unattenuated
magnitude $m_2$ can be found as
\begin{eqnarray}
m_1 - m_2 &=& -2.5 \log_{10} \exp(-\tau_{\lambda}) = 2.5 \tau_{\lambda} \log_{10} e = 1.086\tau_{\lambda}\\
m_1 &=& m_2 + 1.086\tau_{\lambda}\\
\therefore m_1 &=& m_2 + A_{\lambda};~~A_{\lambda} = 1.086\tau_{\lambda},
\end{eqnarray}
\noindent
where $A_{\lambda}$ is called the {\it extinction}.

\section{Problem 3: Sparke \& Gallagher 1.12}

The Milky Way luminosity is $L\approx2\times10^{10}\Lsun$,
and pretend it is a sphere with radius $R\approx5\kpc$.

Equation 1.3 is
\begin{equation}
L=4\pi R^2 \sigma T^4.
\end{equation}
\noindent
We need to find $T$, so
\begin{eqnarray}
T &=& \left( \frac{L}{4\pi R^2 \sigma}\right)^{1/4}\\
T &=& \left( \frac{2\times10^{10}\Lsun \times 3.839\times10^{33}~\erg~\s^{-1}/\Lsun}{4\pi\times(5\kpc \times 3.0857\times10^{21}\cm/\kpc)^2 \times (5.67\times10^{-5}\erg~\s^{-1}~\cm^{-2}~\K^{-4})}\right)^{1/4}\\
T &=& (1131.75\K^{4})^{1/4} = 5.8\K \approx 5\K
\end{eqnarray}

\section{Problem 4: Sparke \& Gallagher 1.16}

Equation 1.25 from the book is
\begin{equation}
\rho_{L}(B_J) = \int_{0}^{\infty}\Phi(L)dL = n_{\star}L_{\star}\Gamma(\alpha+2) \approx 2 \times 10^{8}h\Lsun~\Mpc^{-3}.
\end{equation}
\noindent
Let's be more precise, and use $L_\star = 9\times10^{9}h^{-2}\Lsun$, $n_{\star} = 0.02 h^{3}\Mpc^{-3}$, and $\alpha=-0.46$ such that $\Gamma(2-0.46) = 0.888178$.  We then have
\begin{equation}
\rho_L (B_J) = n_{\star}L_\star\Gamma(2+\alpha) = (9\times10^{9}h^{-2}\Lsun) \times (0.02 h^{3}\Mpc^{-3}) \times (0.888178) = 1.599\times10^{8}h\Lsun~\Mpc^{-3}
\end{equation}
\noindent
For the universe to be at the critical density, we would need
the average mass to light ratio to be
\begin{equation}
M/L = \rhoc/\rho_L = \frac{2.8\times10^{11}h^{2}\Msun\Mpc^{-3}}{1.599\times10^{8}h\Lsun~\Mpc^{-3}} = 1751 h\Msun/\Lsun \approx 1700 h\Msun/\Lsun.
\end{equation}


\end{document}