\documentclass[]{article}
\usepackage[margin=1.0in]{geometry}
\usepackage{amssymb}

%title material
\title{Astronomy 400B Lecture X: YYY}
\author{Brant Robertson}
\date{April, 2015}


%include latex definitions

%average
\newcommand{\ave}[1]{\langle#1\rangle}
%radian
\newcommand{\rad}{\mathrm{rad}}

%astronomical unit
\newcommand{\AU}{\mathrm{AU}}

%centimeter
\newcommand{\cm}{\mathrm{cm}}

%meter
\newcommand{\m}{\mathrm{m}}

%kilometer
\newcommand{\km}{\mathrm{km}}

%parsec
\newcommand{\pc}{\mathrm{pc}}

%kiloparsec
\newcommand{\kpc}{\mathrm{kpc}}

%megaparsec
\newcommand{\Mpc}{\mathrm{Mpc}}

%gigaparsec
\newcommand{\Gpc}{\mathrm{Gpc}}

%light year
\newcommand{\ly}{\mathrm{ly}}

%second
\newcommand{\s}{\mathrm{s}}
\newcommand{\yr}{\mathrm{yr}}
\newcommand{\Gyr}{\mathrm{Gyr}}

%solar mass
\newcommand{\Msun}{M_{\odot}}

%grams
\newcommand{\g}{\mathrm{g}}

%erg
\newcommand{\erg}{\mathrm{erg}}

%solar luminosity
\newcommand{\Lsun}{L_{\odot}}

%jansky 
\newcommand{\Jy}{\mathrm{Jy}}

%flux density
\newcommand{\Fnu}{F_{\nu}}
\newcommand{\Flambda}{F_{\lambda}}
\newcommand{\Lnu}{L_{\nu}}

%hertz 
\newcommand{\Hz}{\mathrm{Hz}}

%angstrom
\newcommand{\Ang}{\mathrm{\r{A}}}

%solar radius
\newcommand{\Rsun}{R_{\odot}}

%stefan-boltzmann
\newcommand{\sigmaSB}{\sigma_{\mathrm{SB}}}

%boltzmann
\newcommand{\kB}{k_{\mathrm{B}}}

%kelvin
\newcommand{\K}{\mathrm{K}}

%T_bandpass
\newcommand{\TBP}{T_{\mathrm{BP}}}

%magnitude
\newcommand{\mg}{\mathrm{mag}}


%arcsecond
\newcommand{\arcsec}{\mathrm{arcsec}}

%begin the document
\begin{document}

%make the title, goes after document begins
\maketitle

%first section
\section{Tidal Torques}

Peculiar velocities grow as $\vv\propto t^{1/3}$ while distances
grow as $d\propto a(t) \propto t^{2/3}$. In the linear regime,
angular momentum then grows as $d \times v\propto t$. The
region stops accruing angular momentum once it turns around
and begins to collapse. Therefore, more overdense regions 
tend to have less time to spin up.  However, tidal torques
are also stronger in dense regions and so objects
acquire the same average angular momentum in relation to their
mass or energy.

A galaxy of radius $R$, mass $M$, and angular momentum $L$
will rotate an angular speed
\begin{equation}
\omega \sim L / (MR^2).
\end{equation}
\noindent
The angular speed of a circular orbit at radius $R$ is
\begin{equation}
\omega_c^2 R \sim G M/R^2.
\end{equation}
The energy is $E\sim - GM^2/R$.  We therefore have 
\begin{equation}
\frac{\omega}{\omega_c}\equiv\lambda = \frac{L}{MR^2} \times \frac{R^{3/2}}{\sqrt{GM}} = \frac{L|E|^{1/2}}{GM^{5/2}}.
\end{equation}
\noindent
From N-body simulations, we expect that $\lambda\sim$ a  few percent.
Ellipticals have about this spin, but the Milky Way has $\lambda \approx 0.5$.
Since the MW is a disk, energy dissipation can help amplify $\lambda$.

This also argues for a dark halo, as otherwise the disk would
not have time to form.  Without a halo, $L$ and $M$ remain
fixed as the disk moves in.  The radius must decrease by $100\times$
for $E$ to increase proportionally by the same amount.
Disk material near the Sun would need to originate $800~\kpc$
from the center, but $M(<R)$ would lie interior to the Sun's orbit.
The orbital period of the Sun would be $1000\times$ longer than
it's observed to be ($240~\Gyr$).  It would take many times the
age of the universe to make the disk.

Since the milky way has a large DM halo, the gas in the MW disk
originates from a radius closer by a factor $M_d/M_{DM}$. The
decrease in size is only a factor of 10.  Shrinking at
$200~\km~\s^{-1}$ from a radius $80~\kpc$, the disk could
have formed in $\lesssim2~\Gyr$.




\end{document}