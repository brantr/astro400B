\documentclass[]{article}
\usepackage[margin=1.0in]{geometry}
\usepackage{amssymb}

%title material
\title{Astronomy 400B Lecture 12: Structure Formation Cont.}
\author{Brant Robertson}
\date{April, 2015}


%include latex definitions

%average
\newcommand{\ave}[1]{\langle#1\rangle}
%radian
\newcommand{\rad}{\mathrm{rad}}

%astronomical unit
\newcommand{\AU}{\mathrm{AU}}

%centimeter
\newcommand{\cm}{\mathrm{cm}}

%meter
\newcommand{\m}{\mathrm{m}}

%kilometer
\newcommand{\km}{\mathrm{km}}

%parsec
\newcommand{\pc}{\mathrm{pc}}

%kiloparsec
\newcommand{\kpc}{\mathrm{kpc}}

%megaparsec
\newcommand{\Mpc}{\mathrm{Mpc}}

%gigaparsec
\newcommand{\Gpc}{\mathrm{Gpc}}

%light year
\newcommand{\ly}{\mathrm{ly}}

%second
\newcommand{\s}{\mathrm{s}}
\newcommand{\yr}{\mathrm{yr}}
\newcommand{\Gyr}{\mathrm{Gyr}}

%solar mass
\newcommand{\Msun}{M_{\odot}}

%grams
\newcommand{\g}{\mathrm{g}}

%erg
\newcommand{\erg}{\mathrm{erg}}

%solar luminosity
\newcommand{\Lsun}{L_{\odot}}

%jansky 
\newcommand{\Jy}{\mathrm{Jy}}

%flux density
\newcommand{\Fnu}{F_{\nu}}
\newcommand{\Flambda}{F_{\lambda}}
\newcommand{\Lnu}{L_{\nu}}

%hertz 
\newcommand{\Hz}{\mathrm{Hz}}

%angstrom
\newcommand{\Ang}{\mathrm{\r{A}}}

%solar radius
\newcommand{\Rsun}{R_{\odot}}

%stefan-boltzmann
\newcommand{\sigmaSB}{\sigma_{\mathrm{SB}}}

%boltzmann
\newcommand{\kB}{k_{\mathrm{B}}}

%kelvin
\newcommand{\K}{\mathrm{K}}

%T_bandpass
\newcommand{\TBP}{T_{\mathrm{BP}}}

%magnitude
\newcommand{\mg}{\mathrm{mag}}


%arcsecond
\newcommand{\arcsec}{\mathrm{arcsec}}

%begin the document
\begin{document}

%make the title, goes after document begins
\maketitle

%first section
\section{Tidal Torques}

Peculiar velocities grow as $\vv\propto t^{1/3}$ while distances
grow as $d\propto a(t) \propto t^{2/3}$. In the linear regime,
angular momentum then grows as $d \times v\propto t$. The
region stops accruing angular momentum once it turns around
and begins to collapse. Therefore, more overdense regions 
tend to have less time to spin up.  However, tidal torques
are also stronger in dense regions and so objects
acquire the same average angular momentum in relation to their
mass or energy.

A galaxy of radius $R$, mass $M$, and angular momentum $L$
will rotate an angular speed
\begin{equation}
\omega \sim L / (MR^2).
\end{equation}
\noindent
The angular speed of a circular orbit at radius $R$ is
\begin{equation}
\omega_c^2 R \sim G M/R^2.
\end{equation}
The energy is $E\sim - GM^2/R$.  We therefore have 
\begin{equation}
\frac{\omega}{\omega_c}\equiv\lambda = \frac{L}{MR^2} \times \frac{R^{3/2}}{\sqrt{GM}} = \frac{L|E|^{1/2}}{GM^{5/2}}.
\end{equation}
\noindent
From N-body simulations, we expect that $\lambda\sim$ a  few percent.
Ellipticals have about this spin, but the Milky Way has $\lambda \approx 0.5$.
Since the MW is a disk, energy dissipation can help amplify $\lambda$.

This also argues for a dark halo, as otherwise the disk would
not have time to form.  Without a halo, $L$ and $M$ remain
fixed as the disk moves in.  The radius must decrease by $100\times$
for $E$ to increase proportionally by the same amount.
Disk material near the Sun would need to originate $800~\kpc$
from the center, but $M(<R)$ would lie interior to the Sun's orbit.
The orbital period of the Sun would be $1000\times$ longer than
it's observed to be ($240~\Gyr$).  It would take many times the
age of the universe to make the disk.

Since the milky way has a large DM halo, the gas in the MW disk
originates from a radius closer by a factor $M_d/M_{DM}$. The
decrease in size is only a factor of 10.  Shrinking at
$200~\km~\s^{-1}$ from a radius $80~\kpc$, the disk could
have formed in $\lesssim2~\Gyr$.

\section{Jeans Mass}

The potential energy of a uniform sphere of radius $r$ and
density $\rho$ is
\begin{equation}
PE \equiv -\frac{1}{2}\int \rho(\vx)\Phi(\vx)d^3\vx \approx - \frac{16\pi^2}{15}G\rho^2 r^5.
\end{equation}
\noindent
The kinetic energy of such a sphere made of gas with sound
speed $c_s$ is
\begin{equation}
KE \approx \frac{3c_s^2}{2}\frac{4\pi r^3}{3}\rho.
\end{equation}
In virial equilibrium, we have $|PE| = 2 KE$.  The
cloud's thermal motions couldn't support its weight if
the $KE$ is less than this, and cloud would collapse.
This always happens if the cloud is big enough!  In
fact $KE < |PE|/2$ when
\begin{equation}
2r \gtrsim \sqrt{\frac{15}{\pi}}\sqrt{\frac{c_s^2}{G\rho}}\approx\lambda_J,~\mathrm{where}~\lambda_J\equiv c_s\sqrt{\frac{\pi}{G\rho}}.
\end{equation}
\noindent
We call $\lambda_J$ the {\it Jeans length}.  If the cloud
is larger than $\lambda_J$, then perturbations or slight compressions
will cause the cloud to collapse under gravity.

\subsection{Early Universe}
Early in the radiation-dominated era, the
density $\rho_r = a_B T^4 / c^2$ is low and the
pressure is high, with $c_s = c/\sqrt{3}$.  We
find that
\begin{equation}
\lambda_J = c^2\left(\frac{\pi}{3Ga_BT^4}\right)^{1/2} \propto T^{-2}.
\end{equation}
\noindent
The {\it Jeans mass} $M_J$ is the amount of matter in a sphere of 
diameter $\lambda_J$:
\begin{equation}
M_J \equiv \frac{\pi}{6} \lambda_J^3 \rho_m,
\end{equation}
\noindent
where $\rho_m$ refers only to the matter density.
During radiation-domination, we have $M_J\propto\rho_m T^{-6}$,
with $T\propto 1/R(t)$ and $\rho_m \propto R^{-3}$.  The
Jeans mass then grows as $M_J \propto R^3(t)$.

At matter-radiation equality, the temperature is $T_{eq}$ and
the matter density is $\rho_m = \rho_r = a_B T_{eq}^4/c^2$.
Radiation still provides most of the pressure and
$p\approx c^2\rho_r/3$.  The jeans mass is
\begin{equation}
M_J(t_{eq}) = \frac{\pi}{6}\rho_m(t_{eq})\left(\frac{\pi c^4/3}{Ga_B T_{eq}^4}\right)^{3/2} = \frac{\pi^{5/2}}{18\sqrt{3}}\frac{c^4}{G^{3/2}a_B^{1/2}}\frac{1}{T_{eq}^2}.
\end{equation}
\noindent
If equality occurs at $1+z_{eq} = 24000\Omega_m h^{2}$, then
\begin{equation}
M_(T_{eq}) = 3.6 \times 10^{16}(\Omega_m h^2)^{-2}~\Msun
\end{equation}
\noindent
which is $\sim 100\times$ the mass of the Virgo cluster, and
today would require the mean density in a volume $50/(\Omega_m h^2)~\Mpc$
on a side.

\subsection{After Matter-Radiation Equality}
While the gas in the early universe is still ionized, after
matter-radiation equality the density is $\rho\approx\rho_m$
and the pressure is still $p\approx c^2 \rho_r/3$.
If a small volume is squeezed adiabatically, the changes
in the matter and radiation densities are
\begin{equation}
4 \Delta \rho_m / \rho_m = 3 \Delta \rho_r/\rho_r.
\end{equation}
\noindent
The sound speed is then
\begin{equation}
c_s^2 = \frac{\partial p}{\partial \rho} = \frac{c^2 \Delta\rho_r/3}{\Delta \rho_m} = \frac{c^2}{3}\frac{4\rho_r}{3\rho_m}\propto \frac{1}{R(t)},~\mathrm{so}~\lambda_J = c_s\sqrt{\frac{\pi}{G\rho_m}}\propto R(t).
\end{equation}
\noindent
The Jeans mass remains constant in this era.

\subsection{Recombination}
By $z\sim1100$ the temperature declined to $T_{rec}\sim 3000~\K$ and
electrons could remain bound to protons as hydrogen atoms.  At this
point, the opacity of the universe declined dramatically and photons
could free-stream, becoming uncoupled to the matter.  The operative
pressure then became the pressure of the gas
\begin{equation}
c_s(t_{rec})\approx\sqrt{\frac{k_BT}{m_p}}\approx 5~\km~\s^{-1}.
\end{equation}
\noindent
The Jeans mass became
\begin{equation}
M_J = \frac{\pi}{6} \rho_m\left(\frac{\pi k_B T_{rec}}{G\rho_m m_p}\right)^{3/2} \approx 5\times 10^4 (\Omega_m h^2)^{-1/2}~\Msun
\end{equation}
\noindent
which is about twelve orders of magnitude less than before
matter-radiation equality.

During matter-domination, overdensities grow as $\delta(t)\propto R(t)$, such
that to reach $\delta\sim1$ by the present day we would need that at $z_{rec}\approx1100$
the overdensity would need to be $\delta\gtrsim10^{-3}$.  But the typical
flucutations in the CMB are $\sim 2\times 10^{-5}$.  How can we possibly generate
structure by the present day?  We need dark matter!
\end{document}