\documentclass[]{article}
\usepackage[margin=1.0in]{geometry}
\usepackage{amssymb}

%title material
\title{Astronomy 400B Lecture 4: The Milky Way}
\author{Brant Robertson}
\date{January 5, 2015}


%include latex definitions

%average
\newcommand{\ave}[1]{\langle#1\rangle}
%radian
\newcommand{\rad}{\mathrm{rad}}

%astronomical unit
\newcommand{\AU}{\mathrm{AU}}

%centimeter
\newcommand{\cm}{\mathrm{cm}}

%meter
\newcommand{\m}{\mathrm{m}}

%kilometer
\newcommand{\km}{\mathrm{km}}

%parsec
\newcommand{\pc}{\mathrm{pc}}

%kiloparsec
\newcommand{\kpc}{\mathrm{kpc}}

%megaparsec
\newcommand{\Mpc}{\mathrm{Mpc}}

%gigaparsec
\newcommand{\Gpc}{\mathrm{Gpc}}

%light year
\newcommand{\ly}{\mathrm{ly}}

%second
\newcommand{\s}{\mathrm{s}}
\newcommand{\yr}{\mathrm{yr}}
\newcommand{\Gyr}{\mathrm{Gyr}}

%solar mass
\newcommand{\Msun}{M_{\odot}}

%grams
\newcommand{\g}{\mathrm{g}}

%erg
\newcommand{\erg}{\mathrm{erg}}

%solar luminosity
\newcommand{\Lsun}{L_{\odot}}

%jansky 
\newcommand{\Jy}{\mathrm{Jy}}

%flux density
\newcommand{\Fnu}{F_{\nu}}
\newcommand{\Flambda}{F_{\lambda}}
\newcommand{\Lnu}{L_{\nu}}

%hertz 
\newcommand{\Hz}{\mathrm{Hz}}

%angstrom
\newcommand{\Ang}{\mathrm{\r{A}}}

%solar radius
\newcommand{\Rsun}{R_{\odot}}

%stefan-boltzmann
\newcommand{\sigmaSB}{\sigma_{\mathrm{SB}}}

%boltzmann
\newcommand{\kB}{k_{\mathrm{B}}}

%kelvin
\newcommand{\K}{\mathrm{K}}

%T_bandpass
\newcommand{\TBP}{T_{\mathrm{BP}}}

%magnitude
\newcommand{\mg}{\mathrm{mag}}


%arcsecond
\newcommand{\arcsec}{\mathrm{arcsec}}

%begin the document
\begin{document}

%make the title, goes after document begins
\maketitle

Today's class will regard the Milky Way!

%first section
\section{Solar Neighborhood}

Stars in the local neighborhood provide a lot
of our information about the Milky Way and,
frankly, a lot of what we infer about any galaxy.
Learning about the stars requires knowing
something about their intrinsic luminosities
and masses. This requirement results in the
need to have robust distance estimates for
a large number of stars.

\subsection{Parallax}

{\it Trignometric parallax} provides a direct distance
estimate for very nearby stars.  Trignometric
parallax is the shift in the angular position on
the sky of a star as viewed from different
locations of the Earth's orbit around the Sun.

For an object at distance $d$, the parallax $p$
is
\begin{equation}
\frac{1~\AU}{d} = \tan p \approx p
\end{equation}
\noindent
The $\pc$ is the distance where the parallax
of a star would be $p=1~\arcsec$. The closest
star is Proxima Centuari, with $p=0.8~\arcsec$
and a $d=1.3~\pc$.

\subsection{Distance Modulus}

For nearby (Galactic) objects, we can relate the
difference between the apparent and absolute
magnitude of an object with its distance or
parallax through the {\it distance modulus}
equation
\begin{equation}
(m-M) = 5 \log \left(\frac{d}{10~\pc}\right) = 5 \log \left(\frac{0.1~\arcsec}{p}\right)
\end{equation}




\end{document}















