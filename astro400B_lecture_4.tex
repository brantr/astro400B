\documentclass[]{article}
\usepackage[margin=1.0in]{geometry}
\usepackage{amssymb}

%title material
\title{Astronomy 400B Lecture 4: The Milky Way}
\author{Brant Robertson}
\date{January 5, 2015}


%include latex definitions

%average
\newcommand{\ave}[1]{\langle#1\rangle}
%radian
\newcommand{\rad}{\mathrm{rad}}

%astronomical unit
\newcommand{\AU}{\mathrm{AU}}

%centimeter
\newcommand{\cm}{\mathrm{cm}}

%meter
\newcommand{\m}{\mathrm{m}}

%kilometer
\newcommand{\km}{\mathrm{km}}

%parsec
\newcommand{\pc}{\mathrm{pc}}

%kiloparsec
\newcommand{\kpc}{\mathrm{kpc}}

%megaparsec
\newcommand{\Mpc}{\mathrm{Mpc}}

%gigaparsec
\newcommand{\Gpc}{\mathrm{Gpc}}

%light year
\newcommand{\ly}{\mathrm{ly}}

%second
\newcommand{\s}{\mathrm{s}}
\newcommand{\yr}{\mathrm{yr}}
\newcommand{\Gyr}{\mathrm{Gyr}}

%solar mass
\newcommand{\Msun}{M_{\odot}}

%grams
\newcommand{\g}{\mathrm{g}}

%erg
\newcommand{\erg}{\mathrm{erg}}

%solar luminosity
\newcommand{\Lsun}{L_{\odot}}

%jansky 
\newcommand{\Jy}{\mathrm{Jy}}

%flux density
\newcommand{\Fnu}{F_{\nu}}
\newcommand{\Flambda}{F_{\lambda}}
\newcommand{\Lnu}{L_{\nu}}

%hertz 
\newcommand{\Hz}{\mathrm{Hz}}

%angstrom
\newcommand{\Ang}{\mathrm{\r{A}}}

%solar radius
\newcommand{\Rsun}{R_{\odot}}

%stefan-boltzmann
\newcommand{\sigmaSB}{\sigma_{\mathrm{SB}}}

%boltzmann
\newcommand{\kB}{k_{\mathrm{B}}}

%kelvin
\newcommand{\K}{\mathrm{K}}

%T_bandpass
\newcommand{\TBP}{T_{\mathrm{BP}}}

%magnitude
\newcommand{\mg}{\mathrm{mag}}


%arcsecond
\newcommand{\arcsec}{\mathrm{arcsec}}

%begin the document
\begin{document}

%make the title, goes after document begins
\maketitle

Today's class will regard the Milky Way!

%first section
\section{Solar Neighborhood}

Stars in the local neighborhood provide a lot
of our information about the Milky Way and,
frankly, a lot of what we infer about any galaxy.
Learning about the stars requires knowing
something about their intrinsic luminosities
and masses. This requirement results in the
need to have robust distance estimates for
a large number of stars.

\subsection{Parallax}

{\it Trignometric parallax} provides a direct distance
estimate for very nearby stars.  Trignometric
parallax is the shift in the angular position on
the sky of a star as viewed from different
locations of the Earth's orbit around the Sun.

For an object at distance $d$, the parallax $p$
is
\begin{equation}
\frac{1~\AU}{d} = \tan p \approx p
\end{equation}
\noindent
The $\pc$ is the distance where the parallax
of a star would be $p=1~\arcsec$. The closest
star is Proxima Centuari, with $p=0.8~\arcsec$
and a $d=1.3~\pc$.

\subsection{Distance Modulus}

For nearby (Galactic) objects, we can relate the
difference between the apparent and absolute
magnitude of an object with its distance or
parallax through the {\it distance modulus}
equation
\begin{equation}
m-M = 5 \log \left(\frac{d}{10~\pc}\right) = 5 \log \left(\frac{0.1~\arcsec}{p}\right).
\end{equation}

\subsection{Luminosity and Mass Functions}
The luminosity and mass functions of stars enable
us to learn about the star formation history of
the Galaxy and about the star formation process
itself.

{\bf See Figure 2.3 of Sparke and Gallagher}

The luminosity and mass functions reflect combinations
of the numbers of stars per unit mass and the
mass to light ratio of stars as a function of mass.

The luminosity function is
\begin{equation}
\Phi(x) = \frac{\mathrm{number~of~stars~per~unit~magnitude}}{\mathrm{detectable~volume}}
\end{equation}

\subsubsection{Initial Luminosity Function}
Stars have a finite lifetime, so the observed luminosity
function defined above is a time-evolved distribution
where some fraction of stars have left the main sequence
after their MS lifetime $\tau_{\mathrm{MS}}$.
If the star formation rate of the disk has been roughly
constant, the initial luminosity function
is related to the observed luminosity function as
\begin{eqnarray}
\Psi(M_{V}) &=& \Phi_{\mathrm{MS}}(M_V)~~\mathrm{for~}\tau_{\mathrm{MS}}(M_V)\ge\tau_{\mathrm{gal}} \\
&=& \Phi_{\mathrm{MS}}(M_V)\times\frac{\tau_{\mathrm{gal}}}{\tau_{\mathrm{MS}}(M_V)}~~\mathrm{when~}\tau_{\mathrm{MS}}(M_V)<\tau_{\mathrm{gal}} \nonumber
\end{eqnarray}
\noindent
where $\tau_\mathrm{gal}\approx8-10~\Gyr$ is the star formation timescale
of the disk.

\subsubsection{Initial Mass Function}

The stellar initial mass function, or IMF for short,
is one of the most important distributions in astronomy
and represents an interesting combination of physics in
molecular gas, cooling, and gravity. The IMF represents
the initial number of stars per unit mass that forms
in a typical volume.  We write
\begin{equation}
\xi(M) dM = \xi_0 (M/\Msun)^{-2.35}\frac{dM}{\Msun}
\end{equation}
\noindent
The power-law slope $-2.35$ is called the {\it Salpeter}
slope.

{\bf See Figure 2.5 of Sparke and Gallagher}

\section{Stars in the Galaxy}

Understanding the structure of the Galaxy is
tightly connected to understanding the distances
to stars at all distances.

\subsection{Distances from Motions}

The tangential and radial velocities of stars
can inform us about their distances.  The radial
(line-of-sight) velocities can be measured via
the Doppler shift we discussed previously. The
tangential velocity is related to the apparent
proper motion $\mu$ on the sky of an object and
its distance $d$.  If the relation between
tangential and radial velocities is known, then
by measuring the radial velocity and proper motion
we can find an object's distance.

The tangential velocity is
\begin{equation}
v_{t} = \mu~(\mathrm{radians/time})\times d,~or~\mu~(0.001\arcsec/\yr)=\frac{v_t~(\km~\s^{-1})}{4.74\times d~(\kpc)}
\end{equation}

We can identify stars that orbit the supermassive black hole
at the center of the Galaxy. The orbits of these stars allow 
us to measure the mass of the Galactic SMBH, and since we can
compute their physical tangential velocity their distances can
be inferred. It turns out we are $d=8.46\pm0.4~\kpc$ away from
the Galactic center.

\end{document}















