\documentclass[]{article}
\usepackage[margin=1.0in]{geometry}
\usepackage{amssymb}

%title material
\title{Astronomy 400B Lecture 4: The Milky Way}
\author{Brant Robertson}
\date{January 5, 2015}


%include latex definitions

%average
\newcommand{\ave}[1]{\langle#1\rangle}
%radian
\newcommand{\rad}{\mathrm{rad}}

%astronomical unit
\newcommand{\AU}{\mathrm{AU}}

%centimeter
\newcommand{\cm}{\mathrm{cm}}

%meter
\newcommand{\m}{\mathrm{m}}

%kilometer
\newcommand{\km}{\mathrm{km}}

%parsec
\newcommand{\pc}{\mathrm{pc}}

%kiloparsec
\newcommand{\kpc}{\mathrm{kpc}}

%megaparsec
\newcommand{\Mpc}{\mathrm{Mpc}}

%gigaparsec
\newcommand{\Gpc}{\mathrm{Gpc}}

%light year
\newcommand{\ly}{\mathrm{ly}}

%second
\newcommand{\s}{\mathrm{s}}
\newcommand{\yr}{\mathrm{yr}}
\newcommand{\Gyr}{\mathrm{Gyr}}

%solar mass
\newcommand{\Msun}{M_{\odot}}

%grams
\newcommand{\g}{\mathrm{g}}

%erg
\newcommand{\erg}{\mathrm{erg}}

%solar luminosity
\newcommand{\Lsun}{L_{\odot}}

%jansky 
\newcommand{\Jy}{\mathrm{Jy}}

%flux density
\newcommand{\Fnu}{F_{\nu}}
\newcommand{\Flambda}{F_{\lambda}}
\newcommand{\Lnu}{L_{\nu}}

%hertz 
\newcommand{\Hz}{\mathrm{Hz}}

%angstrom
\newcommand{\Ang}{\mathrm{\r{A}}}

%solar radius
\newcommand{\Rsun}{R_{\odot}}

%stefan-boltzmann
\newcommand{\sigmaSB}{\sigma_{\mathrm{SB}}}

%boltzmann
\newcommand{\kB}{k_{\mathrm{B}}}

%kelvin
\newcommand{\K}{\mathrm{K}}

%T_bandpass
\newcommand{\TBP}{T_{\mathrm{BP}}}

%magnitude
\newcommand{\mg}{\mathrm{mag}}


%arcsecond
\newcommand{\arcsec}{\mathrm{arcsec}}

%begin the document
\begin{document}

%make the title, goes after document begins
\maketitle

Today's class will regard the Milky Way!

%first section
\section{Solar Neighborhood}

Stars in the local neighborhood provide a lot
of our information about the Milky Way and,
frankly, a lot of what we infer about any galaxy.
Learning about the stars requires knowing
something about their intrinsic luminosities
and masses. This requirement results in the
need to have robust distance estimates for
a large number of stars.

\subsection{Parallax}

{\it Trignometric parallax} provides a direct distance
estimate for very nearby stars.  Trignometric
parallax is the shift in the angular position on
the sky of a star as viewed from different
locations of the Earth's orbit around the Sun.

For an object at distance $d$, the parallax $p$
is
\begin{equation}
\frac{1~\AU}{d} = \tan p \approx p
\end{equation}
\noindent
The $\pc$ is the distance where the parallax
of a star would be $p=1~\arcsec$. The closest
star is Proxima Centuari, with $p=0.8~\arcsec$
and a $d=1.3~\pc$.

\subsection{Distance Modulus}

For nearby (Galactic) objects, we can relate the
difference between the apparent and absolute
magnitude of an object with its distance or
parallax through the {\it distance modulus}
equation
\begin{equation}
m-M = 5 \log \left(\frac{d}{10~\pc}\right) = 5 \log \left(\frac{0.1~\arcsec}{p}\right).
\end{equation}

\subsection{Luminosity and Mass Functions}
The luminosity and mass functions of stars enable
us to learn about the star formation history of
the Galaxy and about the star formation process
itself.

{\bf See Figure 2.3 of Sparke and Gallagher}

The luminosity and mass functions reflect combinations
of the numbers of stars per unit mass and the
mass to light ratio of stars as a function of mass.

The luminosity function is
\begin{equation}
\Phi(x) = \frac{\mathrm{number~of~stars~per~unit~magnitude}}{\mathrm{detectable~volume}}
\end{equation}

\subsubsection{Initial Luminosity Function}
Stars have a finite lifetime, so the observed luminosity
function defined above is a time-evolved distribution
where some fraction of stars have left the main sequence
after their MS lifetime $\tau_{\mathrm{MS}}$.
If the star formation rate of the disk has been roughly
constant, the initial luminosity function
is related to the observed luminosity function as
\begin{eqnarray}
\Psi(M_{V}) &=& \Phi_{\mathrm{MS}}(M_V)~~\mathrm{for~}\tau_{\mathrm{MS}}(M_V)\ge\tau_{\mathrm{gal}} \\
&=& \Phi_{\mathrm{MS}}(M_V)\times\frac{\tau_{\mathrm{gal}}}{\tau_{\mathrm{MS}}(M_V)}~~\mathrm{when~}\tau_{\mathrm{MS}}(M_V)<\tau_{\mathrm{gal}} \nonumber
\end{eqnarray}
\noindent
where $\tau_\mathrm{gal}\approx8-10~\Gyr$ is the star formation timescale
of the disk.

\subsubsection{Initial Mass Function}

The stellar initial mass function, or IMF for short,
is one of the most important distributions in astronomy
and represents an interesting combination of physics in
molecular gas, cooling, and gravity. The IMF represents
the initial number of stars per unit mass that forms
in a typical volume.  We write
\begin{equation}
\xi(M) dM = \xi_0 (M/\Msun)^{-2.35}\frac{dM}{\Msun}
\end{equation}
\noindent
The power-law slope $-2.35$ is called the {\it Salpeter}
slope.

{\bf See Figure 2.5 of Sparke and Gallagher}

\section{Stars in the Galaxy}

Understanding the structure of the Galaxy is
tightly connected to understanding the distances
to stars at all distances.

\subsection{Distances from Motions}

The tangential and radial velocities of stars
can inform us about their distances.  The radial
(line-of-sight) velocities can be measured via
the Doppler shift we discussed previously. The
tangential velocity is related to the apparent
proper motion $\mu$ on the sky of an object and
its distance $d$.  If the relation between
tangential and radial velocities is known, then
by measuring the radial velocity and proper motion
we can find an object's distance.

The tangential velocity is
\begin{equation}
v_{t} = \mu~(\mathrm{radians/time})\times d,~or~\mu~(0.001\arcsec/\yr)=\frac{v_t~(\km~\s^{-1})}{4.74\times d~(\kpc)}
\end{equation}

We can identify stars that orbit the supermassive black hole
at the center of the Galaxy. The orbits of these stars allow 
us to measure the mass of the Galactic SMBH, and since we can
compute their physical tangential velocity their distances can
be inferred. It turns out we are $d=8.46\pm0.4~\kpc$ away from
the Galactic center.

\subsection{Spectroscopic and Photometric Parallax, and Galactic Structure}

The shapes and depths of spectral lines of stars
depend on properties that correlate with their
mass and luminosity. Given a measurement of the
stellar flux, the absolute distances to stars
can be estimated from their spectra. We call
this technique the {\it spectroscopic parallax}.
Similarly, based on the color of star one may
infer a temperature and then use other information
to classify the star as a dwarf or giant and
constrain the mass, intrinsic luminosity, and
distance. This
method, called the {\it photometric parallax},
is less reliable but still widely employed.

Much of what we know about the structure of
the stellar disk of the Milky Way is inferred
using these methods. The proper motions of
many stars in the Milky Way are too small
to be reliably measured, so the only
distance estimates that we have come from
either spectra or photometry. Obviously, 
photometry is available for many, many more stars!

The shape of the stellar disk can be expressed
in terms of the radius $R$, vertical height
$z$, and spectral type $S$ as
\begin{equation}
n(R,z,S) = n(0,0,S) \exp\left[-R/h_{R}(S)\right]\exp\left[-|z|/h_{z}(S)\right].
\end{equation}
\noindent
Here, $h_R$ is the disk scale length and $h_z$ is the scale height.
Sensibly, the scale length and height do depend strongly on the stellar
type.  The scale height of older and lower mass stars is larger 
($\sim350~\pc$) than for 
young and massive stars ($\sim200~\pc$. HI and molecular gas have even 
smaller scale heights.

{\bf See Figure 2.8 of Sparke and Gallagher.}
{\bf See Table 2.1 of Sparke and Gallagher.}

\subsection{Star Formation Rate of the Milky Way}

We can use the stellar luminosity of the disk and
a typical mass to light ratio to infer the disk stellar
mass. For $L_{\mathrm{disk}}\sim 1.5\times10^{10}\Lsun$
and $M/L\sim2$ for the most abundant stars, we find
a stellar mass of $M_{\star}\sim3 \times 10^{10}\Msun$.
The disk is about $\sim10~\Gyr$ old, and stars loose
about half their mass in winds over this time when
averaged over the typical IMF. That means the star
formation rate in the disk is $\sim(3-5)~\Msun~\yr^{-1}$.
The cold gas in the Milky Way disk can sustain this
star formation rate for only a few gigayears.

\subsection{Velocity Dispersion}

The non-zero height of the disk implies that there
has to be kinematical support of disk stars relative to
the disk gravitational potential. The vertical velocity
dispersion is
\begin{equation}
\sigma_z^2 \equiv \left\langle v_z^2 - \ave{v_z}^2\right\rangle.
\end{equation}
The velocity dispersion of stars tends to increase with their
age (in conjunction with their scale height). This fattening
of the disk results from scattering processes that arise from
the non-uniformity of the disk. There are velocity dispersions
in the other polar directions as well, typically with
$\sigma_R \ge \sigma_\phi \ge \sigma_z$.

We can also estimate the {\it asymmetric drift} of a star, which
reflects the orbital velocity lag of the star relative to a 
circular orbit at the Sun's position. This property will be
discussed more later.

\section{Stellar Clusters}

Stars in galaxies typically form in {\it clusters}, collections
of stars that are gravitationally bound or structurally associated
(loosely bound).
Stars in a cluster are roughly co-eval, so there is a lot we
can learn from them!

The Hertzsprung-Russell diagram of the clusters reflect an
{\it isochrone}, a line corresponding to a single age stellar
population. The location of the {\it main sequence turn-off}
indicates the age of the stellar population.

{\bf See Figure 2.12 of Sparke and Gallagher.}

\subsection{Open Clusters}

There are $>1000$ collections of (usually) recently-formed stars
called {\it open clusters} consisting of stars with
ages $\lesssim1~\Gyr$, total luminosities of 
$L\sim100-10,000\Lsun$, and several hundred total stars.

{\bf See Table 2.2 of Sparke and Gallagher.}


\subsection{Globular Clusters}

There are much denser, more massive, and typically older
collections of stars called {\it globular clusters}.
Globulars may have luminosities of $10^5-10^6~\sim\Lsun$ 
and $10^{5}-10^6$ stars. Most of the globulars are more
than $10\sim\Gyr$ old and very metal poor.

{\bf See Table 2.3 of Sparke and Gallagher.}

{\bf See Figure 2.14 of Sparke and Gallagher.}


\section{Galactic Rotation}

The disk of the Milky Way rotates, and it rotates {\it differentially}
such that the stars in the inner parts of the galaxy orbit the
center of the galaxy more rapidly than stars in the outer disk do.
Differential rotation by looking at the proper motions of disk
stars, and seeing that inner stars were passing us while
outer stars were trailing. For stars near the Sun, proper
motions showed a $\cos(2l)$ dependence on Galactic longitude that
can be understood via a differentially rotating disk.

\subsection{Local Standard of Rest}

The sun lies about 15$\pc$ out of the disk plane, and the orbit
of the Sun about the GC is not purely circular. We define the
{\it local standard of rest} by the average motions of stars
near the Sun correcting for asymmetric drift. Relative to the
standard, the Sun is moving toward the GC, leading the 
rotation, and moving up relative to the disk plane.

In 1985, the IAU adopted a LSR radius of $R_{0}=8.5~\kpc$ from
the GC and $V_{0}=220~\km~\s^{-1}$ for its circular velocity.
Current estimates are closer to $R_0\sim8~\kpc$ and $V_0 \sim 200$.

\subsection{Measuring Galactic Rotation}

For disk stars, we can determine their radial velocity
relative to the Sun assuming circular orbits.

{\bf See Figure 2.19 of Sparke and Gallagher}.

The Sun is at $R_0$, $V_0$, while the star has a radius $R$
and orbital speed $V(R)$.  The radial velocity relative to
the Sun is
\begin{equation}
V_{r} = V\cos\alpha - V_0 \sin l.
\end{equation}
\noindent
But $\sin l/R - \sin(\alpha + \pi/2)/R_0$, so
\begin{equation}
V_r = R_0 \sin l \left(\frac{V}{R} - \frac{V_0}{R_0}\right).
\end{equation}

For solid body rotation, $V(R)=constant$ and the relative 
radial velocity would be zero. Instead, $V/R$ decreases
with $R$.

{\bf See Figure 2.20 of Sparke and Gallagher}

As a function of Galactic longitude, we have
\begin{itemize}
\item For $0<l<90$, $V_r$ is positive for nearby objects,
and negative for distant objects on the other side of the
Galaxy with $R>R_0$.
\item For $90<l<180$, $V_r$ is always negative.
\item For $180<l<270$, $V_r$ is always positive.
\item For $270<l<360$, the first quadrant pattern is repeated with opposite sign.
\end{itemize}

\subsection{Oort Constants}
When an object is near the Sun, $R\approx R_0 - d \cos l$ and
we can approximate the radial velocity as
\begin{equation}
V_r \approx d \sin(2l) A
\end{equation}
\noindent
where
\begin{equation}
A \equiv \left[-\frac{R}{2} \frac{dV}{dR}\right]_{R_0}
\end{equation}
\noindent
is called the {\it Oort} constant after Jan Oort. The $A$ constant
measures the local shearing motions in the disk.
For the Milky Way, $A = 14.8\pm0.8~\km~s^{-1}~\kpc^{-1}$.

The tangential velocity of a star relative to the Sun is
\begin{equation}
V_t = V \sin \alpha - V_0 \cos l
\end{equation}
\noindent
But $R_0 \cos l = R\sin\alpha + d$, and we have
\begin{equation}
V_t = R_0 \cos l \left(\frac{V}{R} - \frac{V_0}{R_0}\right) - V\frac{d}{R}
\end{equation}
\noindent
Near the Sun, $R_0-R \approx d \cos l$, so
\begin{equation}
V_t \approx d [A \cos(2l) + B]
\end{equation}
\noindent
where
\begin{equation}
B = -\frac{1}{2}\left[\frac{1}{R}\frac{d (RV)}{dR}\right]
\end{equation}
\noindent
where $B$ is the second Oort constant that measures vorticity.
$B = -12.4\pm0.6~\km~\s^{-1}~\kpc^{-1}$ in the Solar neighborhood.

\subsection{Tangent Point Method}

In the inner Galaxy $R<R_0$, the {\it tangent point method} can be
used to find the rotation curve.  Since $V/R$ decreases with
radius, if we look at regions near the Galactic center in projection
($0<l<90$), then the radial speed $V_r(l,R)$ is maximized at the
tangent point where the
line of sight is closest to the GC.  There, we have
\begin{equation}
R = R_0 \sin l
\end{equation}
\noindent
and
\begin{equation}
V(R) = V_r + V_0 \sin l
\end{equation}

The galactic rotation curve can then be inferred from
the maximum velocity in Figure 2.20 at each $l$.

{\bf See Figure 2.20 of Sparke and Gallagher}

\subsection{Outer Galaxy}

Measuring the rotation of the outer galaxy is done by
determining distances via spectroscopic or photometric
parallax, and then measuring $V_r$ via emission lines.
We think $V(R)$ is either flat or increasing in the
outer galaxy.


\section{Dark Matter}

For a spherical mass distribution, the velocity of
a circular orbit related to the interior mass by
\begin{equation}
V^2 = \frac{GM(<R)}{R}
\end{equation}
\noindent
If the circular velocity is flat in the outer 
galaxy, we would infer that
\begin{equation}
M(<R) \propto R
\end{equation}
\noindent
Clearly, the exponential disk cannot supply
this mass growth with radius.
But we know how to relate $M(<R)$ to a density
profile through the integral
\begin{equation}
M(<R) = 4\pi\int_0^{R}\rho(r)r^{2}dr
\end{equation}
\noindent
Further if we assume $\rho(r)\propto r^{\alpha}$,
then
\begin{equation}
M(<R) \propto R \propto R^{3 + \alpha}
\end{equation}
\noindent
So we infer that a spherical density profile would 
be declining at $\rho(r)\propto r^{-2}$ in the
regime where the mass was growing linearly and
the circular velocity is flat.

That mass grows with radius implies there is a {\it lot}
of mass in the exterior of the galaxy.

\section{The Gaseous Disk}

The distribution of gas in the galaxy can be determined by
measuring the line of sight velocity of emission lines and
then assuming the gas is on a circular orbit.  We call this
approach determining the {\it kinematical distance} to the
gas. This method can be used on both HI and $H_2$ (well, CO)
to determine the relative distribution of neutral and molecular
gas in the Galaxy.

{\bf See Figure 2.22 of Sparke and Gallagher}

The MW has about $6\times10^{9}\Msun$ of HI and about 
$3\times10^{9}\Msun$ of $H_2$. The molecular hydrogen
is mostly in the solar circle, but the HI extends much
further.  The molecular disk is about $80\pc$ thick
near the Sun, while the HI is about twice as extended.

There is some HI gas outside of the midplane, and much of
that is moving toward us. There are HI gas clouds called
{\it high velocity clouds} because they are moving
at $\sim100\km~\s^{-1}$ toward us.  We don't know what
causes HVCs.

{\bf See Figure 2.22 of Sparke and Gallagher}

Gas above $\sim1\kpc$ is mostly warm or hot ionized media.
This gas absorbs CIV, and the hotter gas emits OVI.

In the inner $3\kpc$ of the galaxy, interior to a molecular
ring, the surface density of galaxy drops again.  In the
inner $200\pc$ of the Galactic Bulge is a gas rich region
with about one hundred million solar masses of gas.

The interstellar medium is comprised of a variety of
phases that are mixed on scales below $\sim1\kpc$:

\begin{itemize}
\item There are a few thousand gian molecular clouds
with about $20\pc$ sizes, $M>10^{5}\Msun$, and
$n\sim200-10^{4}~\cm^{-3}$.
\item The GMCs are surrounded by neutral hydrogen
with $n\sim25$ and $T<80\K$. Near the sun HI can
have $T\sim8000\K$ and $n\sim0.3\cm^{-3}$.
\item Hot diffuse plasmas with $n\sim0.002\cm^{-3}$
and $T\sim10^{6}\K$ surrounds the HI.
\item Around hot stars HII regions can form, with
photodissociation regions located at the boundary
with surrounding dense gas.
\end{itemize}

{\bf See Table 2.4 of Sparke and Gallagher}

GMCs are turbulent, and have supersonic line widths.

\subsection{Dust}

There is a lot of dust in Milky Way gas and in the
gas of many other galaxies.  The dust absorbs about
half the optical and UV light of the galaxy.  There
is about one dust grain per trillion hydrogen atoms,
and the grains are typically $\lesssim0.3$ microns
in size. About $10-20\%$ of the dust is in small
polycyclic aromatic hydrocarbons (PAH) that absorbs
light and emits photoelectrons, which thermalize
and heat the gas (forming the prime heat source for
atomic HI).  Hot dust is at $\sim100K$ and emits at
$30$ microns, while cold dust is at $30\K$ and
emits at $100$ microns.


{\bf See Figure 2.24 of Sparke and Gallagher}

\section{Recombination and the Ionization State of the ISM}

Ionized hydrogen (proton plus electron) can undergo a recombination
to form neutral hydrogen. The rate of this process depends on the
properties of the ionized gas.  We can write the recombination rate
$dn_e/dt$ that changes the number density $n_e$ of free electrons as
\begin{equation}
\frac{d n_{e}}{d t} = n_{e}^2 \alpha(T_e)
\end{equation}
\noindent
where $T_e$ is the temperature of the gas and $\alpha(T_e)$ is 
the ``recombination coefficient''
\begin{equation}
\alpha(T_e) \approx 2 \times 10^{-13} \left( \frac{T_e}{10^4~\K}\right)^{-3/4}~\cm^{3}~\s^{-1}.
\end{equation}
\noindent
There's a lot hidden in the coefficient!

We can also define the recombination time $t_{rec}$ that characterizes
the timescale over which the ionized gas will significantly increase
its neutrality.  The recombination time is given by
\begin{equation}
t_{rec} = \frac{n_e}{|dn_e/dt|} = \frac{1}{n_e \alpha(T_e)} \approx 1500~\yr \times \left(\frac{T_e}{10^{4}K}\right)^{3/4}\left(\frac{100~\cm^{-3}}{n_e}\right)
\end{equation}

In HII regions, $t_{rec}$ is a few thousand years.  In the diffuse ionized ISM, the recombination time is a few million years.

\subsection{Cooling Rate and Time}

Gas has an internal thermal energy of $E \propto nT$. If the gas radiates this thermal energy with an energy $L$, the
cooling timescale for the gas is $t_{cool} \propto nT/L$. For optically thin gas, we can write
\begin{equation}
L = n^{2} \Lambda(T)
\end{equation}
\noindent
and
\begin{equation}
t_{cool}  \propto T/[n\Lambda(T)].
\end{equation}
The quantity $\Lambda(T)$ is called the ``cooling function'' and depends only on temperature.

{\bf See Table 2.5 and Figure 2.25 of Sparke and Gallagher.}

At high temperatures $T>10^{7}~\K$, free-free cooling dominates such that $\Lambda(T) \propto T^{1/2}$
and $t_{cool} \propto \sqrt{T}/n$.

\section{Jeans Mass}

Since gas can cool, eventually the thermal random motions of the gas will be unable to support
it against its own self-gravity. The condition for this self-gravitational collapse is called the
Jeans criterion. A gas cloud of density $\rho$ and temperature $T$ will collapse if its 
diameter is
\begin{equation}
\lambda_J = c_s \sqrt{\frac{\pi}{G\rho}}
\end{equation}
\noindent
where $c_s$ is the sound speed of the gas. We can express the sound speed
in terms of the temperature and mean molecular weight of the gas
\begin{equation}
c_s^{2} = \kB T / \mu m_{H}
\end{equation}
\noindent
where $m_H$ is the mass of the hydrogen atom.  The mass within the Jeans diameter
\begin{equation}
M_J \equiv \frac{\pi}{6} \lambda_J^3 \rho = \left(\frac{1}{\mu m_H}\right)^2 \left(\frac{\kB T}{G}\right)^{3/2} \left(\frac{4\pi n}{3}\right)^{-1/2} \frac{\pi^3}{3\sqrt{3}}
\end{equation}
\begin{equation}
M_j \approx 20 \left(\frac{T}{10\K}\right)^{3/2} \left(\frac{100~\cm^{-3}}{n}\right)^{1/2}~\Msun.
\end{equation}
\noindent
If the cloud does collapse, it does so on the free fall time
\begin{equation}
t_{ff} = \sqrt{\frac{1}{G\rho}} \approx \frac{10^8}{\sqrt{n_H}}~\yr.
\end{equation}
\noindent
If the cooling time is much less than the free fall time, the gas cloud will fragment in to smaller clouds during the collapse.
Something must moderate this process because the star formation rate you would infer from converting the molecular gas into stars on a free fall time is many times larger than the observed star formation rate.

\end{document}













