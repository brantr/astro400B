\documentclass[]{article}
\usepackage[margin=1.0in]{geometry}
\usepackage{amssymb}

%title material
\title{Astronomy 400B Lecture 4: The Milky Way}
\author{Brant Robertson}
\date{January 5, 2015}


%include latex definitions

%average
\newcommand{\ave}[1]{\langle#1\rangle}
%radian
\newcommand{\rad}{\mathrm{rad}}

%astronomical unit
\newcommand{\AU}{\mathrm{AU}}

%centimeter
\newcommand{\cm}{\mathrm{cm}}

%meter
\newcommand{\m}{\mathrm{m}}

%kilometer
\newcommand{\km}{\mathrm{km}}

%parsec
\newcommand{\pc}{\mathrm{pc}}

%kiloparsec
\newcommand{\kpc}{\mathrm{kpc}}

%megaparsec
\newcommand{\Mpc}{\mathrm{Mpc}}

%gigaparsec
\newcommand{\Gpc}{\mathrm{Gpc}}

%light year
\newcommand{\ly}{\mathrm{ly}}

%second
\newcommand{\s}{\mathrm{s}}
\newcommand{\yr}{\mathrm{yr}}
\newcommand{\Gyr}{\mathrm{Gyr}}

%solar mass
\newcommand{\Msun}{M_{\odot}}

%grams
\newcommand{\g}{\mathrm{g}}

%erg
\newcommand{\erg}{\mathrm{erg}}

%solar luminosity
\newcommand{\Lsun}{L_{\odot}}

%jansky 
\newcommand{\Jy}{\mathrm{Jy}}

%flux density
\newcommand{\Fnu}{F_{\nu}}
\newcommand{\Flambda}{F_{\lambda}}
\newcommand{\Lnu}{L_{\nu}}

%hertz 
\newcommand{\Hz}{\mathrm{Hz}}

%angstrom
\newcommand{\Ang}{\mathrm{\r{A}}}

%solar radius
\newcommand{\Rsun}{R_{\odot}}

%stefan-boltzmann
\newcommand{\sigmaSB}{\sigma_{\mathrm{SB}}}

%boltzmann
\newcommand{\kB}{k_{\mathrm{B}}}

%kelvin
\newcommand{\K}{\mathrm{K}}

%T_bandpass
\newcommand{\TBP}{T_{\mathrm{BP}}}

%magnitude
\newcommand{\mg}{\mathrm{mag}}


%arcsecond
\newcommand{\arcsec}{\mathrm{arcsec}}

%begin the document
\begin{document}

%make the title, goes after document begins
\maketitle

Today's class will regard the Milky Way!

%first section
\section{Solar Neighborhood}

Stars in the local neighborhood provide a lot
of our information about the Milky Way and,
frankly, a lot of what we infer about any galaxy.
Learning about the stars requires knowing
something about their intrinsic luminosities
and masses. This requirement results in the
need to have robust distance estimates for
a large number of stars.

\subsection{Parallax}

{\it Trignometric parallax} provides a direct distance
estimate for very nearby stars.  Trignometric
parallax is the shift in the angular position on
the sky of a star as viewed from different
locations of the Earth's orbit around the Sun.

For an object at distance $d$, the parallax $p$
is
\begin{equation}
\frac{1~\AU}{d} = \tan p \approx p
\end{equation}
\noindent
The $\pc$ is the distance where the parallax
of a star would be $p=1~\arcsec$. The closest
star is Proxima Centuari, with $p=0.8~\arcsec$
and a $d=1.3~\pc$.

\subsection{Distance Modulus}

For nearby (Galactic) objects, we can relate the
difference between the apparent and absolute
magnitude of an object with its distance or
parallax through the {\it distance modulus}
equation
\begin{equation}
m-M = 5 \log \left(\frac{d}{10~\pc}\right) = 5 \log \left(\frac{0.1~\arcsec}{p}\right).
\end{equation}

\subsection{Luminosity and Mass Functions}
The luminosity and mass functions of stars enable
us to learn about the star formation history of
the Galaxy and about the star formation process
itself.

{\bf See Figure 2.3 of Sparke and Gallagher}

The luminosity and mass functions reflect combinations
of the numbers of stars per unit mass and the
mass to light ratio of stars as a function of mass.

The luminosity function is
\begin{equation}
\Phi(x) = \frac{\mathrm{number~of~stars~per~unit~magnitude}}{\mathrm{detectable~volume}}
\end{equation}

\subsubsection{Initial Luminosity Function}
Stars have a finite lifetime, so the observed luminosity
function defined above is a time-evolved distribution
where some fraction of stars have left the main sequence
after their MS lifetime $\tau_{\mathrm{MS}}$.
If the star formation rate of the disk has been roughly
constant, the initial luminosity function
is related to the observed luminosity function as
\begin{eqnarray}
\Psi(M_{V}) &=& \Phi_{\mathrm{MS}}(M_V)~~\mathrm{for~}\tau_{\mathrm{MS}}(M_V)\ge\tau_{\mathrm{gal}} \\
&=& \Phi_{\mathrm{MS}}(M_V)\times\frac{\tau_{\mathrm{gal}}}{\tau_{\mathrm{MS}}(M_V)}~~\mathrm{when~}\tau_{\mathrm{MS}}(M_V)<\tau_{\mathrm{gal}} \nonumber
\end{eqnarray}
\noindent
where $\tau_\mathrm{gal}\approx8-10~\Gyr$ is the star formation timescale
of the disk.

\subsubsection{Initial Mass Function}

The stellar initial mass function, or IMF for short,
is one of the most important distributions in astronomy
and represents an interesting combination of physics in
molecular gas, cooling, and gravity. The IMF represents
the initial number of stars per unit mass that forms
in a typical volume.  We write
\begin{equation}
\xi(M) dM = \xi_0 (M/\Msun)^{-2.35}\frac{dM}{\Msun}
\end{equation}
\noindent
The power-law slope $-2.35$ is called the {\it Salpeter}
slope.

{\bf See Figure 2.5 of Sparke and Gallagher}

\section{Stars in the Galaxy}

Understanding the structure of the Galaxy is
tightly connected to understanding the distances
to stars at all distances.

\subsection{Distances from Motions}

The tangential and radial velocities of stars
can inform us about their distances.  The radial
(line-of-sight) velocities can be measured via
the Doppler shift we discussed previously. The
tangential velocity is related to the apparent
proper motion $\mu$ on the sky of an object and
its distance $d$.  If the relation between
tangential and radial velocities is known, then
by measuring the radial velocity and proper motion
we can find an object's distance.

The tangential velocity is
\begin{equation}
v_{t} = \mu~(\mathrm{radians/time})\times d,~or~\mu~(0.001\arcsec/\yr)=\frac{v_t~(\km~\s^{-1})}{4.74\times d~(\kpc)}
\end{equation}

We can identify stars that orbit the supermassive black hole
at the center of the Galaxy. The orbits of these stars allow 
us to measure the mass of the Galactic SMBH, and since we can
compute their physical tangential velocity their distances can
be inferred. It turns out we are $d=8.46\pm0.4~\kpc$ away from
the Galactic center.

\subsection{Spectroscopic and Photometric Parallax, and Galactic Structure}

The shapes and depths of spectral lines of stars
depend on properties that correlate with their
mass and luminosity. Given a measurement of the
stellar flux, the absolute distances to stars
can be estimated from their spectra. We call
this technique the {\it spectroscopic parallax}.
Similarly, based on the color of star one may
infer a temperature and then use other information
to classify the star as a dwarf or giant and
constrain the mass, intrinsic luminosity, and
distance. This
method, called the {\it photometric parallax},
is less reliable but still widely employed.

Much of what we know about the structure of
the stellar disk of the Milky Way is inferred
using these methods. The proper motions of
many stars in the Milky Way are too small
to be reliably measured, so the only
distance estimates that we have come from
either spectra or photometry. Obviously, 
photometry is available for many, many more stars!

The shape of the stellar disk can be expressed
in terms of the radius $R$, vertical height
$z$, and spectral type $S$ as
\begin{equation}
n(R,z,S) = n(0,0,S) \exp\left[-R/h_{R}(S)\right]\exp\left[-|z|/h_{z}(S)\right].
\end{equation}
\noindent
Here, $h_R$ is the disk scale length and $h_z$ is the scale height.
Sensibly, the scale length and height do depend strongly on the stellar
type.  The scale height of older and lower mass stars is larger 
($\sim350~\pc$) than for 
young and massive stars ($\sim200~\pc$. HI and molecular gas have even 
smaller scale heights.

{\bf See Figure 2.8 of Sparke and Gallagher.}
{\bf See Table 2.1 of Sparke and Gallagher.}

\subsection{Star Formation Rate of the Milky Way}

We can use the stellar luminosity of the disk and
a typical mass to light ratio to infer the disk stellar
mass. For $L_{\mathrm{disk}}\sim 1.5\times10^{10}\Lsun$
and $M/L\sim2$ for the most abundant stars, we find
a stellar mass of $M_{\star}\sim3 \times 10^{10}\Msun$.
The disk is about $\sim10~\Gyr$ old, and stars loose
about half their mass in winds over this time when
averaged over the typical IMF. That means the star
formation rate in the disk is $\sim(3-5)~\Msun~\yr^{-1}$.
The cold gas in the Milky Way disk can sustain this
star formation rate for only a few gigayears.

\subsection{Velocity Dispersion}

The non-zero height of the disk implies that there
has to be kinematical support of disk stars relative to
the disk gravitational potential. The vertical velocity
dispersion is
\begin{equation}
\sigma_z^2 \equiv \left\langle v_z^2 - \ave{v_z}^2\right\rangle.
\end{equation}
The velocity dispersion of stars tends to increase with their
age (in conjunction with their scale height). This fattening
of the disk results from scattering processes that arise from
the non-uniformity of the disk. There are velocity dispersions
in the other polar directions as well, typically with
$\sigma_R \ge \sigma_\phi \ge \sigma_z$.

We can also estimate the {\it asymmetric drift} of a star, which
reflects the orbital velocity lag of the star relative to a 
circular orbit at the Sun's position. This property will be
discussed more later.

\section{Stellar Clusters}

Stars in galaxies typically form in {\it clusters}, collections
of stars that are gravitationally bound or structurally associated
(loosely bound).
Stars in a cluster are roughly co-eval, so there is a lot we
can learn from them!

The Hertzsprung-Russell diagram of the clusters reflect an
{\it isochrone}, a line corresponding to a single age stellar
population. The location of the {\it main sequence turn-off}
indicates the age of the stellar population.

{\bf See Figure 2.12 of Sparke and Gallagher.}

\subsection{Open Clusters}

There are $>1000$ collections of (usually) recently-formed stars
called {\it open clusters} consisting of stars with
ages $\lesssim1~\Gyr$, total luminosities of 
$L\sim100-10,000\Lsun$, and several hundred total stars.
{\bf See Table 2.2 of Sparke and Gallagher.}


\subsection{Globular Clusters}

There are much denser, more massive, and typically older
collections of stars called {\it globular clusters}.
Globulars may have luminosities of $10^5-10^6~\sim\Lsun$ 
and $10^{5}-10^6$ stars. Most of the globulars are more
than $10\sim\Gyr$ old and very metal poor.

{\bf See Table 2.3 of Sparke and Gallagher.}
{\bf See Table 2.14 of Sparke and Gallagher.}


\end{document}













