\documentclass[]{article}
\usepackage[margin=1.0in]{geometry}
\usepackage{amssymb}

%title material
\title{Astronomy 400B Homework \#5}
\author{Please Show Your Work for Full Credit}
\date{Due April 30, 2015 by 9:35am}


%include latex definitions

%average
\newcommand{\ave}[1]{\langle#1\rangle}
%radian
\newcommand{\rad}{\mathrm{rad}}

%astronomical unit
\newcommand{\AU}{\mathrm{AU}}

%micron
\newcommand{\mum}{\mu\mathrm{m}}

%millimeter
\newcommand{\mm}{\mathrm{mm}}

%centimeter
\newcommand{\cm}{\mathrm{cm}}

%meter
\newcommand{\m}{\mathrm{m}}

%kilometer
\newcommand{\km}{\mathrm{km}}

%parsec
\newcommand{\pc}{\mathrm{pc}}

%kiloparsec
\newcommand{\kpc}{\mathrm{kpc}}

%megaparsec
\newcommand{\Mpc}{\mathrm{Mpc}}

%gigaparsec
\newcommand{\Gpc}{\mathrm{Gpc}}

%light year
\newcommand{\ly}{\mathrm{ly}}

%second
\newcommand{\s}{\mathrm{s}}
\newcommand{\yr}{\mathrm{yr}}
\newcommand{\Gyr}{\mathrm{Gyr}}

%solar mass
\newcommand{\Msun}{M_{\odot}}

%grams
\newcommand{\g}{\mathrm{g}}

%erg
\newcommand{\erg}{\mathrm{erg}}

%solar luminosity
\newcommand{\Lsun}{L_{\odot}}

%jansky 
\newcommand{\Jy}{\mathrm{Jy}}

%flux density
\newcommand{\Fnu}{F_{\nu}}
\newcommand{\Flambda}{F_{\lambda}}
\newcommand{\Lnu}{L_{\nu}}

%hertz 
\newcommand{\Hz}{\mathrm{Hz}}

%angstrom
\newcommand{\Ang}{\mathrm{\r{A}}}

%solar radius
\newcommand{\Rsun}{R_{\odot}}

%stefan-boltzmann
\newcommand{\sigmaSB}{\sigma_{\mathrm{SB}}}

%boltzmann
\newcommand{\kB}{k_{\mathrm{B}}}

%kelvin
\newcommand{\K}{\mathrm{K}}

%T_bandpass
\newcommand{\TBP}{T_{\mathrm{BP}}}

%magnitude
\newcommand{\mg}{\mathrm{mag}}


%arcsecond
\newcommand{\arcsec}{\mathrm{arcsec}}

%critical density
\newcommand{\rhoc}{\rho_{\mathrm{crit}}}

%proton mass
\newcommand{\mproton}{m_{\mathrm{p}}}

%electron volt
\newcommand{\eV}{\mathrm{eV}}

%kiloelectron volt
\newcommand{\keV}{\mathrm{keV}}

%megaelectron volt
\newcommand{\MeV}{\mathrm{MeV}}

%gigaelectron volt
\newcommand{\GeV}{\mathrm{GeV}}

%vector velocity
\newcommand{\vv}{\mathbf{v}}

%vector radius
\newcommand{\vr}{\mathbf{r}}

%vector position
\newcommand{\vx}{\mathbf{x}}

%vector force
\newcommand{\vF}{\mathbf{F}}

%vector surface
\newcommand{\vS}{\mathbf{S}}

%vector angular momentum
\newcommand{\vL}{\mathbf{L}}

%script I -- integral of motion
\newcommand{\cI}{\mathcal{I}}

%effective potential
\newcommand{\Phieff}{\Phi_\mathrm{eff}}


%begin the document
\begin{document}

%make the title, goes after document begins
\maketitle

\section{Sparke \& Gallagher Problem 8.7}

The quantity $\ave{\Delta_k^2}^{1/2}$ gives the expected fractional deviation
$|\delta(\vx)|$ from the mean density in an overdense of diffuse region of
size $1/k$.  Write $\delta(\vx)$ and $\Delta \Phi(\vx)$ as Fourier transforms and
use Poisson's equation
\begin{equation}
\nabla^2 \Phi(\vx) = 4\pi G \rho(\vx)
\end{equation}
\noindent
to show that these lumps and voids cause fluctuations $\Delta\Phi_k$ in the
gravitational potential, where $k^2|\Delta\Phi_k|\sim4\pi G\bar{\rho}\ave{\Delta_k^2}^{1/2}$.
Show that, when $P(k)\propto k$, the {\it Harrison-Zel'dovich} spectrum, $|\Phi_k|$ does
not depend on $k$: the potential is equally `rippled' on all spatial scales.

\subsection{Solution}

We write the density as
\begin{equation}
\rho = \bar{\rho}[1 + \delta]
\end{equation}
\noindent
and the potential as
\begin{equation}
\Phi = \bar{\Phi} + \Delta \Phi.
\end{equation}
\noindent
The Poisson equation gives us
\begin{eqnarray}
\nabla^2 \Phi &=& 4\pi G \rho\\
\nabla^2 \bar{\Phi} + \nabla^2 \Delta \Phi &=& 4\pi G \bar{\rho}[1 + \delta].
\end{eqnarray}
\noindent
Defining $\nabla^2{} \bar{\Phi} = 4 \pi G \bar{\rho}$, we then have
\begin{equation}
\label{eqn:fpoisson}
\nabla^2 \Delta \Phi = 4\pi G \bar{\rho} \delta.
\end{equation}
\noindent
Let's write $\Delta \Phi$ and $\delta$ as Fourier transforms, meaning
\begin{equation}
\Delta \Phi = \int d^3 k \Delta \Phi_k e^{i \vk \cdot \vx}
\end{equation}
\noindent
and
\begin{equation}
\delta = \int d^3 k \tilde{\delta}(\vk) e^{i \vk \cdot \vx}.
\end{equation}
\noindent
Applying Equation \ref{eqn:fpoisson} we have
\begin{eqnarray}
\nabla^2 \Delta \Phi &=& 4\pi G \bar{\rho} \delta \\
\nabla^2 \int d^3 k \Delta \Phi_k e^{i \vk \cdot \vx} &=& 4\pi G \bar{\rho}\int d^3 k \tilde{\delta}(\vk) e^{i \vk \cdot \vx} \\
- \int d^3 k k^2 \Delta \Phi_k e^{i \vk \cdot \vx}&=& 4\pi G \bar{\rho}\int d^3 k \tilde{\delta}(\vk) e^{i \vk \cdot \vx} \\
k^2 |\Delta \Phi_k| &=& 4\pi G \bar{\rho} |\tilde{\delta}|.
\end{eqnarray}
\noindent
But since $\ave{\Delta_k}^{1/2} \sim | \tilde{\delta} |$, we have
\begin{equation}
k^2 |\Delta \Phi_k| \sim 4\pi G \bar{\rho} |\ave{\Delta_k}^{1/2}|.
\end{equation}


If $P(k) \propto k$, and $\ave{\Delta_k^2}^{1/2} \propto (k^3 P(k))^{1/2} \propto k^2$, then $|\Delta \Phi_k| \propto k^2 \times k^{-2} \propto~\mathrm{const}$.  Note that the total $\Phi_k$, the Fourier transform of the total potential $\Phi = \bar{\Phi} + \Delta \Phi$ can have no additional $k$ dependence since $\bar{\rho}$ is a constant.


\section{Sparke \& Gallagher Problem 8.9}

By substituting into the equation
\begin{equation}
\ddot{R}(t) = -\frac{4\pi G}{3} R(t)\left[ \rho(t) + \frac{3p(t)}{c^2}\right]
\end{equation}
\noindent
show that, when vacuum energy dominates the expansion, we have $R(t) \propto \exp(t\sqrt{\Lambda/3})$.

\subsection{Solution}

The energy density of vacuum energy is 
\begin{equation}
\rho_{\Lambda} = \frac{\Lambda}{8\pi G}.
\end{equation}
\noindent
The pressure of vacuum energy is
\begin{equation}
p_{\Lambda} = -\frac{\Lambda c^2}{8\pi G}
\end{equation}
\noindent
The Friedmann equation then reads
\begin{equation}
\ddot{R}(t) = -\frac{4\pi G}{3} R(t)\left[ \frac{\Lambda}{8\pi G} - \frac{3}{c^2}\frac{\Lambda c^2}{8\pi G}\right] = R(t) \frac{\Lambda}{3}.
\end{equation}
\noindent
Multiplying both sides by $\dot{R}(t)/2$, we have
\begin{equation}
\frac{1}{2}\dot{R}(t)\ddot{R}(t) = \frac{1}{2} \dot{R}(t) R(t) \frac{\Lambda}{3}.
\end{equation}
\noindent
Integrating, we have
\begin{eqnarray}
\dot{R}^2(t) &=& R^2(t)\frac{\Lambda}{3} \\
\frac{dR}{dt} &=& R\sqrt{\frac{\Lambda}{3}}\\
\frac{dR}{R} &=&dt\sqrt{\frac{\Lambda}{3}}\\
\therefore R(t) &\propto& \exp\left[t\sqrt{\frac{\Lambda}{3}}\right].
\end{eqnarray}

\section{Sparke \& Gallagher Problem 8.11}

Blackbody radiation and relativistic particles provide most of the
energy density at $t\ll t_{eq}$. Show that the equation
\begin{equation}
H^2(t) = H_0^2[\Omega_r(1+z)^4 + \Omega_m(1+z)^3 + (1 -\Omega_{tot})(1+z)^2 + \Omega_{\Lambda}]
\end{equation}
\noindent
then implies that $H(t)=1/(2t)$. Early on, the leftmost term of the
equation
\begin{equation}
-\frac{kc^2}{R^2(t_0)} = H_0^2(1-\Omega_{tot}) = a^2(t)\left[H^2(t) - \frac{8\pi G}{3} \rho(t)\right]
\end{equation}
\noindent
is tiny, so $H^2(t)\approx8\pi G\rho(t)/3$: show that the temperature $T(t)$ is
given by
\begin{equation}
T = \left(\frac{3c^2}{32 \pi G a_B t^2}\right)^{1/4},
\end{equation}
where $a_B = 7.56\times 10^{-16}~\mathrm{J}~\mathrm{m}^{-3}~\K^{-4}$ is the blackbody constant.

\subsection{Solution}

In the radiation-dominated era, we have
\begin{eqnarray}
H^{2}(t) &=& H_0^2\left[\Omega_r(1+z)^4\right]\\
\left(\frac{\dot{a}}{a}\right)^2&\propto& a^{-4} \\
\dot{a}/a &\propto& a^{-2}\\
\dot{a} &\propto& a^{-1}\\
a \times da &\propto& dt\\
a^2 / 2 &\propto& t\\
\therefore R(t) \propto a &\propto& t^{1/2},~\mathrm{and}~H(t)\propto1/t.
\end{eqnarray}
\noindent
Evaluating $H(t)$, we have
\begin{equation}
H(t) = \frac{\dot{a}}{a} = \frac{\frac{d}{dt} a}{a} = \frac{\frac{1}{2}t^{-1/2}}{t^{1/2}} = \frac{1}{2t}.
\end{equation}
\noindent
Early on we have
\begin{equation}
H^{2}(t) = \frac{1}{4t^2} = \frac{8\pi G \rho}{3},
\end{equation}
\noindent
but the energy density of black body radiation is $\rho c^2 = a_B T^{4}$.
We then have
\begin{eqnarray}
\frac{1}{4t^2} = \frac{8\pi G \rho}{3} &=& \frac{8\pi G a_B T^4}{3c^2}\\
T^{4} &=& \frac{3 c^2}{32\pi G a_B t^2}\\
\therefore T &=& \left(\frac{3 c^2}{32\pi G a_B t^2}\right)^{1/4}.
\end{eqnarray}



\section{Sparke \& Gallagher Problem 8.13}

Show that, when cool matter accounts for most of the energy density, and the universe
is flat with $k=0$, we have
\begin{equation}
\dot{a}\propto a^{-1/2},~\mathrm{and}~a(t)\propto t^{2/3}.
\end{equation}

\subsection{Solution}
During matter domination, we have
\begin{eqnarray}
H^2(t) \propto (1+z)^3 &\propto& a^{-3}\\
\dot{a}^2 &\propto& a^{-1}\\
\therefore \dot{a} &\propto& a^{-1/2}\\
\mathrm{and}~ a^{1/2} da &\propto& dt\\
a^{3/2} &\propto& t\\
\therefore a&\propto& t^{2/3}.
\end{eqnarray}
\end{document}