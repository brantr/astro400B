\documentclass[]{article}
\usepackage[margin=1.0in]{geometry}
\usepackage{amssymb}

%title material
\title{Astronomy 400B Lecture 10: Galaxy Groups and Clusters}
\author{Brant Robertson}
\date{March, 2015}


%include latex definitions

%average
\newcommand{\ave}[1]{\langle#1\rangle}
%radian
\newcommand{\rad}{\mathrm{rad}}

%astronomical unit
\newcommand{\AU}{\mathrm{AU}}

%centimeter
\newcommand{\cm}{\mathrm{cm}}

%meter
\newcommand{\m}{\mathrm{m}}

%kilometer
\newcommand{\km}{\mathrm{km}}

%parsec
\newcommand{\pc}{\mathrm{pc}}

%kiloparsec
\newcommand{\kpc}{\mathrm{kpc}}

%megaparsec
\newcommand{\Mpc}{\mathrm{Mpc}}

%gigaparsec
\newcommand{\Gpc}{\mathrm{Gpc}}

%light year
\newcommand{\ly}{\mathrm{ly}}

%second
\newcommand{\s}{\mathrm{s}}
\newcommand{\yr}{\mathrm{yr}}
\newcommand{\Gyr}{\mathrm{Gyr}}

%solar mass
\newcommand{\Msun}{M_{\odot}}

%grams
\newcommand{\g}{\mathrm{g}}

%erg
\newcommand{\erg}{\mathrm{erg}}

%solar luminosity
\newcommand{\Lsun}{L_{\odot}}

%jansky 
\newcommand{\Jy}{\mathrm{Jy}}

%flux density
\newcommand{\Fnu}{F_{\nu}}
\newcommand{\Flambda}{F_{\lambda}}
\newcommand{\Lnu}{L_{\nu}}

%hertz 
\newcommand{\Hz}{\mathrm{Hz}}

%angstrom
\newcommand{\Ang}{\mathrm{\r{A}}}

%solar radius
\newcommand{\Rsun}{R_{\odot}}

%stefan-boltzmann
\newcommand{\sigmaSB}{\sigma_{\mathrm{SB}}}

%boltzmann
\newcommand{\kB}{k_{\mathrm{B}}}

%kelvin
\newcommand{\K}{\mathrm{K}}

%T_bandpass
\newcommand{\TBP}{T_{\mathrm{BP}}}

%magnitude
\newcommand{\mg}{\mathrm{mag}}


%arcsecond
\newcommand{\arcsec}{\mathrm{arcsec}}

%begin the document
\begin{document}

%make the title, goes after document begins
\maketitle

%first section
\section{Galaxy Groups}

Galaxies often are found in dynamically-connected
collections called groups, with the largest such
structures called clusters. These groups and
clusters are gravitationally bound amalgamations
of individual galaxies that act dynamically as
a single object.  For instance, we can weigh
a group of cluster using the virial theorem,
assuming some mass profile for the system.  For
a group with a $1-D$ radial
velocity dispersion of $\sigma_r =310~\km~\s^{-1}$
and typical radius of $a_p=100~\kpc$, then assuming
a Plummer sphere (with $PE = -3\pi G M^2/32a_p$) for the potential the 
virial theorem says that
\begin{equation}
\frac{3M\sigma_r^2}{2} = KE = -\frac{PE}{2} = \frac{3\pi}{64}\frac{GM^2}{a_p}.
\end{equation}
\noindent
Recall that a convenient set of units for the gravitational
constant is $G=4.301183\times10^{-6}~(\km/\s)^2~\Msun^{-1}~\kpc$.
We can then solve for the mass as
\begin{equation}
M = \frac{64\sigma_r^2 a_p}{\pi G} = \frac{32\times(310~\km/\s)^2\times(100~\kpc)}{\pi \times [4.301183\times10^{-6}~(\km/\s)^2~\Msun^{-1}~\kpc]} \approx 2.3\times10^{13}\Msun.
\end{equation}

Typically, the random motions of gas will also come in 
virial equilibrium with the gravitational potential of a
group or cluster.  For very large systems, the gas can be
in excess of $T>10^{7}\K$, where it loses energy primarily
by {\it free-free} emission, also called {\it thermal bremsstrahlung}.
If so, the luminosity of a gas of volume $V$ with density $n$ will be
\begin{equation}
L_{\mathrm{X}} = n^2 \Lambda(T_{\mathrm{X}}) V,~\mathrm{where}~\Lambda\approx3\times10^{-27}T_{\mathrm{X}}^{1/2}~\erg~\s^{-1}~\cm^{3}.
\end{equation}
\noindent
If the gas is in hydrostatic equilibrium, then
the pressure gradient in the gas will balance the gravitational
force pulling the gas toward the center of the group or cluster.
We can then write that
\begin{equation}
\nabla P = -\rho(r) \frac{GM(<r)}{r}
\end{equation}
\noindent
where the fraction on the RHS is the gravitational acceleration.
We can related the pressure to the properties of an ideal gas
if we assume that the average mass of a gas particle for 
gas has a weight $m = \mu m_{p}$, where for fully-ionized
hydrogen $\mu\sim0.5$ and for solar metallicity gas $\mu\sim0.6$.
We can then write the pressure as
\begin{equation}
P = \frac{\rho}{\mu m_{p}}kT
\end{equation}
\noindent
and
\begin{equation}
M(<r) = \frac{k}{\mu m_p}\frac{r}{G \rho(r)} \frac{d}{dr}(-\rho T).
\end{equation}
\noindent
We can infer the density profile from the gas surface brightness
distribution, and the temperature by taking an x-ray spectrum of the
gas.  That then allows us to infer the interior mass profile.

\section{Dynamical Friction}

One of the most important phenomena in groups and clusters is
{\it dynamical friction}, which is the process by which the
galaxies in groups and clusters lose energy and angular momentum
to the surrounding sea of dark matter and stars in the intragroup
or intracluster medium. Dynamical friction causes galaxies to ``sink''
to the central regions of the group or cluster and thereby be
assimilated into the larger system.  So how does this process work?

As with the impluse approximation we studied before, consider a
{\it galaxy} of mass $M$ moving past a star (or clump of dark matter)
with a mass $m$ at an impact parameter $b$.  During the passage, the
galaxy has a change of velocity in the direction perpendicular to the
direction motion of
\begin{equation}
\Delta V_{\perp} = \frac{2Gm}{bV}.
\end{equation}
\noindent
This equation applies when the impact parameter $b$ is larger than the
size of the galaxy with mass $M$, and when the velocity $V$ is large
enough at separation $b$ such that $\Delta V_{\perp}\ll V$.  We then
require that
\begin{equation}
b \gg \frac{2G(M+m)}{V^2} \equiv 2 r_s.
\end{equation}
\noindent
Note this radius differs from the strong encounter radius we used
previously because $M\ne m$ in this case.

The perturber must obtain an equal and opposite momentum, so the
total kinetic energy in the perpendicular motion becomes
\begin{equation}
\Delta KE_{\perp} = \frac{M}{2}\left(\frac{2Gm}{bV}\right)^2 + \frac{m}{2}\left(\frac{2GM}{bV}\right)^2 = \frac{2G^2mM(M+m)}{b^2 V^2}.
\end{equation}
\noindent
The smaller object acquires most of the energy, which must be depleted from the motion by an amount $\Delta V_{\parallel}$ of the
larger object.  Once the galaxy and the perturber are far away (and before the encounter) the kinetic energies 
before and after the encounter must be the same.  We then have that
\begin{equation}
\frac{M}{2}V^2 = \Delta KE_{\perp} + \frac{M}{2}(V + \Delta V_{\parallel})^2 + \frac{m}{2}\left(\frac{M}{m} \Delta V_{\parallel} \right)^2
\end{equation}
\noindent
If we take $\Delta V_{\parallel} \ll V$, then the terms proportional to $\Delta V_{\parallel}^2$ are very small and
can be ignored.  We can then find the amount by which each perturber reduces the velocity of the galaxy with
mass $M$ as
\begin{equation}
- \Delta V_{\parallel} \approx \frac{\Delta KE_{\perp}}{M V} = \frac{2G^2 m(M+m)}{b^2 V^3}.
\end{equation}

OK, now what if there are a number density per cubic parsec $n$ perturbers of mass $m$?  
We then have to integrate through the cylinder of radius $b$ to find the total
effect on the galaxy of mass $M$.  We find that
\begin{equation}
-\frac{dV}{dt} = \int_{b_{\mathrm{min}}}^{b_{\mathrm{max}}} n V \frac{2G^2 m (M+m)}{b^2 V^3} 2\pi b db = \frac{4\pi G^2(M+m)}{V^2} n m \ln \Lambda
\end{equation}
\noindent
where $\Lambda\equiv b_{\mathrm{max}}/b_{\mathrm{min}}$.

Some interesting things to note
\begin{enumerate}
\item The slower the galaxy $M$ moves, the larger its deceleration.
\item If $V\ll\sigma$, where $\sigma$ is the velocity dispersion of perturbers, we find that $dV/dt\propto -V$.  This is the same as what happens to a parachutist.
\item The net effect is to lower the total kinetic motions of the galaxies over time.
Before the encounter, the kinetic energy of one of the galaxies is
\begin{equation}
E_0 = KE_0 + PE_0 = - KE_0
\end{equation}
\noindent
since the potential energy is $ PE_0 = - 2KE_0$ in virial equilibrium.
Dynamical friction increases the energy in random motions and the internal kinetic
energy by $\Delta KE$.  After the system again reaches virial equilibrium,
the kinetic energy is less than before
\begin{equation}
KE_1  = - (E_0 + \Delta KE) = KE_0 - \Delta KE
\end{equation}
\noindent
Stars that gain the most kinetic energy are ejected.
\item Groups have lower relative velocties, so the dynamical
friction should operate more efficiently.
\end{enumerate}

\section{Galaxy Mergers}

Show a movie!

During the merger, most of the energy released by newly formed
stars will be absorbed by dust and re-radiated in the infrared.
This leads to {\it ultraluminous infrared galaxies} called
{\it ULIRGs}.  The relation between the star formation rate and
the far-infrared emission is about
\begin{equation}
\dot{M}_{\star} \approx \frac{L_{FIR}}{6\times10^9\Lsun} \Msun~\yr^{-1}.
\end{equation}

{\bf Show Figure 7.7 of SG.}

\section{Gas Cooling in Clusters}

The x-ray emitting gas in clusters radiates, and must
therefore cool if there is no additional 
heat source.  The time for the gas to cool 
depends on the density $n$ and the thermal
energy $2n \times (3kT/2)$ as
\begin{equation}
t_{\mathrm{cool}} \frac{3nkT}{3\times 10^{-27}n^2\sqrt{T}}~\s\approx 14\left(\frac{10^{-3}\cm^{-3}}{n}\right)\left(\frac{T}{10^{7}\K}\right)^{1/2}~\Gyr
\end{equation}
\noindent
In the centers of clusters, the gas density is $n\sim10^{-2}\cm^{-3}$ so
the gas should cool on $\Gyr$ timescales.  However, we don't see star
formation in the central cluster galaxies so we think the gas is somehow
heated again (perhaps from SN or AGN).  Gas in the outer regions
will remain hot effectively forever, with densities of $n\sim10^{-4}\cm^{-3}$.

\section{Gravitational Lensing}

Einstein predicted that light passing a distance $b$
past a mass $M$ would be deflected by an angle
\begin{equation}
\label{eqn:alpha_grav_lens}
\alpha \approx \frac{4GM}{bc^2} = \frac{2R_s}{b}
\end{equation}
\noindent
where $R_s = 2 GM/c^2$ is the {\it Schwarzschild radius}.
For the Sun, $R_s\sim3~\km$. The approximation
holds only for small deflections $\alpha\ll1$.
Note this is exactly twice the deflection determined by 
the impluse approximation.

{\bf Show figure 7.14 of SG.}

Without the lense $L$, the object would appear at a distance
$\beta = y/d_{S}$ as long as $d_{S} \gg y$.
The light is bent by $\alpha$, so the object instead appears
at an angle $\theta \approx x/d_S$ (for $d_S \gg x$).
When the bending is small $x-y = \alpha d_{LS}$.
The impact parameter in the lens plane is $b = \theta d_{L}$
as long as $d_S \gg b$.

If we divide Equation \ref{eqn:alpha_grav_lens} by $d_S$
we find
\begin{equation}
\theta - \beta = \frac{\alpha d_{LS}}{d_S} = \frac{1}{\theta} \frac{4 GM}{c^2} \frac{d_{LS}}{d_L d_S} \equiv \frac{1}{\theta} \theta_E^2
\end{equation}
\noindent
where $\theta_E$ is called the {\it Einstein radius}.
We have a quadratic relation between the angular distance $\theta$ between
$L$ and the object's position as
\begin{equation}
\theta^2 - \beta \theta - \theta_E^2 = 0,~\mathrm{so}~\theta_{\pm} = \frac{\beta \pm \sqrt{\beta^2 + 4 \theta_E^2}}{2}.
\end{equation}
\noindent
A star directly behind the lense with $\beta=0$ will be seen as a circle
on the sky with radius $\theta_E$.  When $\beta>0$, the image at $\theta_{+}$
is further away from the lense with $\theta_{+}>\beta$ and is outside
the Einstein radius with $\theta_{+}>\theta_E$ (these were the images
seen in the Sun's lensing).  The image at $\theta_{-}$ is inverted, on
the other side of the lens, and is within the Einstein radius.

Stars in the disk plane of the Milky Way will lense each other.  We
often cannot resolve the lense and the source separately, but we will
see the magnification of the star -- we call this microlensing.

Gravitational lensing leaves the surface brightness unchanged by increases
the area of an extended source on the sky.  The increase of the apparent
brightness is then just proportional to the increase in the area.

{\bf Show figure 7.15 of SG.}

Consider an annulus of width $S'$ centered on $L$ between
radius $y$ and $y+\Delta y$.  An image $I$ of
$S'$ occupies the same angle $\delta \Phi$ but the distance
from the center is expanded or contracted as $x/y = \theta/\beta$
while $\delta x/\delta y= d\theta/d\beta$.  The ratio of the areas
is 
\begin{equation}
\frac{A_{\pm}(image)}{A(source)} = \left| \frac{\theta}{\beta}\frac{d\theta}{d\beta}\right| = \frac{1}{4}\left( \frac{\beta}{\sqrt{\beta^2 + 4 \theta_E^2}} + \frac{\sqrt{\beta^2 + 4 \theta_E^2}}{\beta}\pm 2 \right)
\end{equation}
\noindent
The image at $\theta_{+}$ is always brighter than the source and is stretched in the
tangential direction.  The closer image is dimmer unless
\begin{equation}
\beta^2 < (3-2\sqrt{2})\theta_E^2 / \sqrt{2}
\end{equation}
\noindent
or
\begin{equation}
\beta \lesssim 0.348 \theta_E.
\end{equation}

\subsection{Extended Lens}

If the lense is a galaxy cluster, then we can
think of the system as a collection of point lenses.
Let's re-write Equation \ref{eqn:alpha_grav_lens} as
\begin{equation}
\alpha(b) \equiv \frac{d \psi_L}{db},~\mathrm{where}~\psi_L = \frac{4 GM}{c^2}\ln b
\end{equation}
\noindent
We can then sum the effect of all the mass within the lense.
If the lens size is small compared with $d_L$ and $d_{LS}$ then
the deflection only depends on the surface density $\Sigma(\vx)$
of the lens.  We then have
\begin{equation}
\alpha(\vb) \equiv \nabla \psi_L(\vb),~\mathrm{where}~\psi_L(\vb) = \frac{4G}{c^2}\int \Sigma(\vb') \ln |\vb - \vb'| dS'
\end{equation}
\noindent
If the lens is axisymmetric, we can evaluate this integral as
\begin{equation}
\alpha(b) = \frac{4G}{bc^2}\int_{0}^{b} \Sigma(R) 2 \pi R dR = \frac{4G}{c^2} \frac{M(<b)}{b}
\end{equation}
\noindent
The change in the position angle is
\begin{equation}
\theta - \beta = \alpha(\theta)\frac{d_{LS}}{d_S} \frac{1}{\theta} \cdot \frac{4GM(<b)}{c^2}\frac{d_{LS}}{d_L d_S}
\end{equation}
\noindent
but since $b = \theta d_L$, we can re-write this in terms of the critical density for lensing
\begin{equation}
\beta = \theta\left[1 - \frac{1}{\Sigma_{\mathrm{crit}}}\frac{M(<b)}{\pi b^2}\right],~\mathrm{where}~\Sigma_{\mathrm{crit}} \equiv \frac{c^2}{4\pi G}\frac{d_S}{d_L d_LS}
\end{equation}
\noindent
The quantity $M(<b)/\pi b^2$ is just the average surface density within $b$.

If the central density exceeds $\Sigma_{\mathrm{crit}}$ then the image of a
source at $\beta=0$ will be a thin circular Einstein ring of angular size $\theta_{E} = b_E/d_L$
where
\begin{equation}
\frac{M(<b_E)}{\pi b_E^2} = \Sigma_{\mathrm{crit}}
\end{equation}
defines the radius $b_E$ where the average surface density falls to $\Sigma_{\mathrm{crit}}$. If
the surface density is not larger than $\Sigma_\mathrm{crit}$ no rings or multiple images are produced.
\end{document}