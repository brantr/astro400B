\documentclass[]{article}
\usepackage[margin=1.0in]{geometry}
\usepackage{amssymb}

%title material
\title{Astronomy 400B Lecture 10: Galaxy Groups and Clusters}
\author{Brant Robertson}
\date{March, 2015}


%include latex definitions

%average
\newcommand{\ave}[1]{\langle#1\rangle}
%radian
\newcommand{\rad}{\mathrm{rad}}

%astronomical unit
\newcommand{\AU}{\mathrm{AU}}

%micron
\newcommand{\mum}{\mu\mathrm{m}}

%millimeter
\newcommand{\mm}{\mathrm{mm}}

%centimeter
\newcommand{\cm}{\mathrm{cm}}

%meter
\newcommand{\m}{\mathrm{m}}

%kilometer
\newcommand{\km}{\mathrm{km}}

%parsec
\newcommand{\pc}{\mathrm{pc}}

%kiloparsec
\newcommand{\kpc}{\mathrm{kpc}}

%megaparsec
\newcommand{\Mpc}{\mathrm{Mpc}}

%gigaparsec
\newcommand{\Gpc}{\mathrm{Gpc}}

%light year
\newcommand{\ly}{\mathrm{ly}}

%second
\newcommand{\s}{\mathrm{s}}
\newcommand{\yr}{\mathrm{yr}}
\newcommand{\Gyr}{\mathrm{Gyr}}

%solar mass
\newcommand{\Msun}{M_{\odot}}

%grams
\newcommand{\g}{\mathrm{g}}

%erg
\newcommand{\erg}{\mathrm{erg}}

%solar luminosity
\newcommand{\Lsun}{L_{\odot}}

%jansky 
\newcommand{\Jy}{\mathrm{Jy}}

%flux density
\newcommand{\Fnu}{F_{\nu}}
\newcommand{\Flambda}{F_{\lambda}}
\newcommand{\Lnu}{L_{\nu}}

%hertz 
\newcommand{\Hz}{\mathrm{Hz}}

%angstrom
\newcommand{\Ang}{\mathrm{\r{A}}}

%solar radius
\newcommand{\Rsun}{R_{\odot}}

%stefan-boltzmann
\newcommand{\sigmaSB}{\sigma_{\mathrm{SB}}}

%boltzmann
\newcommand{\kB}{k_{\mathrm{B}}}

%kelvin
\newcommand{\K}{\mathrm{K}}

%T_bandpass
\newcommand{\TBP}{T_{\mathrm{BP}}}

%magnitude
\newcommand{\mg}{\mathrm{mag}}


%arcsecond
\newcommand{\arcsec}{\mathrm{arcsec}}

%critical density
\newcommand{\rhoc}{\rho_{\mathrm{crit}}}

%proton mass
\newcommand{\mproton}{m_{\mathrm{p}}}

%electron volt
\newcommand{\eV}{\mathrm{eV}}

%kiloelectron volt
\newcommand{\keV}{\mathrm{keV}}

%megaelectron volt
\newcommand{\MeV}{\mathrm{MeV}}

%gigaelectron volt
\newcommand{\GeV}{\mathrm{GeV}}

%vector velocity
\newcommand{\vv}{\mathbf{v}}

%vector radius
\newcommand{\vr}{\mathbf{r}}

%vector position
\newcommand{\vx}{\mathbf{x}}

%vector force
\newcommand{\vF}{\mathbf{F}}

%vector surface
\newcommand{\vS}{\mathbf{S}}

%vector angular momentum
\newcommand{\vL}{\mathbf{L}}

%script I -- integral of motion
\newcommand{\cI}{\mathcal{I}}

%effective potential
\newcommand{\Phieff}{\Phi_\mathrm{eff}}


%begin the document
\begin{document}

%make the title, goes after document begins
\maketitle

%first section
\section{Galaxy Groups}

Galaxies often are found in dynamically-connected
collections called groups, with the largest such
structures called clusters. These groups and
clusters are gravitationally bound amalgamations
of individual galaxies that act dynamically as
a single object.  For instance, we can weigh
a group of cluster using the virial theorem,
assuming some mass profile for the system.  For
a group with a $1-D$ radial
velocity dispersion of $\sigma_r =310~\km~\s^{-1}$
and typical radius of $a_p=100~\kpc$, then assuming
a Plummer sphere for the potential the 
virial theorem says that
\begin{equation}
\frac{3M\sigma_r^2}{2} = KE = -\frac{PE}{2} = \frac{3\pi}{64}\frac{GM^2}{a_p}.
\end{equation}
\noindent
Recall that a convenient set of units for the gravitational
constant is $G=4.301183\times10^{-6}~(\km/\s)^2~\Msun^{-1}~\kpc$.
We can then solve for the mass as
\begin{equation}
M = \frac{64\sigma_r^2 a_p}{\pi G} = \frac{32\times(310~\km/\s)^2\times(100~\kpc)}{\pi \times [4.301183\times10^{-6}~(\km/\s)^2~\Msun^{-1}~\kpc]} \approx 2.3\times10^{13}\Msun.
\end{equation}

Typically, the random motions of gas will also come in 
virial equilibrium with the gravitational potential of a
group or cluster.  For very large systems, the gas can be
in excess of $T>10^{7}\K$, where it looses energy primarily
by {\it free-free} emission, also called {\it thermal bremsstrahlung}.
If so, the luminosity of a gas with density $n$ will be
\begin{equation}
L_{\mathrm{X}} = n^2 \Lambda(T_{\mathrm{X}}),~\mathrm{where}~\Lambda\approx3\times10^{-27}T_{\mathrm{X}}^{1/2}~\erg~\s^{-1}.
\end{equation}
\noindent
If the gas is in hydrostatic equilibrium, then
the pressure gradient in the gas will balance the gravitational
force pulling the gas toward the center of the group or cluster.
We can then write that
\begin{equation}
\nabla P = -\rho(r) \frac{GM(<r)}{r}
\end{equation}
\noindent
where the fraction on the RHS is the gravitational acceleration.
We can related the pressure to the properties of an ideal gas
if we assume that the average mass of a gas particle for 
gas has a weight $m = \mu m_{p}$, where for fully-ionized
hydrogen $\mu\sim0.5$ and for solar metallicity gas $\mu\sim0.6$.
We can then write the pressure as
\begin{equation}
P = \frac{\rho}{\mu m_{p}}kT
\end{equation}
\noindent
and
\begin{equation}
M(<r) = \frac{k}{\mu m_p}\frac{r}{G \rho(r)} \frac{d}{dr}(-\rho T).
\end{equation}
\noindent
We can infer the density profile from the gas surface brightness
distribution, and the temperature by taking an x-ray spectrum of the
gas.  That then allows us to infer the interior mass profile.

\section{Dynamical Friction}

One of the most important phenomena in groups and clusters is
{\it dynamical friction}, which is the process by which the
galaxies in groups and clusters loose energy and angular momentum
to the surrounding sea of dark matter and stars in the intragroup
or intracluster medium. Dynamical friction causes galaxies to ``sink''
to the central regions of the group or cluster and thereby be
assimilated into the larger system.  So how does this process work?

As with the impluse approximation we studied before, consider a
{\it galaxy} of mass $M$ moving past a star (or clump of dark matter)
with a mass $m$ at an impact parameter $b$.  During the passage, the
galaxy has a change of velocity in the direction perpendicular to the
direction motion of
\begin{equation}
\Delta V_{\perp} = \frac{2Gm}{bV}.
\end{equation}
\noindent
This equation applies when the impact parameter $b$ is larger than the
size of the galaxy with mass $M$, and when the velocity $V$ is large
enough at separation $b$ such that $\Delta V_{\perp}\ll V$.  We then
require that
\begin{equation}
b \gg \frac{2G(M+m)}{V^2} \equiv 2 r_s.
\end{equation}
\noindent
Note this radius differs from the strong encounter radius we used
previously because $M\ne m$ in this case.

The perturber must obtain an equal and opposite momentum, so the
total kinetic energy in the perpendicular motion becomes
\begin{equation}
\Delta KE_{\perp} = \frac{M}{2}\left(\frac{2Gm}{bV}\right)^2 + \frac{m}{2}\left(\frac{2GM}{bV}\right)^2 = \frac{2G^2mM(M+m)}{b^2 V^2}.
\end{equation}
\noindent
The smaller object acquires most of the energy, which must be depleted from the motion by an amount $\Delta V_{\parallel}$ of the
larger object.  Once the galaxy and the perturber are far away (and before the encounter) the kinetic energies 
before and after the encounter must be the same.  We then have that
\begin{equation}
\frac{M}{2}V^2 = \Delta KE_{\perp} + \frac{M}{2}(V + \Delta V_{\parallel})^2 + \frac{m}{2}\left(\frac{M}{m} \Delta V_{\parallel} \right)^2
\end{equation}
\noindent
If we take $\Delta V_{\parallel} \ll V$, then the terms proportional to $\Delta V_{\parallel}^2$ are very small and
can be ignored.  We can then find the amount by which each perturber reduces the velocity of the galaxy with
mass $M$ as
\begin{equation}
- \Delta V_{\parallel} \approx \frac{\Delta KE_{\perp}}{M V} = \frac{2G^2 m(M+m)}{b^2 V^3}.
\end{equation}

OK, now what if there are a number density per cubic parsec $n$ perturbers of mass $m$?  
We then have to integrate through the cylinder of radius $b$ to find the total
effect on the galaxy of mass $M$.  We find that
\begin{equation}
-\frac{dV}{dt} = \int_{b_{\mathrm{min}}}^{b_{\mathrm{max}}} n V \frac{2G^2 m (M+m)}{b^2 V^3} 2\pi b db = \frac{4\pi G^2(M+m)}{V^2} n m \ln \Lambda
\end{equation}
\noindent
where $\Lambda\equiv b_{\mathrm{max}}/b_{\mathrm{min}}$.

Some interesting things to note
\begin{enumerate}
\item The slower the galaxy $M$ moves, the larger its deceleration.
\item If $V\ll\sigma$, where $\sigma$ is the velocity dispersion of perturbers, we find that $dV/dt\propto -V$.  This is the same as what happens to a parachutist.
\end{enumerate}




\end{document}