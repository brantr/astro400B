\documentclass[]{article}
\usepackage[margin=1.0in]{geometry}
\usepackage{amssymb}

%title material
\title{Astronomy 400B Homework \#2}
\author{Please Show Your Work for Full Credit}
\date{Due Feb 26, 2015 by 9:35am}


%include latex definitions

%average
\newcommand{\ave}[1]{\langle#1\rangle}
%radian
\newcommand{\rad}{\mathrm{rad}}

%astronomical unit
\newcommand{\AU}{\mathrm{AU}}

%centimeter
\newcommand{\cm}{\mathrm{cm}}

%meter
\newcommand{\m}{\mathrm{m}}

%kilometer
\newcommand{\km}{\mathrm{km}}

%parsec
\newcommand{\pc}{\mathrm{pc}}

%kiloparsec
\newcommand{\kpc}{\mathrm{kpc}}

%megaparsec
\newcommand{\Mpc}{\mathrm{Mpc}}

%gigaparsec
\newcommand{\Gpc}{\mathrm{Gpc}}

%light year
\newcommand{\ly}{\mathrm{ly}}

%second
\newcommand{\s}{\mathrm{s}}
\newcommand{\yr}{\mathrm{yr}}
\newcommand{\Gyr}{\mathrm{Gyr}}

%solar mass
\newcommand{\Msun}{M_{\odot}}

%grams
\newcommand{\g}{\mathrm{g}}

%erg
\newcommand{\erg}{\mathrm{erg}}

%solar luminosity
\newcommand{\Lsun}{L_{\odot}}

%jansky 
\newcommand{\Jy}{\mathrm{Jy}}

%flux density
\newcommand{\Fnu}{F_{\nu}}
\newcommand{\Flambda}{F_{\lambda}}
\newcommand{\Lnu}{L_{\nu}}

%hertz 
\newcommand{\Hz}{\mathrm{Hz}}

%angstrom
\newcommand{\Ang}{\mathrm{\r{A}}}

%solar radius
\newcommand{\Rsun}{R_{\odot}}

%stefan-boltzmann
\newcommand{\sigmaSB}{\sigma_{\mathrm{SB}}}

%boltzmann
\newcommand{\kB}{k_{\mathrm{B}}}

%kelvin
\newcommand{\K}{\mathrm{K}}

%T_bandpass
\newcommand{\TBP}{T_{\mathrm{BP}}}

%magnitude
\newcommand{\mg}{\mathrm{mag}}


%arcsecond
\newcommand{\arcsec}{\mathrm{arcsec}}

%begin the document
\begin{document}

%make the title, goes after document begins
\maketitle

\section{Sparke \& Gallagher Problem 2.20}

Consider the spherical density distribution $\rho_H(r)$ with
\begin{equation}
4\pi G\rho_H(r) = \frac{V_H^2}{r^2 +a_H^2},
\end{equation}
where $V_H$ and $a_H$ are constants; what is the mass $M(<r)$ contained within
a radius $r$? Use the equation
\begin{equation}
M(<R) = RV^2/G
\end{equation}
to show that the speed $V(r)$ of a circular orbit at radius $r$
is given by
\begin{equation}
V^2(r) = V_H^2[1-(a_H/r)\arctan(r/a_H)],
\end{equation}
\noindent
and sketch $V(r)$ as a function of radius. This density law is sometimes
used to represent the mass of a galaxy's dark halo -- why?

\section{Sparke \& Gallagher Problem 2.24}

We can estimate the size of an HII region around a massive star that radiates $S_\star$ photons with energy above 13.6eV each second. Assume that the
gas within radius $r_\star$ absorbs all these photons, becoming almost
completely ionized so that $n_e \approx n_H$, the density of H nuclei. In a
steady state atoms recombine as fast as they are ionized, so the star ionizes
a mass of gas $M_g$, where
\begin{equation}
S_\star = (4r_\star^3/3)n_H^2\alpha(T_e) = (M_g/m_p)n_H\alpha(T_e).
\end{equation}
Use the equation
\begin{equation}
-\frac{dn_e}{dt} = n_e^2 \alpha(T_e)~~\mathrm{with}~~\alpha(T_e) \approx 2\times 10^{-13}\left(\frac{T_e}{10^4~\K}\right)^{-3/4}~\cm^3~\s^{-1}
\end{equation}
\noindent
to show that a mid-O star radiating $S_\star=10^{49}~\s^{-1}$ into gas of
density $10^{3}~\cm^{-3}$ creates an HII region of radius $0.67~\pc$,
containing $\sim30\Msun$ of gas (assume that $T_e=10^4\K$). What
is $r_\star$ if the density is ten times larger? Show that only a tenth
as much gas is ionized. How large is the HII region around a B1
star with $n_H=10^3 \cm^{-3}$ but only $S_\star = 3\times10^{47}~\s^{-1}$?

\pagebreak

\section{Sparke \& Gallagher Problem 3.7}

The {\it Navarro-Frenk-White} (NFW) model describes the
halos of cold dark matter that form in cosmological simulations.
Show that the potential corresponding to the density
\begin{equation}
\rho_{NFW}(r) = \frac{\rho_N}{(r/a_N)(1+r/a_N)^2}~\mathrm{is}~\Phi_{NFW}(r) = -\sigma_N^2\frac{\ln(1+r/a_N)}{(r/a_N)},
\end{equation}
\noindent
where $\sigma_N^2 = 4\pi G \rho_N a_N^2$. The density rises steeply
at the center, but less so than in the singular isothermal sphere;
at large radii $\rho(r)\propto r^{-3}$. Show that the speed
$V$ of a circular orbit at radius $r$ is given
by
\begin{equation}
V^2(r) = \sigma_N^2\left[\frac{\ln(1+r/a_N)}{(r/a_N)} - \frac{1}{(1+r/a_N)}\right].
\end{equation}

\section{Sparke \& Gallagher Problem 3.12}

Show that for the Plummer sphere model with density profile
\begin{equation}
\rho_P(r) = \frac{3a_P^2}{4\pi}\frac{M}{(r^2 + a_P^2)^{5/2}}
\end{equation}
\noindent
and potential
\begin{equation}
\Phi_P(r) = -\frac{GM}{\sqrt{r^2 +a_P^2}}
\end{equation}
\noindent
the potential energy is
\begin{equation}
PE = - \frac{3\pi}{32} \frac{GM^2}{a_P}.
\end{equation}


\end{document}