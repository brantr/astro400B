\documentclass[12pt]{article}
\usepackage[margin=1.0in]{geometry}
\usepackage{amssymb}

%title material
\title{Astronomy 400B Lecture 1: Review of Stellar Astrophysics}
\author{Brant Robertson}
\date{January 5, 2015}


%include latex definitions

%average
\newcommand{\ave}[1]{\langle#1\rangle}
%radian
\newcommand{\rad}{\mathrm{rad}}

%astronomical unit
\newcommand{\AU}{\mathrm{AU}}

%centimeter
\newcommand{\cm}{\mathrm{cm}}

%meter
\newcommand{\m}{\mathrm{m}}

%kilometer
\newcommand{\km}{\mathrm{km}}

%parsec
\newcommand{\pc}{\mathrm{pc}}

%kiloparsec
\newcommand{\kpc}{\mathrm{kpc}}

%megaparsec
\newcommand{\Mpc}{\mathrm{Mpc}}

%gigaparsec
\newcommand{\Gpc}{\mathrm{Gpc}}

%light year
\newcommand{\ly}{\mathrm{ly}}

%second
\newcommand{\s}{\mathrm{s}}
\newcommand{\yr}{\mathrm{yr}}
\newcommand{\Gyr}{\mathrm{Gyr}}

%solar mass
\newcommand{\Msun}{M_{\odot}}

%grams
\newcommand{\g}{\mathrm{g}}

%erg
\newcommand{\erg}{\mathrm{erg}}

%solar luminosity
\newcommand{\Lsun}{L_{\odot}}

%jansky 
\newcommand{\Jy}{\mathrm{Jy}}

%flux density
\newcommand{\Fnu}{F_{\nu}}
\newcommand{\Flambda}{F_{\lambda}}
\newcommand{\Lnu}{L_{\nu}}

%hertz 
\newcommand{\Hz}{\mathrm{Hz}}

%angstrom
\newcommand{\Ang}{\mathrm{\r{A}}}

%solar radius
\newcommand{\Rsun}{R_{\odot}}

%stefan-boltzmann
\newcommand{\sigmaSB}{\sigma_{\mathrm{SB}}}

%boltzmann
\newcommand{\kB}{k_{\mathrm{B}}}

%kelvin
\newcommand{\K}{\mathrm{K}}

%T_bandpass
\newcommand{\TBP}{T_{\mathrm{BP}}}

%magnitude
\newcommand{\mg}{\mathrm{mag}}


%arcsecond
\newcommand{\arcsec}{\mathrm{arcsec}}

%begin the document
\begin{document}

%make the title, goes after document begins
\maketitle

%distance measures
\section{Distance Measures}

Astronomy is a science based on relative measurements, 
especially for angular separations on the sky and apparent brightness.
The {\it angular separation} $\alpha$ of objects on the sky
is often measured
using the {\it small angle} formula, since the relation between
the physical line-of-sight distance $d$ and the physical separation 
$\Delta x$ in the plane of the sky in terms of {\it radians} ($\rad$) is
\begin{equation}
\label{eqn:small_angle}
\frac{\Delta x}{d} = \sin \alpha \approx \alpha (\ll 1 \rad).
\end{equation}
Angular separations can be measured in degrees (deg; 360 deg around the sky = $2\pi~\rad$), 
arcmin ('; = 1/60 deg),
and arcseconds (''; =1/60 arcmin).

Distances on the scale of solar systems are often given in terms
of the {\it Astronomical Unit} ($\AU$), which is the mean orbital
radius of Earth about the Sun (in {\it centimeters}; $\cm$):
\begin{equation}
\label{eqn:AU}
1 \AU = 1.49597871 \times 10^{13} \cm.
\end{equation}
\noindent
The {\it parsec} ($\pc$) unit is defined to be the line-of-sight distance $d=1 \pc$
at which two objects separated by a physical distance $\Delta x=1\AU$ in the
plane of the sky have an apparent angular separation on the sky of $\alpha=1''$.
In terms of other units,
\begin{equation}
\label{eqn:parsec}
1\pc = 2.06264806 \times 10^{5} \AU = 3.08567758 \times 10^{18} \cm = 3.2616334~\ly
\end{equation}
\noindent
where a {\it light year} ($\ly$) is
\begin{equation}
1\ly = 9.4605284\times10^{17}\cm
\end{equation}
\noindent
and is defined relative to the {\it speed of light} ($c$)
\begin{equation}
\label{eqn:speed_of_light}
c = 2.9979245800\times10^{10} \cm~\s^{-1}.
\end{equation}
\noindent
Note that astronomers typically prefer $\pc$ to $\ly$. We also
use the units {\it megaparsecs} ($\Mpc = 10^6 \pc$) and {\it gigaparsecs}
($\Gpc = 10^9 \pc$).

%mass, luminosity and temperature
\section{Stellar Mass, Luminosity, Flux, Radius, and Temperature}
Stellar masses are often expressed in terms of a {\it solar mass} ($\Msun$; mass of the Sun)
\begin{equation}
1\Msun = 1.9891\times10^{33}\g
\end{equation}
\noindent
in grams ($\g$). The most massive stars are $\approx100\Msun$ while the
least massive stars are $\approx0.075\Msun$.

Stellar luminosities (ergs of energy emitted per second; $1\erg = 1~\g~\cm^{2}~\s^{-2}$) are also 
often expressed in terms of the total (bolometric) {\it solar luminosity} ($\Lsun$; luminosity of the Sun)
\begin{equation}
1\Lsun = 3.846\times10^{33} \erg~\s^{-1}
\end{equation}
\noindent
Stars range in luminosity from $10^{6}\Lsun$ to less than $10^{-4}\Lsun$.

The {\it flux} $F$ is the energy per unit second per unit area ($\erg~\s^{-1}~\cm^{-2}$)
that is received from an object a distance $d$ away, and is
given by the {\it inverse square law}
\begin{equation}
\label{eqn:inverse_square}
F = \frac{L}{4\pi d^{2}}.
\end{equation}
\noindent

The {\it flux density} ($\Fnu \equiv dF/d\nu$) of an object is best measured in {\it janskys}
\begin{equation}
1\Jy = 10^{-23} \erg~\s^{-1}~\cm^{-2}~\Hz^{-1}.
\end{equation}
\noindent
Note that the {\it hertz} ($1\Hz = 1 s^{-1}$) is the unit of frequency $\nu$.
Astronomers will also use a flux density ($\Flambda = dF/d\lambda$) 
defined relative to wavelength. Typically, the units of $\Flambda$
are in $\erg~\cm^{-2}~\s^{-1}~\Ang^{-1}$, where $\Ang$ is the {\it angstrom}
($1\Ang = 10^{-8}\cm$). The wavelength and frequency of light are related by
$c = \nu \lambda$, and the two flux densities are therefore 
related by $\Flambda = (c/\lambda^2)\Fnu$.
The total flux is related to the flux density by
\begin{equation}
F = \int \Fnu d \nu = \int \Flambda d\lambda.
\end{equation}
\noindent
If the distance of an object is known, astronomers will also use the term
{\it luminosity density} (e.g., $\Lnu = dL/d\nu$).

The luminosity, radius, and temperature of an object are related. For stellar
objects, we often use the {\it solar radius} ($\Rsun$) to express sizes
\begin{equation}
1\Rsun = 6.995 \times 10^{10} \cm
\end{equation}
\noindent
If the luminosity $L$ and radius $R$ of an object are known, we can define
the effective temperature $T$ in {\it Kelvin} ($K$)
of an object through the {\it Stefan-Boltzmann Law}
\begin{equation}
L = 4\pi R^{2}\sigmaSB T^{4}
\end{equation}
\noindent
where the {\it Stefan-Boltzmann constant} is
\begin{equation}
\sigmaSB = 5.670373\times^{-5} \erg~\cm^{-2}~\s^{-1}~\K^{-4}.
\end{equation}
The Stefan-Boltzmann constant can be computed theoretically in
terms of the speed of light, the {\it Boltzmann constant}
\begin{equation}
\kB = 1.380658\times10^{-16} \erg~\K^{-1}
\end{equation}
\noindent
and the {\it Planck constant}
\begin{equation}
h = 6.6260755\times10^{-27}\erg~\s,
\end{equation}
\noindent
which gives the definition
\begin{equation}
\sigmaSB \equiv \frac{2\pi^5 \kB^4}{15 h^3 c^2}.
\end{equation}

The effective temperature is typically a measure of the temperature
of the star at its {\it photosphere}, or the radius where the optical
depth of the star's atmosphere is $\tau\approx1$ and it becomes opaque. 
For the Sun,
$T\approx5780\K$. The temperature is also related to the wavelength
at the peak of the black body curve via {\it Wien's Displacement Law}
\begin{equation}
\lambda_{\mathrm{max}} = \frac{2.897756\times10^{7} \Ang~\K}{T},
\end{equation}
which gives $\lambda_{\mathrm{max}}\approx5000\Ang$ (yellow)
for the Sun.

\subsection{The Main Sequence}

Based on their temperatures, stars are classified into {\it spectral types}.
The spectral types are OBAFGKMLTY, and are usually subclassified into {\it early}
(OBA) and {\it late} (FGKM) type stars. O type stars are hot ($T>3\times10^4\K$) while
M type stars are relatively cool ($M\approx3000\K$), and each type is
subsubclassified by decreasing temperature
from $0-9$ (e.g., the Sun is a G2 star). Deviations from a black body are
determined by the atomic and molecular species absorption in the stellar atmospheres,
with strong Hydrogen absorption in A star spectra and strong molecular bands appearing
in M star spectra.

{\bf See Figure 1.1 of Sparke and Gallagher.}

The {\it main sequence} of stars traces a band in the diagram of luminosity and
temperature (the {\it Hertzsprung-Russell or HR diagram}), and consists of stars that
shine by burning hydrogen through nuclear fusion. Main sequence stars range in
radius from about $R\approx0.1\Rsun$ to $R\approx25\Rsun$, with a scaling of
\begin{equation}
R\sim \Rsun\left(\frac{M}{\Msun}\right)^{0.7}.
\end{equation}
\noindent
The corresponding scaling of stellar luminosity near the main sequence is
\begin{equation}
\label{eqn:luminosity_vs_mass}
L\sim\Lsun\left(\frac{M}{\Msun}\right)^{\beta}
\end{equation}
\noindent
with $\beta\approx5$ for $M\lesssim\Msun$ and 
$\beta\approx3.9$ for $\Msun\lesssim M \lesssim 10\Msun$. For stars
with $M\gtrsim10\Msun$, $L\sim50\Lsun(M/\Msun)^{2.2}$.

{\it Giant} and {\it supergiant} stars have evolved off the main sequence, and
have radii and luminosities that are typically much larger than main sequence
stars of the same effective temperature. Supergiants can have radii of $1000\Rsun$
and luminosities of $10^{6}\Rsun$.

{\it White dwarf} stars are not on the main sequence, but instead are the
remains of the core of an evolved star that is supported through electron
degeneracy pressure. White dwarfs have typical sizes of $0.01\Rsun$.

{\it Neutron stars} are the remnants of certain supernovae that leave behind
a core supported by neutron degeneracy pressure. Neutron stars have radii 
of only $R\sim10\mathrm{km}$, about the size of a large city.

{\bf See Figure 1.4 of Sparke and Gallagher.}

The main sequence lifetime of a star is predominately determined by its
mass-to-light ratio, since hydrogen mass is being burned into helium at a 
rate that must fuel the stellar luminosity. For $\ave{\beta} \approx 3.5$ in
Equation \ref{eqn:luminosity_vs_mass}, we have that
\begin{equation}
\label{eqn:main_sequence_lifetime_basic}
\tau_{\mathrm{MS}} = \tau_{\mathrm{MS},\odot}\frac{M/L}{\Msun/\Lsun} \sim 10\Gyr\left(\frac{M}{\Msun}\right)^{-2.5}\sim 10\Gyr\left(\frac{L}{\Lsun}\right)^{-5/7}.
\end{equation}
Sparke and Gallagher give a more accurate approximation as
\begin{equation}
\label{eqn:main_sequence_lifetime}
\log(\tau_{\mathrm{MS}}/1\Gyr) = 1.015 - 3.49\log(M/\Msun) +0.83(\log[M/\Msun])^2
\end{equation}
\noindent
(note the factor of 10 error in Sparke and Gallagher Equation 1.9).

\subsection{Chemical Abundance}

The presence of atomic and molecular features in the spectra of stars enable us
to measure the amount of metals (elements heavier than Helium) in their atmospheres.
We typically measure the metallicity $Z$ on a log scale relative to the solar abundance,
with the definition
\begin{equation}
[A/B]\equiv \log \left[ \frac{(\mathrm{number\,\,of\,\,A\,\,atoms/number\,\,of\,\,B\,\,atoms})_{\star}}{(\mathrm{number\,\,of\,\,A\,\,atoms/number\,\,of\,\,B\,\,atoms})_{\odot}} \right].
\end{equation}
For the Sun, the total metallicity is $[Z/\mathrm{H}] = 0$ by definition, while for a star with $10\times$ the
solar metallicity we have $[Z/\mathrm{H}] = 1$. Often, astronomers abbreviate metallicity 
with $[\mathrm{Fe}/\mathrm{H}]$ even though O is typically the most abundant metal by number. 
The relative abundance of metals in stars is set by a combination of {\it Big Bang Nucleosynthesis}
that determined the initial abundance of H, He, and Li (primarily) and {\it Stellar Nucleosynthesis}
that determined the abundance of heavier elements through a variety of stellar fusion and supernovae-related
processes. The presence of metals in a star can effect its evolution by increasing the atmospheric opacity.
Low metallicity stars have lower opacity in their atmospheres, leading to a denser, hotter, and bluer star
at fixed mass.

{\bf See Figure 1.3 of Sparke and Gallagher.}

\subsection{Typical life of a star}

The typical star goes through several phases during its life:

\begin{enumerate}

\item {\it Pre-main sequence}. The initial formation of stars is not well-understood, but
the general picture involves a collapsing gas cloud with some initial angular momentum. As
the cloud cools, the angular momentum causes the gas cloud to form a disk. The disk eventually
forms a central bound object that continues to collapse roughly spherically.  This cloud does
shine (in the infrared) by converting gravitational potential energy to thermal energy via a
virialization process. The pre-main sequence phase ends once the star is dense and hot enough in
its core to undergo nuclear fusion.

\item {\it Main sequence}. The hydrogen-burning phase of stellar life. All stars with masses
below about $0.6\Msun$ have main sequence lifetimes longer than the age of the universe and
remain on the main sequence.  The fate of heavier stars depends on their mass.

\item {\it Post-main sequence for low-mass stars} ($0.6\Msun\lesssim M\lesssim 2\Msun$).
{\bf See Figure 1.4 of Sparke and Gallagher.} 
The cores of these stars contracts as hydrogen is burned, and the outer layers puff up and
become cooler as the star becomes a {\it subgiant}. The temperature around the core
becomes hot enough to burn hydrogen in a shell around the exterior of the core, leading to
the {\it red giant} phase. The helium by-product of the hydrogen shell burning falls on the
core, which causes the core to contract and get hotter. The shell burns even brighter, causing
the star to become very luminous and ascend the {\it red giant branch} of the HR diagram.
The (now) helium core eventually contracts until He can burn to C through the
{\it triple $\alpha$ process}, which happens quickly in what's called the {\it helium flash}.
The helium cores of stars with $M\lesssim 2\Msun$ have about the same helium core mass
and therefore the same luminosity at the tip of the giant branch. Once the core helium is mostly
burned the core contracts again until both He and H are burned in shells, corresponding to the
{\it asymptotic giant branch} (AGB) phase. In the AGB phase, the star looses its outer envelope
in a {\it planetary nebula} phase where the expelled gas is ionized by UV light from the hot
central core. This core is a degenerate white dwarf, which gradually emits as a black body (typically)
modified by H and He absorption. The white dwarf cooling time (to $T\sim0$) is longer than the age
of the universe, and most white dwarfs are around ten thousand $\K$ at their surface.

\item {\it Post-main sequence for intermediate-mass stars} ($2\Msun\lesssim M\lesssim 8\Msun$).
The post-main sequence lives of these stars are similar to low-mass stars, but their helium
cores can become more massive and they can therefore reach higher luminosities (and are bluer) 
than the red clump where low-mass stars congregate in their giant phases. Some of these stars
regularly pulsate as {\it Cepheid} variables that can be used as distance indicators. The
white dwarfs of these stars may contain some Ne in addition to C and O.

\item {\it Post-main sequence for high-mass stars} ($M\gtrsim 8\Msun$).
The post-main sequence lives of high-mass stars differs substantially from lower mass
populations. The cores of stars above $\sim10\Msun$ can burn through the $\alpha$ process C$\to$O,
O$\to$Ne, all the way up to Fe where the binding energy per nucleon peaks. During this heavy element
burning, the star is a blue or yellow supergiant. But after the core reaches Fe,
nuclear fusion stops since burining iron to heavier elements through fusion is endothermic
and requires (rather than provides) energy. The core then collapses, forms neutrons and hardens,
and then the outer layers bounce and are pushed via neutrinos to form a {\it Supernova Type II}.
The remnant is left as a neutron star. For stars $8\Msun \lesssim M \lesssim10\Msun$, the core probably
exceeds the {\it Chandrasekar limit} of $1.4\Msun$ where degeneracy pressure can no longer resist gravity
and the star also explodes as a Type II supernova. For stars larger than $\sim40\Msun$ less is 
known, but some shed their outer layers and are seen as {\it Wolf-Rayet} stars with strong winds. Some
heavier stars may have cores that collapse to form black holes.

\end{enumerate}




\end{document}
