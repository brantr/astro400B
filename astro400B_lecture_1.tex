\documentclass[]{article}
\usepackage[margin=1.0in]{geometry}
\usepackage{amssymb}

%title material
\title{Astronomy 400B Lecture 1: Review of Stellar Astrophysics}
\author{Brant Robertson}
\date{January 5, 2015}


%include latex definitions

%average
\newcommand{\ave}[1]{\langle#1\rangle}
%radian
\newcommand{\rad}{\mathrm{rad}}

%astronomical unit
\newcommand{\AU}{\mathrm{AU}}

%centimeter
\newcommand{\cm}{\mathrm{cm}}

%meter
\newcommand{\m}{\mathrm{m}}

%kilometer
\newcommand{\km}{\mathrm{km}}

%parsec
\newcommand{\pc}{\mathrm{pc}}

%kiloparsec
\newcommand{\kpc}{\mathrm{kpc}}

%megaparsec
\newcommand{\Mpc}{\mathrm{Mpc}}

%gigaparsec
\newcommand{\Gpc}{\mathrm{Gpc}}

%light year
\newcommand{\ly}{\mathrm{ly}}

%second
\newcommand{\s}{\mathrm{s}}
\newcommand{\yr}{\mathrm{yr}}
\newcommand{\Gyr}{\mathrm{Gyr}}

%solar mass
\newcommand{\Msun}{M_{\odot}}

%grams
\newcommand{\g}{\mathrm{g}}

%erg
\newcommand{\erg}{\mathrm{erg}}

%solar luminosity
\newcommand{\Lsun}{L_{\odot}}

%jansky 
\newcommand{\Jy}{\mathrm{Jy}}

%flux density
\newcommand{\Fnu}{F_{\nu}}
\newcommand{\Flambda}{F_{\lambda}}
\newcommand{\Lnu}{L_{\nu}}

%hertz 
\newcommand{\Hz}{\mathrm{Hz}}

%angstrom
\newcommand{\Ang}{\mathrm{\r{A}}}

%solar radius
\newcommand{\Rsun}{R_{\odot}}

%stefan-boltzmann
\newcommand{\sigmaSB}{\sigma_{\mathrm{SB}}}

%boltzmann
\newcommand{\kB}{k_{\mathrm{B}}}

%kelvin
\newcommand{\K}{\mathrm{K}}

%T_bandpass
\newcommand{\TBP}{T_{\mathrm{BP}}}

%magnitude
\newcommand{\mg}{\mathrm{mag}}


%arcsecond
\newcommand{\arcsec}{\mathrm{arcsec}}

%begin the document
\begin{document}

%make the title, goes after document begins
\maketitle

%distance measures
\section{Distance Measures}

Astronomy is a science based on relative measurements, 
especially for angular separations on the sky and apparent brightness.
The {\it angular separation} $\alpha$ of objects on the sky
is often measured
using the {\it small angle} formula, since the relation between
the physical line-of-sight distance $d$ and the physical separation 
$\Delta x$ in the plane of the sky in terms of {\it radians} ($\rad$) is
\begin{equation}
\label{eqn:small_angle}
\frac{\Delta x}{d} = \sin \alpha \approx \alpha (\ll 1 \rad).
\end{equation}
Angular separations can be measured in degrees (deg; 360 deg around the sky = $2\pi~\rad$), 
arcmin ('; = 1/60 deg),
and arcseconds (''; =1/60 arcmin).

Distances on the scale of solar systems are often given in terms
of the {\it Astronomical Unit} ($\AU$), which is the mean orbital
radius of Earth about the Sun (in {\it centimeters}; $\cm$):
\begin{equation}
\label{eqn:AU}
1 \AU = 1.49597871 \times 10^{13} \cm.
\end{equation}
\noindent
The {\it parsec} ($\pc$) unit is defined to be the line-of-sight distance $d=1 \pc$
at which two objects separated by a physical distance $\Delta x=1\AU$ in the
plane of the sky have an apparent angular separation on the sky of $\alpha=1''$.
In terms of other units,
\begin{equation}
\label{eqn:parsec}
1\pc = 2.06264806 \times 10^{5} \AU = 3.08567758 \times 10^{18} \cm = 3.2616334~\ly
\end{equation}
\noindent
where a {\it light year} ($\ly$) is
\begin{equation}
1\ly = 9.4605284\times10^{17}\cm
\end{equation}
\noindent
and is defined relative to the {\it speed of light} ($c$)
\begin{equation}
\label{eqn:speed_of_light}
c = 2.9979245800\times10^{10} \cm~\s^{-1}.
\end{equation}
\noindent
Note that astronomers typically prefer $\pc$ to $\ly$. We also
use the units {\it megaparsecs} ($\Mpc = 10^6 \pc$) and {\it gigaparsecs}
($\Gpc = 10^9 \pc$).

%mass, luminosity and temperature
\section{Stellar Mass, Luminosity, Flux, Radius, and Temperature}
Stellar masses are often expressed in terms of a {\it solar mass} ($\Msun$; mass of the Sun)
\begin{equation}
1\Msun = 1.9891\times10^{33}\g
\end{equation}
\noindent
in grams ($\g$). The most massive stars are $\approx100\Msun$ while the
least massive stars are $\approx0.075\Msun$.

Stellar luminosities (ergs of energy emitted per second; $1\erg = 1~\g~\cm^{2}~\s^{-2}$) are also 
often expressed in terms of the total (bolometric) {\it solar luminosity} ($\Lsun$; luminosity of the Sun)
\begin{equation}
1\Lsun = 3.846\times10^{33} \erg~\s^{-1}
\end{equation}
\noindent
Stars range in luminosity from $10^{6}\Lsun$ to less than $10^{-4}\Lsun$.

The {\it flux} $F$ is the energy per unit second per unit area ($\erg~\s^{-1}~\cm^{-2}$)
that is received from an object a distance $d$ away, and is
given by the {\it inverse square law}
\begin{equation}
\label{eqn:inverse_square}
F = \frac{L}{4\pi d^{2}}.
\end{equation}
\noindent

The {\it flux density} ($\Fnu \equiv dF/d\nu$) of an object is best measured in {\it janskys}
\begin{equation}
1\Jy = 10^{-23} \erg~\s^{-1}~\cm^{-2}~\Hz^{-1}.
\end{equation}
\noindent
Note that the {\it hertz} ($1\Hz = 1 s^{-1}$) is the unit of frequency $\nu$.
Astronomers will also use a flux density ($\Flambda = dF/d\lambda$) 
defined relative to wavelength. Typically, the units of $\Flambda$
are in $\erg~\cm^{-2}~\s^{-1}~\Ang^{-1}$, where $\Ang$ is the {\it angstrom}
($1\Ang = 10^{-8}\cm$). The wavelength and frequency of light are related by
$c = \nu \lambda$, and the two flux densities are therefore 
related by $\Flambda = (c/\lambda^2)\Fnu$.
The total flux is related to the flux density by
\begin{equation}
F = \int \Fnu d \nu = \int \Flambda d\lambda.
\end{equation}
\noindent
If the distance of an object is known, astronomers will also use the term
{\it luminosity density} (e.g., $\Lnu = dL/d\nu$).

The luminosity, radius, and temperature of an object are related. For stellar
objects, we often use the {\it solar radius} ($\Rsun$) to express sizes
\begin{equation}
1\Rsun = 6.995 \times 10^{10} \cm
\end{equation}
\noindent
If the luminosity $L$ and radius $R$ of an object are known, we can define
the effective temperature $T$ in {\it Kelvin} ($K$)
of an object through the {\it Stefan-Boltzmann Law}
\begin{equation}
L = 4\pi R^{2}\sigmaSB T^{4}
\end{equation}
\noindent
where the {\it Stefan-Boltzmann constant} is
\begin{equation}
\sigmaSB = 5.670373\times^{-5} \erg~\cm^{-2}~\s^{-1}~\K^{-4}.
\end{equation}
The Stefan-Boltzmann constant can be computed theoretically in
terms of the speed of light, the {\it Boltzmann constant}
\begin{equation}
\kB = 1.380658\times10^{-16} \erg~\K^{-1}
\end{equation}
\noindent
and the {\it Planck constant}
\begin{equation}
h = 6.6260755\times10^{-27}\erg~\s,
\end{equation}
\noindent
which gives the definition
\begin{equation}
\sigmaSB \equiv \frac{2\pi^5 \kB^4}{15 h^3 c^2}.
\end{equation}j

The effective temperature is typically a measure of the temperature
of the star at its {\it photosphere}, or the radius where the optical
depth of the star's atmosphere is $\tau\approx1$ and it becomes opaque. 
For the Sun,
$T\approx5780\K$. The temperature is also related to the wavelength
at the peak of the black body curve via {\it Wien's Displacement Law}
\begin{equation}
\lambda_{\mathrm{max}} = \frac{2.897756\times10^{7} \Ang~\K}{T},
\end{equation}
which gives $\lambda_{\mathrm{max}}\approx5000\Ang$ (yellow)
for the Sun.

Based on their temperatures, stars are classified into {\it spectral types}.
The spectral types are OBAFGKMLTY, and are usually subclassified into {\it early}
(OBA) and {\it late} (FGKM) type stars. O type stars are hot ($T>3\times10^4\K$) while
M type stars are relatively cool ($M\approx3000\K$), and each type is
subsubclassified by decreasing temperature
from $0-9$ (e.g., the Sun is a G2 star). Deviations from a black body are
determined by the atomic and molecular species absorption in the stellar atmospheres,
with strong Hydrogen absorption in A star spectra and strong molecular bands appearing
in M star spectra.

{\bf See Figure 1.1 of Sparke and Gallagher.}

The {\it main sequence} of stars traces a band in the diagram of luminosity and
temperature (the {\it Hertzsprung-Russell diagram}), and consists of stars that
shine by burning hydrogen through nuclear fusion. Main sequence stars range in
radius from about $R\approx0.1\Rsun$ to $R\approx25\Rsun$, with a scaling of
\begin{equation}
R\sim \Rsun\left(\frac{M}{\Msun}\right)^{0.7}.
\end{equation}
\noindent
The corresponding scaling of stellar luminosity near the main sequence is
\begin{equation}
L\sim\Lsun\left(\frac{M}{\Msun}\right)^{\beta}
\end{equation}
\noindent
with $\beta\approx5$ for $M\lesssim\Msun$ and 
$\beta\approx3.9$ for $\Msun\lesssim M \lesssim 10\Msun$. For stars
with $M\gtrsim10\Msun$, $L\sim50\Lsun(M/\Msun)^{2.2}$.

{\it Giant} and {\it supergiant} stars have evolved off the main sequence, and
have radii and luminosities that are typically much larger than main sequence
stars of the same effective temperature. Supergiants can have radii of $1000\Rsun$
and luminosities of $10^{6}\Rsun$.

{\it White dwarf} stars are not on the main sequence, but instead are the
remains of the core of an evolved star that is supported through electron
degeneracy pressure. White dwarfs have typical sizes of $0.01\Rsun$.

{\it Neutron stars} are the remnants of certain supernovae that leave behind
a core supported by neutron degeneracy pressure. Neutron stars have radii 
of only $R\sim10\mathrm{km}$, about the size of a large city.



\end{document}