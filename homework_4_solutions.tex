\documentclass[]{article}
\usepackage[margin=1.0in]{geometry}
\usepackage{amssymb}

%title material
\title{Astronomy 400B Homework \#4}
\author{Please Show Your Work for Full Credit}
\date{Due April 9, 2015 by 9:35am}


%include latex definitions

%average
\newcommand{\ave}[1]{\langle#1\rangle}
%radian
\newcommand{\rad}{\mathrm{rad}}

%astronomical unit
\newcommand{\AU}{\mathrm{AU}}

%centimeter
\newcommand{\cm}{\mathrm{cm}}

%meter
\newcommand{\m}{\mathrm{m}}

%kilometer
\newcommand{\km}{\mathrm{km}}

%parsec
\newcommand{\pc}{\mathrm{pc}}

%kiloparsec
\newcommand{\kpc}{\mathrm{kpc}}

%megaparsec
\newcommand{\Mpc}{\mathrm{Mpc}}

%gigaparsec
\newcommand{\Gpc}{\mathrm{Gpc}}

%light year
\newcommand{\ly}{\mathrm{ly}}

%second
\newcommand{\s}{\mathrm{s}}
\newcommand{\yr}{\mathrm{yr}}
\newcommand{\Gyr}{\mathrm{Gyr}}

%solar mass
\newcommand{\Msun}{M_{\odot}}

%grams
\newcommand{\g}{\mathrm{g}}

%erg
\newcommand{\erg}{\mathrm{erg}}

%solar luminosity
\newcommand{\Lsun}{L_{\odot}}

%jansky 
\newcommand{\Jy}{\mathrm{Jy}}

%flux density
\newcommand{\Fnu}{F_{\nu}}
\newcommand{\Flambda}{F_{\lambda}}
\newcommand{\Lnu}{L_{\nu}}

%hertz 
\newcommand{\Hz}{\mathrm{Hz}}

%angstrom
\newcommand{\Ang}{\mathrm{\r{A}}}

%solar radius
\newcommand{\Rsun}{R_{\odot}}

%stefan-boltzmann
\newcommand{\sigmaSB}{\sigma_{\mathrm{SB}}}

%boltzmann
\newcommand{\kB}{k_{\mathrm{B}}}

%kelvin
\newcommand{\K}{\mathrm{K}}

%T_bandpass
\newcommand{\TBP}{T_{\mathrm{BP}}}

%magnitude
\newcommand{\mg}{\mathrm{mag}}


%arcsecond
\newcommand{\arcsec}{\mathrm{arcsec}}

%begin the document
\begin{document}

%make the title, goes after document begins
\maketitle

\section{Sparke \& Gallagher Problem 7.1}

Suppose that gas atoms and galaxies in a group move at the same average random speed $\sigma$ along each direction. At temperature $T$, the average energy of gas particle is $3k_B T/2$, where $k_B$ is Boltzmann's constant.  If the gas is mainly ionized hydrogen, these particles are protons and electrons; show that, if the atom's kinetic energy $(3 m_p/2)\sigma^2$ is shared equally
between them, then
\begin{equation}
T \approx \frac{(m_p/2)\sigma^2}{k_B} \approx 5 \times 10^6 \left( \frac{\sigma}{300~\km~\s^{-1}}\right)^2~\K.
\end{equation}
\noindent
Hot gas in a group or cluster is usually close to this {\it virial temperature}.

\subsection{Solution}

If the atom's kinetic energy is shared between protons and electrons, then we should equate the thermal energy with half the kinetic energy.  This gives
\begin{eqnarray}
\frac{3}{2} k_B T &\approx& \frac{3}{4} m_p \sigma^2 \\
T &\approx& \frac{m_p}{2} \frac{\sigma^2}{k_B} = \frac{1.672622\times10^{-24}~\g}{2} \frac{\sigma^2}{1.380650\times10^{-16}~\erg~\K^{-1}}\\
T &\approx& 6.1\times 10^{-1} \frac{\sigma^2}{\cm^2~\s^{-2}}~\K \\
\therefore T &\approx& 5.5 \times 10^{6} \left(\frac{\sigma}{300~\km~\s^{-1}}\right)^2~\K
\end{eqnarray}

\section{Sparke \& Gallagher Problem 7.13}

If the lens $L$ is an object of mass $M_{\odot}$ at a distance $d_{\mathrm{Lens}}$ from
us, show that the Einstein radius for a star at distance $d_S = 2 d_{\mathrm{Lens}}$ is
\begin{equation}
\theta_{E} = \sqrt{\frac{R_s}{d_{\mathrm{Lens}}}} \approx 2 \times 10^{-3} \sqrt{\frac{1~\kpc}{d_{\mathrm{Lens}}}}~\mathrm{arcsec}.
\end{equation}

\subsection{Solution}

The Einstein radius is given by
\begin{equation}
\theta_{E} = \left( \frac{4 G M}{c^2} \frac{d_{LS}}{d_L d_S}\right)^{\frac{1}{2}} = \sqrt{ 2 R_s \frac{d_{LS}}{d_L d_S}},
\end{equation}
\noindent
where $R_s = 2 G M c^{-2}$ is the Schwarzschild radius (for the Sun, $R_s =  = 295343.16~\cm = 9.57\times10^{-17}~\kpc$. If $d_s = 2 d_L$, then $d_{LS} = d_L$. We then have
\begin{equation}
\theta_E = \sqrt{ 2 R_s \frac{d_{LS}}{d_L d_S}} = \sqrt{ 2 R_s \frac{d_L}{2 d_L^2}} = \sqrt{\frac{R_s}{d_L}} = \sqrt{9.57\times10^{-17} \frac{1~\kpc}{d_L}} = 9.78\times10^{-9}~\mathrm{rad}~\sqrt{\frac{1~\kpc}{d_L}}
\end{equation}
\begin{equation}
\therefore \theta_E  = 2.02\times10^{-3}~\mathrm{arcsec}~\sqrt{\frac{1~\kpc}{d_L}}.
\end{equation} 

\section{Sparke \& Gallagher Problem 7.17}

If a lens at distance $d_{\mathrm{Lens}}$ bends the light of a much more
distant galaxy, so that $d_S$ and $d_{LS}\gg d_{\mathrm{Lens}}$, show that the
critical density is
\begin{equation}
\Sigma_{\mathrm{crit}} \approx 2 \times 10^4 \left( \frac{100~\Mpc}{d_{\mathrm{Lens}}}\right)~M_{\odot}~\pc^{-2},
\end{equation}
\noindent
and that the mass projected within angle $\theta_E$ of the center is
\begin{equation}
M(<\theta_E) \approx \left(\frac{d_{\mathrm{Lens}}}{100~\Mpc}\right)\left(\frac{\theta_E}{1~\mathrm{arcsec}}\right)^2 ~ 10^{10}~M_{\odot}.
\end{equation}

\subsection{Solution}
The critical surface density is
\begin{equation}
\Sigma_{\mathrm{crit}} = \frac{c^2}{4\pi G} \frac{d_S}{d_L d_{LS}}.
\end{equation}
\noindent
If $d_{LS}\gg d_L$, then $d_S \equiv d_{LS} + d_L \approx d_{LS}$.
Then we have
\begin{eqnarray}
\Sigma_{\mathrm{crit}} &\approx& \frac{c^2}{4\pi G} \frac{d_{LS}}{d_L d_{LS}}\\
\Sigma_{\mathrm{crit}} &\approx& \frac{c^2}{4\pi G} \frac{1}{d_L}\\
\Sigma_{\mathrm{crit}} &\approx& \frac{(3\times10^5~\km~\s)^2}{4\pi \left(4.301\times10^{-3}~\km^2~\s^2~\Msun^{-1}~\pc\right)} \frac{1}{d_L}\\
\Sigma_{\mathrm{crit}} &\approx& 1.67\times10^{12}\frac{1}{d_L}~\Msun~\pc^{-1}\\
\therefore \Sigma_{\mathrm{crit}} &\approx& 1.67\times10^{4}\left(\frac{100~\Mpc}{d_L}\right)~\Msun~\pc^{-2}.
\end{eqnarray}
\noindent
With the definition that $b_E = \theta_E d_L$, the projected mass is
\begin{eqnarray}
M(<\theta_E) = \pi b_E^2 \Sigma_{\mathrm{crit}}&=& \pi \left(\theta_E d_L\right)^2 \Sigma_{\mathrm{crit}}\\
M(<\theta_E) &=& \pi \left(\theta_E d_L\right)^2 \left[ 1.67\times10^{4}\left(\frac{100~\Mpc}{d_L}\right)~\Msun~\pc^{-2}\right]\\
M(<\theta_E) &=& \pi\times10^{16}~\pc^2 \theta_E^2 \left(\frac{d_L}{100~\Mpc}\right)^2 \left[ 1.67\times10^{4}\left(\frac{100~\Mpc}{d_L}\right)~\Msun~\pc^{-2}\right]\\
M(<\theta_E) &=& 5.25\times10^{10} \theta_E^2 \left(\frac{d_L}{100~\Mpc}\right)~10^{10}\Msun\\
M(<\theta_E) &=& 5.25\times10^{10} \left(4.85\times10^{-6}~\rad\right)^2 \left(\frac{\theta_E}{1~\mathrm{arcsec}}\right)^2 \left(\frac{d_L}{100~\Mpc}\right)~10^{10}\Msun\\
M(<\theta_E) &=& 1.23 \left(\frac{\theta_E}{1~\mathrm{arcsec}}\right)^2 \left(\frac{d_L}{100~\Mpc}\right)~10^{10}\Msun\\
\therefore M(<\theta_E) &\approx&  \left(\frac{\theta_E}{1~\mathrm{arcsec}}\right)^2 \left(\frac{d_L}{100~\Mpc}\right)~10^{10}\Msun\\
\end{eqnarray}


\end{document}