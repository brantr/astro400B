\documentclass[]{article}
\usepackage[margin=1.0in]{geometry}
\usepackage{amssymb}
\usepackage{epsfig}

%title material
\title{Astronomy 400B Homework \#2 Solutions}
\author{Brant Robertson}
\date{March 2015}


%include latex definitions

%average
\newcommand{\ave}[1]{\langle#1\rangle}
%radian
\newcommand{\rad}{\mathrm{rad}}

%astronomical unit
\newcommand{\AU}{\mathrm{AU}}

%centimeter
\newcommand{\cm}{\mathrm{cm}}

%meter
\newcommand{\m}{\mathrm{m}}

%kilometer
\newcommand{\km}{\mathrm{km}}

%parsec
\newcommand{\pc}{\mathrm{pc}}

%kiloparsec
\newcommand{\kpc}{\mathrm{kpc}}

%megaparsec
\newcommand{\Mpc}{\mathrm{Mpc}}

%gigaparsec
\newcommand{\Gpc}{\mathrm{Gpc}}

%light year
\newcommand{\ly}{\mathrm{ly}}

%second
\newcommand{\s}{\mathrm{s}}
\newcommand{\yr}{\mathrm{yr}}
\newcommand{\Gyr}{\mathrm{Gyr}}

%solar mass
\newcommand{\Msun}{M_{\odot}}

%grams
\newcommand{\g}{\mathrm{g}}

%erg
\newcommand{\erg}{\mathrm{erg}}

%solar luminosity
\newcommand{\Lsun}{L_{\odot}}

%jansky 
\newcommand{\Jy}{\mathrm{Jy}}

%flux density
\newcommand{\Fnu}{F_{\nu}}
\newcommand{\Flambda}{F_{\lambda}}
\newcommand{\Lnu}{L_{\nu}}

%hertz 
\newcommand{\Hz}{\mathrm{Hz}}

%angstrom
\newcommand{\Ang}{\mathrm{\r{A}}}

%solar radius
\newcommand{\Rsun}{R_{\odot}}

%stefan-boltzmann
\newcommand{\sigmaSB}{\sigma_{\mathrm{SB}}}

%boltzmann
\newcommand{\kB}{k_{\mathrm{B}}}

%kelvin
\newcommand{\K}{\mathrm{K}}

%T_bandpass
\newcommand{\TBP}{T_{\mathrm{BP}}}

%magnitude
\newcommand{\mg}{\mathrm{mag}}


%arcsecond
\newcommand{\arcsec}{\mathrm{arcsec}}

%begin the document
\begin{document}

%make the title, goes after document begins
\maketitle

%first section
\section{Sparke \& Gallagher Problem 2.20}

The mass contained within $r$ is
\begin{eqnarray}
M(<r) &=& \int_{0}^{r} \frac{V_H^2}{4\pi G}\frac{4\pi r'^2}{r'^2 + a_H^2}dr'\\
&=& \frac{V_H^2 r}{G}\left[1 - \frac{a_H}{r}\arctan(r/A_H)\right].
\end{eqnarray}
\noindent
Since $V^2(r) = GM(<r)/r$, we have
\begin{equation}
V^2(r) = \frac{GM(<r)}{r} = V_H^2\left[1 - \frac{a_H}{r}\arctan(r/a_H)\right].
\end{equation}
\noindent
This is sometimes used to model a dark halo because the rotation curve flattens beyond $r/a_H\sim1$.
\begin{figure}
\begin{center}
\includegraphics[width=3.5in]{vrot.eps}
\caption{Rotation curve from Problem 2.20.}
\end{center}
\end{figure}



\section{Sparke \& Gallagher Problem 2.24}
We have that
\begin{eqnarray}
r_{\star} &=& \left(\frac{3S_\star}{4\pi n_H^2 \alpha(T_e)}\right)^{1/3}\\
&=& \left(\frac{3\times 10^{49}~\s^{-1}}{4\pi (10^3~\cm^{-3})^2 (2.6\times10^{-13}~\cm^{3}~\s^{-1})}\right)^{1/3}\\
&=&\left(\frac{3\times10^{49}~\s^{-1}}{3.267\times10^{-6}~\cm^{-3}~\s^{-1}}\right)^{1/3}\\
&=& 2.0852\times10^{18}~\cm \times \frac{1\pc}{3.08568\times10^{18}~\cm} = 0.675~\pc
\end{eqnarray}
\noindent
The mass within this radius is
\begin{eqnarray}
M_g &=& \frac{4\pi}{3} m_p n_H r_{\star}^3\\
&=& 4.189 \times (1.672622\times10^{-24}~\g) \times(10^3~\cm^{-3})\times(2.0852\times10^{18}~\cm)^3\\
&=&6.35\times10^{34}~\g \times \left(\frac{1~\Msun}{1.988920\times10^{33}~\g}\right)\sim 32\Msun
\end{eqnarray}
\noindent
If $n_H\sim10^4~\cm^{-3}$ we have
\begin{eqnarray}
r_{\star} 
&=& \left(\frac{3\times 10^{49}~\s^{-1}}{4\pi (10^4~\cm^{-3})^2 (2.6\times10^{-13}~\cm^{3}~\s^{-1})}\right)^{1/3}\\
&=&\left(\frac{3\times10^{49}~\s^{-1}}{3.267\times10^{-4}~\cm^{-3}~\s^{-1}}\right)^{1/3}\\
&=& 4.4932\times10^{17}~\cm \times \frac{1\pc}{3.08568\times10^{18}~\cm} \approx 0.15~\pc,
\end{eqnarray}
\noindent
and for the mass we have
\begin{eqnarray}
M_g &=& \frac{4\pi}{3} m_p n_H r_{\star}^3\\
&=& 4.189 \times (1.672622\times10^{-24}~\g) \times(10^4~\cm^{-3})\times(4.4932\times10^{17}~\cm)^3\\
&=&6.36\times10^{33}~\g \times \left(\frac{1~\Msun}{1.988920\times10^{33}~\g}\right)\sim 3.2\Msun
\end{eqnarray}
\noindent
Lastly, if instead $S_\star\sim3\times10^{47}~\s^{-1}$ and $n_H\approx10^3\cm^{-3}$ then
\begin{eqnarray}
r_{\star} 
&=& \left(\frac{3\times 3\times10^{47}~\s^{-1}}{4\pi (10^3~\cm^{-3})^2 (2.6\times10^{-13}~\cm^{3}~\s^{-1})}\right)^{1/3}\\
&=&\left(\frac{9\times10^{47}~\s^{-1}}{3.267\times10^{-6}~\cm^{-3}~\s^{-1}}\right)^{1/3}\\
&=& 6.48\times10^{17}~\cm \times \frac{1\pc}{3.08568\times10^{18}~\cm} \approx 0.21~\pc.
\end{eqnarray}

\section{Sparke \& Gallagher Problem 3.7}
The NFW density is
\begin{equation}
\rho_{NFW}(r) = \frac{\rho_N}{(r/a)(1+r/a)^2},
\end{equation}
\noindent
The potential for a spherical system is
\begin{equation}
\Phi_{NFW}(r) = -\left[\frac{GM_{NFW}(<r)}{r} + 4\pi G\int_{r}^{\infty}\rho_{NFW}(r')r'dr'\right]
\end{equation}
\noindent
The first term is
\begin{equation}
\frac{GM_{NFW}(<r)}{r} = \frac{4\pi G}{r} \int_{0}^{r} \rho_{NFW}(r')r'^2 dr' = \frac{4\pi G \rho_{N} a^2}{r/a}\left[ \ln\left(1+\frac{r}{a}\right) - \frac{r}{r+a}\right]
\end{equation}
\noindent
The second term is
\begin{equation}
4\pi G \int_{r}^{\infty} \rho_{NFW}(r')r'dr' = \frac{4\pi G \rho_{N} a^3}{r+a}= \frac{4\pi G \rho_{N} a^2}{r/a}\frac{r}{r+a}
\end{equation}
\noindent
So we have
\begin{eqnarray}
\Phi_{NFW}(r) &=& -\left[\frac{GM_{NFW}(<r)}{r} + 4\pi G\int_{r}^{\infty}\rho_{NFW}(r')r'dr'\right]\\
&=& - \frac{4\pi G \rho_{N} a^2}{r/a}\left[ \ln\left(1+\frac{r}{a}\right) - \frac{r}{r+a} + \frac{r}{r+a}\right]\\
&=& -\sigma_N^2 \left[ \frac{\ln\left(1+\frac{r}{a}\right)}{r/a}\right]
\end{eqnarray}
%According to Poisson's equation, we have
%\begin{equation}
%\nabla^2 \Phi = \frac{1}{r^2}\frac{\partial}{\partial r}\left(r^2 \frac{\partial}{\partial r}\right) \Phi = 4 \pi G \rho_{NFW}(r)
%\end{equation}
%so we have to integrate a couple of times to find the potential
%\begin{eqnarray}
%r^2 \frac{\partial}{\partial r}\Phi(r) &=& 4\pi G\int_{0}^{r} \rho(r') r'^2 dr' = 4\pi G \rho_N a^3 \left[ -\frac{r}{a+r} + \ln\left(1+\frac{r}{a}\right)\right]\\
%\Phi(r) &=& 4\pi G \rho_N a^3 \int_{0}^{r} r'^{-2} \left[ -\frac{r'}{a+r'} + \ln\left(1+\frac{r'}{a}\right)\right] dr'\\
%\Phi(r) &=& -4\pi G \rho_N a^2 \left[ \frac{\ln(1+r/a)}{r/a} - 1 \right] + C
%\end{eqnarray}
%\noindent
%Requiring $\Phi(r)\to0$ as $r\to\infty$ we can set $C$ such that
%\begin{equation}
%\Phi(r) = -\sigma_N^2  \frac{\ln(1+r/a)}{r/a}~~\mathrm{with}~~\sigma_N^2 = 4\pi G \rho_N a^2
%\end{equation}
\noindent
To find the circular velocity, we have
\begin{eqnarray}
V^2(r) &=& \frac{G}{r} \int_{0}^{r} 4\pi \rho_{NFW}(r')r'^2 dr' = \frac{4\pi G}{r} \int_{0}^{r} \frac{r'^2dr'}{(r'/a)(1+r'/a)^2} \\
V^2(r) &=& \frac{4\pi G a^3}{r} \left[ -\frac{r}{a+r} + \ln(1+r/a)\right] = \sigma_N^2 \left[  \frac{\ln(1+r/a)}{r/a} -\frac{1}{1+r/a} \right]
\end{eqnarray}

\section{Sparke \& Gallagher Problem 3.12}

To find the potential energy from the density and potential, we have
\begin{eqnarray}
PE &=& \frac{1}{2} \int_0^{\infty} \rho(r) \Phi(r) dV\\
PE &=& -\frac{3GM^2 a^2}{8\pi} \int_0^{\infty}  \frac{4\pi r^2 dr}{(r^2 +a^2)^{3}}\\
PE &=& -\frac{3GM^2 a^2}{2} \int_0^{\infty}  \frac{ r^2 dr}{(r^2 +a^2)^{3}}\\
PE &=& -\frac{3GM^2 a^2}{2} \frac{\pi}{16a^3}\\
PE &=& -\frac{3GM^2 \pi}{32 a}
\end{eqnarray}

\end{document}