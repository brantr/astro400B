\documentclass[]{article}
\usepackage[margin=1.0in]{geometry}
\usepackage{amssymb}

%title material
\title{Astronomy 400B Lecture 3: Intro to Galaxies and Cosmology}
\author{Brant Robertson}
\date{January 5, 2015}


%include latex definitions

%average
\newcommand{\ave}[1]{\langle#1\rangle}
%radian
\newcommand{\rad}{\mathrm{rad}}

%astronomical unit
\newcommand{\AU}{\mathrm{AU}}

%micron
\newcommand{\mum}{\mu\mathrm{m}}

%millimeter
\newcommand{\mm}{\mathrm{mm}}

%centimeter
\newcommand{\cm}{\mathrm{cm}}

%meter
\newcommand{\m}{\mathrm{m}}

%kilometer
\newcommand{\km}{\mathrm{km}}

%parsec
\newcommand{\pc}{\mathrm{pc}}

%kiloparsec
\newcommand{\kpc}{\mathrm{kpc}}

%megaparsec
\newcommand{\Mpc}{\mathrm{Mpc}}

%gigaparsec
\newcommand{\Gpc}{\mathrm{Gpc}}

%light year
\newcommand{\ly}{\mathrm{ly}}

%second
\newcommand{\s}{\mathrm{s}}
\newcommand{\yr}{\mathrm{yr}}
\newcommand{\Gyr}{\mathrm{Gyr}}

%solar mass
\newcommand{\Msun}{M_{\odot}}

%grams
\newcommand{\g}{\mathrm{g}}

%erg
\newcommand{\erg}{\mathrm{erg}}

%solar luminosity
\newcommand{\Lsun}{L_{\odot}}

%jansky 
\newcommand{\Jy}{\mathrm{Jy}}

%flux density
\newcommand{\Fnu}{F_{\nu}}
\newcommand{\Flambda}{F_{\lambda}}
\newcommand{\Lnu}{L_{\nu}}

%hertz 
\newcommand{\Hz}{\mathrm{Hz}}

%angstrom
\newcommand{\Ang}{\mathrm{\r{A}}}

%solar radius
\newcommand{\Rsun}{R_{\odot}}

%stefan-boltzmann
\newcommand{\sigmaSB}{\sigma_{\mathrm{SB}}}

%boltzmann
\newcommand{\kB}{k_{\mathrm{B}}}

%kelvin
\newcommand{\K}{\mathrm{K}}

%T_bandpass
\newcommand{\TBP}{T_{\mathrm{BP}}}

%magnitude
\newcommand{\mg}{\mathrm{mag}}


%arcsecond
\newcommand{\arcsec}{\mathrm{arcsec}}

%critical density
\newcommand{\rhoc}{\rho_{\mathrm{crit}}}

%proton mass
\newcommand{\mproton}{m_{\mathrm{p}}}

%electron volt
\newcommand{\eV}{\mathrm{eV}}

%kiloelectron volt
\newcommand{\keV}{\mathrm{keV}}

%megaelectron volt
\newcommand{\MeV}{\mathrm{MeV}}

%gigaelectron volt
\newcommand{\GeV}{\mathrm{GeV}}

%vector velocity
\newcommand{\vv}{\mathbf{v}}

%vector radius
\newcommand{\vr}{\mathbf{r}}

%vector position
\newcommand{\vx}{\mathbf{x}}

%vector force
\newcommand{\vF}{\mathbf{F}}

%vector surface
\newcommand{\vS}{\mathbf{S}}

%vector angular momentum
\newcommand{\vL}{\mathbf{L}}

%script I -- integral of motion
\newcommand{\cI}{\mathcal{I}}

%effective potential
\newcommand{\Phieff}{\Phi_\mathrm{eff}}


%begin the document
\begin{document}

%make the title, goes after document begins
\maketitle

%Celestial Coordinates
\section{Galaxy Classification}

Galaxies display a rich variety of morphologies, shapes, sizes, colors, and
luminosities. Based on these bulk properties, astronomers have tried to
develop classification schemes that group galaxies that may have similar
formation mechanisms. The most famous classification scheme is owes to 
{\it Edwin Hubble}, and is sometimes called the {\it tuning fork}.

{\bf See Figure 1.11 of Sparke and Gallagher}

The Hubble scheme separates galaxies primarily into {\it ellipticals}
(aka early type) and
{\it spirals} (aka late type), with an intermediate grouping
called {\it lenticulars}.

Ellipticals, also called {\it spheroidals}, are typically massive
galaxies with stars on non-circular orbits and with low gas content.
Ellipticals tend to be red, old, and lack large amounts of
on-going star formation.
The largest ellipticals, called cD galaxies, are located at the
center of clusters of galaxies. Ellipticals are classified with
E, followed by a number that reflects the ratio of the semi
major to semi minor axes.  There are also {\it dwarf elliptical}
(dE) and {\it dwarf spheroidal} (dSph) galaxies, that have lower
mass. We now think these classifications represent a mass sequence
rather than an evolutionary sequence.

Spirals are disk galaxies, typically containing blue, young stars,
on-going star formation and lots of gas.
Spiral galaxies are classified based on the prevalence of their
bulge, with Sa galaxies having large bulges and Sd galaxies having
very small bulges. We separately classify {\it barred spirals}
based on the presence of a noticeable bar feature.  Again, the
size of the bulge defines the range from SBa to SBd.  The Milky
Way is probably Sc or Sbc, while Andromeda is an Sb galaxy.
These classifications
also represent a mass sequence rather than an evolutionary one.
On the low mass end, there are Sm and SBm galaxies, as well as the
{\it dwarf irregular} (dIrr) systems that are star forming but
do not necessarily have well-defined disks. The ``m'' in Sm and
SBm refer to {\it Magellanic} spirals, with the prototype being
the LMC.

S0 galaxies (the lenticulars) are systems that are mostly a
bulge with a subdominant disk.

The early type vs. late type nomenclature was originally
mean to suggest an evolutionary sequence from ellipticals to
spirals. Instead, we know think that ellipticals form from
spirals in galaxy mergers that disrupt the well-defined 
circular orbits of spirals into the disordered orbital
structure of ellipticals.

\section{Galaxy Catalogues}

Very influential in the discovery and naming of large
nearby galaxies were
\begin{enumerate}

\item The Charles {\it Messier} catalogue from 1784 of 109 
objects that looked ``fuzzy'' in small telescopes. Some
of these objects were nebulae, some were globular clusters,
and some were galaxies. Famous Messier objects include the
Andromeda galaxy M31, the Triangulum galaxy M33, the M51
disk galaxy merging with a spheroidal, and the M101 Pinwheel
Galaxy.

\item In 1888, 1895, and 1908, the {\it New General Catalogue}
contained more than 7000 extended objects including many
galaxies. The catalogue was created by Dreyer, and Caroline,
William, and John Herschel.  The NGC catalogue overlaps with
the Messier catalogue (e.g., Andromeda is NGC 224).

\item Other modern catalogues include the {\it Third Reference
Catalogue} by de Vaucouleurs and the {\it Uppsala General
Catalogue} by Nilson that enumerates many galaxies in the UGC
system.
\end{enumerate}

\section{Galaxy Surface Brightness}

A galaxy with luminosity $L$ at a distance $d$ away will have
a flux $F = L/(4\pi d^2)$. If the galaxy has a physical
size $\Delta x$ in the plane of the sky and an angular
size $\theta = \Delta x/d$, then the surface brightness
$I$ will be
\begin{equation}
I \equiv \frac{F}{\theta^{2}} = \frac{L/(4\pi d^2)}{\Delta x^2 / d^2} = \frac{L}{4\pi\Delta x^2}
\end{equation}
\noindent
Unless cosmological effects come into play, the surface brightness
is independent of distance $d$. The units of surface brightness are
often given in $\mg~\arcsec^{-2}$, such that a square arcsecond area
of the galaxy has the apparent brightness of an object with the
same magnitude. Sensible astronomers also use $\Lsun\pc^{-2}$. Central
regions of galaxies reach $I_{B}\approx 18~\mg~\arcsec^{-2}$, while
the outer regions of disks are typically $I_{B}\approx25~\mg~\arcsec{-2}$.
The radius of the isophote of 25th B-band magnitude, $R_{25}$ is used as
a proxy for galaxy size, as is the {\it Holmberg radius} at the 26.5th
magnitude isophote.

\section{Sky Brightness}

The sky is bright!  In full moon, the sky is brighter than 20$\mg~\arcsec^{-2}$
in the optical and about 13$\mg~\arcsec^{-2}$ in the infrared.  Most of the
infrared sky brightness comes from spectral lines like OH in the atmosphere.
Space is typically $3-4\mg~\arcsec^{-2}$ fainter in the optical and about
$9\mg~\arcsec^{-2}$ fainter in the near IR.  For observations in the infrared,
space can't be beat.

\section{Galaxy Luminosity Function}

We can count the number density of galaxies as a function of their luminosity,
and we appropriately call this distribution the {\it luminosity function}. The
galaxy luminosity function has been found to have a shape close to a parameterized
form called the {\it Schechter} function (after Paul Schechter).  The Schechter
function provides the number density of galaxies in a differential luminosity bin
$dL$ as
\begin{equation}
\label{eqn:schechter_function_luminosity}
\Phi(L)dL = \phi_{\star} \left(\frac{L}{L_{\star}}\right)^{\alpha}\exp\left(-\frac{L}{L_{\star}}\right)\frac{dL}{L_{\star}}
\end{equation}
\noindent
where $L_{\star}$ is a characteristic luminosity of galaxies and $\phi_{\star}$ is 
a typical abundance.  Below $L_\star$, the luminosity function is a power law, and
above $L_\star$ the abundance of galaxies drops exponentially. Sometimes, astronomers
will use $1+\alpha$ as the power law exponent, so beware!

{\bf See Figure 1.16 of Sparke and Gallagher.}

It happens to be the case that $L_{\star}\approx2\times10^{10}\Lsun$, which is close
to the luminosity of the Milky Way. The typical abundance of galaxies is $\phi_\star\approx7\times10^{-3}\Mpc^{-3}$. The faint-end slope of the 2DF luminosity function is $\alpha=-0.46$.
Defined as in Equation \ref{eqn:schechter_function_luminosity}, the number of galaxies
diverges as $L\to0$ if $\alpha<-1$.

The total luminosity density provided by galaxies can be found by integrating 
Equation \ref{eqn:schechter_function_luminosity} as
\begin{equation}
\rho_{L} = \int_{0}^{\infty} \Phi(L) L dL = \phi_\star L_{\star} \Gamma(\alpha + 2)
\end{equation}
\noindent
where $\Gamma$ is the Gamma function, which for an integer $n$ is $\Gamma(n) = (n-1)!$.
It turns out that $\Gamma(1.5)\approx0.886227$, so we have that 
$\rho_{L}\approx1.25\times10^{8}\Lsun\Mpc^{-3}$.

\section{Galaxies Trace the Universal Expansion}

Galaxies are great because they are bright and easy targets for
measuring spectra. By measuring spectra we can determine the 
redshift from identified spectral lines whose rest wavelengths are
known. The redshifts can be converted into line-of-sight velocities $v$.
Sensibly, in 1929 Hubble plotted the velocities of about two dozen
galaxies versus their distances (which were very wrong), and 
found that the recessional velocity increased linearly with distance as
\begin{equation}
v = H_{0} d,
\end{equation}
\noindent
which is known as the {\it Hubble Law}.  The factor $H_0$ is known
as the {\it Hubble parameter} and typically has units 
of $\km~\s^{-1}~\Mpc^{-1}$. Current best estimates are that
$H_0\approx67~\km~\s^{-1}~\Mpc^{-1}$. We often further
parameterize the {\it Hubble parameter} in terms of
\begin{equation}
h = \frac{H_0}{100~\km~\s^{-1}~\Mpc^{-1}},
\end{equation}
\noindent
such that $h\approx0.67$. Often people will just approximate and
use $h=0.7$. Distances found from velocities depend on $h^{-2}$ and
number densities will depend on $h^{3}$, and often you will encounter
the following standard notations
\begin{equation}
R = r~h^{-1}~\Mpc~~\mathrm{(Distance)}
\end{equation}
\begin{equation}
L = l~h^{-2}~\Lsun~~\mathrm{(Luminosity)}
\end{equation}
\begin{equation}
V = v~h^{-3}~\Mpc~~\mathrm{(Volume)}
\end{equation}
\begin{equation}
N = n~h^{3}~\Mpc^{-3}~~\mathrm{(Number~density)}
\end{equation}
\noindent
When you see these notations, you need to plug in the value of the
Hubble parameter $h$ to compute a ``proper'' number.  For instance,
the book has the local luminosity density as 
$\rho_{L}\approx2\times10^{8}h~\Lsun~\Mpc^{-3}$. For $h=0.67$, the
proper luminosity density is then $\rho_L = 1.34\times10^{8}~\Lsun~\Mpc^{-3}$.
You will hear astronomers say ``with little h'' or ``including little h'' --
and sometimes they mean they have or have not included little $h$!

\subsection{Hubble Time}
You may have noticed that $H_0$ has units of inverse time. We can
define the {\it Hubble time} as 
\begin{equation}
t_{H} = \frac{1}{H_0} = 9.78h^{-1}~\Gyr = 15~\Gyr \times \left(\frac{0.67}{h}\right).
\end{equation}
\noindent
If the Hubble Law held to time $t\to0$, then $1/H_0$ would be the age of the
universe since the Big Bang. But the universe does not expand at the
same rate during its history, and $1/H_0$ is simply an approximate age
of the universe (not bad, since the universe is about $13.8~\Gyr$ old).

\subsection{Peculiar Velocities}

As one might imagine, the expansion of the universe is not the only physical
process that might induce a line-of-sight velocity for a galaxy relative to
us. For instance, the random motions of galaxies in a galaxy cluster can induce a
substantial los velocity ($\sim1500~\km~\s^{-1}$) that might change the measured
redshift for the galaxy. We call the deviation of the los velocity from the
Hubble law the {\it peculiar velocity} of the galaxy, and write
\begin{equation}
v = H_0 d + v_{\mathrm{pec}}.
\end{equation}

\section{Critical Density}

We will describe in future classes the dynamics of the expanding universe.
For now, we will discuss some salient features of the universal expansion. An
important quantity is the {\it critical density} $\rhoc$, which describes the
critical energy density that the universe must have to be geometrically flat.
From the equations of general relativity, we can determine that the critical density is
\begin{equation}
\rhoc = \frac{3 H_0^2}{8\pi G} = 1.9\times10^{-29}~h^2~\g~\cm^{-3}~=2.8\times10^{11}~h^2~\Msun~\Mpc^{-3}
\end{equation}
\noindent
For a matter dominated universe with the critical density, the age of the universe is
\begin{equation}
t_{\mathrm{u}} = \frac{2}{3H_0} \approx 10~\Gyr \times \left(\frac{0.67}{h}\right).
\end{equation}
\noindent
But our universe is dark energy dominated (see below)! 
This is useful nonetheless because it's an
exact result.  For our universe, we have to do an integral (later!) to determine the
age of the universe in terms of the Hubble parameter.

\section{Components of the Universe}

Our universe is comprised of a variety of forms of matter and energy.  Owing to
general relativity, both matter and energy provide a source for gravity and so
both effect the dynamical expansion of the universe.  Here is a brief run-down
of the different types of matter and energy in the universe:

\begin{enumerate}
\item {\it Baryonic (normal) matter}. We are made of {\it baryonic} matter, meaning
that we are comprised of forms of matter that are ultimately made from three quarks
(proton and neutrons). There are also enough electrons such that the universe is
electrically neutral, but their mass is $\sim2000\times$ less than the nucleons.
Surprising as it may seem, the universe is only about $4-5\%$ baryonic matter!
\item {\it Radiation} There are a {\bf lot} of photons in the universe! The energy
density associated with the total cosmic microwave background and the extragalactic
background light is quite small today ($<10^{-4}$ of the total energy density). At
early times the universe was actually dominated by radiation, but the fractional
contribution of radiation to the total energy density declines dramatically with
time.
\item {\it Dark matter}. Most of the matter in the universe is dark matter, which
comprises about $\sim25\%$ of the total energy density. Dark matter is non-baryonic,
and likely consists of subatomic particles called {\it Weakly Interacting Massive Particles},
or WIMPs, that only interact through the gravitational and possibly weak nuclear forces.
There are plenty of dark matter candidates from elementary particle theories, but dark
matter has not yet been detected directly via experiment.
\item {\it Dark energy}. This mysterious component of the universe as a negative
equation of state associated with it, such that a universe dominated by dark
energy accelerates its expansion as its volume increases. We think that about $70\%$
of the universal energy density is comprised of dark energy, and it is possibly in
the form of a ``cosmological constant'', meaning that the dark energy density remains
constant with time. Since the universe is expanding and the matter density declines
with volume, the universe is becoming progressively more and more dark energy dominated
as it expands. The result is that the universal expansion will continue to accelerate,
likely without a meaningful bound (the expansion will infact eventually exceed the speed
of light if the dark energy is a cosmological constant). There are natural ways of
estimating the energy density associated with a cosmological constant from fundamental
particle physics, but these methods overestimate the amount of dark energy by $\sim120$
{\it orders of magnitude}(!; depending on how you count). Needless to say, even if the
dark energy is a cosmological constant, we don't really know how to describe it in terms
of a physical model.
\item {\it Curvature}. If the universe does not have the critical density, then there
is a geometric curvature to the universe. There is an energy density associated with this
curvature that can affect the dynamics of the universe, and if the universe is not flat then
this curvature energy must reckon in the accounting for the time-dependent expansion rate.  
Fortunately (I guess), we think the universe is very close to
flat, remarkably so, and most calculations will be performed assuming the curvature energy
is zero.
\end{enumerate}

\section{The Scale Factor and Universal Expansion}

Since the universe is expanding, it is useful to define a quantity called the
{\it scale factor} $a(t)$ that relates the current separation $R(t_0) = r$ of objects to
their separation $R(t) = r\times a(t)$ at a previous time.  Defining $a(t_0)=1$, and calling
$r$ the {\it comoving} separation, we can more easily compare physical (proper)
length scales at different times. As one can infer, the Hubble parameter is related
to the current rate of change of $a(t)$:
\begin{equation}
H_0 = \frac{1}{a(t_0)}\left.\frac{da}{dt}\right|_{t=t_0} = \frac{\dot{a}(t_0)}{a(t_0)}
\end{equation}
\noindent
where we'll use the dot notation to represent a time derivative.
You can also infer that the Hubble parameter is time-dependent, such that
$H(t) = \dot{a}(t)/a(t)$.

The expansion of the universe affects light that is traveling through the
universe over an appreciable distance. As the universe expands, the wavelength
of light will increase, leading to an induced redshift. If the galaxies
recessional velocity is much less than the speed of light ($v\ll c$), then
over the light travel time $\Delta t = d/c$ the ratio of the observed to
emitted wavelength will be
\begin{equation}
\frac{\lambda_{\mathrm{obs}}}{\lambda_{\mathrm{e}}} = 1 + \frac{\Delta \lambda}{\lambda_{\mathrm{e}}} \approx 1+\frac{H_0 d}{c} = 1+ H_0 \Delta t = 1 + \frac{\dot{a}(t_0)}{a(t_0)} \Delta t
\end{equation}
\noindent
To find a more general relation, this
equation can be rewritten to relate the rate of change in wavelength in terms of
the rate of change of the scale factor as
\begin{equation}
\frac{\dot{\lambda}}{\lambda} = \frac{\dot{a}}{a}
\end{equation}
\noindent
We can integrate these equations over time
\begin{equation}
\int_{t_\mathrm{e}}^{t_\mathrm{obs}} \frac{\dot{\lambda}}{\lambda}d t = \int_{t_\mathrm{e}}^{t_\mathrm{obs}} \frac{\dot{a}}{a}d t 
\end{equation}
\begin{equation}
\ln \frac{\lambda(t_{\mathrm{obs}})}{\lambda(t_\mathrm{e})} = \ln \frac{a(t_{\mathrm{obs}})}{a(t_\mathrm{e})} 
\end{equation}
\begin{equation}
1+z = \frac{\lambda(t_{\mathrm{obs}})}{\lambda(t_\mathrm{e})} =  \frac{a(t_{\mathrm{obs}})}{a(t_\mathrm{e})} 
\end{equation}
\noindent
This equation is the {\it cosmological redshift} formula that enables us to compute the 
redshifting of light from the universal expansion as it travels from distant objects
 to us. Note that the cosmological redshift may also be used as a short hand for
 time or relative scale factor.  
 As $z\to\infty$, $t_\mathrm{e}\to0$, and $a(t_\mathrm{e})\to0$.
 The most distant galaxies now known are at $z\sim10$ and the CMB was
 emitted at $z\sim1100$.

\section{The Early Universe}

From the Wein displacement law, we know that the peak emission wavelength of a 
black body varies inversely with its temperature. The wavelength of light in 
the universe will be redshifted by the cosmological expansion, so we expect
that black body radiation filling the cosmos will have its temperature
decline as $T\propto1/a(t)$. A corollary of this relation is that at
early times, the universe was filled with very hot, energetic radiation.
Interacting photons with sufficient energies could produce matter--antimatter
pairs of particles.  Here's how it works -- basically the typical
energy of a photon in a radiation field of temperature $T$ is
\begin{equation}
\mathcal{E} = 4 \kB T.
\end{equation}
\noindent
Roughly speaking, a pair of photons could produce a proton--antiproton pair
if
\begin{equation}
\kB T \gtrsim \mproton c^2
\end{equation}
\noindent
where $\mproton$ is the mass of the proton. We typically talk about
mass energies like $\mproton c^2$ in units of the {\it electron volt} (eV),
which is the energy gained by an electron moving across a potential
difference of 1V. For reference, 
an electon volt $1\eV = 1.6\times10^{-12}~\erg$.
The rest mass energy of the proton is 
$\mproton c^2 = 0.938272 \times 10^9 \eV = 0.938272~\GeV$. It turns
out that this energy corresponds to a universal time of $<10^{-4}\s$
when the universe had a temperature $T\gg10^{13}~\K$. So before this epoch,
protons and antiprotons could be created in pairs freely.  After this epoch,
the temperature of photons were no longer high enough to produce the pairs and
the only possible process was annihilation.

\subsection{Baryon Asymmetry}

A substantial problem in modern physics is explaining why, in the context
of the above picture, there were somehow more protons that survived than
antiprotons! That problem is called {\it baryon asymmetry} and it forms 
an important area of current fundamental physics research. The size of the
asymmetry was only about one part in a billion.

\subsection{Other Processes}

Many different matter--antimatter pairs of particles can be created by 
photons provided the temperature of the radiation is large enough. 
Interesting additional processes beyond the proton--antiproton generation
include
\begin{enumerate}
\item {\it Electron--positron pairs}. The rest mass of the electron
is about $511~\keV$, and they can be generated by photons of temperature
$T\gg10^{10}~\K$. When the electron--positron pair annihilates, it can
also produce {\it electron neutrinos} (and antineutrinos) via
\begin{equation}
\mathrm{e}^{-} + \mathrm{e}^{+} \to \nu_{\mathrm{e}} + \bar{\nu}_{\mathrm{e}}
\end{equation}
\noindent
The reverse reaction is also possible.

\item {\it Neutrons}. The presence of positrons and antineutrinos enables the
production of free neutrons.  Free neutrons are unstable, decaying on 
$\sim15$ minute ($886~\s$) timescales. The relevant reactions are
\begin{equation}
\mathrm{e}^{-} + \mathrm{p} \to \mathrm{n} + \nu_{\mathrm{e}}
\end{equation}
\begin{equation}
\bar{\nu}_{\mathrm{e}} + \mathrm{p} \to \mathrm{n} + \mathrm{e}^{+}
\end{equation}
\noindent
and $\beta$-decays
\begin{equation}
\mathrm{n} \to \mathrm{p} + \mathrm{e}^{-} + \bar{\nu}_{\mathrm{e}}
\end{equation}
\noindent
The inverse processes are also allowed. In equilibrium, the ratio of
neutrons to protons is determined by their mass ratio via the equation
\begin{equation}
\frac{\mathrm{n}}{\mathrm{p}} = \exp\left(-Q/\kB T\right)
\end{equation}
\begin{equation}
Q = (m_{\mathrm{n}} - \mproton)c^2 \approx 1.293~\MeV
\end{equation}
\noindent
The production of free neutrons continued until there were few enough
energetic neutrinos to enable the forward reactions. The temperature when
conditions became unfavorable is $\kB T< 0.8~\MeV$, which corresponds to
about $1~\s$. Using the above equation, we find that the abundance of
neutrons relative to protons at this time was about 1/5.
\end{enumerate}

\subsection{Nucleosynthesis}

The free neutrons would all decay away on a 15 minute timescale, but
the formation of deuterium enabled the subsequent formation of heavier
nuclei. The deuterium reaction is
\begin{equation}
\mathrm{n} + \mathrm{p} \to \mathrm{D} + \gamma
\end{equation}
\noindent
where the photon $\gamma$ carries off the excess kinetic energy of the
neutron and proton that become bound. The deuterium could then easily
form He$^{4}$ and He$^{3}$. Additional trace amounts of Li and B also
formed, but at very small levels.

The amount of deuterium left over at the end of nucleosynthesis is 
a strong function of the amount of total baryonic material that exists
in the early universe.  Too little baryons, the collision rate of deuterium
is low and much of it would survive. With a lot of baryons, the collision
rate would be large and the amount of surviving deuterium would be very low.
Deuterium can be burned into helium in stars, but accounting for that
we can constrain the amount of baryons based on the surviving abundance
of deuterium at relatively late times.  Based on observations, we find
that as a fraction of the critical density the abundance of baryons is
\begin{equation}
0.02h^{-2} \lesssim \rho_{\mathrm{B}} / \rhoc \lesssim 0.025h^{-2}.
\end{equation}
Independent observations of properties of the CMB give a consistent
value of $\rho_{\mathrm{B}} / \rhoc = 0.02222\pm0.00023 \times h^{-2}$.

\subsection{Recombination}

When the universe cooled sufficiently such that free electrons recombined on 
protons to form neutral hydrogen, at a temperature of about 
$T\approx10^{4}~\K$, the opacity of the universe decreased dramatically and 
photons were allowed to free stream. These photons redshifted to form the
CMB that we see today as black body radiation with a temperature 
$T=2.728\pm0.002~\K$. There are a few billion photons per nucleon in the
universe in the present day. The energy density of these CMB photons is
comparable to the star light energy density in the exterior regions of the
Milky Way.

\end{document}










