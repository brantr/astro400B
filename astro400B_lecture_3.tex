\documentclass[]{article}
\usepackage[margin=1.0in]{geometry}
\usepackage{amssymb}

%title material
\title{Astronomy 400B Lecture 3: Intro to Galaxies and Cosmology}
\author{Brant Robertson}
\date{January 5, 2015}


%include latex definitions

%average
\newcommand{\ave}[1]{\langle#1\rangle}
%radian
\newcommand{\rad}{\mathrm{rad}}

%astronomical unit
\newcommand{\AU}{\mathrm{AU}}

%centimeter
\newcommand{\cm}{\mathrm{cm}}

%meter
\newcommand{\m}{\mathrm{m}}

%kilometer
\newcommand{\km}{\mathrm{km}}

%parsec
\newcommand{\pc}{\mathrm{pc}}

%kiloparsec
\newcommand{\kpc}{\mathrm{kpc}}

%megaparsec
\newcommand{\Mpc}{\mathrm{Mpc}}

%gigaparsec
\newcommand{\Gpc}{\mathrm{Gpc}}

%light year
\newcommand{\ly}{\mathrm{ly}}

%second
\newcommand{\s}{\mathrm{s}}
\newcommand{\yr}{\mathrm{yr}}
\newcommand{\Gyr}{\mathrm{Gyr}}

%solar mass
\newcommand{\Msun}{M_{\odot}}

%grams
\newcommand{\g}{\mathrm{g}}

%erg
\newcommand{\erg}{\mathrm{erg}}

%solar luminosity
\newcommand{\Lsun}{L_{\odot}}

%jansky 
\newcommand{\Jy}{\mathrm{Jy}}

%flux density
\newcommand{\Fnu}{F_{\nu}}
\newcommand{\Flambda}{F_{\lambda}}
\newcommand{\Lnu}{L_{\nu}}

%hertz 
\newcommand{\Hz}{\mathrm{Hz}}

%angstrom
\newcommand{\Ang}{\mathrm{\r{A}}}

%solar radius
\newcommand{\Rsun}{R_{\odot}}

%stefan-boltzmann
\newcommand{\sigmaSB}{\sigma_{\mathrm{SB}}}

%boltzmann
\newcommand{\kB}{k_{\mathrm{B}}}

%kelvin
\newcommand{\K}{\mathrm{K}}

%T_bandpass
\newcommand{\TBP}{T_{\mathrm{BP}}}

%magnitude
\newcommand{\mg}{\mathrm{mag}}


%arcsecond
\newcommand{\arcsec}{\mathrm{arcsec}}

%begin the document
\begin{document}

%make the title, goes after document begins
\maketitle

%Celestial Coordinates
\section{Galaxy Classification}

Galaxies display a rich variety of morphologies, shapes, sizes, colors, and
luminosities. Based on these bulk properties, astronomers have tried to
develop classification schemes that group galaxies that may have similar
formation mechanisms. The most famous classification scheme is owes to 
{\it Edwin Hubble}, and is sometimes called the {\it tuning fork}.

{\bf See Figure 1.11 of Sparke and Gallagher}

The Hubble scheme separates galaxies primarily into {\it ellipticals}
(aka early type) and
{\it spirals} (aka late type), with an intermediate grouping
called {\it lenticulars}.

Ellipticals, also called {\it spheroidals}, are typically massive
galaxies with stars on non-circular orbits and with low gas content.
Ellipticals tend to be red, old, and lack large amounts of
on-going star formation.
The largest ellipticals, called cD galaxies, are located at the
center of clusters of galaxies. Ellipticals are classified with
E, followed by a number that reflects the ratio of the semi
major to semi minor axes.  There are also {\it dwarf elliptical}
(dE) and {\it dwarf spheroidal} (dSph) galaxies, that have lower
mass. We now think these classifications represent a mass sequence
rather than an evolutionary sequence.

Spirals are disk galaxies, typically containing blue, young stars,
on-going star formation and lots of gas.
Spiral galaxies are classified based on the prevalence of their
bulge, with Sa galaxies having large bulges and Sd galaxies having
very small bulges. We separately classify {\it barred spirals}
based on the presence of a noticeable bar feature.  Again, the
size of the bulge defines the range from SBa to SBd.  The Milky
Way is probably Sc or Sbc, while Andromeda is an Sb galaxy.
These classifications
also represent a mass sequence rather than an evolutionary one.
On the low mass end, there are Sm and SBm galaxies, as well as the
{\it dwarf irregular} (dIrr) systems that are star forming but
do not necessarily have well-defined disks. The ``m'' in Sm and
SBm refer to {\it Magellanic} spirals, with the prototype being
the LMC.

S0 galaxies (the lenticulars) are systems that are mostly a
bulge with a subdominant disk.

The early type vs. late type nomenclature was originally
mean to suggest an evolutionary sequence from ellipticals to
spirals. Instead, we know think that ellipticals form from
spirals in galaxy mergers that disrupt the well-defined 
circular orbits of spirals into the disordered orbital
structure of ellipticals.

\section{Galaxy Catalogues}

Very influential in the discovery and naming of large
nearby galaxies were
\begin{enumerate}

\item The Charles {\it Messier} catalogue from 1784 of 109 
objects that looked ``fuzzy'' in small telescopes. Some
of these objects were nebulae, some were globular clusters,
and some were galaxies. Famous Messier objects include the
Andromeda galaxy M31, the Triangulum galaxy M33, the M51
disk galaxy merging with a spheroidal, and the M101 Pinwheel
Galaxy.

\item In 1888, 1895, and 1908, the {\it New General Catalogue}
contained more than 7000 extended objects including many
galaxies. The catalogue was created by Dreyer, and Caroline,
William, and John Herschel.  The NGC catalogue overlaps with
the Messier catalogue (e.g., Andromeda is NGC 224).

\item Other modern catalogues include the {\it Third Reference
Catalogue} by de Vaucouleurs and the {\it Uppsala General
Catalogue} by Nilson that enumerates many galaxies in the UGC
system.
\end{enumerate}

\section{Galaxy Surface Brightness}

A galaxy with luminosity $L$ at a distance $d$ away will have
a flux $F = L/(4\pi d^2)$. If the galaxy has a physical
size $\Delta x$ in the plane of the sky and an angular
size $\theta = \Delta x/d$, then the surface brightness
$I$ will be
\begin{equation}
I \equiv \frac{F}{\theta^{2}} = \frac{L/(4\pi d^2)}{\Delta x^2 / d^2} = \frac{L}{4\pi\Delta x^2}
\end{equation}
\noindent
Unless cosmological effects come into play, the surface brightness
is independent of distance $d$. The units of surface brightness are
often given in $\mg~\arcsec^{-2}$, such that a square arcsecond area
of the galaxy has the apparent brightness of an object with the
same magnitude. Sensible astronomers also use $\Lsun\pc^{-2}$. Central
regions of galaxies reach $I_{B}\approx 18~\mg~\arcsec^{-2}$, while
the outer regions of disks are typically $I_{B}\approx25~\mg~\arcsec{-2}$.
The radius of the isophote of 25th B-band magnitude, $R_{25}$ is used as
a proxy for galaxy size, as is the {\it Holmberg radius} at the 26.5th
magnitude isophote.

\section{Sky Brightness}





\end{document}