\documentclass[]{article}
\usepackage[margin=1.0in]{geometry}
\usepackage{amssymb}

%title material
\title{Astronomy 400B Midterm Review}
\author{Brant Robertson}
\date{March, 2015}


%include latex definitions

%average
\newcommand{\ave}[1]{\langle#1\rangle}
%radian
\newcommand{\rad}{\mathrm{rad}}

%astronomical unit
\newcommand{\AU}{\mathrm{AU}}

%centimeter
\newcommand{\cm}{\mathrm{cm}}

%meter
\newcommand{\m}{\mathrm{m}}

%kilometer
\newcommand{\km}{\mathrm{km}}

%parsec
\newcommand{\pc}{\mathrm{pc}}

%kiloparsec
\newcommand{\kpc}{\mathrm{kpc}}

%megaparsec
\newcommand{\Mpc}{\mathrm{Mpc}}

%gigaparsec
\newcommand{\Gpc}{\mathrm{Gpc}}

%light year
\newcommand{\ly}{\mathrm{ly}}

%second
\newcommand{\s}{\mathrm{s}}
\newcommand{\yr}{\mathrm{yr}}
\newcommand{\Gyr}{\mathrm{Gyr}}

%solar mass
\newcommand{\Msun}{M_{\odot}}

%grams
\newcommand{\g}{\mathrm{g}}

%erg
\newcommand{\erg}{\mathrm{erg}}

%solar luminosity
\newcommand{\Lsun}{L_{\odot}}

%jansky 
\newcommand{\Jy}{\mathrm{Jy}}

%flux density
\newcommand{\Fnu}{F_{\nu}}
\newcommand{\Flambda}{F_{\lambda}}
\newcommand{\Lnu}{L_{\nu}}

%hertz 
\newcommand{\Hz}{\mathrm{Hz}}

%angstrom
\newcommand{\Ang}{\mathrm{\r{A}}}

%solar radius
\newcommand{\Rsun}{R_{\odot}}

%stefan-boltzmann
\newcommand{\sigmaSB}{\sigma_{\mathrm{SB}}}

%boltzmann
\newcommand{\kB}{k_{\mathrm{B}}}

%kelvin
\newcommand{\K}{\mathrm{K}}

%T_bandpass
\newcommand{\TBP}{T_{\mathrm{BP}}}

%magnitude
\newcommand{\mg}{\mathrm{mag}}


%arcsecond
\newcommand{\arcsec}{\mathrm{arcsec}}

%begin the document
\begin{document}

%make the title, goes after document begins
\maketitle

%first section
\section{Chapter 1}

\subsection{Stefan-Boltzmann Law}

Stellar masses are often expressed in terms of a {\it solar mass} ($\Msun$; mass of the Sun)
\begin{equation}
1\Msun = 1.9891\times10^{33}\g
\end{equation}
\noindent
in grams ($\g$). The most massive stars are $\approx100\Msun$ while the
least massive stars are $\approx0.075\Msun$.

Stellar luminosities (ergs of energy emitted per second; $1\erg = 1~\g~\cm^{2}~\s^{-2}$) are also 
often expressed in terms of the total (bolometric) {\it solar luminosity} ($\Lsun$; luminosity of the Sun)
\begin{equation}
1\Lsun = 3.846\times10^{33} \erg~\s^{-1}
\end{equation}
\noindent
Stars range in luminosity from $10^{6}\Lsun$ to less than $10^{-4}\Lsun$.

The {\it flux} $F$ is the energy per unit second per unit area ($\erg~\s^{-1}~\cm^{-2}$)
that is received from an object a distance $d$ away, and is
given by the {\it inverse square law}
\begin{equation}
\label{eqn:inverse_square}
F = \frac{L}{4\pi d^{2}}.
\end{equation}
\noindent

The {\it flux density} ($\Fnu \equiv dF/d\nu$) of an object is best measured in {\it janskys}
\begin{equation}
1\Jy = 10^{-23} \erg~\s^{-1}~\cm^{-2}~\Hz^{-1}.
\end{equation}
\noindent
Note that the {\it hertz} ($1\Hz = 1 s^{-1}$) is the unit of frequency $\nu$.
Astronomers will also use a flux density ($\Flambda = dF/d\lambda$) 
defined relative to wavelength. Typically, the units of $\Flambda$
are in $\erg~\cm^{-2}~\s^{-1}~\Ang^{-1}$, where $\Ang$ is the {\it angstrom}
($1\Ang = 10^{-8}\cm$). The wavelength and frequency of light are related by
$c = \nu \lambda$, and the two flux densities are therefore 
related by $\Flambda = (c/\lambda^2)\Fnu$.
The total flux is related to the flux density by
\begin{equation}
F = \int \Fnu d \nu = \int \Flambda d\lambda.
\end{equation}
\noindent
If the distance of an object is known, astronomers will also use the term
{\it luminosity density} (e.g., $\Lnu = dL/d\nu$).

The luminosity, radius, and temperature of an object are related. For stellar
objects, we often use the {\it solar radius} ($\Rsun$) to express sizes
\begin{equation}
1\Rsun = 6.995 \times 10^{10} \cm
\end{equation}
\noindent
If the luminosity $L$ and radius $R$ of an object are known, we can define
the effective temperature $T$ in {\it Kelvin} ($K$)
of an object through the {\it Stefan-Boltzmann Law}
\begin{equation}
L = 4\pi R^{2}\sigmaSB T^{4}
\end{equation}
\noindent
where the {\it Stefan-Boltzmann constant} is
\begin{equation}
\sigmaSB = 5.670373\times10^{-5} \erg~\cm^{-2}~\s^{-1}~\K^{-4}.
\end{equation}
The Stefan-Boltzmann constant can be computed theoretically in
terms of the speed of light, the {\it Boltzmann constant}
\begin{equation}
\kB = 1.380658\times10^{-16} \erg~\K^{-1}
\end{equation}
\noindent
and the {\it Planck constant}
\begin{equation}
h = 6.6260755\times10^{-27}\erg~\s,
\end{equation}
\noindent
which gives the definition
\begin{equation}
\sigmaSB \equiv \frac{2\pi^5 \kB^4}{15 h^3 c^2}.
\end{equation}

The effective temperature is typically a measure of the temperature
of the star at its {\it photosphere}, or the radius where the optical
depth of the star's atmosphere is $\tau\approx1$ and it becomes opaque. 
For the Sun,
$T\approx5780\K$. The temperature is also related to the wavelength
at the peak of the black body curve via {\it Wien's Displacement Law}
\begin{equation}
\lambda_{\mathrm{max}} = \frac{2.897756\times10^{7} \Ang~\K}{T},
\end{equation}
which gives $\lambda_{\mathrm{max}}\approx5000\Ang$ (yellow)
for the Sun.

\subsection{The Milky Way}

The Milky Way is a fantastic galaxy, and not just because we live there.
It has a terrific richness of structure and complexity, and simply put we
currently do not know how such a galaxy forms.

The main structures of the Milky Way are ({\bf See Figure 1.8 of Sparke and Gallagher})

\begin{enumerate}


\item The dark matter halo, consisting of (presumably) subatomic particles that
participate only in weak (Weak, gravitational force) interactions. The mass of
the Milky Way halo is about $10^{12}\Msun$. The radius of the dark matter halo
is roughly $300\kpc$. The dark matter halo is thought to obey a roughly broken
power-law density profile, with $\rho\propto r^{-1}$ within about $30\kpc$ and
$\rho\propto r^{-3}$ in the exterior.

\item The stellar halo, consisting of old, metal-poor stars and globular clusters
(old clusters of $\sim10^{5}-10^{6}$ stars). The stellar halo is only about
$10^{9}\Msun$.

\item The central bulge of the galaxy is a pseudospheroidal distribution of
stars with a luminosity of $L\approx 5\times 10^{9}\Lsun$ 
and mass $2\times10^{10}\Msun$.

\item The supermassive black hole at the center of the Milky Way, which has a
mass of about $\approx4.1\times10^{6}\Msun$.

\item The stellar disk of the galaxy is roughly exponential, such that
the surface density scales as $\Sigma(r) \propto \exp(-r/h)$, 
with a scale length $h\approx3\kpc$. The total luminosity of the disk
is about $2\times10^{10}\Lsun$ and with a mass in stars of
about $6\times10^{10}\Msun$. The stellar disk has a {\it thin disk} 
component with
a scale height of about $300\pc$ containing $95\%$ of the mass, and a
{\it thick disk} containing about $5\%$ with a scale height of about $1\kpc$.

\item The gaseous disk of the galaxy is a thin layer about a $100\pc$ thick
consisting mostly of neutral hydrogen (HI) and molecular hydrogen (H$_2$) gas.
The gaseous disk also contains a warm ionized and hot ionized interstellar medium 
(ISM; see below).
The gaseous disk is also very dusty!

\end{enumerate}


\subsection{Galaxy Luminosity Function}

We can count the number density of galaxies as a function of their luminosity,
and we appropriately call this distribution the {\it luminosity function}. The
galaxy luminosity function has been found to have a shape close to a parameterized
form called the {\it Schechter} function (after Paul Schechter).  The Schechter
function provides the number density of galaxies in a differential luminosity bin
$dL$ as
\begin{equation}
\label{eqn:schechter_function_luminosity}
\Phi(L)dL = \phi_{\star} \left(\frac{L}{L_{\star}}\right)^{\alpha}\exp\left(-\frac{L}{L_{\star}}\right)\frac{dL}{L_{\star}}
\end{equation}
\noindent
where $L_{\star}$ is a characteristic luminosity of galaxies and $\phi_{\star}$ is 
a typical abundance.  Below $L_\star$, the luminosity function is a power law, and
above $L_\star$ the abundance of galaxies drops exponentially. Sometimes, astronomers
will use $1+\alpha$ as the power law exponent, so beware!

{\bf See Figure 1.16 of Sparke and Gallagher.}

It happens to be the case that $L_{\star}\approx2\times10^{10}\Lsun$, which is close
to the luminosity of the Milky Way. The typical abundance of galaxies is $\phi_\star\approx7\times10^{-3}\Mpc^{-3}$. The faint-end slope of the 2DF luminosity function is $\alpha=-0.46$.
Defined as in Equation \ref{eqn:schechter_function_luminosity}, the number of galaxies
diverges as $L\to0$ if $\alpha<-1$.

The total luminosity density provided by galaxies can be found by integrating 
Equation \ref{eqn:schechter_function_luminosity} as
\begin{equation}
\rho_{L} = \int_{0}^{\infty} \Phi(L) L dL = \phi_\star L_{\star} \Gamma(\alpha + 2)
\end{equation}
\noindent
where $\Gamma$ is the Gamma function, which for an integer $n$ is $\Gamma(n) = (n-1)!$.
It turns out that $\Gamma(1.5)\approx0.886227$, so we have that 
$\rho_{L}\approx1.25\times10^{8}\Lsun\Mpc^{-3}$.

For a partial integral, we have
\begin{equation}
\rho_{L}(L>L_{\mathrm{min}}) = \int_{L_{\mathrm{min}}}^{\infty} \Phi(L) L dL = \phi_\star L_{\star} \Gamma(\alpha + 2, L_{\mathrm{min}}/L_{\star}),
\end{equation}
\noindent
where $\Gamma(a,x)$ is the incomplete gamma function.


\section{Components of the Universe}

Our universe is comprised of a variety of forms of matter and energy.  Owing to
general relativity, both matter and energy provide a source for gravity and so
both effect the dynamical expansion of the universe.  Here is a brief run-down
of the different types of matter and energy in the universe:

\begin{enumerate}
\item {\it Baryonic (normal) matter}. We are made of {\it baryonic} matter, meaning
that we are comprised of forms of matter that are ultimately made from three quarks
(proton and neutrons). There are also enough electrons such that the universe is
electrically neutral, but their mass is $\sim2000\times$ less than the nucleons.
Surprising as it may seem, the universe is only about $4-5\%$ baryonic matter!
\item {\it Radiation} There are a {\bf lot} of photons in the universe! The energy
density associated with the total cosmic microwave background and the extragalactic
background light is quite small today ($<10^{-4}$ of the total energy density). At
early times the universe was actually dominated by radiation, but the fractional
contribution of radiation to the total energy density declines dramatically with
time.
\item {\it Dark matter}. Most of the matter in the universe is dark matter, which
comprises about $\sim25\%$ of the total energy density. Dark matter is non-baryonic,
and likely consists of subatomic particles called {\it Weakly Interacting Massive Particles},
or WIMPs, that only interact through the gravitational and possibly weak nuclear forces.
There are plenty of dark matter candidates from elementary particle theories, but dark
matter has not yet been detected directly via experiment.
\item {\it Dark energy}. This mysterious component of the universe as a negative
equation of state associated with it, such that a universe dominated by dark
energy accelerates its expansion as its volume increases. We think that about $70\%$
of the universal energy density is comprised of dark energy, and it is possibly in
the form of a ``cosmological constant'', meaning that the dark energy density remains
constant with time. Since the universe is expanding and the matter density declines
with volume, the universe is becoming progressively more and more dark energy dominated
as it expands. The result is that the universal expansion will continue to accelerate,
likely without a meaningful bound (the expansion will infact eventually exceed the speed
of light if the dark energy is a cosmological constant). There are natural ways of
estimating the energy density associated with a cosmological constant from fundamental
particle physics, but these methods overestimate the amount of dark energy by $\sim120$
{\it orders of magnitude}(!; depending on how you count). Needless to say, even if the
dark energy is a cosmological constant, we don't really know how to describe it in terms
of a physical model.
\item {\it Curvature}. If the universe does not have the critical density, then there
is a geometric curvature to the universe. There is an energy density associated with this
curvature that can affect the dynamics of the universe, and if the universe is not flat then
this curvature energy must reckon in the accounting for the time-dependent expansion rate.  
Fortunately (I guess), we think the universe is very close to
flat, remarkably so, and most calculations will be performed assuming the curvature energy
is zero.
\end{enumerate}

\section{Chapter 2}

\subsection{Dark Matter}

For a spherical mass distribution, the velocity of
a circular orbit related to the interior mass by
\begin{equation}
V^2 = \frac{GM(<R)}{R}
\end{equation}
\noindent
If the circular velocity is flat in the outer 
galaxy, we would infer that
\begin{equation}
M(<R) \propto R
\end{equation}
\noindent
Clearly, the exponential disk cannot supply
this mass growth with radius.
But we know how to relate $M(<R)$ to a density
profile through the integral
\begin{equation}
M(<R) = 4\pi\int_0^{R}\rho(r)r^{2}dr
\end{equation}
\noindent
Further if we assume $\rho(r)\propto r^{\alpha}$,
then
\begin{equation}
M(<R) \propto R \propto R^{3 + \alpha}
\end{equation}
\noindent
So we infer that a spherical density profile would 
be declining at $\rho(r)\propto r^{-2}$ in the
regime where the mass was growing linearly and
the circular velocity is flat.

That mass grows with radius implies there is a {\it lot}
of mass in the exterior of the galaxy.


\section{Chapter 3}

\subsection{Continuous Matter Distributions}
Now consider a continuous distribution of matter density $\rho(\vx)$.  The
potential generated by $\rho(\vx)$ is given by
\begin{equation}
\label{eqn:potential}
\Phi(\vx) = - \int \frac{G\rho(\vx')}{|\vx - \vx'|}d^{3}\vx'
\end{equation}
\noindent
Note that the integral is performed over $\vx'$.  The force $\vF$ per unit mass
is
\begin{equation}
\vF(\vx) = - \nabla \Phi(\vx) = - \int \frac{G\rho(\vx')(\vx-\vx')}{|\vx-\vx'|^{3}}d^{3}\vx'
\end{equation}

\subsection{Poisson's Equation}

Take Equation \ref{eqn:potential} and apply the Laplacian operator 
\begin{equation}
\nabla^{2} \equiv \nabla \cdot \nabla = \left[\frac{\partial^{2}}{\partial x^{2}}+\frac{\partial^{2}}{\partial y^{2}}+\frac{\partial^{2}}{\partial z^{2}}\right]
\end{equation}
\noindent
to both sides.  Remembering that the operator acts on $\vx$ and not $\vx'$, we have
\begin{equation}
\label{eqn:pois_init}
\nabla^{2} \Phi(\vx) = - \int G \rho(\vx') \nabla^{2} \left(\frac{1}{|\vx-\vx'|}\right)d^{3}\vx'.
\end{equation}
\noindent
We can evaluate this by noting that
\begin{equation}
\label{eqn:nabla}
\nabla\left(\frac{1}{|\vx - \vx'|}\right) = -\frac{\vx-\vx'}{|\vx-\vx'|^3}, \nabla^{2}\left(\frac{1}{|\vx - \vx'|}\right) = 0.
\end{equation}
\noindent
So we conclude that outside of a very small region around $\vx$, $\nabla^{2}\Phi(\vx)=0$.
Let's take a spherical region $S(\epsilon)$ of radius $\epsilon$ centered on $\vx$. In proceeding, let's 
note that 
\begin{equation}
\nabla^{2} f(|\vx-\vx'|) = \nabla_{\vx'}^{2} f(|\vx-\vx'|)
\end{equation}
\noindent
for any function $f(|\vx - \vx'|)$. If we take $\epsilon$ to be small enough such 
that $\rho(\vx)\approx$ a constant, then we can write
\begin{eqnarray}
\label{eqn:pois_med}
\nabla^{2}\Phi(\vx) &\approx& - G\rho(\vx) \int_{S(\epsilon)} \nabla^2 \left( \frac{1}{|\vx - \vx'|} \right)d^{3}\vx' \nonumber \\
&=& - G \rho(\vx)  \int_{S(\epsilon)} \nabla_{\vx'}^2 \left( \frac{1}{|\vx - \vx'|} \right)d V'.
\end{eqnarray}
\noindent
Now we get to use the {\it divergence} theorem
\begin{equation}
\int \nabla^{2} f dV = \oint \nabla f \cdot dS,
\end{equation}
\noindent
which allows us to write Equation \ref{eqn:pois_med} as
\begin{equation}
- G \rho(\vx)  \int_{S(\epsilon)} \nabla_{\vx'}^2 \left( \frac{1}{|\vx - \vx'|} \right)d V' = - G \rho(\vx) \oint_{S(\epsilon)} \nabla_{\vx'}\left(\frac{1}{|\vx-\vx'|}\right) \cdot d\vS'
\end{equation}
\noindent
By applying Equation \ref{eqn:nabla} and the identity $\nabla_{\vx'} f = -\nabla f$, we have
\begin{eqnarray}
- G \rho(\vx) \oint_{S(\epsilon)} \nabla_{\vx'}\left(\frac{1}{|\vx-\vx'|}\right) \cdot d\vS' &=&
-G \rho(\vx) \oint_{S(\epsilon)} \left(\frac{\vx-\vx'}{|\vx-\vx'|^{3}}\right) \cdot d\vS' \nonumber \\
\therefore \nabla^2 \Phi &=& 4 \pi G \rho(\vx)
\end{eqnarray}



\section{Distribution Function}

The {\it distribution function} $f(\vx,\vv,t)$ gives the probability density
in six-dimensional {\it phase space} $(\vx,\vv)$ of having an object
in the volume $d\vx d\vv$.  The number density $n(\vx,t)$ of objects is the
volume integral of the distribution function
\begin{equation}
n(\vx,t) = \int_{-\infty}^{\infty}\int_{-\infty}^{\infty}\int_{-\infty}^{\infty} f(\vx,\vv,t) dv_{x}dv_y,dv_z.
\end{equation}
\noindent
We can use this expression to define moments of the velocity distribution, such as
the average velocity 
\begin{equation}
\ave{\vv(\vx,t)} = \frac{1}{n(\vx,t)} \int_{-\infty}^{\infty}\int_{-\infty}^{\infty}\int_{-\infty}^{\infty} \vv f(\vx,\vv,t) dv_{x}dv_y,dv_z.
\end{equation}


\section{Chapter 4}

\subsection{A Simple Example of the Tidal Limit}

Consider the satellite to have mass $m$ and the main galaxy to have mass $M$, separated by distance $D$,
orbiting the center of mass $C$ with angular speed $\Omega$.  Take $x$ as the coordinate
along the ray between $m$ toward $M$.  The center of mass is then at $x=DM/(M+m)$.  The
effective potential is then
\begin{equation}
\Phieff(x) = -\frac{GM}{|D-x|} -\frac{Gm}{|x|} - \frac{\Omega^2}{2}\left(x-\frac{DM}{M+m}\right)^2.
\end{equation}

The effective potential has three maxima known as the {\it Lagrange points}, [$L_1$, $L_2$, $L_3$].  Let's find them by setting
the derivative of the effective potential to zero.
\begin{equation}
\label{eqn:find_phieff_max}
\frac{\partial\Phieff}{\partial x} = 0 = -\frac{GM}{(D-x)^2} \pm \frac{Gm}{x^2} - \Omega^2\left(x-\frac{DM}{M+m}\right)
\end{equation}
\noindent
The acceleration $\Omega^2DM/(M+m)$ of $m$ as it circles $C$ owes to the gravitational
attraction of $M$. We can then write
\begin{equation}
\Omega\frac{DM}{M+m}=\frac{GM}{D^2},~~\mathrm{so}~\Omega^2=\frac{G(M+m)}{D^3}
\end{equation}
\noindent
If the satellite is much less massive than the main galaxy, $L_1$ and $L_2$ will lie close to $m$.
Substitute $\Omega^2$ into Equation \ref{eqn:find_phieff_max}, and expand in powers of $x/D$ to find
\begin{equation}
0 \approx -\frac{GM}{D^2} - 2\frac{GM}{D^3}x \pm \frac{Gm}{x^2} - \frac{G(M+m)}{D^3}\left(x-\frac{DM}{M+m}\right)
\end{equation}
\noindent
At the Lagrange points $L_1$ and $L_2$ we have
\begin{equation}
x = \pm r_J, ~~\mathrm{where}~r_J = D\left(\frac{m}{3M+m}\right)^{1/3}
\end{equation}
\noindent
Stars that cannot stray further from the satellite than $r_J$, the {\it Jacobi radius} or {\it Roche limit},
will remain bound to it.  The radius $L_1$ is not where the gravitational force from the satellite and 
main galaxy are the same, but where the effective potential has a minimum and lies further from the satellite.
This radius is also where expanding stars lose mass to their companion.

When $M\gg m$, then the mean density within $r_J$ is three times the mean density within $D$ of the main galaxy.
A star orbiting the satellite near $r_J$ will have an orbital period comparable to the orbit period of the
satellite around the main galaxy.

When satellites are not on circular orbits, the relevant $r_J$ is determined at the pericenter. If the
satellite is orbiting within the dark matter halo of the main galaxy, the relevant radius is
\begin{equation}
r_J = D\left[\frac{m}{2M(<D)}\right]^{1/3}
\end{equation}
where $M(<D)$ is the mass within D.

What's the Jacobi radius of the LMC-Milky Way system? The LMC is at a distance of $\sim50\kpc$ where
the speed of a circular orbit is about the same as it is at the solar circle, or about $\sim200\km~\s^{-1}$.
The mass of the Milky Way within the LMC's orbit is about $5\times10^{11}\Msun$.  The LMC mass is about
$10^{10}\Msun$, so we have
\begin{equation}
r_J\approx 50\kpc~\times~\left(\frac{10^{10}\Msun}{2\times5\times10^{11}\Msun}\right)^{1/3} \approx 11\kpc.
\end{equation}
The LMC disk lies within this radius, but the SMC is too far away to stay bound to the LMC.


\section{Chapter 5}

\subsection{Rotation Curve}

For a galaxy at a systemic velocity $V_{sys}$ inclined at an angle $i$ to face-on, the rotation curve we measure as a function of radial distance $R$ and azimuth $\phi$ is
\begin{equation}
V_r(R,i) = V_{sys} + V(R)\sin i \cos \phi
\end{equation}
\noindent
so to determine $V(R)$ we must determine the inclination and perhaps determine an angular average over azimuth.

In determining the rotation curve, we are often trying to weigh the galaxy.  This is straightforward to do in
some limiting cases.

For a thin exponential disk that supplies its own gravity, the rotation curve can be written in terms of Bessel
functions as
\begin{equation}
V^2(R) = 4\pi G \Sigma_0 h_R y^2 [I_0(y) K_0(y) - I_1(y) K_1(y)]
\end{equation}
\noindent
where $\Sigma_0$ is the central mass surface density, $y\equiv R/2h_R$,
 and $I$ and $K$ are modified Bessel functions (mind the subscript!). We can
relate this to the total disk mass $M_d = 2\pi\Sigma_0 h_R^2$, such that
\begin{equation}
V^2(R) = \frac{2GM_d}{h_R}f(y).
\end{equation}

Otherwise, if we examine the regions of a disk where the dark halo dominates the potential, than we can just use
the radial force equation
\begin{equation}
\frac{V^2(R)}{R} = \frac{GM(<R)}{R^2}
\end{equation}
\noindent 
to estimate the interior mass from the rotation curve.

\subsection{Tully-Fisher Relation}

In 1977, Tully and Fisher showed that the luminosity of a disk galaxy was correlated with a
power of the maximum rotational velocity.  They found that
\begin{equation}
\frac{L_I}{4\times10^{10} L_{I,\odot}} \approx \left(\frac{V_{max}}{200~\km~\s^{-1}}\right)^4
\end{equation}
\noindent
It turns out that this relationship is perplexing if the circular velocity is dominated
by the dark halo, and the disk and halo must have some connection to force $L\propto V^4$.

\end{document}